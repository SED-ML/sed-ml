%%%%%%%%%%%%%%%%%%%%%%%%%%%%%%%%%%%%%%%%%%%%%%%%%%%%%%%%%%%%%%%%%%
%%  Commands
%%%%%%%%%%%%%%%%%%%%%%%%%%%%%%%%%%%%%%%%%%%%%%%%%%%%%%%%%%%%%%%%%%

\newcommand{\code}[1]{\texttt{#1}}
\newcommand{\token}[1]{\texttt{#1}}
\newcommand{\concept}[1]{\textcolor{blue}{#1}}
\newcommand{\element}[1]{\texttt{#1}}
\newcommand{\alert}[1]{\textcolor{red}{#1}}
\newcommand{\note}[1]{\paragraph*{} \emph{\scshape{\alert{Please Note}}: #1} \newline}
\newcommand{\mailto}[1]   {\link{mailto:#1}{#1}}
\newcommand{\link}[2]     {\literalFont{\href{#1}{#2}}}
\newcommand{\literalFont}[1]{\textup{\texttt{#1}}}
\newcommand{\version}{4\xspace}
\newcommand{\level}{1\xspace}
\newcommand{\LoneVone}{Level~1 Version~1\xspace}
\newcommand{\LoneVtwo}{Level~1 Version~2\xspace}
\newcommand{\LoneVthree}{Level~1 Version~3\xspace}
\newcommand{\LoneVfour}{Level~1 Version~4\xspace}
\newcommand{\currentLV}{Level~1 Version~4\xspace}
\newcommand{\previousLV}{Level~1 Version~3\xspace}
\newcommand{\biom}{BioModels Database\xspace}
% attribute table layout
\newcommand{\attribute}{attribute\xspace}
\newcommand{\desc}{description\xspace}
\newcommand{\subelements}{sub-elements\xspace}

\newcommand{\SedModel}{\hyperref[class:model]{Model}\xspace}
\newcommand{\SedDataSource}{\hyperref[class:dataSource]{DataSource}\xspace}
\newcommand{\SedDataDescription}{\hyperref[class:dataDescription]{DataDescription}\xspace}
\newcommand{\SedSlice}{\hyperref[class:slice]{Slice}\xspace}

\newcommand{\refpage}[1]{\hyperref[#1]{page \pageref{#1}}} % to hyperref to a particular page in the spec
\newcommand{\tabcap}[1]{  % to create table captions for overview tables for each SED-ML class
Attributes and nested elements for \concept{#1}. \emph{$^{o}$}denotes optional elements and attributes.
}

\newcommand{\tabtext}[2]{ % to create the introducing table text for each table reference
\tab{#1}~shows all attributes and sub-elements for the \concept{#2} element. % as defined by the SED-ML \currentLV XML Schema.
}

\newcommand{\lsttext}[2]{ % to create the introducing listing text for each listing reference
  Listing~\vref{lst:#1} shows the use of the \element{#2} element. % in a SED-ML file.
}

\newcommand{\lsttexta}[2]{ % to create the introducing listing text for each listing reference
  Listing~\vref{lst:#1} shows the use of the \element{#2} attribute in a SED-ML file.}

%
\newcommand{\chap}[1]     {Chapter~\protect\ref{chap:#1}\xspace}
\newcommand{\sect}[1]     {Section~\protect\ref{sec:#1}\xspace}
\newcommand{\fig}[1]      {Figure~\protect\vref{fig:#1}\xspace}
\newcommand{\tab}[1]      {Table~\protect\vref{tab:#1}\xspace}
\newcommand{\lst}[1]      {Listing~\protect\ref{lst:#1}\xspace}
\newcommand{\eg}          {e.\,g.,\xspace}
\newcommand{\ie}          {i.\,e.,\xspace}

\newcommand{\tickYes}{\hspace{1pt}\ding{52}}
\newcommand{\tickNo}{\hspace{1pt}\ding{56}}

%%%%%%%%%%%%%%%%%%%%%%%%%%%%%%%%%%%%%%%%%%%%%%%%%%%%%%%%%%%%%%%%%%
%%  environments
%%%%%%%%%%%%%%%%%%%%%%%%%%%%%%%%%%%%%%%%%%%%%%%%%%%%%%%%%%%%%%%%%%

% standard figure layout
\newcommand{\sedfig}[4][]
	{\begin{figure}[H]\begin{center}{\includegraphics[width=0.9\textwidth,#1]{#2}}\caption{#3}\label{#4}\end{center}\end{figure}}

\newcommand{\sedfigX}[4][]
	{\begin{figure}[H]\begin{center}{\includegraphics[#1]{#2}}\caption{#3}\label{#4}\end{center}\end{figure}}

% standard XML listing layout
\lstnewenvironment{myXmlLst}[2]
	{\lstset{basicstyle=\ttfamily\scriptsize, caption={#1},label={#2}, keywordstyle=\color{blue}\bfseries, stringstyle=\color{blue}, commentstyle=\color{red}, captionpos=b, breaklines=true, xleftmargin=1.5em, xrightmargin=1.5em, numbers=left, numberstyle=\ttfamily\tiny, numbersep=5pt, tabsize=4, showstringspaces=false, language=XML}} %, float=!h
	{}

% listings in appendixes:
\newcommand{\myXmlImport}[3]	{\lstinputlisting[basicstyle=\ttfamily\scriptsize,caption={#1},label={#2},%
	keywordstyle=\color{blue}\bfseries, stringstyle=\color{blue}, commentstyle=\color{red}, captionpos=b, breaklines=true, xleftmargin=1.5em, xrightmargin=1.5em, numbers=left, numberstyle=\ttfamily\tiny, numbersep=5pt, tabsize=4, showstringspaces=false, language=XML, stepnumber=1]{#3}} %float=h!

\newcommand{\changedColor}{Maroon}

\newcommand{\changed}[1]{\textcolor{\changedColor}{#1}}
\newcommand{\val}[1]{``\token{#1}''}

\newcommand{\normal} {\hyperref[fun:normal]{normal}\xspace}
\newcommand{\lognormal} {\hyperref[fun:lognormal]{lognormal}\xspace}
\newcommand{\poisson} {\hyperref[fun:poisson]{poisson}\xspace}
\newcommand{\sedgamma} {\hyperref[fun:gamma]{gamma}\xspace}
\newcommand{\sedmin} {\hyperref[fun:min]{min}\xspace}
\newcommand{\sedmax} {\hyperref[fun:max]{max}\xspace}
\newcommand{\sedsum} {\hyperref[fun:sum]{sum}\xspace}
\newcommand{\product} {\hyperref[fun:product]{product}\xspace}
\newcommand{\sedcount} {\hyperref[fun:count]{count}\xspace}
\newcommand{\mean} {\hyperref[fun:mean]{mean}\xspace}
\newcommand{\stdev} {\hyperref[fun:stdev]{stdev}\xspace}
\newcommand{\variance} {\hyperref[fun:variance]{variance}\xspace}
\newcommand{\uniform} {\hyperref[fun:uniform]{uniform}\xspace}

\newcommand{\kisaoID} {\hyperref[sec:kisaoid]{\element{kisaoID}}\xspace}

\newcommand{\AxisType} {\hyperref[type:axisType]{AxisType}\xspace}
\newcommand{\CurveType} {\hyperref[type:curveType]{CurveType}\xspace}
\newcommand{\SurfaceType} {\hyperref[type:surfaceType]{SurfaceType}\xspace}
\newcommand{\SedColor} {\hyperref[type:sedColor]{SedColor}\xspace}
\newcommand{\LineType} {\hyperref[type:lineType]{LineType}\xspace}
\newcommand{\MarkerType} {\hyperref[type:markerType]{MarkerType}\xspace}
\newcommand{\SIdRef} {\hyperref[type:sidref]{SIdRef}\xspace}
\newcommand{\SId} {\hyperref[type:sid]{SId}\xspace}
\newcommand{\ExperimentType} {\hyperref[type:experimentType]{ExperimentType}\xspace}
\newcommand{\MappingType} {\hyperref[type:mappingType]{MappingType}\xspace}
\newcommand{\ScaleType} {\hyperref[type:scaleType]{ScaleType}\xspace}
\newcommand{\XPath} {\hyperref[sec:xpath]{XPath}\xspace}
\newcommand{\Target} {Target\xspace}


\newcommand{\AbstractCurve} {\hyperref[class:abstractCurve]{\emph{AbstractCurve}}\xspace}
\newcommand{\AbstractTask} {\hyperref[class:abstractTask]{\emph{AbstractTask}}\xspace}
\newcommand{\AddXML} {\hyperref[class:addXml]{AddXML}\xspace}
\newcommand{\AdjustableParameter} {\hyperref[class:adjustableParameter]{AdjustableParameter}\xspace}
\newcommand{\AdjustableParameters} {\hyperref[class:adjustableParameter]{AdjustableParameters}\xspace}
\newcommand{\AlgorithmParameter} {\hyperref[class:algorithmParameter]{AlgorithmParameter}\xspace}
\newcommand{\AlgorithmParameters} {\hyperref[class:algorithmParameter]{AlgorithmParameters}\xspace}
\newcommand{\Algorithm} {\hyperref[class:algorithm]{Algorithm}\xspace}
\newcommand{\Annotation} {\hyperref[class:annotation]{Annotation}\xspace}
\newcommand{\Axis} {\hyperref[class:axis]{Axis}\xspace}
\newcommand{\Bounds} {\hyperref[class:bounds]{Bounds}\xspace}
\newcommand{\Calculation} {\hyperref[class:calculation]{Calculation}\xspace}
\newcommand{\Change} {\hyperref[class:change]{Change}\xspace}
\newcommand{\ChangeXML} {\hyperref[class:changeXml]{ChangeXML}\xspace}
\newcommand{\ChangeAttribute} {\hyperref[class:changeAttribute]{ChangeAttribute}\xspace}
\newcommand{\ComputeChange} {\hyperref[class:computeChange]{ComputeChange}\xspace}
\newcommand{\CSV} {\hyperref[sec:dataFormatCSV]{CSV}\xspace}
\newcommand{\Curve} {\hyperref[class:curve]{Curve}\xspace}
\newcommand{\DataDescription} {\hyperref[class:dataDescription]{DataDescription}\xspace}
\newcommand{\DataGenerators} {\hyperref[class:dataGenerator]{DataGenerators}\xspace}
\newcommand{\DataGenerator} {\hyperref[class:dataGenerator]{DataGenerator}\xspace}
\newcommand{\DataRange} {\hyperref[class:dataRange]{DataRange}\xspace}
\newcommand{\DataSet} {\hyperref[class:dataSet]{DataSet}\xspace}
\newcommand{\DataSource} {\hyperref[class:dataSource]{DataSource}\xspace}
\newcommand{\DependentVariable} {\hyperref[class:dependentVariable]{DependentVariable}\xspace}
\newcommand{\ExperimentRef} {\hyperref[class:experimentRef]{ExperimentRef}\xspace}
\newcommand{\Figure} {\hyperref[class:figure]{Figure}\xspace}
\newcommand{\Fill} {\hyperref[class:fill]{Fill}\xspace}
\newcommand{\FitExperiments} {\hyperref[class:fitExperiment]{FitExperiments}\xspace}
\newcommand{\FitExperiment} {\hyperref[class:fitExperiment]{FitExperiment}\xspace}
\newcommand{\FitMapping} {\hyperref[class:fitMapping]{FitMapping}\xspace}
\newcommand{\FitMappings} {\hyperref[class:fitMapping]{FitMappings}\xspace}
\newcommand{\FunctionalRange} {\hyperref[class:functionalRange]{FunctionalRange}\xspace}
\newcommand{\HDF} {\hyperref[sec:dataFormatHDF5]{HDF5}\xspace}
\newcommand{\LeastSquareObjectiveFunction} {\hyperref[class:leastSquareObjectiveFunction]{LeastSquareObjectiveFunction}\xspace}
\newcommand{\Line} {\hyperref[class:line]{Line}\xspace}
\newcommand{\ListOfAdjustableParameters} {\hyperref[class:listOfAdjustableParameters]{ListOfAdjustableParameters}\xspace}
\newcommand{\ListOfAlgorithmParameters} {\hyperref[class:listOfAlgorithmParameters]{ListOfAlgorithmParameters}\xspace}
\newcommand{\ListOfChanges} {\hyperref[sec:changesModel]{ListOfChanges}\xspace}
\newcommand{\ListOfCurves} {\hyperref[class:listOfCurves]{ListOfCurves}\xspace}
\newcommand{\ListOfDataSets} {\hyperref[class:listOfDataSets]{ListOfDataSets}\xspace}
\newcommand{\ListOfExperimentRefs} {\hyperref[class:listOfExperimentRefs]{ListOfExperimentRefs}\xspace}
\newcommand{\ListOfFitExperiments} {\hyperref[class:listOfFitExperiments]{ListOfFitExperiments}\xspace}
\newcommand{\ListOfFitMappings} {\hyperref[class:listOfFitMappings]{ListOfFitMappings}\xspace}
\newcommand{\ListOfParameters} {\hyperref[class:listOfParameters]{ListOfParameters}\xspace}
\newcommand{\ListOfRemainingDimensions} {\hyperref[class:listOfRemainingDimensions]{ListOfRemainingDimensions}\xspace}
\newcommand{\ListOfStyles} {\hyperref[class:listOfStyles]{ListOfStyles}\xspace}
\newcommand{\ListOfSubPlots} {\hyperref[class:listOfSubPlots]{ListOfSubPlots}\xspace}
\newcommand{\ListOfSurfaces} {\hyperref[class:listOfSurfaces]{ListOfSurfaces}\xspace}
\newcommand{\ListOfTasks} {\hyperref[class:listOfTasks]{ListOfTasks}\xspace}
\newcommand{\ListOfVariables} {\hyperref[class:listOfVariables]{ListOfVariables}\xspace}
\newcommand{\Marker} {\hyperref[class:marker]{Marker}\xspace}
\newcommand{\Math} {\hyperref[sec:math]{Math}\xspace}
\newcommand{\Models} {\hyperref[class:model]{Models}\xspace}
\newcommand{\Model} {\hyperref[class:model]{Model}\xspace}
\newcommand{\Notes} {\hyperref[class:notes]{Notes}\xspace}
\newcommand{\Objective} {\hyperref[class:objective]{Objective}\xspace}
\newcommand{\Output} {\hyperref[class:output]{\emph{Output}}\xspace}
\newcommand{\ParameterEstimationReport} {\hyperref[class:parameterEstimationReport]{ParameterEstimationReport}\xspace}
\newcommand{\ParameterEstimationResultsPlot} {\hyperref[class:parameterEstimationResultsPlot]{ParameterEstimationResultsPlot}\xspace}
\newcommand{\ParameterEstimationTask} {\hyperref[class:parameterEstimationTask]{ParameterEstimationTask}\xspace}
\newcommand{\Parameter} {\hyperref[class:parameter]{Parameter}\xspace}
\newcommand{\PlotThree} {\hyperref[class:plot3D]{Plot3D}\xspace}
\newcommand{\PlotTwo} {\hyperref[class:plot2D]{Plot2D}\xspace}
\newcommand{\Plot} {\hyperref[class:plot]{\emph{Plot}}\xspace}
\newcommand{\Range} {\hyperref[class:range]{Range}\xspace}
\newcommand{\RemainingDimension} {\hyperref[class:remainingDimension]{RemainingDimension}\xspace}
\newcommand{\RemoveXML} {\hyperref[class:removeXml]{RemoveXML}\xspace}
\newcommand{\RepeatedTask} {\hyperref[class:repeatedTask]{RepeatedTask}\xspace}
\newcommand{\Report} {\hyperref[class:report]{Report}\xspace}
\newcommand{\SEDBase} {\hyperref[class:sedBase]{\emph{SEDBase}}\xspace}
\newcommand{\SedBase} {\hyperref[class:sedBase]{\emph{SEDBase}}\xspace}
\newcommand{\SedDocument} {\hyperref[class:sed-ml]{SED-ML Document}\xspace}
\newcommand{\SedML} {\hyperref[class:sed-ml]{SedML}\xspace}
\newcommand{\SetValue} {\hyperref[class:setValue]{SetValue}\xspace}
\newcommand{\ShadedArea} {\hyperref[class:shadedArea]{ShadedArea}\xspace}
\newcommand{\SimpleRepeatedTask} {\hyperref[class:simpleRepeatedTask]{SimpleRepeatedTask}\xspace}
\newcommand{\Simulations} {\hyperref[class:simulation]{Simulations}\xspace}
\newcommand{\Simulation} {\hyperref[class:simulation]{Simulation}\xspace}
\newcommand{\Slice} {\hyperref[class:slice]{Slice}\xspace}
\newcommand{\Style} {\hyperref[class:style]{Style}\xspace}
\newcommand{\SubPlot} {\hyperref[class:subPlot]{SubPlot}\xspace}
\newcommand{\SubTask} {\hyperref[class:subTask]{SubTask}\xspace}
\newcommand{\Surface} {\hyperref[class:surface]{Surface}\xspace}
\newcommand{\Task} {\hyperref[class:task]{Task}\xspace}
\newcommand{\UniformTimeCourse} {\hyperref[class:uniformTimeCourse]{UniformTimeCourse}\xspace}
\newcommand{\UniformRange} {\hyperref[class:uniformRange]{UniformRange}\xspace}
\newcommand{\Variables} {\hyperref[class:variable]{Variables}\xspace}
\newcommand{\Variable} {\hyperref[class:variable]{Variable}\xspace}
\newcommand{\VectorRange} {\hyperref[class:vectorRange]{VectorRange}\xspace}
\newcommand{\WaterfallPlot} {\hyperref[class:waterfallPlot]{WaterfallPlot}\xspace}
%\newcommand{\Variable} {\hyperref[class:variable]{Variable}\xspace}


\newenvironment{blockChanged}{\color{\changedColor}}{\color{black}}


\def\sedmltableofcontents{%
  \if@notoc
  \else
    \begingroup
      \small%
      % Tighten spacing between lines within an entry.
      % This assumes 10 pt font!
      \setlength{\baselineskip}{9.7pt}%
      % Now adjust inter-entry spacing.
      \addtolength{\parskip}{-1.35 ex}%
      \if@twocolumntoc
        \setlength{\columnsep}{16pt}%
        \begin{multicols}{2}
      \fi
      \tableofcontents%
      \if@twocolumntoc
        \end{multicols}
      \fi
      \normalsize%
      \addtolength{\parskip}{1.45 ex}%
    \endgroup
  \fi}



%%% Local Variables: 
%%% mode: latex
%%% TeX-master: "../sed-ml-L1V3"
%%% End: 


