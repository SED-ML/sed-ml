\subsection{\element{DataSource}}
% datasource class
\label{class:dataSource}
The \concept{DataSource} class (\fig{sedDataSource}) extracts chunks out of the data file provided by the outer \SedDataDescription element. 

The \concept{DataSource} class introduces three attributes: the required attribute \token{id} and the optional attribute \token{indexSet} and \token{name}. Additionally the \concept{listOfSlices} element is defined. 

% Fig: sed model
\sedfig[width=0.9\textwidth]{pdf/dataSourceClass}{The SED-ML DataSource class}{fig:sedDataSource}
%

\tabtext{dataSource}{dataSource}
%
\begin{table}[ht]
\center
\begin{tabular}{|l|l|}
\hline
\textbf{\attribute} & \textbf{\desc}\\
\hline
metaid$^{o}$ & \refpage{sec:metaID}\\
id & \refpage{sec:id} \\
name$^{o}$ & \refpage{sec:name}\\
\hline
indexSet & \refpage{sec:indexSet}\\
\hline
\hline
\textbf{\subelements} & \textbf{\desc}\\
\hline
notes$^{o}$ & \refpage{class:notes}\\
annotation$^{o}$ & \refpage{class:annotation}\\
\hline
listOfSlices$^{o}$ & \refpage{class:listOfSlices}\\
\hline
\end{tabular}
\caption{\tabcap{dataSource}}
\label{tab:dataSource}
\end{table}
%

\subsubsection{The \token{id} and \token{name} attributes}
The attribute \token{id} of type \token{SId} is meant to uniquely identify the \token{dataSource} element, while the optional \token{name} attribute of type \token{string}, is there to provide a human readable representation if desired.

\subsubsection{The \token{indexSet} attribute}
\label{sec:indexSet}
Since data elements in NuML are either values, or indices, the \SedDataSource element needs two ways of addressing those elements. The \token{indexSet} attribute allows to address all indices privided by NuML elements with \token{indexType}. For example in for the \token{time} \element{componentDescription} above, a \token{dataSource}:

%
\begin{myXmlLst}{}{lst:indexSet}
        <dataSource id="dataTime" indexSet="time" />
\end{myXmlLst} 
%

would extract the set of all timepoints stored in the index. Similarly: 


%
\begin{myXmlLst}{}{lst:indexSet2}
        <dataSource id="allIds" indexSet="SpeciesIds" />
\end{myXmlLst} 
%

would extract all the species id strings stored in that index set. Valid values for \token{indexSet} are all NuML Id elements declared in the \token{dimensionDescription}. If the \token{indexSet} attribute is specified the corresponding \token{dataSource} may not define any \token{slice} elements.

\subsubsection{\element{listOfSlices}}
\label{class:listOfSlices}
The \concept{listOfSlices} contains one or more \SedSlice elements that are then used in the remainder of the SED-ML document.

\subsubsection{Using the \token{dataSource} elements}
Once the \SedDataSource elements are defined, they can be reused anywhere in the SED-ML Description. Specifically their \token{id} attribute can be referenced within the \element{listOfVariables} of \element{DataGenerators}, \element{computeChange} or \element{setValue} objects. Here an example that re-uses the data source \token{dataS1}:


%
\begin{myXmlLst}{}{lst:indexSet2}
   <dataGenerator id="dgDataS1" name="S1 (data)">
     <listOfVariables>
       <variable id="varS1" modelReference="model1" target="#dataS1" />
     </listOfVariables>
     <math xmlns="http://www.w3.org/1998/Math/MathML">
       <ci> varS1 </ci>
     </math>
   </dataGenerator>
\end{myXmlLst} 
%

This represents a change from \LoneVone and \LoneVtwo, in which a \token{taskReference} was always present for a \token{variable} in a data generator.

To indicate that the target is an entity defined within the current SED-ML description the hashtag (\#) with the reference to an \token{id} was used. Additionally, this example uses the \token{modelReference}, in order to facilitate a mapping of the data encoded in the NuML document with a given model. 





%%% Local Variables: 
%%% mode: plain-tex
%%% TeX-master: "../sed-ml-L1V3"
%%% End: 