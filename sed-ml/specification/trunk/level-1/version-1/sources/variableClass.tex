% variable class
\label{class:variable}
Variables in SED-ML are references to already existing constituents in one of the defined \hyperref[class:model]{models}. A variable always is placed inside a \hyperref[class:listOfVariables]{listOfVariables}.
%
\sedfig[width=0.35\textwidth]{variableClass}{The Variable class}{fig:variable}
%
Each instance of the \concept{Variable} class  (see \fig{variable}) has a required \hyperref[sec:id]{id} and an optional \hyperref[sec:name]{name}. 
Variables are used to either refer to a model constituent, i.\,e. a model observable, such as an SBML species, or to refer to am implicit variable. 
The referenced constituent is specified through the mandatory \hyperref[sec:target]{target} attribute in the first case, and through a \hyperref[sec:symbol]{symbol} holding a MIRIAM URI in the second case. 

Listing \ref{lst:variable} shows a model with a listOfVariables declared that holds the variable definitions.
%
\begin{myXmlLst}{SED-ML \code{variable} definition}{lst:variable}
<model [..]>
 <listOfVariables>
   [VARIABLE DEFINITIONS FOLLOWING]
 </listOfVariables>
 [..]
</model>
\end{myXmlLst}
%

\tabtext{variable}{Variable}
%
\begin{table}[ht]
\center
\begin{tabular}{|l|l|}
\hline
\textbf{\attribute} & \textbf{\desc}\\
\hline
metaid$^{o}$ & \refpage{sec:metaID}\\
id & \refpage{sec:id} \\
name$^{o}$ & \refpage{sec:name}\\
\hline
target & \refpage{sec:target}\\
symbol & \refpage{sec:symbol}\\
\hline
taskReference & \refpage{sec:taskReference}\\
modelReference & \refpage{sec:modelReference}\\
\hline
\hline
\textbf{\subelements} & \textbf{\desc}\\
\hline
notes$^{o}$ & \refpage{class:notes}\\
annotation$^{o}$ & \refpage{class:annotation}\\
\hline
\end{tabular}
\label{tab:variable}
\caption{\tabcap{Variable}}
\end{table}

The \hyperref[sec:reference]{reference} to an object of the \concept{Variable} class may occur in two different places of the SED-ML document: 

First, it may occur in the \hyperref[class:change]{Change} class where it is used for describing the mathematical computation of a change of a model's observable, using other observables existing in a defined \hyperref[class:model]{model}.

The second use of the \concept{Variable} class is for defining a \hyperref[class:dataGenerator]{DataGenerator}. Here, a variable in an existing model or an implicit variable might be used to define the post-processing of the simulation.

\subsubsection{The \element{target} attribute}
\label{sec:target}
An instance of \concept{Variable} refers to a model constituent inside a particular \hyperref[class:model]{model} through an \concept{XPath} expression stored in the required \concept{target} attribute. 

XPath allows to unambiguously identify an element or attribute in an XML file.

An example for a variable definition is given in listing \ref{lst:variable}.
%
\begin{myXmlLst}{SED-ML \code{target} definition}{lst:target}
<model id="m0001" language="urn:sedml:language:sbml" source="urn:miriam:biomodels.db:BIOMD0000000012">
 <listOfChanges>
  <computeChange>
   <listOfVariables>
    <variable id="v1" name="Tet Repressor protein" taskreference="t1"  target="/sbml/listOfSpecies/species[@id="PY"]" />
   </listOfVariables>
   [CHANGE DEFINITION FOLLOWING]
  </computeChange>
 </listOfChanges>
</model>
\end{myXmlLst}
%
Please note that the identifier and names inside the SED-ML document do not have to comply with the identifiers and names that the model and its constituents carry in the model definition. In the above example \ref{lst:variable}, the variable with ID \code{v1} is defined. It is described as the \code{TetR protein}. The reference points to a species in the referenced SBML model. The particular species can be identified through its ID in the SBML model, namely \code{PY}. However, SED-ML does not forbid to use identical identifiers and names as in the referenced models neither. The following is the same valid SED-ML example for the specification of a variable as the above in listing \ref{lst:variable}, but with different naming:
%
\begin{myXmlLst}{SED-ML variable definition using the original model identifier and name in SED-ML}{}
<model id="m0001" language="urn:sedml:language:sbml" source="urn:miriam:biomodels.db:BIOMD0000000012">
 <listOfVariables>
  <variable id="PY" name="TetR protein" target="/sbml/listOfSpecies/species[@id="PY"]" />
 </listOfVariables>
 [..]
</model>
\end{myXmlLst}
%

The XPath expression used in the \concept{\code{target}} attribute unambiguously leads to the particular place in the XML SBML model -- the species is to be found in the \emph{sbml} element, and there inside the \emph{listOfSpecies}:
%
\begin{myXmlLst}{Species definition in the referenced model (extracted from \url{urn:miriam:biomodels.db:BIOMD0000000012})}{}
<sbml [..]>
 <listOfSpecies]
  <species metaid="PY" id="PY" name="TetR protein" [..]>
   [..]
  </species>
 </listOfSpecies>
 [..]
</sbml>
\end{myXmlLst}
%

\subsubsection{The \element{symbol} attribute}
\label{sec:symbol}

\concept{Symbols} are predefined, implicit variables that can be called in a SED-ML file by referring to the defined URNs representing that variable's concept. The notion of implicit variables is explained in section \ref{sec:implicitVariable} on page \refpage{sec:implicitVariable}.

An example for a \concept{symbol} definition is given in listing \ref{lst:symbol}.
%
\begin{myXmlLst}{SED-ML \code{symbol} definition}{lst:symbol}
<model id="m0001" language="urn:sedml:language:sbml" source="urn:miriam:biomodels.db:BIOMD0000000012">
 <listOfVariables>
  <variable id="t1" name="time" symbol="urn:sedml:symbol:time" />
 </listOfVariables>
 [..]
</model>
\end{myXmlLst}
%


%\alert{What about the following points?}
%\begin{itemize}
%\item {implicit/explicit notion of time in different formats.}
%\item {reserved words}
%\end{itemize}

%%% Local Variables: 
%%% mode: latex
%%% TeX-master: "../sed-ml-L1V1"
%%% End: 
