\subsection{\element{Slice}}
% datasource class
\label{class:slice}
If a \SedDataSource does not define the \token{indexSet} attribute, it will contain \SedSlice elements. Each slice removes one dimension from the data hypercube.

The \concept{Slice} class introduces two required attributes: \token{reference} and \token{value}. 

% Fig: sed model
\sedfig[width=0.5\textwidth]{pdf/sedSlice}{The SED-ML Slice class}{fig:sedSlice}
%

\tabtext{slice}{slice}
%
\begin{table}[ht]
\center
\begin{tabular}{|l|l|}
\hline
\textbf{\attribute} & \textbf{\desc}\\
\hline
metaid$^{o}$ & \refpage{sec:metaID}\\
\hline
reference & \refpage{sec:sliceReference}\\
value & \refpage{sec:sliceValue}\\
\hline
\hline
\textbf{\subelements} & \textbf{\desc}\\
\hline
notes$^{o}$ & \refpage{class:notes}\\
annotation$^{o}$ & \refpage{class:annotation}\\
\hline
\end{tabular}
\caption{\tabcap{slice}}
\label{tab:slice}
\end{table}
%

\subsubsection{The \token{reference} attribute}
\label{sec:sliceReference}
The \token{reference} attribute references one of the indices described in the \token{dimensionDescription}. In the example above, valid values would be: \token{time} and \token{SpeciesIds}.

\subsubsection{The \token{value} attribute}
\label{sec:sliceValue}
The \token{value} attribute takes the value of a specific index in the referenced set of indices. For example:

%
\begin{myXmlLst}{}{lst:sliceValue1}
        <dataSource id="dataS1">
          <listOfSlices>
            <slice reference="SpeciesIds" value="S1" />
					</listOfSlices>
        </dataSource>
\end{myXmlLst} 
%

would isolate the index set of all species ids specified, to only the single entry for \token{S1}, however over the full range of the \token{time} index set. As stated before, there could be multiple slice elements present, so it would be feasible to slice the data again, to obtain a single time point, for example the initial one:

%
\begin{myXmlLst}{}{lst:sliceValue2}
        <dataSource id="dataS1">
          <listOfSlices>
            <slice reference="time" value="0" />
            <slice reference="SpeciesIds" value="S1" />
          </listOfSlices>
        </dataSource>
\end{myXmlLst} 
%

%%% Local Variables: 
%%% mode: plain-tex
%%% TeX-master: "../sed-ml-L1V3"
%%% End: 