% Change Class
  \subsection[Change]{\element{Change}}
\label{class:change}
SED-ML not only allows to use the sole model for simulation, but also enables the description of \concept{changes} to be made on the model before simulation  (\fig{sedChange}). Changes can be of three distinct types:
\begin{enumerate}
 \item{Changes on attributes of the model's XML representation (\hyperref[class:changeAttribute]{ChangeAttribute})}
 \item{Changes on any XML snippet of the model's XML representation (\hyperref[class:addXml]{AddXML}, \hyperref[class:changeXml]{ChangeXML}, \hyperref[class:removeXml]{RemoveXML})}
 \item{Changes based on mathematical calculations (\hyperref[class:computeChange]{ComputeChange})} 
 \end{enumerate}

The \concept{Change} class is abstract and serves as the container for different types of changes. Therefore, a SED-ML document will only contain the derived classes, i.\,e. \hyperref[class:changeAttribute]{ChangeAttribute}, \hyperref[class:addXml]{AddXML}, \hyperref[class:changeXml]{ChangeXML}, \hyperref[class:removeXml]{RemoveXML}, or \hyperref[class:computeChange]{ComputeChange}.
%
\sedfig[width=\textwidth]{changeClass}{The SED-ML Change class}{fig:sedChange}
%

\tabtext{change}{change}
%
\begin{table}[h!]
\center
\begin{tabular}{|l|l|}
\hline
\textbf{\attribute} & \textbf{\desc}\\
\hline
metaid$^{o}$ & \refpage{sec:metaID}\\
id & \refpage{sec:id} \\
name$^{o}$ & \refpage{sec:name}\\
\hline
target & \refpage{sec:target}\\
\hline
\hline
\textbf{\subelements} & \textbf{\desc}\\
\hline
notes$^{o}$ & \refpage{class:notes}\\
annotation$^{o}$ & \refpage{class:annotation}\\
\hline
addXML$^{o}$ & \refpage{class:addXml}\\
changeXML$^{o}$ & \refpage{class:changeXml}\\
removeXML$^{o}$ & \refpage{class:removeXml}\\
changeAttribute$^{o}$ & \refpage{class:changeAttribute}\\
computeChange$^{o}$ & \refpage{class:computeChange}\\
\hline
\end{tabular}
\caption{\tabcap{change}}
\label{tab:change}
\end{table}
%

Each Change has a \hyperref[sec:target]{target} attribute that holds a valid XPath expression pointing to the XML element or XML attribute that is to undergo the defined changes.
Except for the cases of \hyperref[class:changeXml]{ChangeXML} and \hyperref[class:removeXml]{RemoveXML}, this XPath expression must always select a \emph{single} element or attribute within the relevant model.

%A typical example for a model update (or change) is the assignment of new parameter values to the model. 

  \label{sec:newXml}

The \code{newXML} element provides a piece of XML code (\fig{newXml}). 
\code{NewXML} must hold a valid piece of XML which after insertion into the original model must lead to a valid model file, according to the model language specification (as given by the \hyperref[sec:language]{language} attribute).

\sedfig[width=0.35\textwidth]{newXml}{The \code{NewXML} class}{fig:newXml}.

\tabtext{newXml}{newXml}
%
\begin{table}[ht]
\center
\begin{tabular}{|l|l|}
\hline
\textbf{\attribute} & \textbf{\desc}\\
\hline
metaid$^{o}$ & \refpage{sec:metaID}\\
\hline
\hline
\textbf{\subelements} & \textbf{\desc}\\
\hline
notes$^{o}$ & \refpage{class:notes}\\
annotation$^{o}$ & \refpage{class:annotation}\\
\hline
\end{tabular}
\label{tab:newXml}
\caption{\tabcap{newXML}}
\end{table}
%


The \code{newXML} element is used at two different places inside SED-ML \LoneVone:
%
\begin{enumerate}
\item{If it is used as a sub-element of the \hyperref[class:addXML]{addXML} element, then it is to be \emph{inserted as a child} of the XML element addressed by the XPath.}
\item{It it is used as a sub-element of the \hyperref[class:changeXML]{changeXML} element, then it is to \emph{replace} the XML element addressed by the XPath.}
\end{enumerate}
%
Examples are given in the according change class definitions.



%%% Local Variables: 
%%% mode: latex
%%% TeX-master: "../sed-ml-L1V1"
%%% End: 


  % ChangeAttribute Class
\label{class:addXml}
The \concept{AddXML} class specifies a snippet of XML that is to be added as a child of the specified XPath \hyperref[sec:target]{target} attribute (see again section \ref{sec:target} on page \refpage{sec:target}). 

\tabtext{addXml}{addXml}
%
\begin{table}[ht]
\center
\begin{tabular}{|l|l|}
\hline
\textbf{\attribute} & \textbf{\desc}\\
\hline
metaid$^{o}$ & \refpage{sec:metaID}\\
id & \refpage{sec:id} \\
name$^{o}$ & \refpage{sec:name}\\
target & \refpage{sec:target}\\
\hline
\hline
\textbf{\subelements} & \textbf{\desc}\\
\hline
notes$^{o}$ & \refpage{class:notes}\\
annotation$^{o}$ & \refpage{class:annotation}\\
\hline
newXML & \refpage{sec:newXml}\\
\hline
\end{tabular}
\label{tab:addXml}
\caption{\tabcap{addXML}}
\end{table}
%

\subsubsection{The \element{newXML} element}
\label{sec:newXml}

The new piece of XML code that is to be inserted as child of the the XML element addressed by the XPath is provided in the required \code{newXML} element. 

An example for a change that adds an additional parameter to a model is given in listing \ref{lst:addXML}.
%
\begin{myXmlLst}{The \code{addXML} element with its \code{newXML} sub-element}{lst:addXML}
<model [..]>
 <listOfChanges>
  <addXML target="/sbml:sbml/sbml:model/sbml:listOfParameters" >
   <newXML>
     <parameter metaid="metaid_0000010" id="V_mT" value="0.7" />
  </newXML>
  </addXML>
 </listOfChanges>
</model>
\end{myXmlLst}
%
The code of the model is changed in the way that its parameter with ID \code{V\_mT} is added to its list of parameters.



%%% Local Variables: 
%%% mode: latex
%%% TeX-master: "../sed-ml-L1V1"
%%% End: 


  % ChangeAttribute Class
  \subsubsection{\element{ChangeXML}}
\label{class:changeXml}
The \concept{ChangeXML} class defines changes of any XML element in the model that can be addressed by a valid XPath expression (\fig{changeXml}). 
%
\sedfig[width=0.75\textwidth]{changeXmlClass}{The \code{ChangeXML} class}{fig:changeXml}
%
The XPath is specified in the required \hyperref[sec:target]{target} attribute (see again section \ref{sec:target} on page \refpage{sec:target}). 
The change of XML is specified in the \hyperref[sec:newXml]{NewXML} class.

\tabtext{changeXml}{changeXml}
%
\begin{table}[ht]
\center
\begin{tabular}{|l|l|}
\hline
\textbf{\attribute} & \textbf{\desc}\\
\hline
metaid$^{o}$ & \refpage{sec:metaID}\\
id & \refpage{sec:id} \\
name$^{o}$ & \refpage{sec:name}\\
target & \refpage{sec:target}\\
\hline
\hline
\textbf{\subelements} & \textbf{\desc}\\
\hline
notes$^{o}$ & \refpage{class:notes}\\
annotation$^{o}$ & \refpage{class:annotation}\\
\hline
newXML & \refpage{sec:newXml}\\
\hline
\end{tabular}
\caption{\tabcap{changeXML}}
\label{tab:changeXml}
\end{table}
%

An example for a change that adds an additional parameter to a model is given in listing \ref{lst:changeXML}.
%
\begin{myXmlLst}{The \code{changeXML} element}{lst:changeXML}
<model [..]>
 <listOfChanges>
  <changeXML target="/sbml:sbml/sbml:model/sbml:listOfParameters/sbml:parameter[@id='V_mT']" >
   <newXML>
     <parameter metaid="metaid_0000010" id="V_mT_1" value="0.7" />
     <parameter metaid="metaid_0000050" id="V_mT_2" value="4.6"> />
   </newXML>
  </changeXML>
 </listOfChanges>
</model>
\end{myXmlLst}
%
The code of the model is changed in the way that its parameter with ID \code{V\_mT} is substituted by two other parameters \code{V\_mT\_1} and \code{V\_mT\_2}.
The \code{target} attribute defines that the parameter with ID \code{V\_mT} is to be changed. The \code{newXML} element then specifies the XML that is to be  exchanged for  that parameter.


%%% Local Variables: 
%%% mode: latex
%%% TeX-master: "../sed-ml-L1V1"
%%% End: 


  % ChangeAttribute Class
\subsubsection{\element{RemoveXML}}
\label{class:removeXml}
The \concept{RemoveXML} class can be used to delete the XML element of the model that is addressed by the XPath expression (\fig{removeXml}).
%
\sedfig[width=0.75\textwidth]{removeXmlClass}{The \code{RemoveXML} class}{fig:removeXml}
%

The XPath is specified in the required \hyperref[sec:target]{target} attribute. 

\tabtext{removeXml}{removeXml}
%
\begin{table}[ht]
\center
\begin{tabular}{|l|l|}
\hline
\textbf{\attribute} & \textbf{\desc}\\
\hline
metaid$^{o}$ & \refpage{sec:metaID}\\
id & \refpage{sec:id} \\
name$^{o}$ & \refpage{sec:name}\\
target & \refpage{sec:target}\\
\hline
\hline
\textbf{\subelements} & \textbf{\desc}\\
\hline
notes$^{o}$ & \refpage{class:notes}\\
annotation$^{o}$ & \refpage{class:annotation}\\
\hline
\end{tabular}
\caption{\tabcap{removeXML}}
\label{tab:removeXml}
\end{table}
%

An example for the removal of an XML element from a model is given in Listing~\ref{lst:removeXML}.
%
\begin{myXmlLst}{The \code{removeXML} element}{lst:removeXML}
<model [..]>
 <listOfChanges>
  <removeXML target="/sbml:sbml/sbml:model/sbml:listOfReactions/sbml:reaction[@id='J1']" />
 </listOfChanges>
</model>
\end{myXmlLst}
%

The code of the model is changed by deleting the reaction with ID \code{V\_mT} from the model's list of reactions.


%%% Local Variables: 
%%% mode: latex
%%% TeX-master: "../sed-ml-L1V1"
%%% End: 


  % ChangeAttribute Class
\label{class:changeAttribute}
The \concept{ChangeAttribute} class allows to define updates on the XML attribute values of the corresponding model. 
The \concept{ChangeXML} class covers the possibilities provided by the \hyperref[class:changeAttribute]{ChangeAttribute} class. That is, everything that can be expressed by a \hyperref[class:changeAttribute]{ChangeAttribute} construct can also be expressed by a \concept{ChangeXML}. However, both concepts exist to allow for being very specific in defining changes. It is recommended to use the \concept{ChangeAttribute} for any changes of an XML attribute, and to use the more general \hyperref[class:changeXml]{ChangeXML} for all other cases.

\tabtext{changeAttribute}{changeAttribute}
%
\begin{table}[ht]
\center
\begin{tabular}{|l|l|}
\hline
\textbf{\attribute} & \textbf{\desc}\\
\hline
metaid$^{o}$ & \refpage{sec:metaID}\\
id & \refpage{sec:id} \\
name$^{o}$ & \refpage{sec:name}\\
\hline
target & \refpage{sec:target}\\
newValue & \refpage{sec:newValue}\\
\hline
\hline
\textbf{\subelements} & \textbf{\desc}\\
\hline
notes$^{o}$ & \refpage{class:notes}\\
annotation$^{o}$ & \refpage{class:annotation}\\
\hline
\end{tabular}
\label{tab:changeAttribute}
\caption{\tabcap{ChangeAttribute}}
\end{table}
%

One required attribute of the \code{changeAttribute} element is the \hyperref[sec:target]{target} of change, i.\,e. the location of the addressed XML attribute.


\subsubsection{The \code{newValue} attribute}
\label{sec:newValue}
The second required attribute in the \code{changeAttribute} element is \code{newValue}, which assignes a new value to the targeted XML attribute. 

An example for an SBML model is the update of the initial concentration of a certain parameter, as shown in listing  \ref{lst:changeAttribute}.
%
\begin{myXmlLst}{The \code{changeAttribute} element and its \code{newValue} attribute}{lst:changeAttribute}
<model id="model1" name="Circadian Chaos" language="urn:sedml:language:sbml" source="urn:miriam:biomodels.db:BIOMD0000000021">
 <listOfChanges>
  <changeAttribute target="/sbml:sbml/sbml:model/sbml:listOfParameters/sbml:parameter[@id='V_mT']/@value" newValue="0.28"/>
  <changeAttribute target="/sbml:sbml/sbml:model/sbml:listOfParameters/sbml:parameter[@id='V_dT']/@value" newValue="4.8"/>
 </listOfChanges>
</model>
\end{myXmlLst}
%

%%% Local Variables: 
%%% mode: latex
%%% TeX-master: "../sed-ml-L1V1"
%%% End: 


  % ChangeAttribute Class
\label{class:computeChange}
The \concept{ComputeChange} is used to make changes on any element of the XML file addressable by an XPath expression, where the changes are described by mathematical expressions through MathML. 

\tabtext{computeChange}{ComputeChange}
%
\begin{table}[ht]
\center
\begin{tabular}{|l|l|}
\hline
\textbf{\attribute} & \textbf{\desc}\\
\hline
metaid$^{o}$ & \refpage{sec:metaID}\\
id & \refpage{sec:id} \\
name$^{o}$ & \refpage{sec:name}\\
target & \refpage{sec:target}\\
\alert{math} &\refpage{sec:math}\\
%listOfVariables$^{o}$ & \refpage{sec:listOfVariables}\\
%listOfParameter$^{o}$ & \refpage{sec:listOfParameters}\\
\hline
\hline
\textbf{\subelements} & \textbf{\desc}\\
\hline
notes$^{o}$ & \refpage{class:notes}\\
annotation$^{o}$ & \refpage{class:annotation}\\
variable$^{o}$ & \refpage{class:variable}\\
parameter$^{o}$ & \refpage{class:parameter}\\
\hline
\end{tabular}
\label{tab:computeChange}
\caption{\tabcap{ComputeChange}, \alert{math currently is an attribute, probably should be turned into an element (see listing \ref{lst:computeChange})}}
\end{table}
%

The \element{target} attribute contains the XPath addressing the piece of XML that is to be changed. 
It is possible to introduce additional parameters for the mathematics. Therefore, the parameters first need to be defined in the \concept{listOfParameters}. They are then referenced through their ID.
To use model variables for the definition of a mathematical expression, those variables need to be defined in the \concept{listOfVariables} first, and can then be incorporated through their ID.

\paragraph{The \element{math} attribute}
\label{sec:math}

The \element{math} element is used to define mathematical functions. 
If used as an attribute of the \concept{ComputeChange} class, it computes the change of the element or attribute addressed by the \hyperref[sec:target]{target} attribute.

An example is given in listing \ref{lst:computeChange}.
%
\begin{myXmlLst}{The computeChange element}{lst:computeChange}
<model [..]>
    <computeChange target="/sbml/model/listOfParameters/parameter[@id='w']">
      <listOfVariables>
        <variable id="camkii" name="active calcium/calmoduline kinase II" 
                  target="/sbml/model[@id='calcium']/listOfSpecies/species[@id='KII']" />
        <variable id="w" name="synaptic weight"
                  target="/sbml/model[@id='synapse']/listOfParameters/parameter[@id='w']" />
      <listOfVariables/>
      <listOfParameters>
        <parameter id="w0" name="synaptic weight change" value="1">
        <parameter id="n" name="utrasensitivity to calcium" value="2">
        <parameter id="K" name="sensitivity to calcium" value="1e-6">
      <listOfParameters/>
      <math>
         <apply>
           <plus />
           <ci>w</ci>
           <apply>
             <times />
             <ci>w0</ci>
             <apply>
               <divide />
               <apply>
                 <power />
                 <ci>camkii</ci>
                 <ci>n</ci>
               </apply>
               <apply>
                 <plus />
                 <apply>
                   <power />
                   <ci>K</ci>
                   <ci>n</ci>
                 </apply>
                 <apply>
                   <power />
                   <ci> camkii </ci>
                   <ci>n</ci>
                 </apply>
               </apply>
             </apply>
           </apply> 
         </apply>
      </math>
    </computeChange>
  </listOfChanges>
</model>
\end{myXmlLst}

\LoneVone supports the subset of MathML 2.0 shown in section \ref{sec:mathML}.
%

%A problem arises, because the individual supported model exchange languages allow different subsets of MathML. Thus, when an instance of ComputeChange replaces a %mathematical expression of  an SBML reaction, only the MathML subset allowed by SBML should be used here.


%%% Local Variables: 
%%% mode: latex
%%% TeX-master: "../sed-ml-L1V1"
%%% End: 



%%% Local Variables: 
%%% mode: latex
%%% TeX-master: "../sed-ml-L1V2"
%%% End: 
