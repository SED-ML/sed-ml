% ~~~~~~~~~~~~~~~~~~~~~~~~~~~~~~~~~~~~~~~~
% DATA DESCRIPTION
% ~~~~~~~~~~~~~~~~~~~~~~~~~~~~~~~~~~~~~~~~
\subsection{\element{DataDescription}}
\label{class:dataDescription}

The \concept{DataDescription} class (\fig{dataDescription}) references a file containing data points, along with a description on how to access that file, and what information to extract from it. 

The \concept{DataDescription} class introduces three attributes: the required attributes \token{id} and  \hyperref[sec:data_source]{\code{source}} and the optional attribute \token{name}. Additionally two elements are defined: \hyperref[sec:dimensionDescription]{\code{dimensionDescription}} and \hyperref[sec:listOfDataSources]{\code{listOfDataSources}}. 

\sedfig[width=0.8\textwidth]{images/uml/dataDescription}{The SED-ML DataDescription class}{fig:dataDescription}

\tabtext{dataDescription}{dataDescription}

\begin{table}[ht]
\center
\begin{tabular}{ll}
\toprule
\textbf{\attribute} & \textbf{\desc}\\
\midrule
metaid$^{o}$ & \refpage{sec:metaID}\\
id & \refpage{sec:id} \\
name$^{o}$ & \refpage{sec:name}\\
\midrule
source & \refpage{sec:data_source}\\
\midrule
\textbf{\subelements} & \textbf{\desc}\\
\midrule
notes$^{o}$ & \refpage{class:notes}\\
annotation$^{o}$ & \refpage{class:annotation}\\
\midrule
dimensionDescription$^{o}$ & \refpage{sec:dimensionDescription}\\
listOfDataSources$^{o}$ & \refpage{sec:listOfDataSources}\\
\bottomrule
\end{tabular}
\caption{\tabcap{dataDescription}}
\label{tab:dataDescription}
\end{table}


% ~~~ ID & NAME ~~~
\subsubsection{id and name}
The attribute \token{id} of type \token{SId} is meant to uniquely identify the \token{dataDescription} element, while the optional \token{name} attribute of type \token{string}, is there to provide a human readable representation if desired.


% ~~~ SOURCE ~~~
\subsubsection{source}
\label{sec:data_source}
Analog to how the \token{source} attribute on the \SedModel is handled, this attribute provides a location of a data file. In order to resolve the \token{source} attribute, the same mechanisms are allowed as for \SedModel element: be it a local file system, a relative link or an online resource. In the \currentLV only source files encoded in either NuML or CSV are allowed, with NuML being the recommended data format.

In case of CSV as source encoding the file
\begin{itemize}
	\item must contain a header row which defines the ids
	\item must only contain numerical values 
	\item must use the comma "," as field separator
	\item must use the dot "." as separator in numbers
	\item may contain comment rows which start with "\#"
	\item the dimensionDescription of the CSV dataDescription must be two dimensional and correspond to the content of the CSV file 
\end{itemize}

\lsttext{dataDescription}{dataDescription}
\begin{myXmlLst}{SED-ML \code{dataDescription} element}{lst:dataDescription}
<dataDescription id="Data1" name="Oscli Time Course Data" 
	source="http://svn.code.sf.net/p/libsedml/code/trunk/Samples/data/oscli.numl" >
    [...]
</dataDescription>
\end{myXmlLst} 


% ~~~ DIMENSION DESCRIPTION ~~~
\subsubsection{\element{dimensionDescription}}
\label{sec:dimensionDescription}
The \concept{dimensionDescription} element is the data description from an NuML file. Consider for example:

\begin{myXmlLst}{SED-ML \code{dimensionDescription} element}{lst:dimensionDescription}
<dimensionDescription>
	<compositeDescription indexType="double" id="time" name="time" 
		xmlns="http://www.numl.org/numl/level1/version1">
		<compositeDescription indexType="string" id="SpeciesIds" name="SpeciesIds">
			<atomicDescription valueType="double" name="Concentration" />
		</compositeDescription>
	</compositeDescription>
</dimensionDescription>
\end{myXmlLst} 

Here a nested NuML \token{compositeDescription} with \token{time} spanning one dimension and \token{SpeciesIds} another dimension. This two dimensional space is then filled with \token{double} values representing concentrations.


% ~~~ LIST OF DATA SOURCES ~~~
\subsubsection{\element{listOfDataSources}}
\label{sec:listOfDataSources}
The \concept{listOfDataSources} contains one or more \SedDataSource elements that are then used in the remainder of the SED-ML document.


% ~~~~~~~~~~~~~~~~~~~~~~~~~~~~~~~~~~~~~~~~
% DATA SOURCE
% ~~~~~~~~~~~~~~~~~~~~~~~~~~~~~~~~~~~~~~~~
\subsection{\element{DataSource}}
\label{class:dataSource}
The \concept{DataSource} class (\fig{dataDescription}) extracts chunks out of the data file provided by the outer \SedDataDescription element. 

The \concept{DataSource} class introduces three attributes: the required attribute \token{id} and the optional attributes \hyperref[sec:indexSet]{\code{indexSet}} and \token{name}. Additionally the \hyperref[sec:listOfSlices]{\code{listOfSlices}} element is defined (\fig{dataDescription}). 

\tabtext{dataSource}{dataSource}

\begin{table}[ht]
\center
\begin{tabular}{ll}
\toprule
\textbf{\attribute} & \textbf{\desc}\\
\midrule
metaid$^{o}$ & \refpage{sec:metaID}\\
id & \refpage{sec:id} \\
name$^{o}$ & \refpage{sec:name}\\
\midrule
indexSet & \refpage{sec:indexSet}\\
\midrule
\textbf{\subelements} & \textbf{\desc}\\
\midrule
notes$^{o}$ & \refpage{class:notes}\\
annotation$^{o}$ & \refpage{class:annotation}\\
\midrule
listOfSlices$^{o}$ & \refpage{sec:listOfSlices}\\
\bottomrule
\end{tabular}
\caption{\tabcap{dataSource}}
\label{tab:dataSource}
\end{table}


% ~~~ ID & NAME ~~~
\subsubsection{id and name}
The attribute \token{id} of type \token{SId} is meant to uniquely identify the \token{dataSource} element, while the optional \token{name} attribute of type \token{string}, is there to provide a human readable representation if desired.


%% ~~~ INDEX SET ~~~
\subsubsection{indexSet}
\label{sec:indexSet}
Since data elements in NuML are either values, or indices, the \SedDataSource element needs two ways of addressing those elements. The \token{indexSet} attribute allows to address all indices provided by NuML elements with \token{indexType}. For example in for the \token{time} \element{componentDescription} above, a \token{dataSource}:

\begin{myXmlLst}{}{lst:indexSet}
<dataSource id="dataTime" indexSet="time" />
\end{myXmlLst} 

would extract the set of all timepoints stored in the index. Similarly: 

\begin{myXmlLst}{}{lst:indexSet2}
<dataSource id="allIds" indexSet="SpeciesIds" />
\end{myXmlLst} 

would extract all the species id strings stored in that index set. Valid values for \token{indexSet} are all NuML Id elements declared in the \token{dimensionDescription}. If the \token{indexSet} attribute is specified the corresponding \token{dataSource} may not define any \token{slice} elements.


% ~~~ LIST OF SLICES ~~~
\subsubsection{\element{listOfSlices}}
\label{sec:listOfSlices}
The \concept{listOfSlices} contains one or more \hyperref[class:slice]{Slice} elements.
The \concept{listOfSlices} container holds the \hyperref[class:slice]{slice} definitions of a \hyperref[class:dataSource]{dataSource} (\fig{dataDescription}). The \concept{listOfSlices} is optional and may contain zero to many outputs.

% ~~~ USING DATA SOURCE ~~~
\subsubsection{Using the \token{dataSource} elements}
Once the \SedDataSource elements are defined, they can be reused anywhere in the SED-ML Description. Specifically their \token{id} attribute can be referenced within the \element{listOfVariables} of \element{DataGenerators}, \element{computeChange} or \element{setValue} objects. Here an example that re-uses the data source \token{dataS1}:

\begin{myXmlLst}{}{lst:indexSource}
<dataGenerator id="dgDataS1" name="S1 (data)">
	<listOfVariables>
		<variable id="varS1" modelReference="model1" target="#dataS1" />
	</listOfVariables>
	<math xmlns="http://www.w3.org/1998/Math/MathML">
		<ci> varS1 </ci>
	</math>
</dataGenerator>
\end{myXmlLst} 

This represents a change from \LoneVone and \LoneVtwo, in which a \token{taskReference} was always present for a \token{variable} in a data generator.

To indicate that the target is an entity defined within the current SED-ML description the hashtag (\#) with the reference to an \token{id} was used. Additionally, this example uses the \token{modelReference}, in order to facilitate a mapping of the data encoded in the NuML document with a given model. 


% ~~~~~~~~~~~~~~~~~~~~~~~~~~~~~~~~~~~~~~~~
%% SLICE
% ~~~~~~~~~~~~~~~~~~~~~~~~~~~~~~~~~~~~~~~~
\subsection{\element{Slice}}
\label{class:slice}
If a \SedDataSource does not define the \token{indexSet} attribute, it will contain \SedSlice elements. Each slice removes one dimension from the data hypercube.

The \concept{Slice} class introduces two required attributes: \hyperref[sec:sliceReference]{\token{reference}} and \hyperref[sec:sliceValue]{\token{value}} (\fig{dataDescription}).

% \sedfig[width=0.4\textwidth]{images/uml/slice}{The SED-ML Slice class}{fig:sedSlice}

\tabtext{slice}{slice}

\begin{table}[ht]
\center
\begin{tabular}{ll}
\toprule
\textbf{\attribute} & \textbf{\desc}\\
\midrule
metaid$^{o}$ & \refpage{sec:metaID}\\
\midrule
reference & \refpage{sec:sliceReference}\\
value & \refpage{sec:sliceValue}\\
\midrule
\textbf{\subelements} & \textbf{\desc}\\
\midrule
notes$^{o}$ & \refpage{class:notes}\\
annotation$^{o}$ & \refpage{class:annotation}\\
\bottomrule
\end{tabular}
\caption{\tabcap{slice}}
\label{tab:slice}
\end{table}

%% ~~~ SLICE:REFERENCE ~~~
\subsubsection{The \token{reference} attribute}
\label{sec:sliceReference}
The \token{reference} attribute references one of the indices described in the \token{dimensionDescription}. In the example above, valid values would be: \token{time} and \token{SpeciesIds}.

%% ~~~ SLICE:VALUE ~~~
\subsubsection{The \token{value} attribute}
\label{sec:sliceValue}
The \token{value} attribute takes the value of a specific index in the referenced set of indices. For example:

\begin{myXmlLst}{}{lst:sliceValue1}
<dataSource id="dataS1">
	<listOfSlices>
		<slice reference="SpeciesIds" value="S1" />
	</listOfSlices>
</dataSource>
\end{myXmlLst} 

would isolate the index set of all species ids specified, to only the single entry for \token{S1}, however over the full range of the \token{time} index set. As stated before, there could be multiple slice elements present, so it would be feasible to slice the data again, to obtain a single time point, for example the initial one:

\begin{myXmlLst}{}{lst:sliceValue2}
<dataSource id="dataS1">
	<listOfSlices>
		<slice reference="time" value="0" />
		<slice reference="SpeciesIds" value="S1" />
	</listOfSlices>
</dataSource>
\end{myXmlLst} 
