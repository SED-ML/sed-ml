% on the MathML subset used in SED-ML
\label{sec:mathML}
The SED-ML specification allows for the pre-processing of computational models, 
as well as post processing of the simulation results. The corresponding 
mathematical expressions are encoded using MathML 2.0. MathML is an 
international standard for encoding mathematical expressions using XML and is 
used as representation of mathematical expressions in SBML and CellML, two of 
the languages supported by SED-ML. A problem arises, because the individual 
supported model exchange languages allow different subsets of MathML. Thus, 
when for example a ChangeXML element replaces a mathematical expression of  an 
SBML reaction, only the MathML subset allowed by SBML should be used here. 

In order to make the SED-ML format easier to adopt, at the beginning we 
restrict the MathML subset to the following operations: 

\begin{itemize}\setlength{\parskip}{-0.1ex}

\item \emph{token}: \token{cn}, \token{ci}, \token{csymbol},
  \token{sep}
  
\item \emph{general}: \token{apply}, \token{piecewise},
  \token{piece}, \token{otherwise}, \token{lambda} 

\item \emph{relational operators}: \token{eq}, \token{neq},
  \token{gt}, \token{lt}, \token{geq}, \token{leq}

\item \emph{arithmetic operators}: \token{plus}, \token{minus},
  \token{times}, \token{divide}, \token{power}, \token{root},
  \token{abs}, \token{exp}, \token{ln}, \token{log},
  \token{floor}, \token{ceiling}, \token{factorial}

\item \emph{logical operators}: \token{and}, \token{or},
  \token{xor}, \token{not}

\item \emph{qualifiers}: \token{degree}, \token{bvar},
  \token{logbase}

\item \emph{trigonometric operators}: \token{sin}, \token{cos},
  \token{tan}, \token{sec}, \token{csc}, \token{cot},
  \token{sinh}, \token{cosh}, \token{tanh}, \token{sech},
  \token{csch}, \token{coth}, \token{arcsin}, \token{arccos},
  \token{arctan}, \token{arcsec}, \token{arccsc}, \token{arccot},
  \token{arcsinh}, \token{arccosh}, \token{arctanh},
  \token{arcsech}, \token{arccsch}, \token{arccoth}

\item \emph{constants}: \token{true}, \token{false},
  \token{notanumber}, \token{pi}, \token{infinity},
  \token{exponentiale}

\item \emph{MathML annotations}: \token{semantics},
  \token{annotation}, \token{annotation-xml}

\end{itemize}

It should be noted, that all the operations listed above, only operate on 
singular values. However, as one of SED-ML's aim is to provide post processing 
on the results of simulation experiments, we need to enhance this basic set of 
operations by some aggregate functions. 

\subsubsection{MathML Symbols}

To start enabling for post processing in SED-ML, a defined set of MathML symbols that 
represent vector values are supported by SED-ML \LoneVone. 

%To simplify things for SED-ML L1V1 the only symbols to be used are the identifiers of variables defined in the listOfVariables of DataGenerators. These variables represent the data collected from the simulation experiment with the associated task. 

The following aggregate functions are available for use with \hyperref[class:dataGenerator]{DataGenerator} variables: 

\begin{itemize}\setlength{\parskip}{-0.1ex}

\item \emph{min}: Where the minimum of a variable represents the smallest value 
the simulation experiment yielded. Example: 

\begin{verbatim} <min> <ci> variableId </ci></min> \end{verbatim}.

\item \emph{max}: Where the maximum of a variables represents the largest value 
the simulation experiment yielded. Example: 

\begin{verbatim} <max> <ci> variableId </ci></max> \end{verbatim}.

\item \emph{sum}: All values of the variable returned by the simulation 
experiment are added up. Example: 

\begin{verbatim} <sum> <ci> variableId </ci></sum> \end{verbatim}.

\item \emph{product}: All values of the variable returned by the simulation 
experiment are multiplied. Example: 

\begin{verbatim} <product> <ci> variableId </ci></product> \end{verbatim}.

\end{itemize}

These represent the only exceptions. At this point SED-ML does not define a complete algebra of vector values. For more information see the description of the \hyperref[class:dataGenerator]{DataGenerator} class.

%%% Local Variables: 
%%% mode: latex
%%% TeX-master: "../sed-ml-L1V1"
%%% End: 
