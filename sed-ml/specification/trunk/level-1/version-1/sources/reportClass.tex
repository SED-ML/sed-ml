\label{class:report}
% Fig: Report
\sedfig[width=\textwidth]{report.png}{The SED-ML Report class}{fig:report}
%

\tabtext{report}{Report}
%
\begin{table}[ht]
\center
\begin{tabular}{|l|l|}
\hline
\textbf{\attribute} & \textbf{\desc}\\
\hline
metaid$^{o}$ & \refpage{sec:metaID}\\
id & \refpage{sec:id} \\
name$^{o}$ & \refpage{sec:name}\\
\hline
\hline
\textbf{\subelements} & \textbf{\desc}\\
\hline
notes$^{o}$ & \refpage{class:notes}\\
annotation$^{o}$ & \refpage{class:annotation}\\
dataSet & \refpage{class:dataSet}\\
\hline
\end{tabular}
\label{tab:report}
\caption{\tabcap{Report}}
\end{table}
%

The \concept{Report} class defines an output type that returns the simulation result in actual \emph{numbers}. The particular columns of the report table are defined by creating a \hyperref[class:dataSet]{DataSet} for each column. 

Listing \ref{lst:listOfDataSets} provides an example for the definition of such a report with one column.
%
\begin{myXmlLst}{The \code{report} element with the nested \code{listOfDataSets} element}{lst:listOfDataSets}
<report>
 <listOfDataSets>
  <dataSet>
   [DATA REFERENCE FOLLOWING]
  </dataSet>
 </listOfDataSets>
</report>
\end{myXmlLst}
%

The simulation result itself, i.\,e. concrete result numbers, are not stored in SED-ML, but the directive how to \emph{calculate} them from the output of the simulator is provided through the \concept{dataGenerator}.

%%% Local Variables: 
%%% mode: latex
%%% TeX-master: "../sed-ml-L1V1"
%%% End: 
