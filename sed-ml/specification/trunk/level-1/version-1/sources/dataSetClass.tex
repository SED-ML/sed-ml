\label{class:dataSet}

% Fig: Report
\sedfig[width=0.3\textwidth]{dataSetClass}{The SED-ML DataSet class}{fig:dataSet}
%

\tabtext{dataSet}{DataSet}
%
\begin{table}[ht]
\center
\begin{tabular}{|l|l|}
\hline
\textbf{\attribute} & \textbf{\desc}\\
\hline
metaid$^{o}$ & \refpage{sec:metaID}\\
id & \refpage{sec:id} \\
name$^{o}$ & \refpage{sec:name}\\
dataReference & \refpage{sec:dataReference}\\
\hline
\hline
\textbf{\subelements} & \textbf{\desc}\\
\hline
notes$^{o}$ & \refpage{class:notes}\\
annotation$^{o}$ & \refpage{class:annotation}\\
\hline
\end{tabular}
\label{tab:dataSet}
\caption{\tabcap{DataSet}}
\end{table}
%

\subsubsection{The \element{dataReference} attribute}
\label{sec:dataReference}

tbw

An example for the definition of a data set in form of a table is given in the XML snippet in listing \ref{lst:dataSet}.
%
\begin{myXmlLst}{The SED-ML dataSet element, defining the output  showing the result of the referenced task}{lst:dataSet}
<listOfDataSets>
  <dataSet id="d1" name="v1 over time" dataReference="dg1">
</listOfDataSets>
\end{myXmlLst}

%%% Local Variables: 
%%% mode: latex
%%% TeX-master: "../sed-ml-L1V1"
%%% End: 
