\subsection{\element{name}}
\label{sec:name}
%

Besides an \hyperref[sec:id]{id}, a SED-ML constituent may carry an optional \concept{name}. However, names do not have identifying character;  several SED-ML constituents may carry the same name. The purpose of the \code{name} attribute is to keep a human-readable name of the constituent, e.\,g.\ for display to the user. In the XML Schema representation, names are of the data type \code{String}.

Listing \ref{lst:name} extends the model definition in listing \ref{lst:id} by a model name.
%
\begin{myXmlLst}{SED-ML name definition, e.\,g.\ for a model}{lst:name}
<model id="m00001" name="Circadian oscillator" language="urn:sedml:language:sbml" source="urn:miriam:biomodels.db:BIOMD0000000012">
 [MODEL DEFINITION]
</model>
\end{myXmlLst}
%

%%% Local Variables: 
%%% mode: latex
%%% TeX-master: "../sed-ml-L1V2"
%%% End: 
