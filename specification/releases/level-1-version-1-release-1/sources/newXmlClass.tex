  \subsubsection{\element{NewXML}}
\label{sec:newXml}

The \code{newXML} element provides a piece of XML code (\fig{sedChange}). 
\code{NewXML} must hold a valid piece of XML which after insertion into the original model must lead to a valid model file, according to the model language specification (as given by the \hyperref[sec:language]{language} attribute).

%\sedfig[width=0.35\textwidth]{newXml}{The \code{NewXML} class}{fig:newXml}

\tabtext{newXML}{newXML}

%
\begin{table}[h!]
\center
\begin{tabular}{|l|l|}
\hline
\textbf{\attribute} & \textbf{\desc}\\
\hline
\emph{none} & \\
\hline
\hline
\textbf{\subelements} & \textbf{\desc}\\
\hline
\emph{anyXML} & \\
\hline
\end{tabular}
\caption{\tabcap{newXML}}
\label{tab:newXML}
\end{table}
%


The \code{newXML} element is used at two different places inside SED-ML \LoneVone:
%
\begin{enumerate}
\item{If it is used as a sub-element of the \hyperref[class:addXML]{addXML} element, then the XML it contains  it is to be \emph{inserted as a child} of the XML element addressed by the XPath.}
\item{If it is used as a sub-element of the \hyperref[class:changeXML]{changeXML} element, then the XML it contains is to \emph{replace} the XML element addressed by the XPath.}
\end{enumerate}
%
Examples are given in the relevant change class definitions.



%%% Local Variables: 
%%% mode: latex
%%% TeX-master: "../sed-ml-L1V1"
%%% End: 
