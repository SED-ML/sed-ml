 \subsection{\element{Task}}
\label{class:task}

A task in SED-ML links a \hyperref[class:model]{model} to a certain \hyperref[class:simulation]{simulation} description via their respective identifiers (\fig{sedTask}), using the \hyperref[sec:modelReference]{modelReference} and the \hyperref[sec:simulationReference]{simulationReference}.
%
\sedfig[width=0.35\textwidth]{taskClass}{The SED-ML Task class}{fig:sedTask}
%
In SED-ML \LoneVone it is only possible to link one simulation description to one model at a time. However, one can define as many tasks as needed within one experiment description. 
Please note that the tasks may be executed in any order, as XML does not have an ordering concept.

\tabtext{task}{task}
%
\begin{table}[ht]
\center
\begin{tabular}{|l|l|}
\hline
\textbf{\attribute} & \textbf{\desc}\\
\hline
metaid$^{o}$ & \refpage{sec:metaID}\\
id & \refpage{sec:id} \\
name$^{o}$ & \refpage{sec:name}\\
\hline
modelReference & \refpage{sec:modelReference}\\
simulationReference & \refpage{sec:simulationReference}\\
\hline
\hline
\textbf{\subelements} & \textbf{\desc}\\
\hline
notes$^{o}$ & \refpage{class:notes}\\
annotation$^{o}$ & \refpage{class:annotation}\\
\hline
\end{tabular}
\caption{\tabcap{task}}
\label{tab:task}
\end{table}
%

\lsttext{task}{task}

%
\begin{myXmlLst}{The \code{task} element}{lst:task}
<listOfTasks>
  <task id="t1" name="task definition" modelReference="model1" 
        simulationReference="simulation 1" />
  <task id="t2" name="another task definition" modelReference="model2" 
        simulationReference="simulation 1" />
</listOfTasks>
\end{myXmlLst}
%
In the example, a simulation setting \emph{simulation1} is applied first to \emph{model1} and then is applied to \emph{model2}. 


%%% Local Variables: 
%%% mode: latex
%%% TeX-master: "../sed-ml-L1V1"
%%% End: 