% simulation class
\label{class:simulation}

A simulation is the execution of some defined algorithm(s). Therefore, a simulation is defined through it's \code{id}, an optional \code{name}, and the \concept{simulation algorithm} used to run the simulation. 
SED-ML makes use of the \hyperref[sec:kisao]{KiSA ontology} to refer to a term in the controlled vocabulary identifying the particular simulation algorithm to be used in the simulation. 

% Fig: sed simulation
\sedfig[width=0.5\textwidth]{simulationClass}{The SED-ML Simulation class}{fig:sedSimulation}
%

In SED-ML \version every simulation must contain a reference to a simulation algorithm that shall be used to perform the defined simulation.


\tabtext{simulation}{Simulation}
%
\begin{table}[ht]
\center
\begin{tabular}{|l|l|}
\hline
\textbf{\attribute} & \textbf{\desc}\\
\hline
metaid & \refpage{sec:metaID}\\
id & \refpage{sec:id} \\
name & \refpage{sec:name}\\
algorithm & \refpage{sec:kisao}\\
\hline
\hline
\textbf{\subelements} & \textbf{\desc}\\
\hline
notes & \refpage{class:notes}\\
annotation & \refpage{class:annotation}\\
\hline
\end{tabular}
\label{tab:simulation}
\caption{\tabcap{Simulation}}
\end{table}

%

An example for the definition of a set of simulations is given in the XML snippet in listing \ref{lst:simulation}.
%
\begin{myXmlLst}{The SED-ML listOfSimulations element, defining three different simulations}{lst:simulation}
<listOfSimulations>
  <timeCourse id="s1" name="time course simulation of variable v1 over 100 minutes" 
   algorithm ="KiSAO:ID ">
  <timeCourse id="s2" name="time course definition for concentration of p" algorithm ="KiSAO:ID ">
  <timeCourse id="s3" name="time course definition for concentration of p" algorithm ="KiSAO:ID ">
</listOfSimulations>
\end{myXmlLst}
%
%%% Local Variables: 
%%% mode: plain-tex
%%% TeX-master: "../sed-ml-L1V1"
%%% End: 
