% ChangeAttribute Class
  \subsubsection{\element{ChangeAttribute}}
\label{class:changeAttribute}
The \concept{ChangeAttribute} class allows to define updates on the XML attribute values of the corresponding model (\fig{changeAttribute}).
%
\sedfig[width=0.75\textwidth]{changeAttributeClass}{The \code{ChangeAttribute} class}{fig:changeAttribute}
%
 
The \concept{ChangeXML} class covers the possibilities provided by the \hyperref[class:changeAttribute]{ChangeAttribute} class. That is, everything that can be expressed by a \hyperref[class:changeAttribute]{ChangeAttribute} construct can also be expressed by a \concept{ChangeXML}. However, both concepts exist to allow for being very specific in defining changes. It is recommended to use the \concept{ChangeAttribute} for any changes of an XML attribute's value, and to use the more general \hyperref[class:changeXml]{ChangeXML} for all other cases.

\concept{ChangeAttribute} requires to specify the \hyperref[sec:target]{target} of change, i.\,e.\ the location of the addressed XML attribute, and also the \hyperref[sec:newValue]{new value} of that attribute.


\tabtext{changeAttribute}{changeAttribute}
%
\begin{table}[h!]
\center
\begin{tabular}{|l|l|}
\hline
\textbf{\attribute} & \textbf{\desc}\\
\hline
metaid$^{o}$ & \refpage{sec:metaID}\\
id & \refpage{sec:id} \\
name$^{o}$ & \refpage{sec:name}\\
\hline
target & \refpage{sec:target}\\
newValue & \refpage{sec:newValue}\\
\hline
\hline
\textbf{\subelements} & \textbf{\desc}\\
\hline
notes$^{o}$ & \refpage{class:notes}\\
annotation$^{o}$ & \refpage{class:annotation}\\
\hline
\end{tabular}
\caption{\tabcap{ChangeAttribute}}
\label{tab:changeAttribute}
\end{table}
%


\paragraph{\element{newValue}}
\label{sec:newValue}
The mandatory \code{newValue} attribute assignes a new value to the targeted XML attribute. 

The example in Listing~\ref{lst:changeAttribute} shows the update of the initial concentration of two parameters inside an SBML model.
%
\begin{myXmlLst}{The \code{changeAttribute} element and its \code{newValue} attribute}{lst:changeAttribute}
<model id="model1" name="Circadian Chaos" language="urn:sedml:language:sbml" 
       source="urn:miriam:biomodels.db:BIOMD0000000021">
 <listOfChanges>
  <changeAttribute target="/sbml:sbml/sbml:model/sbml:listOfParameters/sbml:parameter[@id='V_mT']/@value" newValue="0.28"/>
  <changeAttribute target="/sbml:sbml/sbml:model/sbml:listOfParameters/sbml:parameter[@id='V_dT']/@value" newValue="4.8"/>
 </listOfChanges>
</model>
\end{myXmlLst}
%

%%% Local Variables: 
%%% mode: latex
%%% TeX-master: "../sed-ml-L1V1"
%%% End: 
