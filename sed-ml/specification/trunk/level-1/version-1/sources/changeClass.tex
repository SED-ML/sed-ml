% Change Class
\label{class:change}
SED-ML not only allows to use the sole model for simlation, but on the contrary enables the description of \concept{changes} to be made on the model before simulation  (\fig{sedChange}). Changes can be of three different types:
\begin{enumerate}
 \item{Changes on attributes of the model's XML representation (\hyperref[class:changeAttribute]{ChangeAttribute})}
 \item{Changes on any XML snippet of the model's XML representation (\hyperref[class:addXml]{AddXML}, \hyperref[class:changeXml]{ChangeXML}, \hyperref[class:removeXml]{RemoveXML})}
 \item{Changes based on mathematical calculations (\hyperref[class:computeChange]{ComputeChange})} 
 \end{enumerate}

The \concept{Change} class is abstract and serves as the container for the different types of changes. Therefore, a SED-ML document will only contain the derived classes, i.\,e. \hyperref[class:changeAttribute]{ChangeAttribute}, \hyperref[class:addXml]{AddXML}, \hyperref[class:changeXml]{ChangeXML}, \hyperref[class:removeXml]{RemoveXML}, or \hyperref[class:computeChange]{ComputeChange}.
%
\sedfig{changeClass}{The SED-ML Change class}{fig:sedChange}
%

\tabtext{change}{change}
%
\begin{table}[ht]
\center
\begin{tabular}{|l|l|}
\hline
\textbf{\attribute} & \textbf{\desc}\\
\hline
metaid$^{o}$ & \refpage{sec:metaID}\\
id & \refpage{sec:id} \\
name$^{o}$ & \refpage{sec:name}\\
\hline
target & \refpage{sec:target}\\
\hline
\hline
\textbf{\subelements} & \textbf{\desc}\\
\hline
notes$^{o}$ & \refpage{class:notes}\\
annotation$^{o}$ & \refpage{class:annotation}\\
\hline
addXML$^{o}$ & \refpage{class:addXml}\\
changeXML$^{o}$ & \refpage{class:changeXml}\\
removeXML$^{o}$ & \refpage{class:removeXml}\\
changeAttribute$^{o}$ & \refpage{class:changeAttribute}\\
computeChange$^{o}$ & \refpage{class:computeChange}\\
\hline
\end{tabular}
\label{tab:change}
\caption{\tabcap{change}}
\end{table}
%

Each Change has a \hyperref[sec:target]{target} attribute that holds a valid XPath expression pointing to the XML element or XML attribute that is to undergo the defined changes.

%A typical example for a model update (or change) is the assignment of new parameter values to the model. 

%%% Local Variables: 
%%% mode: latex
%%% TeX-master: "../sed-ml-L1V1"
%%% End: 
