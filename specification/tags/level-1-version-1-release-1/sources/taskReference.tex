\subsubsection{taskReference}
\label{sec:taskReference}
\hyperref[class:dataGenerator]{DataGenerator} objects are created to apply post-processing to the simulation results before simulation output. 

For certain types of post-processing \hyperref[class:variable]{Variable} objects need to be created. These link to a defined \hyperref[class:task]{Task} from which the model that contains the variable of interest can be inferred. 
A \concept{taskReference} association is used to realise that link from a \hyperref[class:variable]{Variable} object inside a \hyperref[class:dataGenerator]{DataGenerator} to a \hyperref[class:task]{Task} object. 
Listing \ref{lst:reference3} gives an example.
%
\begin{myXmlLst}{SED-ML \code{taskReference} definition inside a \element{dataGenerator} element}{lst:reference3}
<listOfDataGenerators>
 <dataGenerator id="tim3" name="tim mRNA (difference v1-v2+20)">
  <listOfVariables>
   <variable id="v1" taskReference="t1" [..] />
  </listOfVariables>
  <math [..]/>
 </dataGenerator>
</listOfDataGenerators>
\end{myXmlLst}
%
The example shows the definition of a variable \code{v1} in a \code{dataGenerator} element. The variable appears in the model that is used in task \code{t1}. The task definition of \code{t1} might look as shown in Listing~\ref{lst:taskReferences}.
\begin{myXmlLst}{Use of the reference relations in a task definition}{lst:taskReferences}
<listOfTasks>
  <task id="t1" name="task definition" modelReference="model1" simulationReference="simulation1" />
</listOfTasks>
\end{myXmlLst}
Task \code{t1} references the model \code{model1}. Therefore we can conclude that the variable \code{v1} defined in listing \ref{lst:reference3} targets an element of the model with ID \code{model1}. The targeting process itself will be explained in section \ref{sec:target} on \refpage{sec:target}.
