% introduction
\chapter{Introduction}

The number of available computational models of biological systems is growing at an ever increasing pace. 
At the same time, their size and complexity are also increasing. The need to build on existing studies by reusing models therefore becomes more imperative. 
It is now generally accepted that one needs to be able to exchange the biochemical and mathematical structure of models. 
The efforts to standardise the representation of computational models in various areas of biology, such as the \emph{Systems Biology Markup Language} (SBML, \citep{Hucka:2003}), \emph{CellML} \citep{cuellar:2003} or \emph{NeuroML} \citep{Goddard:2001}, resulted in such an increase of the exchange and re-use of models. 
However, the description of the structure of models is not sufficient to enable the reproduction of simulation results.  
One also needs to describe the procedures the models are subjected to, as described by the \emph{Minimum Information About a Simulation Experiment (MIASE)} \citep{Waltemath:2011}. 

This document presents  \LoneVtwo of the \emph{Simulation Experiment Description Markup Language} (SED-ML), a computer-readable format for encoding simulation experiments. 
SED-ML files are encoded in the \emph{eXtensible Markup Language} (XML) \citep{Bray:2006}. The SED-ML format is defined by an XML Schema \citep{Fallside:2001}. 
\LonVtwo is the successor of \LoneVone, which is described in \citep{WAB+11}.

%% EXAMPLE OF A SIMULATION EXPERIMENT DESCRIPTION (INFORMAL)
\input{sources/motivation}


%%% Local Variables: 
%%% mode: latex
%%% TeX-master: "../sed-ml-L1V2"
%%% End: 
