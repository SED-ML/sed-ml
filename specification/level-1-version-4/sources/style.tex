\begin{blockChanged}
% ~~~~~~~~~~~~~~~~~~~~~~~~~~~~~~~~~~~~~~~~
% STYLE
% ~~~~~~~~~~~~~~~~~~~~~~~~~~~~~~~~~~~~~~~~
\subsection{\element{Style}}
\label{class:style}

The \Style class (\fig{style}) defines a graphical style for use in \Figure or \Plot elements.

\sedfig[width=0.6\textwidth]{images/uml/style}{The SED-ML \Style class}{fig:style}

The \concept{Style} class inherits the attributes and children from \SedBase, extending the \element{id} attribute to be required, adding an optional \element{baseStyle} of type \SIdRef, and allowing up to three optional chidren of type \Line, \Marker, and \Fill.  Collectively, these elements describe a visual style that can be applied to elements of an \Output.

\paragraph*{\element{baseStyle}}
The optional \element{baseStyle} attribute of data type \SIdRef is used to reference a different \Style in the same \SedDocument.  If present, any defined aspect of the referenced \Style is assumed to apply to the current \Style, unless superseded by an element of the current \Style.  For example, if one \Style \val{style1} defines a black line with a blue marker, and a second \Style \val{style2} has a \element{baseStyle} of \val{style1} and defines a red line, applying a \val{style2} would result in a red line with a blue marker.


\subsubsection{\element{Line}}
\label{class:line}

The \Line class inherits the attributes and children of \SedBase, and adds three optional attributes: \element{type} of type \LineType, \element{color} of type \SedColor, and \element{thickness} of type \element{double}.  If any of these attributes are defined, lines presented in the parent \Style should have that type, color, and/or thickness.  If any of the attributes is not defined, it can be defined by the \Style referenced in the \element{baseStyle}, or is undefined and can be anything.

\paragraph*{\element{type}}

The \element{type} attribute defines how lines are to be drawn.  The options are:

\begin{itemize}
        \item \textbf{\element{none}}: The line is not to be displayed at all.
        \item \textbf{\element{solid}}: The line is to be displayed as a continuous line.
        \item \textbf{\element{dash}}: The line is to be displayed as a series of short lines.
        \item \textbf{\element{dot}}: The line is to be displayed as a series of dots.
        \item \textbf{\element{dashDot}}: The line is to be displayed as a series of single lines and single dot combinations.
        \item \textbf{\element{dashDotDot}}: The line is to be displayed as a series of single lines and two dot combinations.
\end{itemize}

\paragraph*{\element{color}}

The \element{color} attribute defines what color the line should be.  See the \SedColor for a description of how colors are defined in SED-ML.

\paragraph*{\element{thickness}}

The \element{thickness} attribute defines the thickness of the line, in pixels (or the equivalent in the application's display environment).


\subsubsection{\element{Marker}}
\label{class:marker}

The \Marker class inherits the attributes and children of \SedBase, and adds five optional attributes: \element{type} of type \MarkerType, \element{size} of type \element{double}, \element{fill} of type \SedColor, \element{lineColor} of type \SedColor, and \element{lineThickness} of type \element{double}.  If any of these attributes are defined, markers presented in the parent \Style should have that attribute.  If any of the attributes is not defined, it can be defined by the \Style referenced in the \element{baseStyle}, or is undefined and can be anything.

\paragraph*{\element{type}}

The \element{type} attribute defines how markers are to be drawn.  The options are:

\begin{itemize}
        \item \textbf{\element{none}}: The marker is not to be displayed at all.
        \item \textbf{\element{square}}: The marker is to be displayed as a square.
        \item \textbf{\element{circle}}: The marker is to be displayed as a circle.
        \item \textbf{\element{diamond}}: The marker is to be displayed as a diamond.
        \item \textbf{\element{xCross}}: The marker is to be displayed as an `x'.
        \item \textbf{\element{plus}}: The marker is to be displayed as a plus.
        \item \textbf{\element{star}}: The marker is to be displayed as a star.
        \item \textbf{\element{triangleUp}}: The marker is to be displayed as an upwards-pointing triangle.
        \item \textbf{\element{triangleDown}}: The marker is to be displayed as a downwards-pointing triangle.
        \item \textbf{\element{triangleLeft}}: The marker is to be displayed as a left-pointing triangle.
        \item \textbf{\element{triangleRight}}: The marker is to be displayed as a right-pointing triangle.
        \item \textbf{\element{hDash}}: The marker is to be displayed as a horizontal dash.
        \item \textbf{\element{vDash}}: The marker is to be displayed as a vertical dash.
\end{itemize}

\paragraph*{\element{size}}

The \element{size} attribute defines what size, in pixels, the marker should be (or the equivalent in the application's display environment.

\paragraph*{\element{fill}}

The \element{fill} attribute defines what color the interior of the marker should be.  See the \SedColor for a description of how colors are defined in SED-ML.

\paragraph*{\element{lineColor}}

The \element{lineColor} attribute defines what color the border of the marker should be.  See the \SedColor for a description of how colors are defined in SED-ML.

\paragraph*{\element{lineThickness}}

The \element{thickness} attribute defines the thickness of the marker's border, in pixels (or the equivalent in the application's display environment).



\subsubsection{\element{Fill}}
\label{class:fill}

The \Fill class inherits the attributes and children of \SedBase, and adds two optional attributes: \element{color} of type \SedColor, and \element{secondColor} of type \SedColor.  If any of these attributes are defined, fills presented in the parent \Style should have that color or colors.  If any of the attributes is not defined, it can be defined by the \Style referenced in the \element{baseStyle}, or is undefined and can be anything.

\paragraph*{\element{color}}

The \element{color} attribute defines what color the fill should be.  See the \SedColor for a description of how colors are defined in SED-ML.

\paragraph*{\element{secondColor}}

The \element{secondColor} attribute defines what the second color of the fill should be.  See the \SedColor for a description of how colors are defined in SED-ML.  By providing a \element{secondColor}, gradients can be specified which run linearly from \element{color} to \element{secondColor}.  If not defined, the fill should be a single color.




\end{blockChanged}
