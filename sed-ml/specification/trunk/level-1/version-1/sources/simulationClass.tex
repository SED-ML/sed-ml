% simulation class
\label{class:simulation}

A simulation is the execution of some defined algorithm(s). Therefore, a simulation is defined through it's \code{id}, an optional \code{name}, and the \concept{simulation algorithm} used to run the simulation. 
Simulations are described differently depending on the type of simulation experiment to be performed. SED-ML \LoneVone does only support \hyperref[class:uniformTimeCourse]{UniformTimeCourse} simulations.

% Fig: sed simulation
\sedfig[width=\textwidth]{simulationClass}{The SED-ML Simulation class}{fig:sedSimulation}
%


\tabtext{simulation}{Simulation}
%
\begin{table}[ht]
\center
\begin{tabular}{|l|l|}
\hline
\textbf{\attribute} & \textbf{\desc}\\
\hline
metaid$^{o}$ & \refpage{sec:metaID}\\
id & \refpage{sec:id} \\
name$^{o}$ & \refpage{sec:name}\\
\hline
\hline
\textbf{\subelements} & \textbf{\desc}\\
\hline
notes$^{o}$ & \refpage{class:notes}\\
annotation$^{o}$ & \refpage{class:annotation}\\
algorithm$^{o}$ & \refpage{class:algorithm}\\
\hline
\end{tabular}
\label{tab:simulation}
\caption{\tabcap{Simulation}}
\end{table}

%

An example for the definition of two different simulations is given in the XML snippet in listing \ref{lst:simulation}.
%
\begin{myXmlLst}{The SED-ML \code{listOfSimulations} element, defining three different simulations}{lst:simulation}
<listOfSimulations>
  <uniformTimeCourse id="s1" name="time course simulation of variable v1 over 100 minutes" [..]>
    <listOfAlgorithms>
      [ALGORITHM DEFINITION FOLLOWING]
    </listOfAlgorithms>
  </uniformTimeCourse>
  <uniformTimeCourse id="s2" name="time course definition for concentration of p" [..]>
    [..]
  </uniformTimeCourse>
</listOfSimulations>
\end{myXmlLst}
%
Two timcourses with uniform range are defined. How to define the concrete algorithm to be used inside the \hyperref[sec:listOfAlgorithms]{listOfAlgorithms} is shown on page \refpage{class:algorithm}.
%%% Local Variables: 
%%% mode: plain-tex
%%% TeX-master: "../sed-ml-L1V1"
%%% End: 
