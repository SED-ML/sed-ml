\label{class:dataGenerator}

% Fig: DG
\sedfig[width=0.9\textwidth]{dataGeneratorClass}{The SED-ML DataGenerator class}{fig:sedDG}
%


\tabtext{dataGenerator}{DataGenerator}
%
\begin{table}[ht]
\center
\begin{tabular}{|l|l|}
\hline
\textbf{\attribute} & \textbf{\desc}\\
\hline
metaid$^{o}$ & \refpage{sec:metaID}\\
id & \refpage{sec:id} \\
name$^{o}$ & \refpage{sec:name}\\
math & \refpage{sec:math}\\
%listOfVariables$^{o}$ & \refpage{sec:listOfVariables}\\
%listOfParameters$^{o}$ & \refpage{sec:listOfParameters}\\
\hline
\hline
\textbf{\subelements} & \textbf{\desc}\\
\hline
notes$^{o}$ & \refpage{class:notes}\\
annotation$^{o}$ & \refpage{class:annotation}\\
variable$^{o}$ & \refpage{class:variable}\\
parameter$^{o}$ & \refpage{class:parameter}\\
\hline
\end{tabular}
\label{tab:dataGenerator}
\caption{\tabcap{DataGenerator}}
\end{table}
%

The \concept{DataGenerator} prepares the simulation data for usage in the \concept{output}. Often, data needs to be post-processed before being returned to the user. Those post-processing steps can be simple normalisation of data, but also mathematical calculations. The data generator describes the rules to build the \concept{Output} from simulation results and existing entities. The data Generator class is shown in Figure \ref{fig:sedDG}.


An example for a data generator is given in listing \ref{lst:listOfDataGenerators}.
%
\begin{myXmlLst}{The listOfDataGenerators element, defining two data generators \emph{time} and \emph{LaCI repressor}}{lst:listOfDataGenerators}
<listOfDatGenerators>
 <dataGenerator id="d1" name="time">
  <listOfVariables>
   <variable id="time" taskReference="task1" target="time" />
  </listOfVariables >
  <listOfParameters />
  <math xmlns="http://www.w3.org/1998/Math/MathML">
   <ci> time </ci>
  </math>
 </dataGenerator>
 <dataGenerator id="LaCI" name="LaCI repressor">
  <listOfVariables>
   <variable id="v1" taskReference="task1" 
    target="/sbml:sbml/sbml:model/sbml:listOfSpecies/
            sbml:species[@id='PX']" />
  </listOfVariables>
  <math:math>
   <math:ci>v1</math:ci>
  </math:math>
 </dataGenerator>
</listOfDataGenerators>
\end{myXmlLst}
%

%% Todo: better example for math

%% Change in schema: The math attribute in the UML would become an attribuet in the XMLS -> maybe better to define a Math class (for consistent conversion and reuse of the Math class)

%% Comment: listOfVariables and listOfParameters should be global lists which are then referred to from inside the listOfDataGenerator/listOfChanges and so on -> propose a change in the XML schema

%% variable has a task reference -> need 2 variable types, as it seems -> propose change in the schema



%%% Local Variables: 
%%% mode: latex
%%% TeX-master: "../sed-ml-L1V1"
%%% End: 
