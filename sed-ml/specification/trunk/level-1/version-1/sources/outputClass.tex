% Output 
\label{class:output}

The ouput container describes how the results of a simulaion should be presented to the user. It does not contain the data itself, but only the type of output and the data generators (\concept{dataGenerator}) which are to be used in the desired output type. Figure \ref{fig:sedOutput} shows the UML class definition of the \concept{Output} class.
% Fig: sed output
\sedfig{outputClass}{The SED-ML Output class}{fig:sedOutput}
%


\tabtext{output}{Output}
%
\begin{table}[ht]
\center
\begin{tabular}{|l|l|}
\hline
\textbf{\attribute} & \textbf{description}\\
\hline
metaid & \refpage{sec:metaID}\\
id & \refpage{sec:id} \\
name & \refpage{sec:name}\\
\hline
\hline
\textbf{sub-elements} & \textbf{description}\\
\hline
notes & \refpage{class:notes}\\
annotation & \refpage{class:annotation}\\
plot2D & \refpage{class:plot2D}\\
plot3D & \refpage{class:plot3D}\\
report & \refpage{class:report}\\
\hline
\end{tabular}
\label{tab:output}
\caption{\tabcap{Output}}
\end{table}
%


The types of output pre-defined in SED-ML are plots and \hyperref[class:report]{reports}, also referrible as data tables. The output can be defined as a \hyperref[class:plot2D]{2D plot} or alternatively as a \hyperref[class:plot3D]{3D plot}. 

%An example for a 2D plot and a report defined in one output is given in listing \ref{lst:listOfOutputs}.
%
%\begin{myXmlLst}{The listOfOutput element}{lst:listOfOutputs}
%<listOfOutputs>
% <plot2D id="plot1" name="sample 2D plot">
%  <listOfCurves> 
%   \alert{[..]}
%  </listOfCurves>
% </plot2D>
%</listOfOutputs>
%\end{myXmlLst}
%


%%% Local Variables: 
%%% mode: latex
%%% TeX-master: "../sed-ml-L1V1"
%%% End: 
