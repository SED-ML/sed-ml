%%%%%%%%%%%%%%%%%%%%%%%%%%%%%%%%%%%%%%%%%%%%%%%%%%%%%%%%%%%%%%%%%%
%%  Commands
%%%%%%%%%%%%%%%%%%%%%%%%%%%%%%%%%%%%%%%%%%%%%%%%%%%%%%%%%%%%%%%%%%

\newcommand{\code}[1]{\texttt{#1}}
\newcommand{\token}[1]{\texttt{#1}}
\newcommand{\concept}[1]{\textcolor{blue}{#1}}
\newcommand{\element}[1]{\texttt{#1}}
\newcommand{\alert}[1]{\textcolor{red}{#1}}
\newcommand{\note}[1]{\paragraph*{} \emph{\scshape{\alert{Please Note}}: #1} \newline}
\newcommand{\mailto}[1]   {\link{mailto:#1}{#1}}
\newcommand{\link}[2]     {\literalFont{\href{#1}{#2}}}
\newcommand{\literalFont}[1]{\textup{\texttt{#1}}}
\newcommand{\version}{1\xspace}
\newcommand{\level}{1\xspace}
\newcommand{\LoneVone}{Level~1 Version~1\xspace}
\newcommand{\biom}{BioModels Database\xspace}
% attribute table layout
\newcommand{\attribute}{attribute\xspace}
\newcommand{\desc}{description\xspace}
\newcommand{\subelements}{sub-elements\xspace}

\newcommand{\refpage}[1]{\hyperref[#1]{page \pageref{#1}}} % to hyperref to a particular page in the spec
\newcommand{\tabcap}[1]{  % to create table captions for overview tables for each SED-ML class
Attributes and nested elements for \concept{#1}. \emph{xy$^{o}$} denotes optional elements and attributes.
}

\newcommand{\tabtext}[2]{ % to create the introducing table text for each table reference
\tab{#1} shows all attributes and sub-elements for the \concept{#2} element as defined by the SED-ML \LoneVone XML Schema.
}

\newcommand{\lsttext}[2]{ % to create the introducing listing text for each listing reference
  Listing \ref{lst:#1} shows the use of the \element{#2} element in a SED-ML file as defined by the SED-ML \LoneVone XML Schema.
}



%figures:
\newcommand{\sedfig}[4][]
	{\begin{figure}[t]\begin{center}{\includegraphics[width=0.9\textwidth,#1]{#2}}\caption{#3}\label{#4}\end{center}\end{figure}}
%
\newcommand{\sect}[1]     {Section~\protect\ref{sec:#1}\xspace}
\newcommand{\fig}[1]      {Figure~\protect\vref{fig:#1}\xspace}
\newcommand{\tab}[1]      {Table~\protect\vref{tab:#1}\xspace}
\newcommand{\eg}          {e.g.,\xspace}
\newcommand{\ie}          {i.e.,\xspace}

\newcommand{\tickYes}{\hspace{1pt}\ding{52}}
\newcommand{\tickNo}{\hspace{1pt}\ding{56}}



%%%%%%%%%%%%%%%%%%%%%%%%%%%%%%%%%%%%%%%%%%%%%%%%%%%%%%%%%%%%%%%%%%
%%  environments
%%%%%%%%%%%%%%%%%%%%%%%%%%%%%%%%%%%%%%%%%%%%%%%%%%%%%%%%%%%%%%%%%%
% standard listings:
\lstnewenvironment{mylisting}[2]
	{\lstset{float,basicstyle=\ttfamily\scriptsize,caption={#1},label={#2},%
	numbers=left,
        stepnumber=1,
        numberstyle=\tiny,
        language=xml,
        keywordstyle=\color[rgb]{0,0,1},
        commentstyle=\color[rgb]{0.133,0.545,0.133}, 
        stringstyle=\color[rgb]{0.627,0.126,0.941}, 
        showspaces=false, 
        showstringspaces=false, 
        identifierstyle=\ttfamily}}{}

\lstnewenvironment{mylistingNoNumbers}[2]
	{\lstset{float,basicstyle=\ttfamily\footnotesize,caption={#1},label={#2}}}
	{}
% listings in appendixes:
\lstnewenvironment{mylistingAppendix}[2]
	{\lstset{float=h,basicstyle=\ttfamily\footnotesize,caption={#1},label={#2},%
	numbers=left,stepnumber=1,numberstyle=\tiny}}
	{}

\lstnewenvironment{myXmlLst}[2]
	{\lstset{basicstyle=\ttfamily\scriptsize, caption={#1},label={#2}, keywordstyle=\color{blue}\bfseries, stringstyle=\color{blue}, commentstyle=\color{red}, captionpos=b, breaklines=true, xleftmargin=1.5em, xrightmargin=1.5em, numbers=left, numberstyle=\ttfamily\tiny, numbersep=5pt, tabsize=4, showstringspaces=false, language=XML, float=h!}}
	{}
