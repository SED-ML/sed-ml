% Change Class
  \subsection[Change]{\element{Change}}
\label{class:change}
SED-ML not only allows to use the sole model for simulation, but also enables the description of \concept{changes} to be made on the model before simulation  (\fig{sedChange}). Changes can be of three distinct types:
\begin{enumerate}
 \item{Changes on attributes of the model's XML representation (\hyperref[class:changeAttribute]{ChangeAttribute})}
 \item{Changes on any XML snippet of the model's XML representation (\hyperref[class:addXml]{AddXML}, \hyperref[class:changeXml]{ChangeXML}, \hyperref[class:removeXml]{RemoveXML})}
 \item{Changes based on mathematical calculations (\hyperref[class:computeChange]{ComputeChange})} 
 \end{enumerate}

The \concept{Change} class is abstract and serves as the base class for different types of changes.
Therefore, a SED-ML document will only contain the derived classes, i.e.\ \hyperref[class:changeAttribute]{ChangeAttribute}, \hyperref[class:addXml]{AddXML}, \hyperref[class:changeXml]{ChangeXML}, \hyperref[class:removeXml]{RemoveXML}, or \hyperref[class:computeChange]{ComputeChange}.
%
\sedfig[width=\textwidth]{pdf/changeClass}{The SED-ML Change class}{fig:sedChange}
%

\tabtext{change}{change}
%
\begin{table}[h!]
\center
\begin{tabular}{|l|l|}
\hline
\textbf{\attribute} & \textbf{\desc}\\
\hline
metaid$^{o}$ & \refpage{sec:metaID}\\
id & \refpage{sec:id} \\
name$^{o}$ & \refpage{sec:name}\\
\hline
target & \refpage{sec:target}\\
\hline
\hline
\textbf{\subelements} & \textbf{\desc}\\
\hline
notes$^{o}$ & \refpage{class:notes}\\
annotation$^{o}$ & \refpage{class:annotation}\\
\hline
\end{tabular}
\caption{\tabcap{change}}
\label{tab:change}
\end{table}
%

Each Change has a \hyperref[sec:target]{target} attribute that holds a valid XPath expression pointing to the XML element or XML attribute that is to undergo the defined changes.
Except for the cases of \hyperref[class:changeXml]{ChangeXML} and \hyperref[class:removeXml]{RemoveXML}, this XPath expression must always select a \emph{single} element or attribute within the relevant model.

    \subsubsection{\element{NewXML}}
\label{sec:newXml}

The \code{newXML} element provides a piece of XML code (\fig{sedChange}). 
\code{NewXML} must hold a valid piece of XML which after insertion into the original model must lead to a valid model file, according to the model language specification (as given by the \hyperref[sec:language]{language} attribute).

%\sedfig[width=0.35\textwidth]{newXml}{The \code{NewXML} class}{fig:newXml}

\tabtext{newXML}{newXML}

%
\begin{table}[h!]
\center
\begin{tabular}{|l|l|}
\hline
\textbf{\attribute} & \textbf{\desc}\\
\hline
\emph{none} & \\
\hline
\hline
\textbf{\subelements} & \textbf{\desc}\\
\hline
\emph{anyXML} & \\
\hline
\end{tabular}
\caption{\tabcap{newXML}}
\label{tab:newXML}
\end{table}
%


The \code{newXML} element is used at two different places inside SED-ML \LoneVtwo:
%
\begin{enumerate}
\item{If it is used as a sub-element of the \hyperref[class:addXML]{addXML} element, then the XML it contains  it is to be \emph{inserted as a child} of the XML element addressed by the XPath.}
\item{If it is used as a sub-element of the \hyperref[class:changeXML]{changeXML} element, then the XML it contains is to \emph{replace} the XML element addressed by the XPath.}
\end{enumerate}
%
Examples are given in the relevant change class definitions.



%%% Local Variables: 
%%% mode: latex
%%% TeX-master: "../sed-ml-L1V2"
%%% End: 


  % ChangeAttribute Class
  \subsubsection{\element{AddXML}}
\label{class:addXml}
The \concept{AddXML} class specifies a snippet of XML that is to be added as a child of the specified XPath \hyperref[sec:target]{target} attribute (\fig{addXMLClass}). 
The new piece of XML code is provided by the \hyperref[sec:newXml]{NewXML} class.
%
\sedfig[width=0.75\textwidth]{addXMLClass}{The SED-ML \code{AddXML} class}{fig:addXMLClass}
%

\tabtext{addXml}{addXml}
%
\begin{table}[ht]
\center
\begin{tabular}{|l|l|}
\hline
\textbf{\attribute} & \textbf{\desc}\\
\hline
metaid$^{o}$ & \refpage{sec:metaID}\\
id & \refpage{sec:id} \\
name$^{o}$ & \refpage{sec:name}\\
target & \refpage{sec:target}\\
\hline
\hline
\textbf{\subelements} & \textbf{\desc}\\
\hline
notes$^{o}$ & \refpage{class:notes}\\
annotation$^{o}$ & \refpage{class:annotation}\\
\hline
newXML & \refpage{sec:newXml}\\
\hline
\end{tabular}
\caption{\tabcap{addXML}}
\label{tab:addXml}
\end{table}
%

An example for a change that adds an additional parameter to a model is given in listing \ref{lst:addXML}.
%
\begin{myXmlLst}{The \code{addXML} element with its \code{newXML} sub-element}{lst:addXML}
<model language="urn:sedml:language:sbml" [..]>
 <listOfChanges>
  <addXML target="/sbml:sbml/sbml:model/sbml:listOfParameters" >
   <newXML>
     <parameter metaid="metaid_0000010" id="V_mT" value="0.7" />
  </newXML>
  </addXML>
 </listOfChanges>
</model>
\end{myXmlLst}
%

The code of the model is changed so that a parameter with ID \code{V\_mT} is added to its list of parameters. The \code{newXML} element adds an additional XML element to the original model. The element's name is \code{parameter} and it is added to the existing parent element \code{listOfParameters} that is addressed by the XPath expression in the \code{target} attribute.

%%% Local Variables: 
%%% mode: latex
%%% TeX-master: "../sed-ml-L1V2"
%%% End: 


  % ChangeAttribute Class
\label{class:changeXml}
The \concept{ChangeXML} class defines changes of any XML element in the model that can be addressed by a valid XPath expression (\fig{changeXml}). 
%
\sedfig[width=0.75\textwidth]{changeXmlClass}{The \code{ChangeXML} class}{fig:changeXml}
%
The XPath is specified in the required \hyperref[sec:target]{target} attribute (see again section \ref{sec:target} on page \refpage{sec:target}). 
The change of XML is specified in the \hyperref[sec:newXml]{NewXML} class.

\tabtext{changeXml}{changeXml}
%
\begin{table}[ht]
\center
\begin{tabular}{|l|l|}
\hline
\textbf{\attribute} & \textbf{\desc}\\
\hline
metaid$^{o}$ & \refpage{sec:metaID}\\
id & \refpage{sec:id} \\
name$^{o}$ & \refpage{sec:name}\\
target & \refpage{sec:target}\\
\hline
\hline
\textbf{\subelements} & \textbf{\desc}\\
\hline
notes$^{o}$ & \refpage{class:notes}\\
annotation$^{o}$ & \refpage{class:annotation}\\
\hline
newXML & \refpage{sec:newXml}\\
\hline
\end{tabular}
\label{tab:changeXml}
\caption{\tabcap{changeXML}}
\end{table}
%

An example for a change that adds an additional parameter to a model is given in listing \ref{lst:changeXML}.
%
\begin{myXmlLst}{The \code{changeXML} element}{lst:changeXML}
<model [..]>
 <listOfChanges>
  <changeXML target="/sbml:sbml/sbml:model/sbml:listOfParameters/sbml:parameter[@id='V_mT']" >
   <newXML>
     <parameter metaid="metaid_0000010" id="V_mT_1" value="0.7" />
     <parameter metaid="metaid_0000050" id="V_mT_2" value="4.6"> />
  </newXML>
  </changeXML>
 </listOfChanges>
</model>
\end{myXmlLst}
%
The code of the model is changed in the way that its parameter with ID \code{V\_mT} is substituted by two other parameters \code{V\_mT\_1} and \code{V\_mT\_2}.
The \code{target} attribute defines that the parameter with ID \code{V\_mT} is to be changed. The \code{newXML} element then specifies the XML that is to be  exchanged for  that parameter.


%%% Local Variables: 
%%% mode: latex
%%% TeX-master: "../sed-ml-L1V1"
%%% End: 


  % ChangeAttribute Class
\subsubsection{\element{RemoveXML}}
\label{class:removeXml}
The \concept{RemoveXML} class can be used to delete the XML element of the model that is addressed by the XPath expression (\fig{removeXml}).
%
\sedfig[width=0.75\textwidth]{removeXmlClass}{The \code{RemoveXML} class}{fig:removeXml}
%

The XPath is specified in the required \hyperref[sec:target]{target} attribute. 

\tabtext{removeXml}{removeXml}
%
\begin{table}[ht]
\center
\begin{tabular}{|l|l|}
\hline
\textbf{\attribute} & \textbf{\desc}\\
\hline
metaid$^{o}$ & \refpage{sec:metaID}\\
id & \refpage{sec:id} \\
name$^{o}$ & \refpage{sec:name}\\
target & \refpage{sec:target}\\
\hline
\hline
\textbf{\subelements} & \textbf{\desc}\\
\hline
notes$^{o}$ & \refpage{class:notes}\\
annotation$^{o}$ & \refpage{class:annotation}\\
\hline
\end{tabular}
\caption{\tabcap{removeXML}}
\label{tab:removeXml}
\end{table}
%

An example for the removal of an XML element from a model is given in Listing~\ref{lst:removeXML}.
%
\begin{myXmlLst}{The \code{removeXML} element}{lst:removeXML}
<model [..]>
 <listOfChanges>
  <removeXML target="/sbml:sbml/sbml:model/sbml:listOfReactions/sbml:reaction[@id='J1']" />
 </listOfChanges>
</model>
\end{myXmlLst}
%

The code of the model is changed by deleting the reaction with ID \code{V\_mT} from the model's list of reactions.


%%% Local Variables: 
%%% mode: latex
%%% TeX-master: "../sed-ml-L1V1"
%%% End: 


  % ChangeAttribute Class
  \subsubsection{\element{ChangeAttribute}}
\label{class:changeAttribute}
The \concept{ChangeAttribute} class allows to define updates on the XML attribute values of the corresponding model (\fig{changeAttribute}).
%
\sedfig[width=0.75\textwidth]{changeAttributeClass}{The \code{ChangeAttribute} class}{fig:changeAttribute}
%
 
The \concept{ChangeXML} class covers the possibilities provided by the \hyperref[class:changeAttribute]{ChangeAttribute} class. That is, everything that can be expressed by a \hyperref[class:changeAttribute]{ChangeAttribute} construct can also be expressed by a \concept{ChangeXML}. However, both concepts exist to allow for being very specific in defining changes. It is recommended to use the \concept{ChangeAttribute} for any changes of an XML attribute, and to use the more general \hyperref[class:changeXml]{ChangeXML} for all other cases.

\concept{ChangeAttribute} requires to specify the \hyperref[sec:target]{target} of change, i.\,e.\ the location of the addressed XML attribute, and also the \hyperref[sec:newValue]{new value} of that attribute.


\tabtext{changeAttribute}{changeAttribute}
%
\begin{table}[h!]
\center
\begin{tabular}{|l|l|}
\hline
\textbf{\attribute} & \textbf{\desc}\\
\hline
metaid$^{o}$ & \refpage{sec:metaID}\\
id & \refpage{sec:id} \\
name$^{o}$ & \refpage{sec:name}\\
\hline
target & \refpage{sec:target}\\
newValue & \refpage{sec:newValue}\\
\hline
\hline
\textbf{\subelements} & \textbf{\desc}\\
\hline
notes$^{o}$ & \refpage{class:notes}\\
annotation$^{o}$ & \refpage{class:annotation}\\
\hline
\end{tabular}
\caption{\tabcap{ChangeAttribute}}
\label{tab:changeAttribute}
\end{table}
%


\subsubsection{The \code{newValue} attribute}
\label{sec:newValue}
The mandatory \code{newValue} attribute assignes a new value to the targeted XML attribute. 

The example in listing \ref{lst:changeAttribute} shows the update of the initial concentration of two parameters inside an SBML model.
%
\begin{myXmlLst}{The \code{changeAttribute} element and its \code{newValue} attribute}{lst:changeAttribute}
<model id="model1" name="Circadian Chaos" language="urn:sedml:language:sbml" 
       source="urn:miriam:biomodels.db:BIOMD0000000021">
 <listOfChanges>
  <changeAttribute target="/sbml:sbml/sbml:model/sbml:listOfParameters/sbml:parameter[@id='V_mT']/@value" newValue="0.28"/>
  <changeAttribute target="/sbml:sbml/sbml:model/sbml:listOfParameters/sbml:parameter[@id='V_dT']/@value" newValue="4.8"/>
 </listOfChanges>
</model>
\end{myXmlLst}
%

%%% Local Variables: 
%%% mode: latex
%%% TeX-master: "../sed-ml-L1V1"
%%% End: 


  % ChangeAttribute Class
  \subsubsection{\element{ComputeChange}}
\label{class:computeChange}
The \concept{ComputeChange} class permits to change, prior to the experiment, the value of any element or attribute of a model addressable by an XPath expression, based on a calculation (\fig{computeChange}). 
%
\sedfig[width=\textwidth]{computeChangeClass}{The \code{ComputeChange} class}{fig:computeChange}
%
The changes are described by mathematical expressions using a \hyperref[sec:mathML]{subset of MathML} (see section \ref{sec:mathML} on \refpage{sec:mathML}). The computation can use the value of variables from any model defined in the simulation experiment. Those \hyperref[class:variable]{variables} need to be defined, and can then be addressed by their ID. A variable used in a \concept{ComputeChange} must carry a modelReference attribute (\refpage{sec:modelReference}) but no taskReference attribute (\refpage{sec:taskReference}). To carry out the calculation it may be necessary to introduce additional parameters, that are not defined in any of the model used by the experiment. This is done through the \hyperref[class:parameter]{parameter} class, thereafter refered to by their ID.  Finally, the change itself is specified using an instance of the \hyperref[sec:math]{Math} class.


\tabtext{computeChange}{computeChange}
%
\begin{table}[ht]
\center
\begin{tabular}{|l|l|}
\hline
\textbf{\attribute} & \textbf{\desc}\\
\hline
metaid$^{o}$ & \refpage{sec:metaID}\\
id & \refpage{sec:id} \\
name$^{o}$ & \refpage{sec:name}\\
\hline
target & \refpage{sec:target}\\
\hline
\hline
\textbf{\subelements} & \textbf{\desc}\\
\hline
notes$^{o}$ & \refpage{class:notes}\\
annotation$^{o}$ & \refpage{class:annotation}\\
\hline
listOfVariables$^{o}$ & \refpage{sec:listOfVariables}\\
listOfParameters$^{o}$ & \refpage{sec:listOfParameters}\\
math &\refpage{sec:math}\\
\hline
\end{tabular}
\caption{\tabcap{computeChange}}
\label{tab:computeChange}
\end{table}
%


\paragraph{\element{Math}}
\label{sec:math}

The \element{Math} element encodes mathematical functions. 
If used as an element of the \concept{ComputeChange} class, it computes the change of the element or attribute addressed by the \hyperref[sec:target]{target} attribute.
\LoneVtwo supports the subset of MathML 2.0 shown in section \ref{sec:mathML}.

\lsttext{computeChange}{computeChange}
%
\begin{myXmlLst}{The computeChange element}{lst:computeChange}
<model [..]>
    <computeChange target="/sbml:sbml/sbml:model/sbml:listOfParameters/sbml:parameter[@id='sensor']">
      <listOfVariables>
        <variable modelReference="model1" id="R" name="regulator" 
                  target="/sbml:sbml/sbml:model/sbml:listOfSpecies/sbml:species[@id='regulator']" />
        <variable modelReference="model2" id="S" name="sensor"
                  target="/sbml:sbml/sbml:model/sbml:listOfParameters/sbml:parameter[@id='sensor']" />
      <listOfVariables/>
      <listOfParameters>
        <parameter id="n" name="cooperativity" value="2">
        <parameter id="K" name="sensitivity" value="1e-6">
      <listOfParameters/>
      <math  xmlns="http://www.w3.org/1998/Math/MathML>
        <apply>
          <times />
          <ci>S</ci>
          <apply>
            <divide />
            <apply>
              <power />
              <ci>R</ci>
              <ci>n</ci>
            </apply>
            <apply>
              <plus />
              <apply>
                <power />
                <ci>K</ci>
                <ci>n</ci>
              </apply>
              <apply>
                <power />
                <ci>R</ci>
                <ci>n</ci>
              </apply>
            </apply> 
          </apply>
        </math>
    </computeChange>
  </listOfChanges>
</model>
\end{myXmlLst}
%

The example in listing \ref{lst:computeChange} computes a change of the variable \code{sensor} of the model \code{model2}. To do so, it uses the value of the variable \code{regulator} coming from model \code{model1}. In addition, the calculation used two additional parameters, the cooperativity \code{n}, and the sensitivity \code{K}.
The mathematical expression in the mathML then computes the new initial value of \code{sensor} using the equation:

\begin{math}
S =  S \times \frac{R^{n}}{K^{n}+R^{n}}
\end{math}
.
%A problem arises, because the individual supported model exchange languages allow different subsets of MathML. Thus, when an instance of ComputeChange replaces a %mathematical expression of  an SBML reaction, only the MathML subset allowed by SBML should be used here.


%%% Local Variables: 
%%% mode: latex
%%% TeX-master: "../sed-ml-L1V2"
%%% End: 



%%% Local Variables: 
%%% mode: latex
%%% TeX-master: "../sed-ml-L1V2"
%%% End: 
