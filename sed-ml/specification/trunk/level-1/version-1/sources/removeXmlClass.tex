% ChangeAttribute Class
\subsubsection{\element{RemoveXML}}
\label{class:removeXml}
The \concept{RemoveXML} class can be used to delete the XML element of the model that is addressed by the XPath expression (\fig{removeXml}).
%
\sedfig[width=0.75\textwidth]{removeXmlClass}{The \code{RemoveXML} class}{fig:removeXml}
%

The XPath is specified in the required \hyperref[sec:target]{target} attribute. 

\tabtext{removeXml}{removeXml}
%
\begin{table}[ht]
\center
\begin{tabular}{|l|l|}
\hline
\textbf{\attribute} & \textbf{\desc}\\
\hline
metaid$^{o}$ & \refpage{sec:metaID}\\
id & \refpage{sec:id} \\
name$^{o}$ & \refpage{sec:name}\\
target & \refpage{sec:target}\\
\hline
\hline
\textbf{\subelements} & \textbf{\desc}\\
\hline
notes$^{o}$ & \refpage{class:notes}\\
annotation$^{o}$ & \refpage{class:annotation}\\
\hline
\end{tabular}
\caption{\tabcap{removeXML}}
\label{tab:removeXml}
\end{table}
%

An example for the removal of an XML element from a model is given in Listing~\ref{lst:removeXML}.
%
\begin{myXmlLst}{The \code{removeXML} element}{lst:removeXML}
<model [..]>
 <listOfChanges>
  <removeXML target="/sbml:sbml/sbml:model/sbml:listOfReactions/sbml:reaction[@id='J1']" />
 </listOfChanges>
</model>
\end{myXmlLst}
%

The code of the model is changed by deleting the reaction with ID \code{V\_mT} from the model's list of reactions.


%%% Local Variables: 
%%% mode: latex
%%% TeX-master: "../sed-ml-L1V1"
%%% End: 
