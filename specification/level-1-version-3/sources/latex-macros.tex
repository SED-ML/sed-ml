%%%%%%%%%%%%%%%%%%%%%%%%%%%%%%%%%%%%%%%%%%%%%%%%%%%%%%%%%%%%%%%%%%
%%  Commands
%%%%%%%%%%%%%%%%%%%%%%%%%%%%%%%%%%%%%%%%%%%%%%%%%%%%%%%%%%%%%%%%%%

\newcommand{\code}[1]{\texttt{#1}}
\newcommand{\token}[1]{\texttt{#1}}
\newcommand{\concept}[1]{\textcolor{blue}{#1}}
\newcommand{\element}[1]{\texttt{#1}}
\newcommand{\alert}[1]{\textcolor{red}{#1}}
\newcommand{\note}[1]{\paragraph*{} \emph{\scshape{\alert{Please Note}}: #1} \newline}
\newcommand{\mailto}[1]   {\link{mailto:#1}{#1}}
\newcommand{\link}[2]     {\literalFont{\href{#1}{#2}}}
\newcommand{\literalFont}[1]{\textup{\texttt{#1}}}
\newcommand{\version}{2\xspace}
\newcommand{\level}{1\xspace}
\newcommand{\LoneVone}{Level~1 Version~1\xspace}
\newcommand{\LoneVtwo}{Level~1 Version~2\xspace}
\newcommand{\LoneVthree}{Level~1 Version~3\xspace}
\newcommand{\currentLV}{Level~1 Version~3\xspace}
\newcommand{\previousLV}{Level~1 Version~2\xspace}
\newcommand{\biom}{BioModels Database\xspace}
% attribute table layout
\newcommand{\attribute}{attribute\xspace}
\newcommand{\desc}{description\xspace}
\newcommand{\subelements}{sub-elements\xspace}

\newcommand{\SedModel}{\hyperref[class:model]{Model}\xspace}
\newcommand{\SedDataSource}{\hyperref[class:dataSource]{DataSource}\xspace}
\newcommand{\SedDataDescription}{\hyperref[class:dataDescription]{DataDescription}\xspace}
\newcommand{\SedSlice}{\hyperref[class:slice]{Slice}\xspace}

\newcommand{\refpage}[1]{\hyperref[#1]{page \pageref{#1}}} % to hyperref to a particular page in the spec
\newcommand{\tabcap}[1]{  % to create table captions for overview tables for each SED-ML class
Attributes and nested elements for \concept{#1}. \emph{xy$^{o}$} denotes optional elements and attributes.
}

\newcommand{\tabtext}[2]{ % to create the introducing table text for each table reference
\tab{#1}~shows all attributes and sub-elements for the \concept{#2} element as defined by the SED-ML \currentLV XML Schema.
}

\newcommand{\lsttext}[2]{ % to create the introducing listing text for each listing reference
  Listing~\vref{lst:#1} shows the use of the \element{#2} element in a SED-ML file as defined by the SED-ML \currentLV XML Schema.
}

\newcommand{\lsttexta}[2]{ % to create the introducing listing text for each listing reference
  Listing~\vref{lst:#1} shows the use of the \element{#2} attribute in a SED-ML file as defined by the SED-ML \currentLV XML Schema.
}

%
\newcommand{\chap}[1]     {Chapter~\protect\ref{chap:#1}\xspace}
\newcommand{\sect}[1]     {Section~\protect\ref{sec:#1}\xspace}
\newcommand{\fig}[1]      {Figure~\protect\vref{fig:#1}\xspace}
\newcommand{\tab}[1]      {Table~\protect\vref{tab:#1}\xspace}
\newcommand{\lst}[1]      {Listing~\protect\ref{lst:#1}\xspace}
\newcommand{\eg}          {e.\,g.,\xspace}
\newcommand{\ie}          {i.\,e.,\xspace}

\newcommand{\tickYes}{\hspace{1pt}\ding{52}}
\newcommand{\tickNo}{\hspace{1pt}\ding{56}}

%%%%%%%%%%%%%%%%%%%%%%%%%%%%%%%%%%%%%%%%%%%%%%%%%%%%%%%%%%%%%%%%%%
%%  environments
%%%%%%%%%%%%%%%%%%%%%%%%%%%%%%%%%%%%%%%%%%%%%%%%%%%%%%%%%%%%%%%%%%

% standard figure layout
\newcommand{\sedfig}[4][]
	{\begin{figure}[H]\begin{center}{\includegraphics[width=0.9\textwidth,#1]{#2}}\caption{#3}\label{#4}\end{center}\end{figure}}

\newcommand{\sedfigX}[4][]
	{\begin{figure}[H]\begin{center}{\includegraphics[#1]{#2}}\caption{#3}\label{#4}\end{center}\end{figure}}

% standard XML listing layout
\lstnewenvironment{myXmlLst}[2]
	{\lstset{basicstyle=\ttfamily\scriptsize, caption={#1},label={#2}, keywordstyle=\color{blue}\bfseries, stringstyle=\color{blue}, commentstyle=\color{red}, captionpos=b, breaklines=true, xleftmargin=1.5em, xrightmargin=1.5em, numbers=left, numberstyle=\ttfamily\tiny, numbersep=5pt, tabsize=4, showstringspaces=false, language=XML}} %, float=!h
	{}

% listings in appendixes:
\newcommand{\myXmlImport}[3]	{\lstinputlisting[basicstyle=\ttfamily\scriptsize,caption={#1},label={#2},%
	keywordstyle=\color{blue}\bfseries, stringstyle=\color{blue}, commentstyle=\color{red}, captionpos=b, breaklines=true, xleftmargin=1.5em, xrightmargin=1.5em, numbers=left, numberstyle=\ttfamily\tiny, numbersep=5pt, tabsize=4, showstringspaces=false, language=XML, stepnumber=1]{#3}} %float=h!

%%% Local Variables: 
%%% mode: latex
%%% TeX-master: "../sed-ml-L1V3"
%%% End: 


