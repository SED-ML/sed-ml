% model class
\label{class:model}
The \concept{Model} class is the container for a model reference (see \fig{sedModel}).
% Fig: sed model
\sedfig[width=0.9\textwidth]{modelClass}{The SED-ML Model class}{fig:sedModel}
%

A model is defined through an unambiguous and mandatory \hyperref[sec:id]{id}. An additional, optional \hyperref[sec:name]{name} may be given to the model. For each model, the \hyperref[sec:language]{language} may be specified, defining the format the model is encoded in, if such a format exists. Example formats are SBML, CellML, or myFancyLanguageML.
The \concept{Model} class refers to the particular model of interest through the \hyperref[sec:source]{source} attribute. The restrictions on the model reference are
\begin{itemize}
 \item{The model must be encoded in an XML format.}
 \item{To refer to the model encoding language, a reference to a valid definition of that XML format must be given.}
 \item{To refer to a particular model in an external resource, an unambiguous reference must be given.}
\end{itemize}


\tabtext{model}{Model}
%
\begin{table}[ht]
\center
\begin{tabular}{|l|l|}
\hline
\textbf{\attribute} & \textbf{\desc}\\
\hline
metaid$^{o}$ & \refpage{sec:metaID}\\
id & \refpage{sec:id} \\
name$^{o}$ & \refpage{sec:name}\\
language & \refpage{sec:language}\\
source & \refpage{sec:source}\\
\hline
\hline
\textbf{\subelements} & \textbf{\desc}\\
\hline
notes$^{o}$ & \refpage{class:notes}\\
annotation$^{o}$ & \refpage{class:annotation}\\
change$^{o}$ & \refpage{class:change}\\
\hline
\end{tabular}
\label{tab:model}
\caption{\tabcap{Model}}
\end{table}
%

A model might need to undergo pre-processings before simulation. Those pre-processings are specified in the SED-ML \hyperref[class:change]{Change} class.

\subsubsection{The \element{language} attribute}
\label{sec:language}
The evaluation of a SED-ML file will decide whether or not it can be used for a particular simulation environment. One crucial criterium is the particular model representation language used to encode the model. A simulation software usually only supports a small subset of the representation formats available to model biological systems computationally. 

To help a software decide whether or not it supports a SED-ML description file, the information on the model encoding for each referenced model can be provided through the \concept{language} attribute. 
%Examples for such  are \element{SBML} denoting the standard format SBML \citep{Hucka:2003}, or \code{CellML} denoting the standard format  CellML \citep{Lloyd:2004}.
%As there is no controlled vocabulary of representation formats for biological models available as of now, the data type for \element{type} is \element{String}. 
As the description of a language name and version through an unrestricted \code{String} is error-prone, SED-ML provides a set of standard URIs to refer to particular language definitions. A prerequisite for a language to be fully supported by SED-ML is that the language definition, e.\,g. the XML Schema, is provided online. One example for a XML Schema location is \alert{tbc}

The \element{language} attribute is mandatory for the XML representation of a SED-ML file. Not only does it help a user to decide whether or not he is able to run the simulation, that is to parse the model referenced in the SED-ML file. The language attribute is also needed to decide how to handle a particular implicit variable in the \hyperref[class:variable]{Variable} class. The interpretation of implicit variables depends on the language of the representation format. The concept of implicit variables has been introduced in section \ref{sec:implicitVariable} on  \refpage{sec:implicitVariable}.


\subsubsection{The \element{source} attribute}
\label{sec:source}
To make the model available during  execution of a SED-ML file, the model \element{source} should be specified through an XLink. 
The XLink should preferably point to a public, consistent URI that contains the model description file and follows the proposed \hyperref[sec:uriScheme]{URI Scheme}.
Consistent URIs ensure the long-term availability of models used in a SED-ML simulation description file. 
Therefore, reference to curated, open model bases are recommended. An example for such a resource is BioModels Database \citep{N+06}. However, any resource registered with MIRIAM resources\footnote{\url{http://www.ebi.ac.uk/miriam/main/}} can easily be used in SED-ML. Even without a MIRIAM URN, SED-ML can be used. The long-term availability of a model, and thus the validity of a simulation experiment cannot be assured in those cases though.


An example for the definition of a model inside the \hyperref[sec:listOfModels]{listOfModels} and using the MIRIAM \hyperref[sec:uriScheme]{URI scheme} is given in the XML snippet in listing \ref{lst:modelA}.
%
\begin{myXmlLst}{The SED-ML model element, using the URI scheme}{lst:modelA}
<listOfModels>
 <model id="m1" name="repressilator" language="urn:sedml:language:sbml" 
  source="urn:miriam:biomodels.db:BIOMD0000000012">
  <listOfChanges>
   [DEFINE MODEL PRE-PROCESSING HERE]
  </listOfChanges>
 </model>
</listOfModels>
\end{myXmlLst}
%
In the example one model is defined. An \element{id} and a \element{name} are given. The \element{language} the model is described in is \element{SBML}, and the model source code is available from \element{urn:miriam:biomodels.db:BIOMD0000000012}. The MIRIAM URN can be resolved into the SBML model stored in BioModels Database under ID \element{BIOMD0000000012}.

An example for the definintion of a model inside the \hyperref[sec:listOfModels]{listOfModels} and using a URL is given in the XML snippet in listing \ref{lst:modelB}.
%
\begin{myXmlLst}{The SED-ML model element, using a URL}{lst:modelB}
<listOfModels>
 <model id="m1" name="repressilator" language="urn:sedml:language:cellml" 
  source="http://models.cellml.org/exposure/bba4e39f2c7ba8af51fd045463e7bdd3/aguda_b_1999.cellml">
  <listOfChanges>
   [DEFINE MODEL PRE-PROCESSING HERE]
  </listOfChanges>
 </model>
</listOfModels>
\end{myXmlLst}
%
In the example one model is defined. An \element{id} and a \element{name} are given. The \element{type} of the model is \element{CellML}. As CellML currently does not provide a MIRIAM URI scheme for model reference, the URL pointing to the model code is used to refer to the model. The URL is given in the \element{source} element.


%%% Local Variables: 
%%% mode: plain-tex
%%% TeX-master: "../sed-ml-L1V1"
%%% End: 
