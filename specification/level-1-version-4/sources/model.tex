% ~~~~~~~~~~~~~~~~~~~~~~~~~~~~~~~~~~~~~~~~
%% MODEL
% ~~~~~~~~~~~~~~~~~~~~~~~~~~~~~~~~~~~~~~~~
\subsection{\element{Model}}
\label{class:model}
The \concept{Model} class defines the models used in a simulation experiment (\fig{model}).

\sedfig[width=0.7\textwidth]{images/uml/model}{The SED-ML Model class}{fig:model}

Each instance of the \concept{Model} class has the mandatory attributes \hyperref[sec:id]{\element{id}} and \hyperref[sec:model_source]{\code{source}}, and the optional attributes \hyperref[sec:name]{\element{name}}, \hyperref[sec:language]{\element{language}}, and \hyperref[sec:changesModel]{\element{listOfChanges}}.

The optional attribute \hyperref[sec:language]{\code{language}} defines the format the model is encoded in.

The \concept{Model} class refers to the particular model of interest through the \hyperref[sec:model_source]{\element{source}} attribute. The restrictions on the model reference are
\begin{itemize}
 \item{The model must be encoded in \changed{a well-defined format}.}
 \item{To refer to the model encoding language, a reference to a valid definition of that format must be given (\hyperref[sec:language]{\element{language}} attribute).}
 \item{To refer to a particular model in an external resource, an unambiguous reference must be given (\hyperref[sec:model_source]{\element{source}} attribute).}
\end{itemize}

A model might need to undergo pre-processing before simulation. Those pre-processing steps are specified in the \hyperref[sec:changesModel]{\element{listOfChanges}} via the \hyperref[class:change]{Change} class.


\lsttext{model}{model}
In the example the \hyperref[class:listOfModels]{\element{listOfModels}} contains three models: The first model \code{m0001} is the Repressilator model from \biom available from \url{urn:miriam:biomodels.db:BIOMD0000000012}. For the SED-ML simulation the model might undergo preprocessing steps described in the \hyperref[sec:changesModel]{\element{listOfChanges}}. Based on the description of the first model \code{m0001}, the second model \code{m00012} is built, which is a modified version of the Repressilator model. \code{m0002} refers to the model \code{m001} in its \hyperref[sec:model_source]{\code{source}} attribute. \code{m0002} might then have additional changes applied to it on top of the changes defined in the pre-processing of \code{m0001}. The third model in the code example is a model in CellML representation. The model \code{m0003} is available from the given URL in the \hyperref[sec:model_source]{\code{source}} attribute. Again, it might have pre-processing steps applied before used in a simulation.

\begin{myXmlLst}{SED-ML \code{model} element}{lst:model}
<listOfModels>
	<model id="m0001" language="urn:sedml:language:sbml" 
		source="urn:miriam:biomodels.db:BIOMD0000000012">
		<listOfChanges>
			<change>
				[MODEL PRE-PROCESSING]
			</change>
		</listOfChanges> 
	</model>
	<model id="m0002" language="urn:sedml:language:sbml" source="m0001">
		<listOfChanges>
			[MODEL PRE-PROCESSING]
		</listOfChange>
	</model>
	<model id="m0003" language="urn:sedml:language:cellml" source="http://models.cellml.org/workspace/leloup_gonze_goldbeter_1999/@@rawfile/d6613d7e1051b3eff2bb1d3d419a445bb8c754ad/leloup_gonze_goldbeter_1999_a.cellml">
		[MODEL PRE-PROCESSING]
	</model>
</listOfModels>
\end{myXmlLst} 

% ~~~ MODEL:LANGUAGE ~~~
\paragraph*{\element{language}}
\label{sec:language}
The optional \concept{\element{language}} attribute of data type \hyperref[type:anyURI]{\code{anyURI}} is used to specify the format of the \hyperref[class:model]{model}. Example formats are SBML (\code{urn:sedml:language:sbml}) or CellML (\code{urn:sedml:language:cellml}). The supported languages are defined in the \hyperref[sec:languageURI]{language references}.

If it is not explicitly defined, the default value for \concept{\element{language}} is \code{urn:sedml:language:xml}, referring to any XML-based model representation. However, the use of the \concept{\element{language}} attribute is strongly encouraged for two reasons. Firstly, it helps to decide whether or not one is able to run the simulation, that is to parse the model referenced in the SED-ML file. Secondly, the language attribute is also needed to decide how to handle the \hyperref[sec:implicitVariable]{Symbols} in the \Variable class, as the interpretation of \hyperref[sec:implicitVariable]{Symbols} depends on the language of the representation format.


% ~~~ MODEL:SOURCE ~~~
\paragraph*{\element{source}}
\label{sec:model_source}
To make a \hyperref[class:model]{model} accessible for the execution of a SED-ML file, the \concept{\element{source}} must be specified through either an URI or a reference to an \code{SId} of an existing \Model. The URI should follow the proposed \hyperref[sec:uriScheme]{URI Scheme} for \hyperref[sec:modelURI]{Model references}.

An example for the definition of a model via an URI is given in Listing~\ref{lst:sourceA}. The example defines one model \code{m1} with the model \concept{\element{source}} available from \code{urn:miriam:biomodels.db:BIOMD0000000012}. The MIRIAM URN can be resolved into the SBML model stored in BioModels Database under the identifier \element{BIOMD0000000012} using the MIRIAM web service. The resulting URL is \url{https://www.ebi.ac.uk/biomodels-main/BIOMD0000000012}.

\begin{myXmlLst}{The SED-ML \code{source} element, using the URI scheme}{lst:sourceA}
<model id="m1" name="repressilator" language="urn:sedml:language:sbml" 
	source="urn:miriam:biomodels.db:BIOMD0000000012">
	<listOfChanges>
		[MODEL PRE-PROCESSING]
	</listOfChanges>
</model>
\end{myXmlLst}

An example for the definition of a model using an URL is given in Listing~\ref{lst:sourceB}. In the example one model is defined. The \element{language} of the model is \element{CellML}. As the CellML model repository currently does not provide a MIRIAM URI for model reference, the \emph{URL} pointing to the model is used in the \concept{\element{source}} attribute.

\begin{myXmlLst}{The SED-ML \code{source} element, using a URL}{lst:sourceB}
<model id="m1" name="repressilator" language="urn:sedml:language:cellml" 
	source="http://models.cellml.org/exposure/bba4e39f2c7ba8af51fd045463e7bdd3/aguda_b_1999.cellml">
	<listOfChanges />
</model>
\end{myXmlLst}



% ~~~ LIST OF CHANGES ~~
\paragraph*{\element{listOfChanges}}
\label{sec:changesModel}
The \concept{\element{listOfChanges}} (\fig{model}) contains the \hyperref[class:change]{Changes} to be applied to a particular \Model. The \concept{\element{listOfChanges}} is optional and may contain zero to many \hyperref[class:change]{Changes}.

\lsttext{listOfChanges}{listOfChanges}
\begin{myXmlLst}{The SED-ML \element{listOfChanges} element, defining a change on a model}{lst:listOfChanges}
<model id="m0001" [..]>
	<listOfChanges>
		[CHANGE DEFINITION]
	</listOfChanges>
</model>
\end{myXmlLst}

% ~~~~~~~~~~~~~~~~~~~~~~~~~~~~~~~~~~~~~~~~
% CHANGE
% ~~~~~~~~~~~~~~~~~~~~~~~~~~~~~~~~~~~~~~~~
\subsection{\element{Change}}
\label{class:change}
The \concept{Change} class allows to describe changes applied to a \hyperref[class:model]{model} before simulation (\fig{sedChange}). \concept{Changes} can be of the following types:
\begin{itemize}
	\item{Changes on \changed{attributes of the model} (\hyperref[class:changeAttribute]{ChangeAttribute})}	
	\item{Changes based on mathematical calculations (\ComputeChange)} 
	\item{\changed{For XML-encoded models, changes on any XML snippet of the model's XML representation} (\hyperref[class:addXml]{AddXML}, \hyperref[class:changeXml]{ChangeXML}, \hyperref[class:removeXml]{RemoveXML})}
\end{itemize}

The \concept{Change} class is abstract and serves as the base class for different types of changes, the \hyperref[class:changeAttribute]{ChangeAttribute}, \hyperref[class:addXml]{AddXML}, \hyperref[class:changeXml]{ChangeXML}, \hyperref[class:removeXml]{RemoveXML}, and \ComputeChange.

The \concept{Change} class has the mandatory attribute \hyperref[sec:changeTarget]{target} which defines the target of the change. The \hyperref[sec:changeTarget]{\element{target}} attribute holds \changed{an unambiguous description of the address of the element or attribute that is to undergo the defined changes, such as a valid \hyperref[sec:xpath]{XPath} expression pointing to the XML element or XML attribute. For \hyperref[class:newXml]{NewXML}, \hyperref[class:addXml]{AddXML}, \hyperref[class:changeXml]{ChangeXML}, and \hyperref[class:removeXml]{RemoveXML}, \hyperref[sec:changeTarget]{target} must be an \hyperref[sec:xpath]{XPath} expression.} Except for the cases of \hyperref[class:changeXml]{ChangeXML} and \hyperref[class:removeXml]{RemoveXML}, this \hyperref[sec:xpath]{XPath} expression must always select a single element or attribute within the relevant model.

\sedfig[width=1.0\textwidth]{images/uml/change}{The SED-ML Change class}{fig:sedChange}


% ~~~ target ~~~
\paragraph*{\element{target}}
\label{sec:changeTarget}
The \concept{\element{target}} attribute holds \changed{an unambiguous description of the address of an element or attribute of a model that is to undergo the defined changes. For XML model languages such as SBML, \hyperref[sec:changeTarget]{\element{target}} should be} a valid \hyperref[sec:xpath]{XPath} expression of data type \hyperref[type:xpath]{\element{xpath}} pointing to the XML element or XML attribute that is to undergo the defined changes.


% ~~~ NewXML ~~~
\subsubsection{\element{NewXML}}
\label{sec:newXml}
The \concept{\element{newXML}} element provides a piece of XML code (\fig{sedChange}). \concept{\element{NewXML}} must hold a valid piece of XML which after insertion into the original model must  result in a valid model file (according to the model language specification as given by the \hyperref[sec:language]{\element{language}} attribute of the \hyperref[class:model]{model}).

The \concept{\element{newXML}} element is used at two different places inside SED-ML \currentLV:

\begin{enumerate}
	\item{If it is used as a sub-element of the \hyperref[class:addXml]{\element{addXML}} element, then the XML it contains \emph{is inserted as a child} of the XML element addressed by the XPath.}
	\item{If it is used as a sub-element of the \hyperref[class:changeXml]{\element{changeXML}} element, then the XML it contains \emph{replaces} the XML element addressed by the XPath.}
\end{enumerate}

Examples are given in the relevant change class definitions.


% ~~~ AddXML ~~~
\subsubsection{\element{AddXML}}
\label{class:addXml}
The \concept{AddXML} class specifies a snippet of XML that is added as a child of the element selected by the XPath expression in the \hyperref[sec:changeTarget]{\element{target}} attribute (\fig{sedChange}). The new piece of XML code is provided by the \hyperref[sec:newXml]{NewXML} class.

An example for a change that adds an additional parameter to a model is given in \lst{addXML}. In the example the model is changed so that a parameter with ID \code{V\_mT} is added to its list of parameters. The \code{newXML} element adds an additional XML element to the original model. The element's name is \code{parameter} and it is added to the existing parent element \code{listOfParameters} that is addressed by the XPath expression in the \code{target} attribute.

\begin{myXmlLst}{The \code{addXML} element with its \code{newXML} sub-element}{lst:addXML}
<model language="urn:sedml:language:sbml" [..]>
	<listOfChanges>
		<addXML target="/sbml:sbml/sbml:model/sbml:listOfParameters" >
			<newXML>
				<parameter metaid="metaid_0000010" id="V_mT" value="0.7" />
			</newXML>
		</addXML>
	</listOfChanges>
</model>
\end{myXmlLst}


% ~~~ ChangeXML ~~~
\subsubsection{\element{ChangeXML}}
\label{class:changeXml}
The \concept{ChangeXML} class allows you to replace any XML element(s) in the model that can be addressed by a valid XPath expression (\fig{sedChange}).

The XPath expression is specified in the required \hyperref[sec:changeTarget]{\element{target}} attribute. The replacement XML content is specified in the \hyperref[sec:newXml]{NewXML} class.

An example for a change that adds an additional parameter to a model is given in \lst{changeXML}. In the example the model is changed in the way that its parameter with ID \code{V\_mT} is substituted by two other parameters \code{V\_mT\_1} and \code{V\_mT\_2}. The \code{target} attribute defines that the parameter with ID \code{V\_mT} is to be changed. The \code{newXML} element then specifies the XML that is to be exchanged for that parameter.

\begin{myXmlLst}{The \code{changeXML} element}{lst:changeXML}
<model [..]>
	<listOfChanges>
		<changeXML target="/sbml:sbml/sbml:model/sbml:listOfParameters/sbml:parameter[@id='V_mT']" >
			<newXML>
				<parameter metaid="metaid_0000010" id="V_mT_1" value="0.7" />
				<parameter metaid="metaid_0000050" id="V_mT_2" value="4.6"> />
			</newXML>
		</changeXML>
	</listOfChanges>
</model>
\end{myXmlLst}


% ~~~ RemoveXML ~~~
\subsubsection{\element{RemoveXML}}
\label{class:removeXml}
The \concept{RemoveXML} class can be used to delete XML elements or attributes in the model that are addressed by the XPath expression (\fig{sedChange}). The XPath is specified in the required \hyperref[sec:changeTarget]{\element{target}} attribute.

An example for the removal of an XML element from a model is given in Listing~\ref{lst:removeXML}. In the example the model is changed by deleting the reaction with ID \code{V\_mT} from the model's list of reactions.

\begin{myXmlLst}{The \code{removeXML} element}{lst:removeXML}
<model [..]>
	<listOfChanges>
		<removeXML target="/sbml:sbml/sbml:model/sbml:listOfReactions/sbml:reaction[@id='J1']" />
	</listOfChanges>
</model>
\end{myXmlLst}


% ~~~ ChangeAttribute ~~~
\subsubsection{\element{ChangeAttribute}}
\label{class:changeAttribute}
The \concept{ChangeAttribute} class allows to define updates on the attribute values of the corresponding model (\fig{sedChange}). \concept{ChangeAttribute} requires to specify the \hyperref[sec:changeTarget]{\element{target}} of the change, i.e., the location of the addressed attribute, and also the \hyperref[sec:newValue]{\element{newValue}} of that attribute. Note that \changed{the \hyperref[sec:changeTarget]{\element{target}} attribute} must select a single attribute within the corresponding model.

The \hyperref[class:changeXml]{ChangeXML} class covers the possibilities provided by the \concept{ChangeAttribute} class, i.e, everything that can be expressed by a \concept{ChangeAttribute} construct can also be expressed by \hyperref[class:changeXml]{ChangeXML}. However, for the common case of changing an attribute value \concept{ChangeAttribute} is easier to use, and so it is recommended to use the \concept{ChangeAttribute} for any changes of an attribute's value, and to use the more general \hyperref[class:changeXml]{ChangeXML} for other cases.


\paragraph*{\element{newValue}}
\label{sec:newValue}
The mandatory \concept{\element{newValue}} attribute of data type \code{string} assignes a new value to the \changed{targeted attribute}.

The example in Listing~\ref{lst:changeAttribute} shows the update of the value of two parameters inside an SBML model.

\begin{myXmlLst}{The \code{changeAttribute} element and its \code{newValue} attribute}{lst:changeAttribute}
<model id="model1" name="Circadian Chaos" language="urn:sedml:language:sbml" 
	source="urn:miriam:biomodels.db:BIOMD0000000021">
	<listOfChanges>
		<changeAttribute target="/sbml:sbml/sbml:model/sbml:listOfParameters/sbml:parameter[@id='V_mT']/@value" newValue="0.28"/>
  		<changeAttribute target="/sbml:sbml/sbml:model/sbml:listOfParameters/sbml:parameter[@id='V_dT']/@value" newValue="4.8"/>
	</listOfChanges>
</model>
\end{myXmlLst}


% ~~~ ComputeChange ~~~
\subsubsection{\element{ComputeChange}}
\label{class:computeChange}
The \concept{ComputeChange} class permits to change, prior to the experiment, the numerical value of any element or attribute of a \Model addressable by \changed{a \hyperref[sec:changeTarget]{\element{target}}}, based on a calculation (\fig{sedChange}).  \changed{It inherits the \element{target} attribute from the \Change abstract base class, as well as the standard \SedBase attributes and children, and its ability to perform a calculation is described in the \Calculation class.  (For implementations, if multiple inheritance is not possible, the children of \Calculation should just be added directly to the \ComputeChange class itself.)}

\changed{The change is calculated from the \Math of the \Calculation, and applied to the \element{target} of the \Change.}

Note that when a \concept{ComputeChange} refers to another model, that model is not allowed to be modified by \concept{ComputeChange}s which directly or indirectly refer to this model. In other words, cycles in the definitions of computed changes are prohibited.

\lsttext{computeChange}{computeChange}

\begin{myXmlLst}{The computeChange element}{lst:computeChange}
<model [..]>
	<listOfChanges>
	<computeChange target="/sbml:sbml/sbml:model/sbml:listOfParameters/sbml:parameter[@id='sensor']">
		<listOfVariables>
			<variable modelReference="model1" id="R" name="regulator" 
				target="/sbml:sbml/sbml:model/sbml:listOfSpecies/sbml:species[@id='regulator']" />
			<variable modelReference="model2" id="S" name="sensor"
				target="/sbml:sbml/sbml:model/sbml:listOfParameters/sbml:parameter[@id='sensor']" />
		<listOfVariables/>
		<listOfParameters>
			<parameter id="n" name="cooperativity" value="2">
			<parameter id="K" name="sensitivity" value="1e-6">
		<listOfParameters/>
		<math  xmlns="http://www.w3.org/1998/Math/MathML">
        <apply>
          <times />
          <ci>S</ci>
          <apply>
            <divide />
            <apply>
              <power />
              <ci>R</ci>
              <ci>n</ci>
            </apply>
            <apply>
              <plus />
              <apply>
                <power />
                <ci>K</ci>
                <ci>n</ci>
              </apply>
              <apply>
                <power />
                <ci>R</ci>
                <ci>n</ci>
              </apply>
            </apply> 
		</apply>
		</math>
	</computeChange>
	</listOfChanges>
</model>
\end{myXmlLst}

The example in \lst{computeChange} computes a change of the variable \code{sensor} of the model \code{model2}. To do so, it uses the value of the variable \code{regulator} coming from model \code{model1}. In addition, the calculation uses two additional parameters, the cooperativity \code{n}, and the sensitivity \code{K}. The mathematical expression in the mathML then computes the new initial value of \code{sensor} using the equation:
\begin{math}
S =  S \times \frac{R^{n}}{K^{n}+R^{n}}
\end{math}
