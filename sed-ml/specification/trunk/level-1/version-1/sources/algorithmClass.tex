% algorithm class
\label{class:algorithm}

SED-ML makes use of the \hyperref[sec:kisao]{KiSA ontology} to refer to a term in the controlled vocabulary identifying the particular simulation algorithm to be used in the simulation. 

One algorithm must be defined for each simulation setup. The instance of the \concept{Algorithm} class must contain a \hyperref[sec:kisao]{KiSAO} reference to a simulation algorithm. The reference should define the  simulation algorithm to be used in the simulation as precisely as possible.


\tabtext{algorithm}{Algorithm}
%
\begin{table}[ht]
\center
\begin{tabular}{|l|l|}
\hline
\textbf{attribute} & \textbf{description}\\
\hline
metaid$^{o}$ & \refpage{sec:metaID}\\
kisaoID & \refpage{sec:kisao}\\
\hline
\hline
\textbf{\subelements} & \textbf{\desc}\\
\hline
notes$^{o}$ & \refpage{class:notes}\\
annotation$^{o}$ & \refpage{class:annotation}\\
\hline
\end{tabular}
\label{tab:algorithm}
\caption{\tabcap{Algorithm}}
\end{table}
%

The example given in code snipped \ref{lst:simulation}, completed by algorithm definitions looks as in listing \ref{lst:algorithm}.
%
\begin{myXmlLst}{The SED-ML \code{algorithm} element, defining the two different algorithms in the two defined simulations}{lst:algorithm}
<listOfSimulations>
  <uniformTimeCourse id="s1" name="time course simulation of variable v1 over 100 minutes" [..]>
      <algorithm kisaoID="KiSAO:0000030" />
  </uniformTimeCourse>
  <uniformTimeCourse id="s2" name="time course definition for concentration of p" [..]>
      <algorithm kisaoID="KiSAO:0000021" />
  </uniformTimeCourse>
</listOfSimulations>
\end{myXmlLst}
%
For both simulations, one algorithm is defined. In the first simulation \code{s1} a deterministic simulation algorithm is used (Euler forward method), in the second simulation \code{s2} a stochastic one is used (Next reaction method).

%%% Local Variables: 
%%% mode: latex
%%% TeX-master: "../sed-ml-L1V1"
%%% End: 
