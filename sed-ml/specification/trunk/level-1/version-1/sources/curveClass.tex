\label{curveClass}
% Fig: Report
\sedfig[width=0.3\textwidth]{curveClass}{The SED-ML Curve class}{fig:curve}
%

\tabtext{curve}{Curve}
%
\begin{table}[ht]
\center
\begin{tabular}{|l|l|}
\hline
\textbf{\attribute} & \textbf{\desc}\\
\hline
metaid & \refpage{sec:metaID}\\
id & \refpage{sec:id} \\
name & \refpage{sec:name}\\
logX & \refpage{sec:logX}\\
xDataReference & \refpage{sec:xDataReference}\\
logY & \refpage{sec:logY}\\
yDataReference & \refpage{sec:yDataReference}\\
\hline
\hline
\textbf{\subelements} & \textbf{\desc}\\
\hline
notes & \refpage{class:notes}\\
annotation & \refpage{class:annotation}\\
\hline
\end{tabular}
\label{tab:curve}
\caption{\tabcap{Curve}}
\end{table}
%

\subsubsection{The \element{logX} attribute}
\label{sec:logX}
\concept{logX} is an optional attribute of the \hyperref[class:curve]{Curve} class and defines whether or not the data output on the x-axis is logarithmic or not. The data type of \concept{logX} is boolean, the standard value is ``false''.
To make the output on the x-axis of a plot logarithmic, \concept{logX} must be set to ``true'', as shown in the sample listing \ref{lst:logX}: 
\begin{myXmlLst}{The SED-ML  logX attribute, defining a logarithmic output on the x-axis of the according output}{lst:logX}
<listOfCurves logX="true">
  <curve id="c1" [..]>
</listOfCurves>
\end{myXmlLst}
\concept{logX} is also used in the definition of a \hyperref[class:surface]{Surface} output.

\subsubsection{The \element{logY} attribute}
\label{sec:logY}
\concept{logY} is an optional attribute of the \hyperref[class:curve]{Curve} class and defines whether or not the data output on the x-axis is logarithmic or not. The data type of \concept{logY} is boolean, the standard value is ``false''.
To make the output on the y-axis of a plot logarithmic, \concept{logY} must be set to ``true'', as shown in the sample listing \ref{lst:logY}: 
\begin{myXmlLst}{The SED-ML  logY attribute, defining a logarithmic output on the y-axis of the according output}{lst:logY}
<listOfCurves logY="true">
  <curve id="c1" [..]>
</listOfCurves>
\end{myXmlLst}
\concept{logY} is also used in the definition of a \hyperref[class:surface]{Surface} output.

\subsubsection{The \element{xDataReference} attribute}
\label{sec:xDataReference}
The \concept{xDataReference} is a mandatory attribute of the \hyperref[class:curve]{Curve} object. It's content refers to a dataGenerator ID which denotes the \hyperref[class:dataGenerator]{DataGenerator} object that is used to generate the output on the x-axis of a \hyperref[class:curve]{Curve} in a \hyperref[class:2DPlot]{2D Plot}. 
\concept{xDataReference} is also used in the definition of the x-axis of a \hyperref[class:surface]{Surface} in a \hyperref[class:3DPlot]{3D Plot}.
The \concept{xDataReference} data type is string. However, the number of valid values for the \concept{xDataReference} is restricted to the IDs of already defined \hyperref[class:dataGenerator]{DataGenerator objects}.

An example for the definition of a curve is given in listing \ref{lst:curve}.

\subsubsection{The \element{yDataReference} attribute}
\label{sec:yDataReference}
The \concept{yDataReference} is a mandatory attribute of the \hyperref[class:curve]{Curve} object. It's content refers to a dataGenerator ID which denotes the \hyperref[class:dataGenerator]{DataGenerator} object that is used to generate the output on the y-axis of a \hyperref[class:curve]{Curve} in a \hyperref[class:2DPlot]{2D Plot}.
\concept{yDataReference} is also used in the definition of the y-axis of a \hyperref[class:surface]{Surface} in a \hyperref[class:3DPlot]{3D Plot}.
The \concept{yDataReference} data type is string. However, the number of valid values for the \concept{yDataReference} is restricted to the IDs of already defined \hyperref[class:dataGenerator]{DataGenerator objects}.

An example for the definition of a curve is given in the XML snippet in listing \ref{lst:curve}.
%
\begin{myXmlLst}{The SED-ML curve element, defining the output curve showing the result of the referenced task}{lst:curve}
<listOfCurves>
  <curve id="c1" name="v1 over time" algorithm ="KiSAO:ID" xDataReference="dg1" yDataReference="dg2">
</listOfCurves>
\end{myXmlLst}
Here, only one curve is created, results shown on the x-axis are generated by the data generator \code{dg1}, results shown on the y-axis are generated by the data generator \code{dg2}. Both \code{dg1} and \code{dg2} need to be already defined in the \hyperref[sec:listOfDataGenerators]{listOfDataGenerators}.



%%% Local Variables: 
%%% mode: latex
%%% TeX-master: "../sed-ml-L1V1"
%%% End: 
