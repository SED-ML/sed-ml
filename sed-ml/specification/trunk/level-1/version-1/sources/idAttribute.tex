\label{sec:id}
%

Most objects in SED-ML carry an \concept{id} attribute. 
The \hyperref[sec:id]{id} attribute, if it exists for an object, is always required and identifies SED-ML constituents unambiguously.   
The \code{id} data type is \code{String}. The following rules govern the scope of an identifier within a SED-ML document.
\begin{itemize}
\item Values of  id attributes for top level elements  \concept{Model}, \concept{Simulation}, \concept{Task}, \concept{DataGenerator}, \concept{Output} ) should be globally unique within a SED-ML document. 
\item Values of id attributes for other identifiable elements need only be unique within their enclosing contained element. For example, a
 \hyperref[class:variable]{Variable} element must be uniquely identifiable within its enclosing  \hyperref[sec:listOfVariables] {ListOfVariables} element, but the same identifier can be used elsewhere in other containers.
\end{itemize}
An example for a defined \code{id} is given in listing \ref{lst:id}.
%
\begin{myXmlLst}{SED-ML identifier definition, e.\,g.\ for a model}{lst:id}
<model id="m00001" language="urn:sedml:language:sbml" source="urn:miriam:biomodels.db:BIOMD0000000012">
 [MODEL DEFINITION]
</model>
\end{myXmlLst}
%
The defined model carries the ID \code{m00001}. If the model is used somewhere else in the SED-ML document, it is referred to by that ID.

%%% Local Variables: 
%%% mode: latex
%%% TeX-master: "../sed-ml-L1V1"
%%% End: 
