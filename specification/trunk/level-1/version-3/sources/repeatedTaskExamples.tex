The \hyperref[class:repeatedTask]{repeatedTask} introduced in \LoneVtwo makes it possible to encode a large number of different simulation experiments. In this section several simulation experiment are presented that use the repeated tasks construct. 

\subsection{One dimensional steady state parameter scan}
Here the repeated task calls out to a \hyperref[class:oneStep]{oneStep} task (performing a steady state computation). Each time a parameter is carried in order to collect different responses. 

In the description below the range to be used in the \hyperref[class:setValue]{setValue} construct use of the \token{range} attribute.

%
\sedfigX[scale=0.6]{xml/v3-example1-repeated-steady-scan-oscli}{The simulation result gained from the simulation description given in \lst{repeated1}}{fig:figrepeated1}
%

\myXmlImport{SED-ML document implementing the one dimensional steady state parameter scan}
{lst:repeated1}
{xml/v3-example1-repeated-steady-scan-oscli.xml}

\subsection{Perturbing a Simulation}
Often it is interesting to see how the dynamic behavior of a model changes when some perturbations are applied to the model. In this example we include one repeated task that makes repeated use of a oneStep task (that advances an ODE integration to the next output step). During the steps one parameter is modified effectively causing the oscillations of a model to stop. Once the value is reset the oscillations recover. 

Note: In the example below we use a functionalRange, although the same result could also be achieved using the setValue element directly.

%
\sedfigX[scale=0.6]{xml/v3-example2-oscli-nested-pulse}{The simulation result gained from the simulation description given in \lst{repeated2}}{fig:figrepeated2}
%

\myXmlImport{SED-ML document implementing the perturbation experiment}
{lst:repeated2}
{xml/v3-example2-oscli-nested-pulse.xml}

\subsection{Repeated Stochastic Simulation}
NOTE: This example produces three dimensional results (time, species concentration, multiple repeats). While \LoneVtwo does not include a way to post-processing these values. So it is left to the implementation on how to display them. One example would be to flatten the values by overlaying them onto the desired plot. 

Running just one stochastic trace does not provide a complete picture of the behavior of a system. A large number of traces are needed to provide a result. This example demonstrates the basic use case of running ten traces of a simulation to. This is achieved by running on repeatedTask running ten uniform time course simulations (each of which performing a stochastic simulation run). 

%
\sedfigX[scale=0.6]{xml/v3-example3-repeated-stochastic-runs}{The simulation result gained from the simulation description given in \lst{repeated3}}{fig:figrepeated3}
%

\myXmlImport{SED-ML document implementing repeated stochastic runs}
{lst:repeated3}
{xml/v3-example3-repeated-stochastic-runs.xml}

\subsection{One dimensional time course parameter scan}
NOTE: This example produces three dimensional results (time, species concentration, multiple repeats). While \LoneVtwo does not include a way to post-processing these values. So it is left to the implementation on how to display them. One example would be to flatten the values by overlaying them onto the desired plot. 

Here one repeatedTask  runs repeated uniform time course simulations (performing a deterministic simulation run) after each run the parameter value is changed.


%
\sedfigX[scale=0.6]{xml/v3-example4-repeated-scan-oscli}{The simulation result gained from the simulation description given in \lst{repeated4}}{fig:figrepeated4}
%

\myXmlImport{SED-ML document implementing the one dimensional time course parameter scan}
{lst:repeated4}
{xml/v3-example4-repeated-scan-oscli.xml}

\subsection{Two dimensional steady state parameter scan}
NOTE: This example produces three dimensional results (time, species concentration, multiple repeats). While \LoneVtwo does not include a way to post-processing these values. So it is left to the implementation on how to display them. One example would be to flatten the values by overlaying them onto the desired plot. 

Here a repeatedTask runs over another repeatedTask which runs over a oneStep task (performing a steady state computation). Each repeated simulation task modifies a different parameter.

%
\sedfigX[scale=0.6]{xml/v3-example5-boris-2d-scan}{The simulation result gained from the simulation description given in \lst{repeated5}}{fig:figrepeated5}
%

\myXmlImport{SED-ML document implementing the one dimensional steady state parameter scan}
{lst:repeated5}
{xml/v3-example5-boris-2d-scan.xml}