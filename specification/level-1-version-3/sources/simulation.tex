% ~~~~~~~~~~~~~~~~~~~~~~~~~~~~~~~~~~~~~~~~
%% SIMULATION
% ~~~~~~~~~~~~~~~~~~~~~~~~~~~~~~~~~~~~~~~~
\subsection{\element{Simulation}}
\label{class:simulation}

A simulation is the execution of some defined algorithm(s). Simulations are described differently depending on the type of simulation experiment to be performed (\fig{sedSimulation}). 

\sedfig[width=0.85\textwidth]{pdf/simulationClass}{The SED-ML Simulation class}{fig:sedSimulation}

\concept{Simulation} is an abstract class and serves as the container for the different types of simulation experiments.
SED-ML \currentLV offers the predefined simulation classes \hyperref[class:uniformTimeCourse]{UniformTimeCourse}, \hyperref[class:oneStep]{OneStep} and \hyperref[class:steadyState]{SteadyState}. Further simulation classes are planned for future versions of SED-ML, including simulation classes for bifurcation analysis. Simulation algorithms used for the execution of a simulation setup are defined in the \hyperref[class:algorithm]{Algorithm} class.

\tabtext{simulation}{simulation}

\begin{table}[ht]
\center
\begin{tabular}{|l|l|}
\hline
\textbf{\attribute} & \textbf{\desc}\\
\hline
metaid$^{o}$ & \refpage{sec:metaID}\\
id & \refpage{sec:id} \\
name$^{o}$ & \refpage{sec:name}\\
\hline
\hline
\textbf{\subelements} & \textbf{\desc}\\
\hline
notes$^{o}$ & \refpage{class:notes}\\
annotation$^{o}$ & \refpage{class:annotation}\\
\hline
algorithm & \refpage{class:algorithm}\\
\hline
\end{tabular}
\caption{\tabcap{simulation}}
\label{tab:simulation}
\end{table}

\lsttext{simulation}{simulation}

\begin{myXmlLst}{The SED-ML \code{listOfSimulations} element, defining two different simulations}{lst:simulation}
<listOfSimulations>
  <uniformTimeCourse [..]>
    [SIMULATION SPECIFICATION]
  </uniformTimeCourse>
  <uniformTimeCourse [..]>
    [SIMULATION SPECIFICATION]
  </uniformTimeCourse>
</listOfSimulations>
\end{myXmlLst}

Two timecourses with uniform range are defined.


%% ~~~ ALGORITHM ~~~
 \subsubsection{\element{Algorithm}}
\label{class:algorithm}

SED-ML makes use of the \hyperref[sec:kisao]{KiSAO ontology} (Section~\ref{sec:kisao} on \refpage{sec:kisao}) to refer to a term in the controlled vocabulary identifying the particular simulation algorithm to be used in the simulation. 

Each instance of the \hyperref[class:simulation]{Simulation} class must contain one reference to a simulation algorithm (\fig{algorithm}). 

\sedfig[width=0.35\textwidth]{pdf/algorithm}{The \code{Algorithm} class}{fig:algorithm}

Each instance of the \concept{Algorithm} class must contain a \hyperref[sec:kisao]{KiSAO} reference to a simulation algorithm. The reference should define the  simulation algorithm to be used in the simulation as precisely as possible, and should be defined in the correct syntax, as defined by the regular expression \code{KISAO:[0-9]\{7\}}.

The \concept{Algorithm} class contains an optional list of parameters (\hyperref[class:algorithmParameter]{algorithmParameter}) that are used to parameterize the particular algorithm used in the simulation. 

\tabtext{algorithm}{Algorithm}

\begin{table}[ht]
\center
\begin{tabular}{|l|l|}
\hline
\textbf{attribute} & \textbf{description}\\
\hline
metaid$^{o}$ & \refpage{sec:metaID}\\
kisaoID & \refpage{sec:kisao}\\
\hline
\hline
\textbf{\subelements} & \textbf{\desc}\\
\hline
notes$^{o}$ & \refpage{class:notes}\\
annotation$^{o}$ & \refpage{class:annotation}\\
algorithmParameter$^{o}$ & \refpage{class:algorithmParameter}\\
\hline
\end{tabular}
\caption{\tabcap{algorithm}}
\label{tab:algorithm}
\end{table}

The example given in code snippet in Listing~\ref{lst:simulation}, completed by algorithm definitions results in the code given in Listing \ref{lst:algorithm}.

\begin{myXmlLst}{The SED-ML \code{algorithm} element for two different time course simulations, defining two different algorithms. KISAO:0000030 refers to the \emph{Euler forward method} ; KISAO:0000021 refers to the \emph{StochSim nearest neighbor algorithm}.}{lst:algorithm}
<listOfSimulations>
 <uniformTimeCourse id="s1" name="time course simulation over 100 minutes" [..]>
  <algorithm kisaoID="KISAO:0000030" />
 </uniformTimeCourse>
 <uniformTimeCourse id="s2" name="time course definition for concentration of p" [..]>
  <algorithm kisaoID="KISAO:0000021" />
 </uniformTimeCourse>
</listOfSimulations>
\end{myXmlLst}

For both simulations, one algorithm is defined. In the first simulation \code{s1} a deterministic approach has been chosen (Euler forward method), in the second simulation \code{s2} a stochastic approach is used (Stochsim nearest neighbor).


%% ~~~ ALGORITHM PARAMETER ~~~
\subsubsection{\element{AlgorithmParameter}}
\label{class:algorithmParameter}

The \concept{AlgorithmParameter} class allows to parameterize a particular simulation algorithm. The set of possible parameters for a particular instance is determined by the algorithm that is referenced by the \concept{kisaoID} of the enclosing \hyperref[class:algorithm]{algorithm} element. Parameters of simulation algorithms are unambiguously referenced by the mandatory \concept{kisaoID} attribute. Their value is set in the mandatory \concept{value} attribute.

\begin{myXmlLst}{The SED-ML \code{algorithmParameter} element setting the parameter value for the simulation algorithm. KISAO:0000032 refers to the \emph{explicit fourth-order Runge-Kutta method}; KISAO:00000211 refers to the \emph{absolute tolerance}.}{lst:algorithmParameter}
<algorithm kisaoID="KISAO:0000032"> 
  <listOfAlgorithmParameters> 
    <algorithmParameter kisaoID="KISAO:0000211" value="23"/> 
  </listOfAlgorithmParameters> 
</algorithm>
\end{myXmlLst}


%% ~~~ UNIFORM TIMECOURSE SIMULATION ~~~
\subsubsection{\element{UniformTimeCourse}}
\label{class:uniformTimeCourse}

\sedfig[width=0.75\textwidth]{pdf/timecourseSimulation}{The \code{UniformTimeCourse} class}{fig:timecourseSimulation}

\tabtext{uniformTimeCourse}{uniformTimeCourse}

\begin{table}[ht]
\center
\begin{tabular}{|l|l|}
\hline
\textbf{attribute} & \textbf{description}\\
\hline
metaid$^{o}$ & \refpage{sec:metaID}\\
id & \refpage{sec:id} \\
name$^{o}$ & \refpage{sec:name}\\
\hline
initialTime & \refpage{sec:initialTime}\\
outputStartTime & \refpage{sec:outputStartTime}\\
outputEndTime & \refpage{sec:outputEndTime}\\
numberOfPoints & \refpage{sec:numberOfPoints}\\
\hline
\hline
\textbf{\subelements} & \textbf{\desc}\\
\hline
notes$^{o}$ & \refpage{class:notes}\\
annotation$^{o}$ & \refpage{class:annotation}\\
\hline
algorithm & \refpage{class:algorithm}\\
\hline
\end{tabular}
\caption{\tabcap{uniformTimeCourse}}
\label{tab:uniformTimeCourse}
\end{table}

\lsttext{timecourse}{uniformTimeCourse}

\begin{myXmlLst}{The SED-ML \code{uniformTimeCourse} element, defining a uniform time course simulation over 2500 time units with 1000 simulation points.}{lst:timecourse}
<listOfSimulations>
 <uniformTimeCourse id="s1"  name="time course simulation of variable v1 over 100 minutes"  
  initialTime="0" outputStartTime="0" outputEndTime="2500" numberOfPoints="1000">
  <algorithm [..] />
 </uniformTimeCourse>
</listOfSimulations>
\end{myXmlLst}

\paragraph{\element{initialTime}}
\label{sec:initialTime}

The attribute \element{initialTime} of type \code{double} represents the time from which to start the simulation. Usually this will be \code{0}. Listing~\ref{lst:timecourse} shows an example. 

\paragraph{\element{outputStartTime}}

\label{sec:outputStartTime}

Sometimes a researcher is not interested in simulation results at the start of the simulation (i.e. the initial time). To accommodate this in SED-ML the \element{uniformTimeCourse} class uses the  attribute \element{outputStartTime} of type \code{double}. To be valid the \element{outputStartTime} cannot be before \element{initialTime}.  For an example, see Listing~\ref{lst:timecourse}. 

\paragraph{\element{outputEndTime}}
\label{sec:outputEndTime}

The attribute \element{outputEndTime} of type \code{double} marks the end time of the simulation. See Listing~\ref{lst:timecourse} for an example. 

\paragraph{\element{numberOfPoints}}
\label{sec:numberOfPoints}

When executed, the \element{uniformTimeCourse} simulation produces output on a regular grid starting with \element{outputStartTime} and ending with \element{outputEndTime}. The attribute  \element{numberOfPoints} of type \code{integer} describes the number of points expected in the result. Software interpreting the \element{uniformTimeCourse} is expected to produce a first outputPoint at time \element{outputStartTime} and then \element{numberOfPoints} output points with the results of the simulation. Thus a total of \code{numberOfPoints + 1} output points will be produced.

Just because the output points lie on the regular grid described above, this does not mean that the simulation algorithm has to work with the same step size. Usually the step size the simulator chooses will be adaptive and much smaller than the required output step size. On the other hand a stochastic simulator might not have any new events occurring between two grid points. Nevertheless the simulator has to produce data on this regular grid. For an example, see Listing~\ref{lst:timecourse}. 


%% ~~~ ONESTEP SIMULATION ~~~
\subsubsection{\element{OneStep}}
\label{class:oneStep}

\sedfig[width=0.3\textwidth]{pdf/oneStep}{The \code{OneStep} class}{fig:oneStepSimulation}

The SED-ML \code{oneStep} calculates one further output step for the model from its current state. Note that this does NOT necessarily equate to one integration step. The simulator is allowed to take as many steps as needed. However, after running oneStep, the desired output time is reached.

\tabtext{oneStep}{oneStep}

\begin{table}[ht]
\center
\begin{tabular}{|l|l|}
\hline
\textbf{attribute} & \textbf{description}\\
\hline
metaid$^{o}$ & \refpage{sec:metaID}\\
id & \refpage{sec:id} \\
name$^{o}$ & \refpage{sec:name}\\
\hline
step & \refpage{sec:step}\\
\hline
\hline
\textbf{\subelements} & \textbf{\desc}\\
\hline
notes$^{o}$ & \refpage{class:notes}\\
annotation$^{o}$ & \refpage{class:annotation}\\
\hline
algorithm & \refpage{class:algorithm}\\
\hline
\end{tabular}
\caption{\tabcap{oneStep}}
\label{tab:oneStep}
\end{table}

\lsttext{oneStep}{oneStep}

\begin{myXmlLst}{The SED-ML \code{oneStep} element, specifying to apply the simulation algorithm for another output step of size 0.1.}{lst:oneStep}
<listOfSimulations> 
  <oneStep id="s1" step="0.1"> 
    <algorithm kisaoID="KISAO:0000019" />
  </oneStep> 
</listOfSimulations>
\end{myXmlLst}

\paragraph{\element{step}}
\label{sec:step}
The \element{oneStep} class has one required attribute \element{step} of type \code{double}.
It defines the next output point that should be reached by the algorithm, by specifying the increment from the current output point. Listing~\ref{lst:oneStep} shows an example. 


%% ~~~ STEADYSTATE SIMULATION ~~~
\subsubsection{\element{SteadyState}}
\label{class:steadyState}
The \element{SteadyState} class represents a steady state computation (as for example implemented by NLEQ or Kinsolve). 

\sedfig[width=0.3\textwidth]{pdf/steadyState}{The \code{SteadyState} class}{fig:steadyStateSimulation}

\tabtext{steadyState}{steadyState}

\begin{table}[ht]
\center
\begin{tabular}{|l|l|}
\hline
\textbf{attribute} & \textbf{description}\\
\hline
metaid$^{o}$ & \refpage{sec:metaID}\\
id & \refpage{sec:id} \\
name$^{o}$ & \refpage{sec:name}\\
\hline
\hline
\hline
\textbf{\subelements} & \textbf{\desc}\\
\hline
notes$^{o}$ & \refpage{class:notes}\\
annotation$^{o}$ & \refpage{class:annotation}\\
\hline
algorithm & \refpage{class:algorithm}\\
\hline
\end{tabular}
\caption{\tabcap{steadyState}}
\label{tab:steadyState}
\end{table}

\lsttext{steadyState}{steadyState}

\begin{myXmlLst}{The SED-ML \code{steadyState} element, defining a steady state simulation with id \code{steady}.}{lst:steadyState}
<listOfSimulations>
  <steadyState id="steady"> 
    <algorithm kisaoID="KISAO:0000282" />
  </steadyState > 
</listOfSimulations>
\end{myXmlLst}


%%% Local Variables: 
%%% mode: plain-tex
%%% TeX-master: "../sed-ml-L1V3"
%%% End: 
