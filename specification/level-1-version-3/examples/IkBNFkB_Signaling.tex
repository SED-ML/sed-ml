\section{IkappaB-NF-kappaB Signaling (SBML)}
The following example provides a SED-ML description for the simulation of the IkappaB-NF-kappaB signaling module based on the publication by Hoffmann, Levchenko, Scott and  Baltimore ``The IkappaB-NF-kappaB signaling module: temporal control and selective gene activation.'' (PubMed ID: 12424381)

This model is referenced by its SED-ML ID \code{model1} and refers to the model with the MIRIAM URN \url{urn:miriam:biomodels.db:BIOMD0000000140}. 
Software applications interpreting this example know how to dereference this URN and access the model in \biom \citep{N+06}.

The simulation description specifies one simulation \code{simulation1}, which is a uniform timecourse simulation that simulates the model for 41 hours. \code{task1} then applies this simulation to the model. 

As output this simulation description collects four parameters: \code{Total\_NFkBn}, \code{Total\_IkBbeta}, \code{Total\_IkBeps} and \code{Total\_IkBalpha}. These variables are to be plotted against the simulation time and displayed in four separate plots, as shown in Figure \ref{fig:ikappabExample}. 

\sedfig[width=0.8\textwidth]{ikappab}{The simulation result gained from the simulation description given in \lst{ikappab}}{fig:ikappabExample}

The SED-ML description of the simulation experiment is given in \lst{ikappab}.

\myXmlImport{IkappaB-NF-kappaB signaling Model Simulation Description in SED-ML}{lst:ikappab}{examples/IkBNFkB_Signaling.xml}


%%% Local Variables: 
%%% mode: latex
%%% TeX-master: "../sed-ml-L1V1"
%%% End: 
