%% Filename:     sbmlpkgspec.cls                         -*- mode: latex -*-
%% Description:  LaTeX style file for SBML Level 3 package specifications
%% Author(s):    Michael Hucka <mhucka@caltech.edu>
%% Organization: California Institute of Technology
%% Date created: 2011-08-11
%%
%% Copyright 2007-2014 California Institute of Technology, Pasadena, CA, USA.
%% 
%% This file is free software; you can redistribute it and/or modify it
%% under the terms of the GNU Lesser General Public License as published
%% by the Free Software Foundation; either version 2.1 of the License, or
%% any later version.
%% 
%% This library is distributed in the hope that it will be useful, but
%% WITHOUT ANY WARRANTY, WITHOUT EVEN THE IMPLIED WARRANTY OF
%% MERCHANTABILITY OR FITNESS FOR A PARTICULAR PURPOSE.  The software and
%% documentation provided hereunder is on an "as is" basis, and the
%% California Institute of Technology has no obligations to provide
%% maintenance, support, updates, enhancements or modifications.  In no
%% event shall the California Institute of Technology be liable to any
%% party for direct, indirect, special, incidental or consequential
%% damages, including lost profits, arising out of the use of this software
%% and its documentation, even if the California Institute of Technology
%% has been advised of the possibility of such damage.  See the
%% GNU Lesser General Public License for more details.
%% 
%% You should have received a copy of the GNU Lesser General Public License
%% along with this library; if not, write to the Free Software Foundation,
%% Inc., 59 Temple Place, Suite 330, Boston, MA 02111-1307 USA.

%\def\filedate{2017/08/16}
%\NeedsTeXFormat{LaTeX2e}
%%\ProvidesClass{sbmlpkgspec}[\filedate\space SBML Level 3
%%Package Specification Style]

%%% ----------------------------------------------------------------------------
%%% Package references and options.
%%% ----------------------------------------------------------------------------

%%% Keep in mind that hyperref needs to be almost the last package loaded.

%\newif\if@draftspec
%\DeclareOption{draftspec}{\global\let\if@draftspec\iftrue}
%\DeclareOption{finalspec}{\global\let\if@draftspec\iffalse}

%\newif\if@notoc
%\DeclareOption{toc}{\global\let\if@notoc\iffalse}
%\DeclareOption{notoc}{\global\let\if@notoc\iftrue}

%\newif\if@twocolumntoc
%\DeclareOption{twocolumntoc}{\global\let\if@twocolumntoc\iftrue}
%\DeclareOption{singlecolumntoc}{\global\let\if@twocolumntoc\iffalse}

%\DeclareOption*{\PassOptionsToClass{\CurrentOption}{article}}

%\ExecuteOptions{toc,twoside}
%\ExecuteOptions{singlecolumntoc}
%\ProcessOptions
%%\LoadClass{article}

%%% ----------------------------------------------------------------------------
%%% Dependencies on other packages.
%%% ----------------------------------------------------------------------------

%\RequirePackage{lastpage}
%\RequirePackage{ifpdf}
%\RequirePackage{booktabs}
%\RequirePackage{multicol}
%\RequirePackage{array}
%\RequirePackage[round,sort]{natbib}
%\RequirePackage{xspace}
%\RequirePackage{scalefnt}
%\RequirePackage{pifont}
%\RequirePackage[pagewise,mathlines,edtable,right]{lineno}
%\RequirePackage{calc}
%\RequirePackage{fancyhdr}
%\RequirePackage{fancybox}
%\RequirePackage{amsmath}
%\RequirePackage{amssymb}
%\RequirePackage{bbding}
%\RequirePackage{wasysym}
%\RequirePackage{enumitem}
%\RequirePackage{accsupp}
%\RequirePackage[T1]{fontenc}
%\RequirePackage{suffix}
%\RequirePackage{varwidth}
%\RequirePackage{etoolbox}

%% We use the "optional" package to indicate whether to generate a grayscale
%% vs a color version of the document without having to alter the .tex file
%% itself.  Unfortunately, the "optional" package doesn't offer an "or" type
%% switch or a way to set a default, so we have to go through some
%% contortions here.  Also, it needs *something* in the optional argument
%% to RequirePackage or UsePackage, hence the "dummyoption" thing below.
%%
%% To produce grayscale output without modifying the source file, invoke
%% latex with a command of the following form:
%%
%%  pdflatex "\newcommand\UseOption{grayscale}\input{sbml-level-2-version-2}"

%\newif\ifgrayscalespec
%\global\let\ifgrayscalespec\iffalse
%\RequirePackage[dummyoption]{optional}
%\opt{grayscale}{\global\let\ifgrayscalespec\iftrue}

%% Load varioref only if not generating HTML, because on HTML
%% pages it doesn't make sense to use varioref.

%% \RequirePackage{html}

%% \latexhtml{
%%begin{latexonly}
%  \RequirePackage[american]{varioref}
%%end{latexonly}
%% }{
%%  \newcommand{\vref}[1]{\ref{#1}}
%% }

%% Hyperref, xcolor, graphicx and possibly others have a flag "pdftex"
%% that needs to be used if pdflatex is being used.  The following puts
%% these inside a conditional for that situation.

%\ifpdf
%  % Case: using pdflatex

%  \RequirePackage[pdftex]{graphicx}

%  % Options get even more complicated.  If we're producing grayscale output,
%  % we don't want to bother with coloring links, but we still want to load
%  % hyperref so that its macros are defined (and we don't have to redefine
%  % everything that uses hyperref).  So:

%  \ifgrayscalespec
%    \RequirePackage[pdftex,breaklinks=true,colorlinks=false,
%    bookmarksnumbered=true]{hyperref}
%  \else
%    \RequirePackage[pdftex,breaklinks=true,colorlinks=true,
%    pdfhighlight=/O,linkcolor={sbmlblue},citecolor={sbmlblue},
%    urlcolor={sbmlblue},anchorcolor={sbmlblue},bookmarksnumbered]{hyperref}
%  \fi

%  % Although it may seem like we wouldn't need to load xcolor if
%  % the grayscale option is given, in fact it doesn't affect anything
%  % to load this and it avoids having to conditionalize other things.
%  % (The colors aren't actually invoked if grayscale is being used.)

%  \RequirePackage[pdftex,rgb,dvipsnames,svgnames,hyperref,table]{xcolor}
%\else
%  % Case: not using pdflatex

%  \RequirePackage{graphicx}
%  % Test whether we're being run from latex or latex2html
%  % \latexhtml{
%    % For latex.
%    \RequirePackage[breaklinks]{hyperref}
%  % }{
%    % For latex2html.
%    % \RequirePackage[latex2html,breaklinks]{hyperref}
%  % }
%  \RequirePackage[rgb,dvipsnames,svgnames,hyperref,table]{xcolor}
%\fi

%% Load listings package & set some values.

%\RequirePackage{listings}
%\lstloadlanguages{bash,csh,XML}

%% Load draftwatermark if this document is a draft version.

%\if@draftspec
%  \RequirePackage[firstpage]{draftwatermark}
%  \SetWatermarkLightness{0.92}
%\fi

%\RequirePackage{overpic}

%% We don't need the rotating package in this file, but there is a strange
%% interaction between the rotating package and something else in sbmlpkgclass
%% (I don't know what).  The effect of the interaction is that sidewaystable
%% environment contents are shifted left on the page (or from the perspective
%% of the rotated table, shiftted up).  The solution is to set \rotFPtop, but
%% asking users of sbmlpkgclass to know that they have to do this is too much
%% to ask.  So, to avoid people experiencing the problem and wasting time
%% trying to figure out what's causing it, I decided it's better to preload
%% rotating.  The problem was first reported to me by Maciej Swat in Aug. 2013
%% and I found the solution at http://tex.stackexchange.com/a/47879/8318

%\RequirePackage{rotating}
%\setlength{\rotFPtop}{0pt plus 1fil}


%%% ----------------------------------------------------------------------------
%%% Font selection.
%%% ----------------------------------------------------------------------------

%% This sets up Helvetica for headings and TX Typewriter for the tt font.
%% The font scaling is because the default Helvetica size is too big.

%\RequirePackage{fourier}
%\RequirePackage{helvet}
%\ifgrayscalespec
%  % In the grayscale version of the spec, Helvetica bold is used for class
%  % names in the text, and the width of the words then is too big unless
%  % we scale down the font even more compared to the regular case below.
%  \def\Hv@scale{0.814}
%\else
%  \def\Hv@scale{0.87}
%\fi

%% The following sets up txtt for the typewriter font.

%\renewcommand{\ttdefault}{txtt}
%\DeclareMathAlphabet{\mathtt}{T1}{txtt}{m}{n}
%\SetMathAlphabet{\mathtt}{bold}{T1}{txtt}{b}{n}

%% The next bit is an adaption of code from ot1phv.fd and adapted to the txtt
%% fonts.  The txtt fonts are just a tad too big, so this tries to rescale
%% them down a tiny bit.  This isn't completely right because I couldn't
%% figure out the right syntax when the DeclareFontShape uses ssub below.
%% (Notice how the ones with ssub don't have the \Txtt@@scale factor.)

%\def\Txtt@scale{0.97}
%\edef\Txtt@@scale{s*[\csname Txtt@scale\endcsname]}%

%\DeclareFontFamily{T1}{txtt}{\hyphenchar \font\m@ne}
%\DeclareFontShape{T1}{txtt}{m}{n}{	%rebular
%     <-> \Txtt@@scale txtt%
%}{}
%\DeclareFontShape{T1}{txtt}{m}{sc}{	%cap & small cap
%     <-> \Txtt@@scale txttsc%
%}{}
%\DeclareFontShape{T1}{txtt}{m}{sl}{	%slanted
%     <-> \Txtt@@scale txttsl%
%}{}
%\DeclareFontShape{T1}{txtt}{m}{it}{	%italic
%     <-> ssub * txtt/m/sl%
%}{}
%\DeclareFontShape{T1}{txtt}{m}{ui}{	%unslanted italic
%     <-> ssub * txtt/m/sl%
%}{}
%\DeclareFontShape{T1}{txtt}{b}{n}{	%bold
%     <-> \Txtt@@scale txbtt%
%}{}
%\DeclareFontShape{T1}{txtt}{b}{sc}{	%bold cap & small cap
%     <-> \Txtt@@scale txbttsc%
%}{}
%\DeclareFontShape{T1}{txtt}{b}{sl}{	%bold slanted
%     <-> \Txtt@@scale txbttsl%
%}{}
%\DeclareFontShape{T1}{txtt}{b}{it}{	%bold italic
%     <-> ssub * txtt/b/sl%
%}{}
%\DeclareFontShape{T1}{txtt}{b}{ui}{	%bold unslanted italic
%     <-> ssub * txtt/b/sl%
%}{}
%\DeclareFontShape{T1}{txtt}{bx}{n}{	%bold extended
%     <-> ssub * txtt/b/n%
%}{}
%\DeclareFontShape{T1}{txtt}{bx}{sc}{	%bold extended cap & small cap
%     <-> ssub * txtt/b/sc%
%}{}
%\DeclareFontShape{T1}{txtt}{bx}{sl}{	%bold extended slanted
%     <-> ssub * txtt/b/sl%
%}{}
%\DeclareFontShape{T1}{txtt}{bx}{it}{	%bold extended italic
%     <-> ssub * txtt/b/sl%
%}{}
%\DeclareFontShape{T1}{txtt}{bx}{ui}{	%bold extended unslanted italic
%     <-> ssub * txtt/b/sl%
%}{}

%% Adjustments due to quirks of the combination of fourier and amsmath:

%% Make the \big( braces regular height (default is too short):
%\delimiterfactor=1000
%\delimitershortfall=1pt

%%% ----------------------------------------------------------------------------
%%% Main code.
%%% ----------------------------------------------------------------------------

%% Local color definitions.

%\definecolor{sbmlblue}{rgb}{0.07,0.50,0.78}
%\definecolor{sbmlgray}{gray}{0.7}
%\definecolor{sbmlrowgray}{gray}{0.96}
%\definecolor{extremelylightgray}{gray}{0.97}
%\definecolor{veryverylightgray}{gray}{0.95}
%\definecolor{verylightgray}{gray}{0.9}
%\definecolor{lightgray}{gray}{0.8}
%\definecolor{mediumgray}{gray}{0.5}
%\definecolor{darkgray}{gray}{0.3}
%\definecolor{almostblack}{gray}{0.23}
%\definecolor{lightyellow}{rgb}{0.98,0.94,0.7}
%\definecolor{verylightyellow}{rgb}{0.97,0.95,0.85}
%\definecolor{darkblue}{rgb}{0.1,0.4,0.55}
%\definecolor{mediumgreen}{rgb}{0.1,0.6,0.3}

%\definecolor{sbmlnormaltextcolor}{gray}{0.27}
%\definecolor{sbmlliteralcolor}{gray}{0}
%\definecolor{sbmllinenumbercolor}{gray}{0.75}
%\definecolor{sbmlchangedcolor}{rgb}{0.69,0.19,0.376}

%% Macros and settings for making consistent font, color, and other
%% selections.
%%
%% If we're outputting grayscale, we don't use the red changed color because
%% the result on most printers is text that a hard-to-read light gray color.

%\ifgrayscalespec
%  \colorlet{@sbmlchangedcolor}{sbmlnormaltextcolor}
%\else
%  \colorlet{@sbmlchangedcolor}{sbmlchangedcolor}
%\fi
%\colorlet{@sectionnumcolor}{sbmlnormaltextcolor}
%\colorlet{@literalcolor}{sbmlliteralcolor}

%\newcommand{\changed}[1]{%
%  \protect\colorlet{@currentliteralcolor}{@literalcolor}%
%  \protect\colorlet{@currentsectionnumcolor}{@sectionnumcolor}%
%  \protect\colorlet{@literalcolor}{sbmlchangedcolor}%
%  \protect\colorlet{@sectionnumcolor}{sbmlchangedcolor}%
%  \textcolor{sbmlchangedcolor}{#1}%
%  \protect\colorlet{@literalcolor}{@currentliteralcolor}%
%  \protect\colorlet{@sectionnumcolor}{@currentsectionnumcolor}}

%\newenvironment{blockChanged}{%
%  \colorlet{@currentliteralcolor}{@literalcolor}%
%  \colorlet{@currentsectionnumcolor}{@sectionnumcolor}%
%  \colorlet{@literalcolor}{sbmlchangedcolor}%
%  \colorlet{@sectionnumcolor}{sbmlchangedcolor}%
%  \color{sbmlchangedcolor}%
%}{%
%  \color{sbmlnormaltextcolor}%
%  \colorlet{@literalcolor}{@currentliteralcolor}%
%  \colorlet{@sectionnumcolor}{@currentsectionnumcolor}}

%\newcommand{\figureFont}[1]{\textsf{\textbf{#1}}}

%\newcommand{\literalFont}[1]{\textcolor{@literalcolor}{\textup{\ttfamily{#1}}}}
%\newcommand{\literalFontNC}[1]{\textup{\ttfamily{#1}}}

%\newcommand{\tightspacing}{\renewcommand{\baselinestretch}{0.85}}
%\newcommand{\regularspacing}{\renewcommand{\baselinestretch}{\normalbaselinestretch}}

%\ifgrayscalespec
%  \newcommand{\defRef}[2]{\textbf{\class{#1}}\xspace}
%  \newcommand{\absDefRef}[2]{\textbf{\abstractclass{#1}}\xspace}
%\else
%  \newcommand{\defRef}[2]{\class{\hyperref[#2]{#1}}\xspace}
%  \newcommand{\absDefRef}[2]{\abstractclass{\hyperref[#2]{#1}}\xspace}
%  \newcommand{\absDefRefUpright}[2]{\abstractclassUpright{\hyperref[#2]{#1}}\xspace}
%\fi

%\renewcommand{\texttt}[1]{\textcolor{@literalcolor}{\ttfamily #1}}

%% Hyperref extras.

%\newcommand{\link}[2]{\literalFontNC{\href{#1}{#2}}}
%\newcommand{\mailto}[1]{\link{mailto:#1}{#1}}

%% 'lineno' package adjustments.
%% The BeginAccSupp business causes the line numbers to be ignored for
%% copy-paste operations in the PDF output, at least in Acrobat.

%\setlength{\linenumbersep}{2.2em}
%\renewcommand{\linenumberfont}{\tiny\sffamily}
%\renewcommand{\thelinenumber}{%
%  \BeginAccSupp{ActualText={}}%
%  \textcolor{sbmllinenumbercolor}{\parbox[b][\height+1.5pt][c]{10pt}{\arabic{linenumber}}}%
%  \EndAccSupp{}%
%}

%% 'booktabs' package adjustments:

%\setlength{\cmidrulewidth}{0.3 pt}
%\setlength{\lightrulewidth}{0.3 pt}
%\setlength{\heavyrulewidth}{0.9 pt}

%% Fix placement of figures & tables.  This keeps latex from shoving big
%% floats to the end of a document when they are somewhat big, which it will
%% do even if you put [htb] as the argument.

%\setcounter{topnumber}{2}               % max num of floats at top of page
%\setcounter{bottomnumber}{2}            % max num of floats at bottom of page
%\renewcommand\topfraction{1.0}          % fraction that a top float can cover
%\renewcommand\bottomfraction{1.0}       % fraction a bottom float can cover
%\renewcommand\textfraction{0.0}        % >5% of a non-float page must be text
%\renewcommand\floatpagefraction{0.9}   % float page must be 95% full

%% Spacing of floats.

%\setlength{\intextsep}{20pt plus 4pt minus 0pt}

%% Make floats that appear alone on a page appear at the top, rather than
%% (as is the LaTeX default) appearing centered vertically on the page.

%\setlength{\@fptop}{5pt}

%% Margin adjustments. I've tried using vmargin.sty, but it interacts badly
%% with page numbers at the bottom of the page, so I've resorted to hardcoding
%% the dimensions like this.

%\setlength{\marginparwidth}{0.77in}
%\setlength{\marginparsep}{4pt}

%\setlength{\oddsidemargin}{0 pt}
%\setlength{\evensidemargin}{0 pt}
%\setlength{\topmargin}{-0.5 in}
%\setlength{\voffset}{0 in}
%\setlength{\hoffset}{0 in}
%\setlength{\textwidth}{6.5 in}
%\setlength{\textheight}{8.95 in}

%% Playing games with line spacing.

%\newcommand{\normalbaselinestretch}{1.04}
%\renewcommand{\baselinestretch}{\normalbaselinestretch}

%% Set the table of contents depth

%\setcounter{tocdepth}{3}

%% Title page macros.  Severly hacked from originals in LaTeX's article.cls.

%\def\packageTitle#1{\title{#1}\gdef\@packageTitle{#1}}
%\def\@packageTitle{\@empty}

%\def\packageGeneralURL#1{\gdef\@packageGeneralURL{#1}}

%\def\packageThisVersionURL#1{\gdef\@packageThisVersionURL{#1}}
%\def\@packageThisVersionURL{\@empty}

%\def\packageVersion#1{\gdef\@packageVersion{#1}}
%\def\@packageVersion{\@empty}

%\def\packageVersionDate#1{\gdef\@packageVersionDate{#1}}
%\def\@packageVersionDate{\@empty}

%\def\frontNotice#1{\gdef\@frontNotice{#1}}

%\def\@puttitle{%
%  \ifx\@packageGeneralURL\undefined
%  \relax
%  \else
%  \vspace*{-1em}
%  {\large SBML Level 3 Package Specification}\\[2em]
%  \fi
%  \scalebox{1}[0.9]{%
%    \fcolorbox{black}{mediumgray}{%
%      \begin{minipage}{\textwidth - 7pt}%
%        \vspace*{6pt}%
%        \centering%
%        \textcolor{white}{\sffamily\bfseries\huge \ifx\@title\@empty\@packageTitle\else\@title\fi}
%      \vspace*{4pt}%
%    \end{minipage}}}}

%\def\@putauthorinfo{%
%  \large
%  \renewcommand{\arraystretch}{0.95}%
%  {\begin{tabular}[t]{c}%
%   \@author
%  \end{tabular}}}

%\def\maketableofcontents{%
%  \thispagestyle{plain}%
%  \if@notoc
%  \else
%    \begingroup
%      \small%
%      % Tighten spacing between lines within an entry.
%      % This assumes 10 pt font!
%      \setlength{\baselineskip}{10pt}%
%      % Now adjust inter-entry spacing.
%      \addtolength{\parskip}{-1.35 ex}%
%      \if@twocolumntoc
%        \setlength{\columnsep}{16pt}%
%        \begin{multicols}{2}
%      \fi
%      \tableofcontents%
%      \if@twocolumntoc
%        \end{multicols}
%      \fi
%      \normalsize%
%      \addtolength{\parskip}{1.45 ex}%
%    \endgroup
%    \clearpage
%  \fi%
%  \linenumbers}

%% Need redefine \tableofcontents to counter effects of hacking \section below.

%\renewcommand\tableofcontents{%
%  \@startsection{section}{0}{0pt}{-1.8ex \@plus -1ex \@minus -.2ex}%
%  {0.8ex}{\normalfont\Large\bfseries\sffamily}*% The star fakes a \section*
%  \contentsname%
%  \@mkboth{\MakeUppercase\contentsname}{\MakeUppercase\contentsname}%
%  \@starttoc{toc}%
%  }

%\renewcommand\maketitle{\par
%  \begingroup
%  \renewcommand\thefootnote{\@fnsymbol\c@footnote}%
%  \def\@makefnmark{\rlap{\@textsuperscript{\normalfont\@thefnmark}}}%
%  \long\def\@makefntext##1{\parindent 1em\noindent
%    \hb@xt@1.8em{%
%      \hss\@textsuperscript{\normalfont\@thefnmark}}##1}%
%  \if@twocolumn
%    \ifnum \col@number=\@ne
%      \@maketitle
%    \else
%      \twocolumn[\@maketitle]%
%    \fi
%  \else
%    \global\@topnum\z@   % Prevents figures from going at top of page.
%    \@maketitle
%  \fi
%  \thispagestyle{empty}\@thanks
%  \endgroup
%  \setcounter{footnote}{0}%
%  \global\let\thanks\relax
%  \global\let\maketitle\relax
%  \global\let\@maketitle\relax
%  \global\let\@thanks\@empty
%  \global\let\@author\@empty
%  \global\let\@date\@empty
%  \global\let\@title\@empty
%  \global\let\@authoremail\@empty
%  \global\let\title\relax
%  \global\let\author\relax
%  \global\let\date\relax
%  \global\let\and\relax
%  }

%\newcommand\maketitlepage{\par
%  \begingroup
%  \renewcommand\thefootnote{\@fnsymbol\c@footnote}%
%  \def\@makefnmark{\rlap{\@textsuperscript{\normalfont\@thefnmark}}}%
%  \long\def\@makefntext##1{\parindent 1em\noindent
%    \hb@xt@1.8em{%
%      \hss\@textsuperscript{\normalfont\@thefnmark}}##1}%
%  \vskip 2em%
%  \if@twocolumn
%    \ifnum \col@number=\@ne
%      \@maketitle
%    \else
%      \twocolumn[\@maketitle]%
%    \fi
%  \else
%    \newpage
%    \global\@topnum\z@   % Prevents figures from going at top of page.
%    \@maketitle
%  \fi
%  \thispagestyle{empty}\@thanks
%  \vfill
%  \centering%
%  \ifx\@frontNotice\undefined
%  \relax
%  \else
%    {\cornersize{0.25}\color{DarkRed}\ovalbox{\begin{varwidth}{0.9\textwidth}\centering\@frontNotice\end{varwidth}}}%
%  \fi
%  \vfill
%  \ifx\@packageGeneralURL\undefined
%  \relax
%  \else
%    The latest release, past releases, and other materials related to this specification are available at\\
%    {\small\url{\@packageGeneralURL}}\\[1em]
%    \emph{This} release of the specification is available at\\
%    {\small\url{\@packageThisVersionURL}}\\
%    \vspace*{2em}
%  \fi
%  \vfill
%  \centerline{\includegraphics[width=1in]{\SBMLbadge}}
%  \clearpage
%  \endgroup
%  \setcounter{footnote}{0}%
%  \global\let\thanks\relax
%  \global\let\maketitle\relax
%  \global\let\@maketitle\relax
%  \global\let\@thanks\@empty
%  \global\let\@author\@empty
%  \global\let\@date\@empty
%  \global\let\@title\@empty
%  \global\let\@authoremail\@empty
%  \global\let\title\relax
%  \global\let\author\relax
%  \global\let\date\relax
%  \global\let\and\relax
%  }

%\def\@maketitle{%
%  \newpage
%  \null
%  \begin{center}%
%    \let \footnote \thanks
%    \@puttitle
%    \vskip 1.8em%
%    {\lineskip .5em%
%      \@putauthorinfo
%      }%
%    \vskip 2em%
%    {\large\fbox{\@packageVersion}}%
%    \vskip 1.5em%
%    {\large\@packageVersionDate}%
%  \end{center}%
%  \par%
%  \vskip 1.5em%
%  \ifpdf
%    \hypersetup{pdftitle={\@title}}
%  \fi%
%  }


%% Including the logo on the front page.  We switch which copy of the logo
%% file we use, depending on various factors, and we switch which file format
%% is used depending on the output we're producing. 

%\ifgrayscalespec
%  \newcommand{\@badgebasefile}{sbml-badge-gray}
%\else
%  \newcommand{\@badgebasefile}{sbml-badge}
%\fi

%\ifpdf
%  % Request the JPG format specifically, because the resulting
%  % quality in the final output is best.
%  \newcommand{\SBMLbadge}{\@badgebasefile.jpg}%
%\else%
%  \newcommand{\SBMLbadge}{\@badgebasefile.eps}%
%\fi


%% Graphics adjustments.  The path setup is so that the \includegraphics
%% in the @puttile definition can find the logo file no matter where the
%% document is located (but obviously, it only works for certain path
%% combinations -- it's a total hack).

%\graphicspath{{./logos/}
%  {../tex/logos/}
%  {../project/tex/logos/}
%  {../../project/tex/logos/}
%  {../../../project/tex/logos/}
%  {../../../../project/tex/logos/}}


%% Set the page footers.
%% Must do this after setting the document title.

%\fancyhf{}

%\newcommand{\@footchpsection}{\sectionlabel~\nouppercase{\leftmark}}
%\newcommand{\@footpage}{Page \thepage\ of \pageref*{LastPage}}

%\renewcommand{\footrule}{}

%\fancypagestyle{headerandfooter}{%
%  \renewcommand{\headrulewidth}{0.25pt}
%  \renewcommand{\headrule}{}
%  \lhead{}%
%  \rhead{\color{mediumgray}\small\textsf{\sectionlabel~\nouppercase{\rightmark}}}%
%  \cfoot{}%
%  \lfoot{\color{mediumgray}\small\textsf{\@footchpsection}}%
%  \rfoot{\color{mediumgray}\small\textsf{\@footpage}}%
%}

%\fancypagestyle{footeronly}{%
%  \renewcommand{\headrule}{}%
%  \lhead{}%
%  \rhead{}%
%  \lfoot{\color{mediumgray}\small\textsf{\@footchpsection}}%
%  \rfoot{\color{mediumgray}\small\textsf{\@footpage}}%
%}

%\fancypagestyle{plain}{%
%  \renewcommand{\headrule}{}%
%  \lhead{}%
%  \rhead{}%
%  \lfoot{}%
%  \rfoot{\color{mediumgray}\small\textsf{\@footpage}}%
%}

%\pagestyle{headerandfooter}


%% Definition of paragraph style.

%\setlength{\parindent}{0 pt}            % Unindented paragraphs, separated ...
%\setlength{\parskip}{1.3 ex}            % ... by roughly one blank line.
%\setlength{\partopsep}{-1ex plus 0.1ex minus -0.2ex}
%\setlength{\itemsep}{-0.25ex plus 0.15ex}

%% \topsep is supposed to affect list environments like itemize,
%% but does nothing there.  Instead, it affects environments like tabular.

%\setlength{\topsep}{0.3ex plus 0.1ex minus -0.2ex}

%\renewcommand{\labelitemi}{\raisebox{0.5pt}{\color{mediumgray}\footnotesize{$\blacksquare$}}}
%\renewcommand{\labelitemii}{\raisebox{-1.5pt}{\color{mediumgray}\LARGE$\bullet$}}
%\renewcommand{\labelitemiii}{\color{mediumgray}\tiny\DiamondSolid}
%\renewcommand{\labelitemiv}{$\diamond$}

%% Definition of section heading style.
%% The "changed" versions put the numbers in red, to indicate a change
%% in section numbers.

%\renewcommand\@seccntformat[1]{%
%  \color{@sectionnumcolor}{\csname the#1\endcsname}\hspace{0.5em}}

%\renewcommand{\section}{\@startsection%
%  {section}{1}{0pt \@plus 1ex}{-3.5ex \@plus -2ex \@minus -.2ex}%
%  {1.2ex}{\clearpage\thispagestyle{footeronly}%
%    \hspace*{-24.25pt}\color{sbmlgray}{\rule[1.8em]{6.84in}{0.6em}}\color{sbmlnormaltextcolor}%
%    \hspace*{-6.84in}\color{sbmlgray}{\rule[-0.8em]{6.84in}{0.1em}}\color{sbmlnormaltextcolor}%
%    \noindent\normalfont\LARGE\bfseries\sffamily\hspace*{-6.84in}}}

%\newcommand{\sectionChanged}[1]{%
%  \colorlet{@currentsectionnumcolor}{@sectionnumcolor}%
%  \colorlet{@sectionnumcolor}{sbmlchangedcolor}%
%  \section{#1}%
%  \colorlet{@sectionnumcolor}{@currentsectionnumcolor}}

%\renewcommand{\subsection}{\@startsection%
%  {subsection}{2}{0pt}{-1.25ex \@plus 2ex \@minus -.2ex}%
%  {0.1ex}{\Large\bfseries\sffamily}}

%\newcommand{\subsectionChanged}[1]{%
%  \colorlet{@sectionnumcolor}{sbmlchangedcolor}%
%  \subsection{#1}%
%  \colorlet{@sectionnumcolor}{@currentsectionnumcolor}}

%\renewcommand{\subsubsection}{\@startsection%
%  {subsubsection}{3}{0pt}{-1ex \@plus 2ex \@minus -.2ex}%
%  {0.4ex}{\large\bfseries\sffamily\slshape}}

%\newcommand{\subsubsectionChanged}[1]{%
%  \colorlet{@sectionnumcolor}{sbmlchangedcolor}%
%  \subsubsection{#1}%
%  \colorlet{@sectionnumcolor}{@currentsectionnumcolor}}

%\renewcommand{\paragraph}{\@startsection%
%  {paragraph}{4}{0pt}{-0.75ex \@plus 2ex \@minus -.2ex}%
%  {0.4ex}{\normalsize\bfseries\sffamily\slshape}}

%\newcommand{\paragraphChanged}[1]{%
%  \colorlet{@sectionnumcolor}{sbmlchangedcolor}%
%  \paragraph{#1}%
%  \colorlet{@sectionnumcolor}{@currentsectionnumcolor}}

%% The References heading needs to be fixed up to account for shift
%% we apply to section headings.  These defs are modified from article.cls.

%\bibliographystyle{apalike}

%\renewenvironment{thebibliography}[1]%
%{\section*{References}%
%  \pagestyle{plain}%
%  \addcontentsline{toc}{section}{\protect\numberline{\refname}}%
%  \list{\@biblabel{\@arabic\c@enumiv}}%
%  {\settowidth\labelwidth{\@biblabel{#1}}%
%    \leftmargin\labelwidth
%    \advance\leftmargin\labelsep
%    \@openbib@code
%    \usecounter{enumiv}%
%    \let\p@enumiv\@empty
%    \renewcommand\theenumiv{\@arabic\c@enumiv}}%
%  \sloppy
%  \clubpenalty4000
%  \@clubpenalty \clubpenalty
%  \widowpenalty4000%
%  \sfcode`\.\@m}
%{\def\@noitemerr
%  {\@latex@warning{Empty `thebibliography' environment}}%
%  \endlist}

%\newcommand{\sectionspecial}{\@startsection%
%  {section}{1}{0pt}{-1.8ex \@plus -1ex \@minus -.2ex}%
%  {0.8ex}{\hspace*{-41.1pt}\colorbox{sbmlgray}{\hspace*{6.99in}}\nopagebreak\\%
%  \hspace*{-20pt}\normalfont\Large\bfseries\sffamily}}

%% The following is a modified version of the macro from article.cls.
%% It adjusts the vertical spacing in TOC for section lines.

%\renewcommand*\l@section[2]{%
%  \ifnum \c@tocdepth >\z@
%    \addpenalty\@secpenalty
%    \addvspace{0.2ex \@plus\p@}%
%    \setlength\@tempdima{1.5em}%
%    \begingroup
%      \parindent \z@ \rightskip \@pnumwidth
%      \parfillskip -\@pnumwidth
%      \leavevmode \bfseries\sffamily
%      \advance\leftskip\@tempdima
%      \hskip -\leftskip
%      #1\nobreak\hfil \nobreak\hb@xt@\@pnumwidth{\hss #2}\par
%    \endgroup
%  \fi}

%\def\@dottedtocline#1#2#3#4#5{%
%  \ifnum #1>\c@tocdepth \else
%    \vskip \z@ \@plus.2\p@
%    {\if@twocolumntoc\else
%      \leftskip #2
%     \fi
%     \relax \rightskip \@tocrmarg \parfillskip -\rightskip
%     \if@twocolumntoc\else
%       \parindent #2
%     \fi
%     \relax\@afterindenttrue
%     \interlinepenalty\@M
%     \leavevmode
%     \@tempdima #3\relax
%     \advance\leftskip \@tempdima \null\nobreak\hskip -\leftskip
%     {\sffamily #4}\nobreak
%     {\color{mediumgray}
%     \leaders\hbox{$\m@th
%        \mkern \@dotsep mu\hbox{.}\mkern \@dotsep
%        mu$}\hfill
%     }\nobreak
%     \hb@xt@\@pnumwidth{\hfil\normalfont\sffamily #5}%
%     \par}%
%  \fi}

%% The following is modified from natbib.sty, version 1999/05/28 7.0.
%% This ensures that the References section gets a TOC entry.

%\renewcommand{\bibsection}{%
%  \addcontentsline{toc}{section}{\protect\textbf{\textsf{\refname}}}%
%  \section*{\refname}}


%% Redefine table & figure to use smaller font, sans serif,
%% and to center contents by default.

%\renewenvironment{table}[1][]  {\@float{table}[#1]\small\sffamily\centering}    {\end@float}
%\renewenvironment{table*}[1][] {\@dblfloat{table}[#1]\small\sffamily\centering} {\end@dblfloat}
%\renewenvironment{figure}[1][] {\@float{figure}[#1]\small\sffamily\centering}   {\end@float}
%\renewenvironment{figure*}[1][]{\@dblfloat{figure}[#1]\small\sffamily\centering}{\end@dblfloat}


%% The following was ripped out of caption.sty, version 1.4b.
%% Copyright (C) 1994-95 Harald Axel Sommerfeldt
%% The first few lines set up the parameters for the layout created
%% by this style file.

%\newcommand{\captionsize}{\small}
%\newcommand{\captionfont}{\captionsize\sffamily\slshape}
%\newcommand{\captionlabelfont}{\captionsize\bfseries\slshape}
%\setlength{\abovecaptionskip}{1em}
%\newlength{\captionmargin}
%\setlength{\captionmargin}{6ex}

%\newsavebox{\as@captionbox}
%\newlength{\as@captionwidth}
%\newcommand{\as@normalcaption}[2]{%
%  #1 #2\par}
%\let\as@caption\as@normalcaption
%\newcommand{\as@centercaption}[2]{%
%  \parbox[t]{\as@captionwidth}{{\centering#1 #2\par}}}
%\let\as@shortcaption\as@centercaption
%\newcommand{\as@makecaption}[2]{%
%  \setlength{\leftskip}{\captionmargin}%
%  \setlength{\rightskip}{\captionmargin}%
%  \addtolength{\as@captionwidth}{-2\captionmargin}%
%  \renewcommand{\baselinestretch}{0.9}
%  \captionfont%
%  \sbox{\as@captionbox}{{\captionlabelfont #1:} #2}%
%  \ifdim \wd\as@captionbox >\as@captionwidth
%    \as@caption{{\captionlabelfont #1:}}{#2}%
%  \else%
%    \as@shortcaption{{\captionlabelfont #1:}}{#2}%
%  \fi}
%\renewcommand{\@makecaption}[2]{%
%  \vskip\abovecaptionskip%
%  \setlength{\as@captionwidth}{\linewidth}%
%  \as@makecaption{#1}{#2}%
%  \vskip\belowcaptionskip}
%\ifx\@makerotcaption\undefined
%\else
%  % Adjustment for the "rotating" package.
%  \renewcommand{\@makerotcaption}[2]{%
%    \renewcommand{\baselinestretch}{0.9}
%    \captionfont%
%    \sbox{\as@captionbox}{{\captionlabelfont #1:} #2}%
%    \ifdim \wd\as@captionbox > .8\vsize
%      \rotatebox{90}{%
%        \setlength{\as@captionwidth}{.8\textheight}%
%        \begin{minipage}{\as@captionwidth}%
%          \as@caption{{\captionlabelfont #1:}}{#2}%
%        \end{minipage}}\par
%    \else%
%      \rotatebox{90}{\usebox{\as@captionbox}}%
%    \fi
%    \hspace{12pt}}
%\fi
%\ifx\floatc@plain\undefined
%\else
%  \typeout{\space\space\space\space\space\space\space\space\space
%           `float' package detected}
%  \renewcommand\floatc@plain[2]{%
%    \setlength{\as@captionwidth}{\linewidth}%
%    \as@makecaption{#1}{#2}}
%  \ifx\as@ruled\undefined
%  \else
%    \renewcommand\floatc@ruled[2]{%
%      \setlength{\as@captionwidth}{\linewidth}%
%      \renewcommand{\baselinestretch}{0.9}
%      \captionfont%
%      \as@caption{{\captionlabelfont #1:}}{#2}}
%  \fi
%\fi

%%
%% Additional new commands.
%%

%% Class and data type formatting macros.

%\newcommand{\class}[1]{\figureFont{#1}}         % only SBML UML classes
%\newcommand{\abstractclass}[1]{\textsl{\figureFont{#1}}} % abstract classes
%\newcommand{\primtype}[1]{\literalFont{#1}}     % double, SId, SBOTerm, etc.
%\newcommand{\primtypeNC}[1]{\literalFontNC{#1}} % same, but without coloring
%\newcommand{\token}[1]{\literalFont{#1}}        % everything else
%\newcommand{\tokenNC}[1]{\literalFontNC{#1}}    % everything else, no color
%\newcommand{\val}[1]{``\token{#1}''}            % a value in XML
\newcommand{\uri}[1]{``\token{#1}''}

%% Macros used for defining validation rules.
%% First, preliminary internal macros.

\newcommand{\vSymbol}{\textcolor{red}{$\Box\mkern-14mu\checkmark$}}
\newcommand{\vSymbolName}{checked box\xspace}
\newcommand{\cSymbol}{\textcolor{Goldenrod}{\scalefont{0.9}\ding{115}}}
\newcommand{\cSymbolName}{triangle\xspace}
\newcommand{\mSymbol}{\textcolor{Green}{$\bigstar$}}
\newcommand{\mSymbolName}{star\xspace}

%% Spacing adjustments when using the symbols inside regular 'description' env.
%% Note: *not* used in \sbmlrule below.

\newcommand{\vsp}{\hspace*{4pt}}
\newcommand{\csp}{\hspace*{6.25pt}}
\newcommand{\msp}{\hspace*{4.9pt}}

%% Now the actual user commands for validation rules.

\newcommand{\sbmlrule}[3]{\begin{enumerate}[labelwidth=4em,leftmargin=6em,align=left]
    \item[\textbf{\textsf{#1}}~~#2] #3
  \end{enumerate}}

\newcommand{\validRule}[2]      {\sbmlrule{#1}{\vSymbol}{#2}}
\newcommand{\consistencyRule}[2]{\sbmlrule{#1}{\cSymbol}{#2}}
\newcommand{\modelingRule}[2]   {\sbmlrule{#1}{\mSymbol}{#2}}


%% Code examples.

%\newcommand{\code}[1]{\literalFont{#1}}
%\newcommand{\codeNC}[1]{\literalFontNC{#1}}

%\lstset{escapechar=|,%
%    language=XML,%
%    columns=fullflexible,%
%    keepspaces=true,%
%    backgroundcolor=\color{veryverylightgray},%
%    rulecolor=\color{sbmlgray},%
%    frame=single,%
%    numberstyle=\tiny\sffamily,%
%    basicstyle=\small\ttfamily\color{black},%
%    keywordstyle=\ttfamily,%
%    xleftmargin=3.5pt,%
%    xrightmargin=3.5pt,%
%    aboveskip=1\baselineskip,%
%    belowskip=0.4\baselineskip,%
%    commentstyle=\footnotesize\itshape\color{mediumgray},%
%    tabsize=2,%
%    captionpos=b}

%\lstdefinestyle{XML}{%
%    language=XML}

%\lstdefinestyle{bash}{%
%    language=bash}%

%\newcommand{\examplespacing}{\renewcommand{\baselinestretch}{0.88}}

%\lstnewenvironment{example}[1][]
%  {\lstset{#1}\examplespacing \csname lst@SetFirstLabel\endcsname}
%  {\regularspacing \csname lst@SaveFirstLabel\endcsname}

%\newcommand{\exampleFile}[2][style=XML]{%
%  \lstset{#1}\examplespacing\lstinputlisting{#2}\regularspacing}


%% Margin note commands.
%% The \reversemarginpar is to put notes in the left margin.

%\reversemarginpar  % Want these be put on the left, not the right.

\newcommand{\notice}{\marginpar{\hspace*{34pt}\raisebox{-0.5ex}{\color{black}\Large\ding{43}}}}
%\newcommand{\warning}{\marginpar{\hspace*{34pt}{\color{red}\large\danger}}}

%% Margin notes.
%%
%% Todonotes is nice and offers things like a list of notes, but it uses TikZ,
%% and that's a heavy package to load every time.  I got annoyed by how much
%% it slowed down my latex runs that I created a simple note alternative
%% instead. Here's the original todonotes version, for posterity:
%% 
%% \RequirePackage[textsize=scriptsize]{todonotes}
%%
%% \if@draftspec
%%   \newcommand{\sbmlshownotes}{}
%% \else
%%   \newcommand{\sbmlshownotes}{disable}
%% \fi
%%
%% \newcommand{\draftnote}[1]{\todo[backgroundcolor=lightyellow,%
%%  bordercolor=lightgray,linecolor=mediumgray,\sbmlshownotes]{\textit{#1}}}

%\newcommand{\draftnoteInternal}[1]{\mbox{}\marginpar{\hspace{0pt}\fcolorbox{Gold}{lightyellow}
%    {\scriptsize\begin{minipage}[t]{0.65in}\raggedright\textit{#1}\end{minipage}}}}

%\if@draftspec
%  \newcommand{\draftnote}[1]{\draftnoteInternal{#1}}
%\else
%  \newcommand{\draftnote}[1]{}
%\fi


%% Line numbered environments.

%\newenvironment{larray}{%
%  \begin{linenomath}\setlength{\arraycolsep}{1.5pt}\begin{eqnarray}}%
%  {\end{eqnarray}\end{linenomath}\par\vspace*{-0.7em}}

%\newenvironment{larray*}{%
%  \begin{linenomath}\setlength{\arraycolsep}{1.5pt}\begin{eqnarray*}}%
%  {\end{eqnarray*}\end{linenomath}\par\vspace*{-0.7em}}


%% Cross-references.
%%
%% We load the varioref package and define a set of macros for referring to
%% floats and sections in a consistent way, such that the entire reference
%% ("Section X", "Figure Y", etc.) is made into a hyperlink -- not only "X"
%% and "Y", as would be the case with using the normal approach of writing
%% "Section~\vref{...}".  Since all figure, table and section number
%% references are always capitalized (they are the proper names of the
%% objects, after all), the problem of creating a macro is simplified.
%% SBMLPkgSpec defines the following for this purpose:
%%
%%   \fig{...}
%%   \tab{...}
%%   \sec{...}
%%
%% In addition, it defines starred versions of the above:
%%
%%   \fig*{...}
%%   \tab*{...}
%%   \sec*{...}
%%
%% The starred versions are useful when you have two or more references in
%% the same paragraph to a float or section that is located elsewhere in the
%% document, to avoid having text like the following happen:
%%
%%   "In Figure 2 on the next page, blah blah blah.  Figure 2 on the
%%   following page also blah blah blah."
%%
%% Using the unstarred version for the first reference and starred version
%% for subsequent references avoids this, and will produce text like this:
%% 
%%   "In Figure 2 on the next page, blah blah blah.  Figure 2 also blah
%%   blah blah."
%%
%% Note: this overrides the definition of \sec{...} from the AMS math
%% package. I decided this is likely to be acceptable to users because
%% in SBML's domain of activity, it is very unlikely that someone would
%% write a mathematical formula using the \sec command in a latex document
%% specifying an SBML Level 3 package.
%%
%% As an alternative, the following plain varioref macros will work:
%%
%% * Use \vref{...} for everything and it will insert Figure~X, Table~Y,
%%   Section~Z, etc., as appropriate for the reference.  You do not need to
%%   add the object name; in other words, don't write "Section~\vref{foo}",
%%   write just "\vref{foo}".
%%
%% * Use \vref*{...} if the reference is in parentheses, like "(\vref*{...})".
%%
%% * At the beginning of a paragraph (and ONLY there), use \Vref*{...}.
%%
%% * For a range, use \vrefrange{startlabel}{endlabel}.
%%
%% To see actual examples of using this, grep for "vref" in *.tex in the
%% JSBML User Guide directory.

%\labelformat{chapter}{Chapter~#1}
%\labelformat{table}{Table~#1}
%\labelformat{figure}{Figure~#1}

%\newcommand{\sectionlabel}{Section}
%\newcommand{\setlabel}[1]{\renewcommand{\sectionlabel}{#1}}
%\newcommand{\normallabels}{\setlabel{Section}}
%\newcommand{\appendixlabels}{\setlabel{Appendix}}

%\labelformat{section}{\sectionlabel~#1}
%\labelformat{subsection}{\sectionlabel~#1}
%\labelformat{subsubsection}{\sectionlabel~#1}
%\labelformat{paragraph}{\sectionlabel~#1}

%\newcommand{\@commonRefNoStar}[1]{\xspace\Vref*{#1}\xspace}
%\newcommand{\@commonRefStar}[1]{\ref{#1}\xspace}

%\newcommand{\fig}[1]{\normallabels\@commonRefNoStar{#1}}
%\WithSuffix\newcommand\fig*[1]{\normallabels\@commonRefStar{#1}}

%\newcommand{\tab}[1]{\normallabels\@commonRefNoStar{#1}}
%\WithSuffix\newcommand\tab*[1]{\normallabels\@commonRefStar{#1}}

%\renewcommand{\sec}[1]{\normallabels\@commonRefNoStar{#1}}
%\WithSuffix\newcommand\sec*[1]{\normallabels\@commonRefStar{#1}}

%\newcommand{\sect}[1]{\normallabels\@commonRefNoStar{#1}}
%\WithSuffix\newcommand\sect*[1]{\normallabels\@commonRefStar{#1}}

%\newcommand{\apdx}[1]{\appendixlabels\@commonRefNoStar{#1}\normallabels}
%\WithSuffix\newcommand\apdx*[1]{\appendixlabels\@commonRefStar{#1}\normallabels}

%\apptocmd{\appendix}{\appendixlabels}{}{}


%% Misc.

%\newcommand{\dblquote}    {\verb|"|} %"



%% 
%% Final settings before the body begins.
%%

%\raggedbottom
%\color{sbmlnormaltextcolor}                % Default body text color.

%%% -----------------------------------------------------------------------------
%%% End of file `cekarticle.cls'.
%%% -----------------------------------------------------------------------------
