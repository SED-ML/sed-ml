\label{sec:listOfModels}
In order to specify a simulation experiment, the participating models have to be defined. SED-ML uses the XML \concept{listOfModels} element as a container for all necessary models (see Figure \fig{listOfModels}. The listOfModels is optional and may contain zero to many models. 

% Fig: sed model
\sedfig{listOfModels}{The SED-ML listOfModels container}{fig:listOfModels}
%

An XML code snippet for the \code{listOfModels} element is shown in listing \ref{lst:listOfModels}.
%
\begin{myXmlLst}{SED-ML listOfModels element}{lst:listOfModels}
<listOfModels>
 <model id="m0001" type="SBML" source="urn:miriam:biomodels.db:BIOMD0000000012">
 [MODEL PRE-PROCESSING]
 </model>
 <model id="m0002" type="SBML" source="m0001">
 [MODEL PRE-PROCESSING]
 </model>
 <model id="m0003" type="CellML" source="http://www.cellml.org/models/leloup_gonze_goldbeter_1999_version02">
 [MODEL PRE-PROCESSING]
 </model>
</listOfModels>
\end{myXmlLst}
%
The above \code{listOfModels} references three models: The first model (\code{m0001}) is the Repressilator model taken from \biom. The model itself is available from \url{urn:miriam:biomodels.db:BIOMD0000000012}. For the SED-ML simulation, the model might undergo pre-processings, described in the \hyperref[class:change]{Change} class.
Based on the description of the first model \code{m0001}, the second model is build. It refers to the model that was originally the \url{urn:miriam:biomodels.db:BIOMD0000000012} model, but had changes applied to it. \code{m0002} might then have even further changes applied to it on top of the changes defined in the pre-processing of \code{m0001}.
The third model in the code example above is a different model in CellML representation. \code{m0003} is the model available from \url{urn:miriam:biomodels.db:BIOMD0000000012}, and might have additional pre-processing applied to it before used in the simulation.
Such pre-processings might include a (new) parametrisation of model constituents, or a model update in terms of revised reaction equations, as well as the substitution of whole model constituents. Further details will be given in the description of the \hyperref[class:change]{Change} class.

A SED-ML description can be a sole storage container for a \emph{general} simulation setting, comparible to an experiment procudure description. In that case, no particular model needs to be related to that description to store the settings for later use in SED-ML format:
%
\begin{myXmlLst}{}{}
<sedML>
 <listOfModels />
 [SIMULATION SETTINGS FOLLOWING]
</sedML>
\end{myXmlLst}
%




%%% Local Variables: 
%%% mode: plain-tex
%%% TeX-master: "../sed-ml-L1V1"
%%% End: 
