%%%%%%%%%%%%%%%%%%%%%%%%%%%%%%%%%%%%%
%% Master file SED-ML specification 
%%%%%%%%%%%%%%%%%%%%%%%%%%%%%%%%%%%%%
\listfiles
\documentclass[pdftex,rgb,dvipsnames,svgnames,hyperref,table]{report}
\usepackage{tocvsec2} 

% layout (initialy done for the SBGN project)
\input{sources/styles/latex-style}

% packages and commands
%%%%%%%%%%%%%%%%%%%%%%%%%%%%%%%%%%%%%%%%%%%%%%%%%%%%%%%%%%%%%%%%%%
%%  Commands
%%%%%%%%%%%%%%%%%%%%%%%%%%%%%%%%%%%%%%%%%%%%%%%%%%%%%%%%%%%%%%%%%%

\newcommand{\code}[1]{\texttt{#1}}
\newcommand{\token}[1]{\texttt{#1}}
\newcommand{\concept}[1]{\textcolor{blue}{#1}}
\newcommand{\element}[1]{\texttt{#1}}
\newcommand{\alert}[1]{\textcolor{red}{#1}}
\newcommand{\note}[1]{\paragraph*{} \emph{\scshape{\alert{Please Note}}: #1} \newline}
\newcommand{\mailto}[1]   {\link{mailto:#1}{#1}}
\newcommand{\link}[2]     {\literalFont{\href{#1}{#2}}}
\newcommand{\literalFont}[1]{\textup{\texttt{#1}}}
\newcommand{\version}{2\xspace}
\newcommand{\level}{1\xspace}
\newcommand{\LoneVone}{Level~1 Version~1\xspace}
\newcommand{\LoneVtwo}{Level~1 Version~2\xspace}
\newcommand{\LoneVthree}{Level~1 Version~3\xspace}
\newcommand{\currentLV}{Level~1 Version~3\xspace}
\newcommand{\previousLV}{Level~1 Version~2\xspace}
\newcommand{\biom}{BioModels Database\xspace}
% attribute table layout
\newcommand{\attribute}{attribute\xspace}
\newcommand{\desc}{description\xspace}
\newcommand{\subelements}{sub-elements\xspace}

\newcommand{\SedModel}{\hyperref[class:model]{Model}\xspace}
\newcommand{\SedDataSource}{\hyperref[class:dataSource]{DataSource}\xspace}
\newcommand{\SedDataDescription}{\hyperref[class:dataDescription]{DataDescription}\xspace}
\newcommand{\SedSlice}{\hyperref[class:slice]{Slice}\xspace}

\newcommand{\refpage}[1]{\hyperref[#1]{page \pageref{#1}}} % to hyperref to a particular page in the spec
\newcommand{\tabcap}[1]{  % to create table captions for overview tables for each SED-ML class
Attributes and nested elements for \concept{#1}. \emph{xy$^{o}$} denotes optional elements and attributes.
}

\newcommand{\tabtext}[2]{ % to create the introducing table text for each table reference
\tab{#1}~shows all attributes and sub-elements for the \concept{#2} element as defined by the SED-ML \currentLV XML Schema.
}

\newcommand{\lsttext}[2]{ % to create the introducing listing text for each listing reference
  Listing~\vref{lst:#1} shows the use of the \element{#2} element in a SED-ML file as defined by the SED-ML \currentLV XML Schema.
}

\newcommand{\lsttexta}[2]{ % to create the introducing listing text for each listing reference
  Listing~\vref{lst:#1} shows the use of the \element{#2} attribute in a SED-ML file as defined by the SED-ML \currentLV XML Schema.
}

%
\newcommand{\chap}[1]     {Chapter~\protect\ref{chap:#1}\xspace}
\newcommand{\sect}[1]     {Section~\protect\ref{sec:#1}\xspace}
\newcommand{\fig}[1]      {Figure~\protect\vref{fig:#1}\xspace}
\newcommand{\tab}[1]      {Table~\protect\vref{tab:#1}\xspace}
\newcommand{\lst}[1]      {Listing~\protect\ref{lst:#1}\xspace}
\newcommand{\eg}          {e.\,g.,\xspace}
\newcommand{\ie}          {i.\,e.,\xspace}

\newcommand{\tickYes}{\hspace{1pt}\ding{52}}
\newcommand{\tickNo}{\hspace{1pt}\ding{56}}

%%%%%%%%%%%%%%%%%%%%%%%%%%%%%%%%%%%%%%%%%%%%%%%%%%%%%%%%%%%%%%%%%%
%%  environments
%%%%%%%%%%%%%%%%%%%%%%%%%%%%%%%%%%%%%%%%%%%%%%%%%%%%%%%%%%%%%%%%%%

% standard figure layout
\newcommand{\sedfig}[4][]
	{\begin{figure}[H]\begin{center}{\includegraphics[width=0.9\textwidth,#1]{#2}}\caption{#3}\label{#4}\end{center}\end{figure}}

\newcommand{\sedfigX}[4][]
	{\begin{figure}[H]\begin{center}{\includegraphics[#1]{#2}}\caption{#3}\label{#4}\end{center}\end{figure}}

% standard XML listing layout
\lstnewenvironment{myXmlLst}[2]
	{\lstset{basicstyle=\ttfamily\scriptsize, caption={#1},label={#2}, keywordstyle=\color{blue}\bfseries, stringstyle=\color{blue}, commentstyle=\color{red}, captionpos=b, breaklines=true, xleftmargin=1.5em, xrightmargin=1.5em, numbers=left, numberstyle=\ttfamily\tiny, numbersep=5pt, tabsize=4, showstringspaces=false, language=XML}} %, float=!h
	{}

% listings in appendixes:
\newcommand{\myXmlImport}[3]	{\lstinputlisting[basicstyle=\ttfamily\scriptsize,caption={#1},label={#2},%
	keywordstyle=\color{blue}\bfseries, stringstyle=\color{blue}, commentstyle=\color{red}, captionpos=b, breaklines=true, xleftmargin=1.5em, xrightmargin=1.5em, numbers=left, numberstyle=\ttfamily\tiny, numbersep=5pt, tabsize=4, showstringspaces=false, language=XML, stepnumber=1]{#3}} %float=h!

%%% Local Variables: 
%%% mode: latex
%%% TeX-master: "../sed-ml-L1V3"
%%% End: 




\begin{document}
\input{sources/A_frontpage}
\tableofcontents
\newpage
% ~~~~~~~~~~~~~~~~~~~~~~~~~~~~~~~~~~~~~~~~
%% INTRODUCTION
% ~~~~~~~~~~~~~~~~~~~~~~~~~~~~~~~~~~~~~~~~
% ~~~~~~~~~~~~~~~~~~~~~~~~~~~~~~~~~~~~~~~~
%% INTRODUCTION
% ~~~~~~~~~~~~~~~~~~~~~~~~~~~~~~~~~~~~~~~~
\chapter{Introduction}
The Simulation Experiment Description Markup Language (SED-ML) is an XML-based format for the description of simulation experiments.

The number of available computational models of biological systems is growing at an ever increasing pace. 
At the same time, their size and complexity are also increasing. The need to build on existing studies by reusing models therefore becomes more imperative. It is now generally accepted that one needs to be able to exchange the biochemical and mathematical structure of models. The efforts to standardise the representation of computational models in various areas of biology, such as the Systems Biology Markup Language (SBML \citep{Hucka:2003}), CellML \citep{cuellar:2003} or NeuroML \citep{Goddard:2001}, resulted in such an increase of the exchange and re-use of models. 

However, the description of the structure of models is not sufficient for the reproduction of simulation results. One also needs to describe the procedures the models are subjected to, as described by the Minimum Information About a Simulation Experiment (MIASE) \citep{Waltemath:2011} which proposes a minimal set of information that should be provided to allow the reproduction of simulation experiments among users and software tools. The increasing use of computational simulation experiments to inform modern biological research creates new challenges to annotate, archive, share and reproduce such experiments. 

SED-ML encodes in a computer-readable exchange format the information required by MIASE to enable reproduction of simulation experiments. It has been developed as a community project and it is defined in this detailed technical specification and in the corresponding XML schema. SED-ML files are encoded in the Extensible Markup Language (XML) \citep{Bray:2006} with the SED-ML format being defined by an XML Schema \citep{Fallside:2001}. 

SED-ML descriptions are independent of the underlying model implementation. SED-ML is a software-independent format for encoding the description of simulation experiments; it is not specific to particular simulation tools.

This document presents \currentLV of SED-ML which is the successor of \previousLV and \LoneVone (described in \citep{WAB+11}).

% ~~~~~~~~~~~~~~~~~~~~~~~~~~~~~~~~~~~~~~~~
%% OVERVIEW
% ~~~~~~~~~~~~~~~~~~~~~~~~~~~~~~~~~~~~~~~~
\section{SED-ML overview}
A SED-ML document specifies for a given simulation experiment

\begin{itemize}
\item what datasets to use
\item which models to use in an simulation experiment
\item which modifications to apply to models before simulation,
\item which simulation procedures to run on each model,
\item what analysis results to plot or report,
\item and how these results should be presented
\end{itemize}

SED-ML containts the following main objects to describe this information: \concept{DataDescription} (added in \LoneVthree), \concept{Model}, \concept{Simulation}, \concept{Task}, the \concept{DataGenerator}, and \concept{Output}.

\paragraph*{\concept{DataDescription}}
\hl{TODO: fill in description}

\paragraph*{\concept{Model}}
The Model class is used to reference the models used in the simulation experiment. SED-ML itself is independent of the model encoding underlying the models. The only requirement is that the model needs to be referenced by using an unambiguous identifier which allows for finding it, for example using a MIRIAM URI. To specify the language in which the model is encoded, a set of predefined language URNs is provided.

The SED-ML \concept{Change} class allows the application of changes to the referenced models, including changes on the XML attributes, e.g. changing the value of an observable, computing the change of a value using mathematics, or general changes on any XML element of the model representation that is addressable by XPath expressions, e.g. substituting a piece of XML by an updated one.

\paragraph*{\concept{Simulation}}
The Simulation class defines the simulation settings and the steps taken during simulation. These include the particular type of simulation and the algorithm used for the execution of the simulation; preferably an unambiguous reference to such an algorithm should be given, using a controlled vocabulary, or ontologies. One example for an ontology of simulation algorithms is the Kinetic Simulation Algorithm Ontology KiSAO. Further information encodable in the Simulation class includes the step size, simulation duration, and other simulation-type dependent information.

\paragraph*{\concept{Task}}
SED-ML makes use of the notion of a Task class to combine a defined model (from the Model class) and a defined simulation setting (from the Simulation class). A task always holds one reference each. To refer to a specific model and to a specific simulation, the corresponding IDs are used.

\paragraph*{\concept{DataGenerator}}
The raw simulation result sometimes does not correspond to the desired output of the simulation, e.g. one might want to normalise a plot before output, or apply post-processing like mean-value calculation. The DataGenerator class allows for the encoding of such post-processings which need to be applied to the simulation result before output. To define data generators, any addressable variable or parameter of any defined model (from instances of the Model class) may be referenced, and new entities might be specified using MathML definitions.

\paragraph*{\concept{Output}}
The Output class defines the output of the simulation, in the sense that it specifies what shall be plotted in the output. To do so, an output type is defined, e.g. 2D-plot, 3D-plot or data table, and the according axes or columns are all assigned to one of the formerly specified instances of the DataGenerator class.

This section provided a high level overview over the content of a SED-ML file. For the detailed technical specification see \hl{??}. 



% ~~~~~~~~~~~~~~~~~~~~~~~~~~~~~~~~~~~~~~~~
%% EXAMPLE SIMULATION
% ~~~~~~~~~~~~~~~~~~~~~~~~~~~~~~~~~~~~~~~~
\section{A simulation experiment}
\label{motivation:example}

\hl{TODO MK: add sample experiment to examples in appendix}
\hl{TODO MK: create simulation results with tellurium and Copasi to demonstrate reproducibility}

The \emph{repressilator} \citep{Elowitz:2000} is a rather small, though famous, model that is capable of displaying rich and variable behaviors. We will use this model to demonstrate how simulation experiments with this model can be described with SED-ML. 

The \emph{repressilator} is a synthetic oscillating network of transcription regulators in Escherichia coli. The network is composed of the three repressor genes Lactose Operon Repressor (lacI), Tetracycline Repressor (tetR) and Repressor CI (cI), which code for proteins binding to the promoter of the other, blocking their transcription. The three inhibitions together in tandem, form a cyclic negative-feedback loop. To describe the interactions of the molecular species involved in the network, the authors built a simple mathematical model of coupled first-order differential equations. All six molecular species included in the network (three mRNAs, three repressor proteins) participated in creation (transcription/translation) and degradation processes. The model was used to determine the influence of the various parameters on the dynamic behavior of the system. In particular, parameter values were sought which induce stable oscillations in the concentrations of the system components. Oscillations in the levels of the three repressor proteins are obtained by numerical integration. 


%% ~~ TIMECOURSE ~~
\subsection{A time-course simulation}
\label{sec:timecourse}
The first simulation experiment we run with the model reproduces the oscillation behavior of the model shown in Figure~1c of the reference publication \citep{Elowitz:2000}. This simulation experiment can be described as:

\begin{enumerate}
 	\item{Import the model identified by the Unified Resource Identifier (URI) \citep{Berners-Lee:2005}\\ 	\url{urn:miriam:biomodels.db:BIOMD0000000012}.}
 	\item {Select a deterministic simulation method.}
 	\item{Run a uniform time course simulation for 1000~min with an output interval of 1~min.}
 	\item{Plot the amount of \code{lacI}, \code{tetR} and \code{cI} against time in a 2D Plot.}
 \end{enumerate}

Following those steps and performing the simulation in the simulation tools supporting SED-ML results in the output depicted in \fig{rep_tc}. \hl{TODO: legend and xaxis label missing, rerun with tellurium and SED-ML webtools.}

\begin{figure}
\centering
\includegraphics[width=0.6\textwidth]{examples/rep_tc.png}
\caption{Time-course simulation of the repressilator model, imported from BioModels Database and simulated in COPASI. The number of repressor proteins lacI, tetR and cI is depicted.}
\label{fig:rep_tc}
\end{figure}


%% ~~ PRE-PROCESSING ~~
\subsection{Applying pre-processing}
\label{sec:preprocessing}
The fine-tuning of the model can be shown by adjusting parameters before simulation. When changing the initial values of the parameters \emph{protein copies per promoter} and \emph{leakiness in protein copies per promoter} the system's behavior switches from sustained oscillation to asymptotic steady-state. The adjustments leading to that behavior may be described as: 

\begin{enumerate}
\item{Import the model as above.}
\item{Change the value of the parameter \code{tps$\_$repr} from “0.0005” to “1.3e-05”.}
\item{Change the value of the parameter \code{tps$\_$active} from “0.5 “ to “ 0.013“.}
\item{Select a deterministic method.}
\item{Run a uniform time course for the duration of 1000~min with an output interval of 1~min.}
\item Plot the amount of lacI, tetR and cI against time in a 2D Plot.
\end{enumerate}

\fig{rep_pre} shows the result of the simulation.

\begin{figure}
\centering
\includegraphics[width=0.7\textwidth]{examples/rep_pre.png}
\caption{Time-course simulation of the repressilator model, imported from BioModels Database and simulated in COPASI after modification of the initial values of the \emph{protein copies per promoter} and the \emph{leakiness in protein copies per promoter}. The number of repressor proteins lacI, tetR and cI is shown}
\label{fig:rep_pre}
\end{figure}

%% ~~ POST-PROCESSING ~~
\subsection{Applying post-processing}
\label{sec:postprocessing}
The raw numerical output of the simulation steps may be subjected to data post-processing before plotting or reporting.  In order to describe the production of a normalized plot of the time-course in the first example (section \ref{sec:intro1}), depicting the influence of one variable on another (in phase-planes), one could define the following further steps:

(Please note that the description steps 1 - 4 remain as given in section \ref{sec:timecourse} above.)
\begin{enumerate}
\item[5.]{Collect lacI(t) , tetR(t) and cI(t).}
\item[6.]{Compute the highest value for each of the repressor proteins,  max(lacI(t)), max(tetR(t)), max(cI(t)).}
\item[7.]{Normalize the data for each of the repressor proteins by dividing each time point by the maximum value, i.\,e.\ lacI(t)/max(lacI(t) ), tetR(t)/max(tetR(t)) , and cI(t)/max(cI(t)).}
\item[8.]{Plot the normalized \code{lacI} protein as a function of the normalized \code{cI}, the normalized \code{cI}  as a function of the normalized \code{tetR} protein, and the normalized \code{tetR} protein against the normalized \code{lacI} protein in a 2D plot.}
\end{enumerate}

\fig{rep_post} illustrates the result of the simulation after post-processing of the output data. 
\begin{figure}
\centering
\includegraphics[width=0.7\textwidth]{examples/rep_post.png}
\caption{Time-course simulation of the repressilator model imported from BioModels Database and simulated with COPASI. Depicted is the normalized temporal evolution of lacI, tetR and cI in phase-plane.}
\label{fig:rep_post}
\end{figure}


%%% Local Variables: 
%%% mode: latex
%%% TeX-master: "../sed-ml-L1V3"
%%% End: 

% ~~~~~~~~~~~~~~~~~~~~~~~~~~~~~~~~~~~~~~~~
%% TECHNICAL SPECIFICATION
% ~~~~~~~~~~~~~~~~~~~~~~~~~~~~~~~~~~~~~~~~
\chapter{SED-ML technical specification}
\label{chp:specification}

This document represents the technical specification of SED-ML. We also provide an XML Schema \citep{xmls} and a UML class diagram representation of that XML Schema (Appendix~\ref{app:uml}). UML class diagrams are a subset of the \emph{Unified Markup Language} notation (UML, \citep{uml22}). Sample experiment descriptions are given as XML snippets that comply with the XML Schema.

It should however be noted that some of the concepts of SED-ML cannot be captured using XML Schema alone. In these cases it is this specification that is considered the normative document. 

%% ~~~ NOTATION CONVENTIONS ~~~
\input{sources/C1_conventions}

%% ~~~ CONCEPTS ~~~
\input{sources/C2_concepts}

%% ~~~ GENERAL LANGUAGE ELEMENTS ~~~
\input{sources/C3_generalElements}

%% ~~~ COMPONENTS ~~~
\settocdepth{subsubsection}

\section{SED-ML Components}
In this section we describe the major components of SED-ML. We use the UML notation presented in section \ref{sec:umlconventions}, and we show the use of SED-ML with XML examples. In addition, we provide an XML Schema in Appendix~\ref{sec:xmlschema}.

%% DATA
\input{sources/01_data}
%% MODEL
\input{sources/02_model}
%% SIMULATION
\input{sources/03_simulation}
%% TASK
% ~~~~~~~~~~~~~~~~~~~~~~~~~~~~~~~~~~~~~~~~
%% ABSTRACT TASK
% ~~~~~~~~~~~~~~~~~~~~~~~~~~~~~~~~~~~~~~~~
\subsection{\element{Abstract Task}}
\label{class:abstractTask}
An abstract task in SED-ML represents the base class for all SED-ML tasks. It is not meant to be instantiated directly.

\sedfig[width=0.80\textwidth]{pdf/abstractTask}{The SED-ML Abstract Task class}{fig:abstractTask}

\tabtext{abstractTask}{abstractTask}

\begin{table}[ht]
\center
\begin{tabular}{|l|l|}
\hline
\textbf{\attribute} & \textbf{\desc}\\
\hline
metaid$^{o}$ & \refpage{sec:metaID}\\
id & \refpage{sec:id} \\
name$^{o}$ & \refpage{sec:name}\\
\hline
\textbf{\subelements} & \textbf{\desc}\\
\hline
notes$^{o}$ & \refpage{class:notes}\\
annotation$^{o}$ & \refpage{class:annotation}\\
\hline
\end{tabular}
\caption{\tabcap{abstractTask}}
\label{tab:abstractTask}
\end{table}


% ~~~~~~~~~~~~~~~~~~~~~~~~~~~~~~~~~~~~~~~~
%% TASK
% ~~~~~~~~~~~~~~~~~~~~~~~~~~~~~~~~~~~~~~~~
\subsection{\element{Task}}
\label{class:task}

A task in SED-ML links a \hyperref[class:model]{model} to a certain \hyperref[class:simulation]{simulation} description via their respective identifiers (\fig{sedTask}), using the \hyperref[sec:modelReference]{modelReference} and the \hyperref[sec:simulationReference]{simulationReference}. The task class receives the id and name attributes from the \hyperref[class:abstractTask]{AbstractTask}.

\sedfig[width=0.75\textwidth]{pdf/taskClass}{The SED-ML Task class}{fig:sedTask}

In SED-ML \currentLV it is only possible to link one simulation description to one model at a time. However, one can define as many tasks as needed within one experiment description. Please note that the tasks may be executed in any order, as determined by the implementation.

\tabtext{task}{task}

\begin{table}[ht]
\center
\begin{tabular}{|l|l|}
\hline
\textbf{\attribute} & \textbf{\desc}\\
\hline
metaid$^{o}$ & \refpage{sec:metaID}\\
id & \refpage{sec:id} \\
name$^{o}$ & \refpage{sec:name}\\
\hline
modelReference & \refpage{sec:modelReference}\\
simulationReference & \refpage{sec:simulationReference}\\
\hline
\hline
\textbf{\subelements} & \textbf{\desc}\\
\hline
notes$^{o}$ & \refpage{class:notes}\\
annotation$^{o}$ & \refpage{class:annotation}\\
\hline
\end{tabular}
\caption{\tabcap{task}}
\label{tab:task}
\end{table}

\lsttext{task}{task}

\begin{myXmlLst}{The \code{task} element}{lst:task}
<listOfTasks>
  <task id="t1" name="task definition" modelReference="model1" 
        simulationReference="simulation 1" />
  <task id="t2" name="another task definition" modelReference="model2" 
        simulationReference="simulation 1" />
</listOfTasks>
\end{myXmlLst}

In the example, a simulation setting \emph{simulation1} is applied first to \emph{model1} and then is applied to \emph{model2}.

% ~~~~~~~~~~~~~~~~~~~~~~~~~~~~~~~~~~~~~~~~
%% SET VALUE
% ~~~~~~~~~~~~~~~~~~~~~~~~~~~~~~~~~~~~~~~~
\subsection{\element{SetValue}}
\label{class:setValue}
The \element{setValue} allow in a \hyperref[class:repeatedTask]{repeatedTask} the modification of values in the model prior to the next execution of the \concept{subTasks}. The changes to the model are hereby listed in the \element{listOfChanges} of the \element{repeatedTask}.

A \element{setValue} element inherits from the \hyperref[class:computeChange]{computeChange} class, which allows it to compute arbitrary expressions involving a number of variables and parameters. The element \element{setValue} adds a mandatory \code{modelReference} attribute, and two optional attributes \code{range} and \code{symbol}.

The value to be changed is identified via the combination of the attributes \code{modelReference} and either \code{symbol} or \code{target}, in order to select an implicit or explicit variable within the referenced model.

As in \hyperref[class:functionalRange]{functionalRange}, the attribute \code{range} may be used as a shorthand to specify the \code{id} of another \concept{Range}. The current value of the referenced range may then be used within the function defining this \concept{FunctionalRange}, just as if that range had been referenced using a \hyperref[class:variable]{variable} element, except that the \code{id} of the range is used directly. In other words, whenever the expression contains a \code{ci} element that contains the value specified in the \code{range} attribute, the value of the referenced range is to be inserted.

The child \element{math} contains the expression computing the value by refering to optional parameters, variables or ranges.
Again as for \hyperref[class:functionalRange]{functionalRange}, variable references always retrieve the \concept{current value} of the model variable or range at the current iteration of the enclosing \element{repeatedTask}. For a model not being simulated by any \element{subTask}, the initial state of the model is used.

\begin{myXmlLst}{A \code{setValue} element setting \code{w} to the values of the range with id \code{current}.}{lst:setValue}
  <listOfChanges>
    <setValue target="/s:sbml/s:model/s:listOfParameters/s:parameter[@id='w']"
              range="current" modelReference="model1">
      <math xmlns="http://www.w3.org/1998/Math/MathML">
        <ci> current </ci>
      </math>
    </setValue>
  </listOfChanges>
\end{myXmlLst}


% ~~~~~~~~~~~~~~~~~~~~~~~~~~~~~~~~~~~~~~~~
%% REPEATED TASK
% ~~~~~~~~~~~~~~~~~~~~~~~~~~~~~~~~~~~~~~~~
\subsection{\element{Repeated Task}}
\label{class:repeatedTask}

The \concept{repeatedTask} class provides a generic looping construct, allowing complex tasks to be represented by composing separate steps. It performs a specified task (or sequence of tasks) multiple times (where the exact number is specified through a \hyperref[sec:ranges]{range} construct), while allowing specific quantities in the model to be altered at each iteration (as defined in the \hyperref[sec:changes]{listOfChanges}).

The \concept{RepeatedTask} inherits from \concept{AbstractTask}. Additionally it has two required attributes \hyperref[sec:rangeAttribute]{range} and \hyperref[sec:resetModel]{resetModel} as well as child elements \hyperref[sec:ranges]{listOfRanges}, \hyperref[sec:changes]{listOfChanges} and \hyperref[class:subTask]{listOfSubTasks}. Of these only \hyperref[sec:changes]{listOfChanges} is optional.

Note that the order of activities within each iteration of a \concept{repeatedTask} is as follows. Firstly the model is reset, if specified by the \element{resetModel} attribute. Secondly any changes to the model specified by \element{setValue} elements are made. Finally, the \element{subTasks} are executed once each in order.

\sedfig[width=.90\textwidth]{pdf/repeatedTaskClass}{The SED-ML RepeatedTask class}{fig:sedRptTask}

\tabtext{repeatedTask}{repeatedTask}

\begin{table}[ht]
\center
\begin{tabular}{|l|l|}
\hline
\textbf{\attribute} & \textbf{\desc}\\
\hline
metaid$^{o}$ & \refpage{sec:metaID}\\
id & \refpage{sec:id} \\
name$^{o}$ & \refpage{sec:name}\\
\hline
range & \refpage{sec:rangeAttribute}\\
resetModel & \refpage{sec:resetModel}\\
\hline
\hline
\textbf{\subelements} & \textbf{\desc}\\
\hline
notes$^{o}$ & \refpage{class:notes}\\
annotation$^{o}$ & \refpage{class:annotation}\\
\hline
range & \refpage{sec:ranges}\\
change$^{o}$ & \refpage{sec:changes}\\
subTask$^{o}$ & \refpage{class:subTask}\\
\hline
\hline
\end{tabular}
\caption{\tabcap{repeatedTask}}
\label{tab:repeatedTask}
\end{table}

\lsttext{repeatedTask}{repeatedTask}

\begin{myXmlLst}{The \code{repeatedTask} element}{lst:repeatedTask}
<task id="task1" modelReference="model1" simulationReference="simulation1" />

<repeatedTask id="task3" resetModel="false" range="current"
    xmlns:s='http://www.sbml.org/sbml/level3/version1/core'>
  <listOfRanges>
    <vectorRange id="current"> 
        <value> 1 </value> 
        <value> 4 </value> 
        <value> 10 </value> 
    </vectorRange> 
  </listOfRanges>
  <listOfChanges>
     <setValue target="/s:sbml/s:model/s:listOfParameters/s:parameter[@id='w']" modelReference="model1">
       <listOfVariables> 
         <variable id="val" name="current range value" target="#current" /> 
       </listOfVariables> 
       <math xmlns="http://www.w3.org/1998/Math/MathML"> 
         <ci> val </ci> 
       </math> 
     </setValue> 
  </listOfChanges>
  <listOfSubTasks>
    <subTask task="task1" />
  </listOfSubTasks>
</repeatedTask>
\end{myXmlLst}

In the example, \code{task1} is repeated three times, each time with a different value for a model parameter \code{w}. 

%% ~~~ REPEATED TASK : RANGE ~~~
\subsubsection{The \element{range} attribute}
\label{sec:rangeAttribute}
The \element{repeatedTask} class has a required attribute \element{range} of type \code{SId}.
It specifies which \hyperref[sec:ranges]{range} defined in the \element{listOfRanges} this repeated task iterates over.
Listing~\ref{lst:repeatedTask} shows an example of a \element{repeatedTask} iterating over a single range comprising the values: \code{1}, \code{4} and \code{10}.
If there are multiple ranges in the \element{listOfRanges}, then only the \concept{master range} identified by this attribute determines how many iterations there will be in the \element{repeatedTask}.
All other ranges must allow for at least as many iterations as the master range, and will be moved through in lock-step; their values can be used in \hyperref[class:setValue]{setValue} constructs.

%% ~~~ REPEATED TASK : RESET MODEL ~~~
\subsubsection{The \element{resetModel} attribute}
\label{sec:resetModel}
The \element{repeatedTask} class has a required attribute \element{resetModel} of type \code{boolean}. It specifies whether the model should be reset to the initial state before processing an iteration of the defined \hyperref[class:subTask]{subTasks}. Here initial state refers to the state of the model as given in the \element{listOfModels}.  In the example in  Listing~\ref{lst:repeatedTask} the repeated task is not to be reset, so a change is made, \code{task1} is carried out, another change is made, then \code{task1} continues from there, another change is applied, and \code{task1} is carried out a last time.

%% ~~~ REPEATED TASK : LIST OF RANGES ~~~
\subsubsection{The \element{listOfRanges}}
\label{sec:ranges}
Ranges represent the iterative element of the nested simulation experiment that provides the collection of values to iterate over. In order to be able to refer to the current value of a range element, an \code{id} attribute is added. When the value of the \code{id} attribute is used in a \element{listOfVariables} within the repeated task class its value is to be replaced with the current value of the range.

There are three different range types permitted in the \element{listOfRanges}: \hyperref[class:uniformRange]{UniformRange}, \hyperref[class:vectorRange]{VectorRange} and \hyperref[class:functionalRange]{FunctionalRange}.
They each inherit from an abstract \hyperref[class:range]{Range} class.

\paragraph{\element{Range}}
\label{class:range}
The \concept{Range} class is abstract and exists solely as the base class for the different types of range. Therefore, a SED-ML document will only contain the derived classes listed below.

\paragraph{\element{UniformRange}}
\label{class:uniformRange}

\sedfig[width=.3\textwidth]{pdf/uniformRange}{The SED-ML UniformRange class}{fig:sedUniformRange}

The \element{UniformRange} is quite similar to what is used in the \hyperref[class:uniformTimeCourse]{UniformTimeCourse} simulation class.
This range is defined through four mandatory attributes: \code{start}, the start value; \code{end}, the end value and \code{numberOfPoints} that contains the number of points the range contains.
A fourth attribute \code{type} that can take the values \code{linear} or \code{log} determines whether to draw the values logarithmically (with a base of $10$) or linearly.

For example:
\begin{myXmlLst}{The \code{UniformRange} element}{lst:uniformRange}
    <uniformRange id="current" start="0.0" end="10.0" numberOfPoints="100" type="linear" /> 
\end{myXmlLst}
As for \hyperref[class:uniformTimeCourse]{UniformTimeCourse}, this range will actually produce 101 values uniformly spaced on the interval $[0, 10]$, in ascending order.

The following logarithmic example generates the three values \code{1}, \code{10} and \code{100}.
\begin{myXmlLst}{The \code{UniformRange} element with a logarithmic range.}{lst:uniformRangeLog}
    <uniformRange id="current" start="1.0" end="100.0" numberOfPoints="2" type="log" />
\end{myXmlLst}

\paragraph{\element{VectorRange}}
\label{class:vectorRange}

\sedfig[width=.3\textwidth]{pdf/vectorRangeClass}{The SED-ML VectorRangeClass class}{fig:sedVectorRangeClass}

A \element{VectorRange} describes an ordered collection of real values, listing them explicitly within child \element{value} elements.
For example, the range below iterates over the values $1$, $4$ and $10$ in that order.
\begin{myXmlLst}{The \code{VectorRange} element}{lst:vectorRange}
    <vectorRange id="current"> 
        <value> 1 </value> 
        <value> 4 </value> 
        <value> 10 </value> 
    </vectorRange> 
\end{myXmlLst}

\paragraph{\element{FunctionalRange}}
\label{class:functionalRange}

\sedfig[width=.9\textwidth]{pdf/functionalRangeClass}{The SED-ML FunctionalRange class}{fig:sedFunctionalRangeClass}

A \element{FunctionalRange} constructs a range through calculations that determine the next value based on the value(s) of other range(s) or model variables. In this it is quite similar to the \hyperref[class:computeChange]{ComputeChange} element, and shares some of the same child elements.
It consists of an optional attribute \code{range}, two optional elements \hyperref[sec:listOfVariables]{listOfVariables} and \hyperref[sec:listOfParameters]{listOfParameters}, and a required element \element{math}.

The optional attribute \code{range} may be used as a shorthand to specify the \code{id} of another \concept{Range}. The current value of the referenced range may then be used within the function defining this \concept{FunctionalRange}, just as if that range had been referenced using a \hyperref[class:variable]{variable} element, except that the \code{id} of the range is used directly.
In other words, whenever the expression contains a \code{ci} element that contains the value specified in the \code{range} attribute, the value of the referenced range is to be inserted.

In the \element{listOfVariables}, \hyperref[class:variable]{variable} elements define identifiers refering to model variables or range values, which may then be used within the \element{math} expression.
These references always retrieve the \concept{current value} of the model variable or range at the current iteration of the enclosing \element{repeatedTask}.
For a model not being simulated by any \element{subTask}, the initial state of the model is used.

The \element{function} encompasses the mathematical expression that is used to compute the value for the functional range at each iteration of the enclosing \element{repeatedTask}.

For example:

\begin{myXmlLst}{An example of a \code{functionalRange} where a parameter \code{w} of model \code{model2} is multiplied by \code{index} each time it is called.}{lst:functionalRange}
  <functionalRange id="current" range="index"
      xmlns:s='http://www.sbml.org/sbml/level3/version1/core'>
    <listOfVariables>
      <variable id="w" name="current parameter value" modelReference="model2"
          target="/s:sbml/s:model/s:listOfParameters/s:parameter[@id='w']" />
    </listOfVariables>
    <math xmlns="http://www.w3.org/1998/Math/MathML">
      <apply>
        <times/>
        <ci> w </ci>
        <ci> index </ci>
      </apply>
    </math>
  </functionalRange>
\end{myXmlLst}

Here is another example, this time using the values in a piecewise expression: 

\begin{myXmlLst}{A \code{functionalRange} element that returns \code{8} if \code{index} is smaller than \code{1}, \code{0.1} if \code{index} is between \code{4} and \code{6}, and \code{8} otherwise.}{lst:functionalRange2}
        <uniformRange id="index" start="0" end="10" numberOfPoints="100" />
        <functionalRange id="current" range="index">
          <math xmlns="http://www.w3.org/1998/Math/MathML">
            <piecewise>
              <piece>
                <cn> 8 </cn>
                <apply>
                  <lt />
                  <ci> index </ci>
                  <cn> 1 </cn>
                </apply>
              </piece>
              <piece>
                <cn> 0.1 </cn>
                <apply>
                  <and />
                  <apply>
                    <geq />
                    <ci> index </ci>
                    <cn> 4 </cn>
                  </apply>
                  <apply>
                    <lt />
                    <ci> index </ci>
                    <cn> 6 </cn>
                  </apply>
                </apply>
              </piece>
              <otherwise>
                <cn> 8 </cn>
              </otherwise>
            </piecewise>
          </math>
        </functionalRange>
\end{myXmlLst}


%% ~~~ REPEATED TASK : LIST OF CHANGES ~~~
\subsubsection{The \element{listOfChanges}}
\label{sec:changes}
The \element{listOfChanges} element, when present, contains one or more \element{setValue} elements. These elements allow the modification of values in the model prior to the next execution of the \concept{subTasks}.

%% ~~~ REPEATED TASK : LIST OF SUBTASKS ~~~
\subsubsection{The \element{listOfSubTasks}}
\label{class:subTask}

The \element{listOfSubTasks} contains one or more \element{subTask} elements that specify what simulations are to be performed by the \element{RepeatedTask}.
All \element{subTask}s have to be carried out sequentially, each continuing from the current model state (i.e.\ as at the end of the previous \code{subTask}, assuming it simulates the same model), and with their results concatenated (thus appearing identical to a single complex simulation).
The \code{subTask} itself has one required attribute \code{task} that references the \code{id} of another task defined in the \code{listOfTasks}.
The order in which to run multiple \code{subTask}s should be specified using \code{order} attributes on the \code{subTask} elements; if these are omitted the ordering is given by the order of the subTask elements.
In order to establish that one \code{subTask} should be carried out before another its \code{order} attribute has to have a lower number (c.f.\ Listing~\ref{lst:subTask}).

\begin{myXmlLst}{The \code{subTask} element. In this example the task \code{task2} has to be carried out before \code{task1}.}{lst:subTask}
  <listOfSubTasks>
    <subTask task="task1" order="2"/> 
    <subTask task="task2" order="1"/> 
  </listOfSubTasks>
\end{myXmlLst}

%%% Local Variables: 
%%% mode: latex
%%% TeX-master: "../sed-ml-L1V3"
%%% End: 
%% DATA GENERATOR
\input{sources/05_dataGenerator}
%% OUTPUT
% ~~~~~~~~~~~~~~~~~~~~~~~~~~~~~~~~~~~~~~~~
%% OUTPUT
% ~~~~~~~~~~~~~~~~~~~~~~~~~~~~~~~~~~~~~~~~ 
\subsection{\element{Output}}
\label{class:output}

The \concept{Ouput} class describes how the results of a simulation should be presented to the user (\fig{sedOutput}). 

\sedfig{pdf/outputClass}{The SED-ML Output class}{fig:sedOutput}

It does not contain the data itself, but the type of output and the \hyperref[class:dataGenerator]{data generators} used to produce a particular output.

The types of output pre-defined in SED-ML \currentLV are plots and \hyperref[class:report]{reports}. The output can be defined as a \hyperref[class:plot2D]{2D plot} or alternatively as a \hyperref[class:plot3D]{3D plot}. 

Note that even though the terms ``2D plot" and ``3D plot" are used, the exact type of plot is not specified. In other words, whether the 3D plot represents a surface plot, or three dimensional lines in space, cannot be distinguished by SED-ML alone. It is expected that applications use \hyperref[class:annotation]{annotations} for this purpose.

\tabtext{output}{output}
%
\begin{table}[ht]
\center
\begin{tabular}{|l|l|}
\hline
\textbf{\attribute} & \textbf{description}\\
\hline
metaid$^{o}$ & \refpage{sec:metaID}\\
id & \refpage{sec:id} \\
name$^{o}$ & \refpage{sec:name}\\
\hline
\hline
\textbf{sub-elements} & \textbf{description}\\
\hline
notes$^{o}$ & \refpage{class:notes}\\
annotation$^{o}$ & \refpage{class:annotation}\\
\hline
plot2D$^{o}$ & \refpage{class:plot2D}\\
plot3D$^{o}$ & \refpage{class:plot3D}\\
report$^{o}$ & \refpage{class:report}\\
\hline
\end{tabular}
\caption{\tabcap{output}}
\label{tab:output}
\end{table}


%% ~~~ PLOT2D ~~~
\subsubsection{\element{Plot2D}}
\label{class:plot2D}
A \concept{2 dimensional plot} (\fig{plot2D}) contains a number of \hyperref[class:curve]{curve} definitions. 

\sedfig[width=0.75\textwidth]{pdf/plot2DClass}{The SED-ML Plot2D class}{fig:plot2D}

\tabtext{plot2D}{plot2D}

\begin{table}[ht]
\center
\begin{tabular}{|l|l|}
\hline
\textbf{\attribute} & \textbf{\desc}\\
\hline
metaid$^{o}$ & \refpage{sec:metaID}\\
id & \refpage{sec:id} \\
name$^{o}$ & \refpage{sec:name}\\
\hline
\hline
\textbf{\subelements} & \textbf{\desc}\\
\hline
notes$^{o}$ & \refpage{class:notes}\\
annotation$^{o}$ & \refpage{class:annotation}\\
\hline
curve & \refpage{class:curve}\\
\hline
\end{tabular}
\caption{\tabcap{plot2D}}
\label{tab:plot2D}
\end{table}

\lsttext{listOfCurves}{listOfCurves}

\begin{myXmlLst}{The \code{plot2D} element with the nested \code{listOfCurves} element}{lst:listOfCurves}
<plot2D>
 <listOfCurves>
  <curve>
    [CURVE DEFINITION]
  </curve>
  [FURTHER CURVE DEFINITIONS]
 </listOfCurves>
</plot2D>
\end{myXmlLst}

The listing shows the definition of a 2 dimensional plot containing one \hyperref[class:curve]{curve} element inside the \code{listOfCurves}. The curve definition follows in Section~\ref{class:curve} on \refpage{class:curve}.


%% ~~~ PLOT3D ~~~
\subsubsection{\element{Plot3D}}
\label{class:plot3D}
A \concept{3 dimensional plot} (\fig{plot3D}) contains a number of \hyperref[class:surface]{surface} definitions.

\sedfig[width=0.75\textwidth]{pdf/plot3DClass}{The SED-ML Plot3D class}{fig:plot3D}

\tabtext{plot3D}{plot3D}

\begin{table}[ht]
\center
\begin{tabular}{|l|l|}
\hline
\textbf{\attribute} & \textbf{\desc}\\
\hline
metaid$^{o}$ & \refpage{sec:metaID}\\
id & \refpage{sec:id} \\
name$^{o}$ & \refpage{sec:name}\\
\hline
\hline
\textbf{\subelements} & \textbf{\desc}\\
\hline
notes$^{o}$ & \refpage{class:notes}\\
annotation$^{o}$ & \refpage{class:annotation}\\
\hline
surface & \refpage{class:surface}\\
\hline
\end{tabular}
\caption{\tabcap{plot3D}}
\label{tab:plot3D}
\end{table}

\lsttext{plot3D}{plot3D}

\begin{myXmlLst}{The \code{plot3D} element with the nested \code{listOfSurfaces} element}{lst:plot3D}
<plot3D>
 <listOfSurfaces>
  <surface> 
   [SURFACE DEFINITION]
  </surface>
  [FURTHER SURFACE DEFINITIONS]
 </listOfSurfaces>
</plot3D>
\end{myXmlLst}

The example defines a \hyperref[class:surface]{surface} for the 3 dimensional plot. The surface definition follows in Section~\ref{class:surface} on \refpage{class:surface}.


%% ~~~ REPORT ~~~
\subsubsection[Report]{The Report class}
\label{class:report}
The \concept{Report} class defines a data table consisting of several single instances of the \hyperref[class:dataSet]{DataSet} class (\fig{report}).
Its output returns the simulation result in actual \emph{numbers}. The particular columns of the report table are defined by creating an instance of the \hyperref[class:dataSet]{DataSet} class for each column. 

\sedfig[width=0.75\textwidth]{pdf/report}{The SED-ML Report class}{fig:report}

\tabtext{report}{report}

\begin{table}[ht]
\center
\begin{tabular}{|l|l|}
\hline
\textbf{\attribute} & \textbf{\desc}\\
\hline
metaid$^{o}$ & \refpage{sec:metaID}\\
id & \refpage{sec:id} \\
name$^{o}$ & \refpage{sec:name}\\
\hline
\hline
\textbf{\subelements} & \textbf{\desc}\\
\hline
notes$^{o}$ & \refpage{class:notes}\\
annotation$^{o}$ & \refpage{class:annotation}\\
\hline
dataSet & \refpage{class:dataSet}\\
\hline
\end{tabular}
\caption{\tabcap{report}}
\label{tab:report}
\end{table}

\lsttext{listOfDataSets}{listOfDataSets}

\begin{myXmlLst}{The \code{report} element with the nested \code{listOfDataSets} element}{lst:listOfDataSets}
<report>
 <listOfDataSets>
  <dataSet>
   [DATA REFERENCE]
  </dataSet>
 </listOfDataSets>
</report>
\end{myXmlLst}

The simulation result itself, i.\,e. concrete result numbers, are not stored in SED-ML, but the directive how to \emph{calculate} them from the output of the simulator is provided through the \concept{dataGenerator}.

The encoding of simulation results is outside the scope of SED-ML, but other efforts exist, for example the \emph{Systems Biology Result Markup Language} (SBRML, \citep{DSM10}).

% ~~~~~~~~~~~~~~~~~~~~~~~~~~~~~~~~~~~~~~~~
%% OUTPUT COMPONENTS
% ~~~~~~~~~~~~~~~~~~~~~~~~~~~~~~~~~~~~~~~~ 
\subsection{Output components}

%% ~~~ CURVE ~~~
\subsubsection{\element{Curve}}
\label{class:curve}
One or more instances of the \concept{Curve} class define a 2D plot. A \concept{curve} needs a data generator reference to refer to the data that will be plotted on the x-axis, using the \concept{xDataReference}. A second data generator reference is needed to refer to the data that will be plotted on the y-axis, using the \concept{yDataReference}. 

\sedfig[width=0.75\textwidth]{pdf/curveClass}{The SED-ML Curve class}{fig:curve}


\tabtext{curve}{curve}

\begin{table}[ht]
\center
\begin{tabular}{|l|l|}
\hline
\textbf{\attribute} & \textbf{\desc}\\
\hline
metaid$^{o}$ & \refpage{sec:metaID}\\
id & \refpage{sec:id} \\
name$^{o}$ & \refpage{sec:name}\\
\hline
logX & \refpage{sec:logX}\\
xDataReference & \refpage{sec:xDataReference}\\
logY & \refpage{sec:logY}\\
yDataReference & \refpage{sec:yDataReference}\\
\hline
\hline
\textbf{\subelements} & \textbf{\desc}\\
\hline
notes$^{o}$ & \refpage{class:notes}\\
annotation$^{o}$ & \refpage{class:annotation}\\
\hline
\end{tabular}
\caption{\tabcap{curve}}
\label{tab:curve}
\end{table}

\lsttext{curve}{curve}

\begin{myXmlLst}{The SED-ML \code{curve} element, defining the output curve showing the result of simulation for the referenced dataGenerators}{lst:curve}
<listOfCurves>
  <curve id="c1" name="v1 / time" xDataReference="dg1" yDataReference="dg2" logX="true" logY="false" />
</listOfCurves>
\end{myXmlLst}
Here, only one curve is created, results shown on the x-axis are generated by the data generator \code{dg1}, results shown on the y-axis are generated by the data generator \code{dg2}. Both \code{dg1} and \code{dg2} need to be already defined in the \hyperref[sec:listOfDataGenerators]{listOfDataGenerators}. The x-axis is plotted logarithmically.

\paragraph{\element{logX}}
\label{sec:logX}
\concept{logX} is a required attribute of the \hyperref[class:curve]{Curve} class and defines whether or not the data output on the x-axis is logarithmic. The data type of \concept{logX} is \code{boolean}. 
To make the output on the x-axis of a plot logarithmic, \concept{logX} must be set to ``true'', as shown in the sample Listing~\ref{lst:curve}.

\concept{logX} is also used in the definition of a \hyperref[class:surface]{Surface} output.

\paragraph{\element{logY}}
\label{sec:logY}
\concept{logY} is a required attribute of the \hyperref[class:curve]{Curve} class and defines whether or not the data output on the y-axis is logarithmic. The data type of \concept{logY} is \code{boolean}. 
To make the output on the y-axis of a plot logarithmic, \concept{logY} must be set to ``true'', as shown in the sample Listing~\ref{lst:curve}. 

\concept{logY} is also used in the definition of a \hyperref[class:surface]{Surface} output.

\paragraph{\element{xDataReference}}
\label{sec:xDataReference}
The \concept{xDataReference} is a mandatory attribute of the \hyperref[class:curve]{Curve} object. Its content refers to a dataGenerator ID which denotes the \hyperref[class:dataGenerator]{DataGenerator} object that is used to generate the output on the x-axis of a \hyperref[class:curve]{Curve} in a \hyperref[class:plot2D]{2D Plot}. 
The \concept{xDataReference} data type is \code{string}. However, the valid values for the \concept{xDataReference} are restricted to the IDs of already defined \hyperref[class:dataGenerator]{DataGenerator objects}.

An example for the definition of a curve is given in Listing~\ref{lst:curve}.
\concept{xDataReference} is also used in the definition of the x-axis of a \hyperref[class:surface]{Surface} in a \hyperref[class:plot3D]{3D Plot}.

\paragraph{\element{yDataReference}}
\label{sec:yDataReference}
The \concept{yDataReference} is a mandatory attribute of the \hyperref[class:curve]{Curve} object. Its content refers to a dataGenerator ID which denotes the \hyperref[class:dataGenerator]{DataGenerator} object that is used to generate the output on the y-axis of a \hyperref[class:curve]{Curve} in a \hyperref[class:plot2D]{2D Plot}.
The \concept{yDataReference} data type is \code{string}. However, the number of valid values for the \concept{yDataReference} is restricted to the IDs of already defined \hyperref[class:dataGenerator]{DataGenerator objects}.

An example for the definition of a curve is given in \lst{curve}.
\concept{yDataReference} is also used in the definition of the y-axis of a \hyperref[class:surface]{Surface} in a \hyperref[class:plot3D]{3D Plot}.


%% ~~~ SURFACE ~~~
\subsubsection{\element{Surface}}
\label{class:surface}
A \concept{surface} is a three-dimensional figure representing a simulation result (\fig{surface}).
 
\sedfig[width=0.85\textwidth]{pdf/surfaceClass}{The SED-ML Surface class}{fig:surface}

Creating an instance of the \concept{Surface} class requires the definition of three different axes, that is which data to plot on which axis and in which way. The aforementioned \hyperref[sec:xDataReference]{xDataReference} and \hyperref[sec:yDataReference]{yDataReference} attributes define the according \hyperref[class:dataGenerator]{data generators} for both the x- and y-axis of a surface. In addition, the \hyperref[sec:zDataReference]{zDataReference} attribute defines the output for the z-axis. All axes might be logarithmic or not. This can be specified through the \hyperref[sec:logX]{logX}, \hyperref[sec:logY]{logY}, and the \hyperref[sec:logZ]{logZ} attributes in the according dataReference elements.

\tabtext{surface}{surface}

\begin{table}[ht]
\center
\begin{tabular}{|l|l|}
\hline
\textbf{\attribute} & \textbf{\desc}\\
\hline
metaid$^{o}$ & \refpage{sec:metaID}\\
id & \refpage{sec:id} \\
name$^{o}$ & \refpage{sec:name}\\
\hline
logX & \refpage{sec:logX}\\
xDataReference & \refpage{sec:xDataReference}\\
logY & \refpage{sec:logY}\\
yDataReference & \refpage{sec:yDataReference}\\
logZ & \refpage{sec:logZ}\\
zDataReference & \refpage{sec:zDataReference}\\
\hline
\hline
\textbf{\subelements} & \textbf{\desc}\\
\hline
notes$^{o}$ & \refpage{class:notes}\\
annotation$^{o}$ & \refpage{class:annotation}\\
\hline
\end{tabular}
\caption{\tabcap{surface}}
\label{tab:surface}
\end{table}

\lsttext{surface}{surface}

\begin{myXmlLst}{The SED-ML \code{surface} element, defining the output showing the result of the referenced task}{lst:surface}
<listOfSurfaces>
  <surface id="s1" name="surface" xDataReference="dg1" yDataReference="dg2" zDataReference="dg3" 
   logX="true"  logY="false" logZ="false" />
  [FURTHER SURFACE DEFINITIONS]
</listOfSurfaces>
\end{myXmlLst}

Here, only one surface is created, results shown on the x-axis are generated by the data generator \code{dg1}, results shown on the y-axis are generated by the data generator \code{dg2}, and results shown on the z-axis are generated by the data generator \code{dg3}. All \code{dg1}, \code{dg2} and \code{dg3} need to be already defined in the \hyperref[sec:listOfDataGenerators]{listOfDataGenerators}.

\paragraph{\element{logZ}}
\label{sec:logZ}
\concept{logZ} is a required attribute of the \hyperref[class:surface]{Surface} class and defines whether or not the data output on the z-axis is logarithmic. The data type of \concept{logZ} is \code{boolean}.
To make the output on the z-axis of a surface plot logarithmic, \concept{logZ} must be set to ``true'', as shown in the sample Listing~\ref{lst:surface}.

\paragraph{\element{zDataReference}}
\label{sec:zDataReference}
The \concept{zDataReference} is a mandatory attribute of the \hyperref[class:surface]{Surface} object. Its content refers to a dataGenerator ID which denotes the \hyperref[class:dataGenerator]{DataGenerator} object that is used to generate the output on the z-axis of a \hyperref[class:plot3D]{3D Plot}.
The \concept{zDataReference} data type is \code{string}. However, the valid values for the \concept{zDataReference} are restricted to the IDs of already defined \hyperref[class:dataGenerator]{DataGenerator objects}.

An example using the \code{zDataReference} is given in Listing~\ref{lst:surface} on \refpage{lst:surface}.


%% ~~~ DATASET ~~~
\subsubsection{\element{DataSet}}
\label{class:dataSet}
The \concept{DataSet} class holds definitions of data to be used in the \hyperref[class:report]{Report} class (\fig{dataSet}).
 
\sedfig[width=0.75\textwidth]{pdf/dataSetClass}{The SED-ML DataSet class}{fig:dataSet}

Data sets are labeled references to instances of the \hyperref[class:dataGenerator]{DataGenerator} class.

\tabtext{dataSet}{dataSet}

\begin{table}[h!t]
\center
\begin{tabular}{|l|l|}
\hline
\textbf{\attribute} & \textbf{\desc}\\
\hline
metaid$^{o}$ & \refpage{sec:metaID}\\
id & \refpage{sec:id} \\
name$^{o}$ & \refpage{sec:name}\\
\hline
dataReference & \refpage{sec:dataReference1}\\
label & \refpage{sec:label}\\
\hline
\hline
\textbf{\subelements} & \textbf{\desc}\\
\hline
notes$^{o}$ & \refpage{class:notes}\\
annotation$^{o}$ & \refpage{class:annotation}\\
\hline
\end{tabular}
\caption{\tabcap{dataSet}}
\label{tab:dataSet}
\end{table}

\paragraph{\element{label}}
\label{sec:label}
Each data set in a \hyperref[class:report]{Report} does have to carry an unambiguous \concept{label}. A label is a human readable descriptor of a data set for use in a  \hyperref[class:report]{report}. For example, for a tabular data set of time series results, the label could be the column heading. 

\paragraph{\element{dataReference}}
\label{sec:dataReference1}

The \concept{dataReference} attribute contains the ID of a \concept{dataGenerator} element and as such represents a link to it. The data produced by that particular data generator fills the according data set in the \hyperref[class:report]{report}.

\lsttext{dataSet}{dataSet}

\begin{myXmlLst}{The SED-ML \code{dataSet} element, defining a data set containing the result of the referenced task}{lst:dataSet}
<listOfDataSets>
  <dataSet id="d1" name="v1 over time" dataReference="dg1" label="_1">
</listOfDataSets>
\end{myXmlLst}


%%% Local Variables: 
%%% mode: latex
%%% TeX-master: "../sed-ml-L1V3"
%%% End: 



% ~~~~~~~~~~~~~~~~~~~~~~~~~~~~~~~~~~~~~~~~
%% COMBINE ARCHIVE
% ~~~~~~~~~~~~~~~~~~~~~~~~~~~~~~~~~~~~~~~~
\chapter{COMBINE archive}
\label{app:archive}

A \concept{COMBINE archive} \citep{Bergmann2014} is a single file that supports the exchange of all the information necessary for a modeling and simulation experiment in biology. A COMBINE archive file is a ZIP container that includes a manifest file, listing the content of the archive, an optional metadata file adding information about the archive and its content, and the files describing the model. The content of a COMBINE Archive consists of files encoded in COMBINE standards whenever possible, but may include additional files defined by an Internet Media Type. Several tools that support the COMBINE Archive are available, either as independent libraries or embedded in modeling software.

The COMBINE archive is described at \url{http://co.mbine.org/documents/archive} and 
in \citep{Bergmann2014}.

COMBINE archives are the recommended means for distributing simulation experiment descriptions in SED-ML, the respective data and model files, and the simulation results (figures and reports).

% ~~~~~~~~~~~~~~~~~~~~~~~~~~~~~~~~~~~~
%% ACKNOWLEDGMENTS
% ~~~~~~~~~~~~~~~~~~~~~~~~~~~~~~~~~~~~
\chapter{Acknowledgements}
\label{sec:acknowledgments}
The SED-ML specification is developed with the input of many people. The following individuals served as past SED-ML Editors and contributed to SED-ML specifications. Their efforts helped shape what SED-ML is today.

\begin{itemize}
\item Richard Adams (editor, 2011-2012)
\item Frank Bergmann (editor, 2011-2014)
\item Jonathan Cooper (editor, 2012-2015)
\item Nicolas Le Novère (editorial advisor, 2011-2012, 2013)
\item Andrew Miller (editor, 2011-2012)
\item Ion Moraru (editor, 2014-2016)
\item Sven Sahle (editor, 2014-2016)
\item Herbert Sauro
\end{itemize}

Moreover, we would like to thank all the participants of the meetings where SED-ML has been discussed as well as the members of the SED-ML community.


\appendix
% ~~~~~~~~~~~~~~~~~~~~~~~~~~~~~~~~~~~~
% UML DIAGRAM
% ~~~~~~~~~~~~~~~~~~~~~~~~~~~~~~~~~~~~
\chapter{SED-ML UML Overview}
\fig{sedML} shows the complete UML diagram of the SED-ML. It gives the full picture of all implemented classes (see the XML Schema definition on page \pageref{lst:schema}).
\label{app:uml}
% SED-ML complete UML
\sedfig[width=\textwidth]{pdf/sedML}{The SED-ML UML class diagram}{fig:sedML}

% ~~~~~~~~~~~~~~~~~~~~~~~~~~~~~~~~~~~~
% XML SCHEMA
% ~~~~~~~~~~~~~~~~~~~~~~~~~~~~~~~~~~~~
\chapter{XML Schema}
Listing \ref{lst:schema} shows the full SED-ML XML Schema. The code is commented inline.
\label{sec:xmlschema}
\myXmlImport{The SED-ML XML Schema definition}{lst:schema}{../../schema/level1/version3/sed-ml-L1-V3.xsd}

% ~~~~~~~~~~~~~~~~~~~~~~~~~~~~~~~~~~~~
% EXAMPLES
% ~~~~~~~~~~~~~~~~~~~~~~~~~~~~~~~~~~~~
\chapter{Examples}
This appendix presents selected SED-ML examples. These examples are only illustrative and do not intend to demonstrate the full capabilities of SED-ML. For a more comprehensive view of the SED-ML features refer to the specification (Chapter~\ref{chp:specification}) and to additional SED-ML examples are available from \url{http://sed-ml.org/}.

The presented examples use models encoded in SBML and CellML. SED-ML is not restricted to those formats, but can be used with models encoded in formats serialized in XML. A list of formats known to have been used with SED-ML is available on \url{http://sed-ml.org/}.

% sed-ml example file
The following example provides a SED-ML description for the simulation of the model based on the publication by Leoup, Gonze and Goldbeter ``Limit Cycle Models for Circadian Rhythms Based on Transcriptional Regulation in Drosophila and Neurospora'' (PubMed ID: 10643740).

This model is referenced by its SED-ML ID  \code{model1} and refers to the model with the MIRIAM URN \url{urn:miriam:biomodels.db:BIOMD0000000021}. 
Software applications interpreting this example know how to dereference this URN and access the model in \biom \citep{N+06}.

A second model is defined in l. 11 of the example, using \code{model1} as a source and applying even further changes to it, in this case updating two model parameters.

One simulation setup is defined in the \code{listOfSimulations}. It is a \code{uniformTimeCourse} over 380 time units, providing 1000 output points. The algorithm used is the CVODE solver, as denoted by the KiSAO ID \code{KiSAO:0000019}.

A number of \code{dataGenerator}s are defined in ll. 23-62. Those are the prerequisite for defining the outputs of the simulation. The first dataGenerator named \code{time} collects the simulation time. \code{tim1} in l. 31 maps on the \code{Mt} entity in the model that is used in \code{task1} which here is the model with ID \code{model1}. The dataGenerator named \code{per-tim1} in l. 39 maps on the \code{Cn} entity in \code{model1}. Finally  the fourth and fifth dataGenerators map on the \code{Mt} and \code{per-tim} entity respectively in the updated model with ID \code{model2}.

The \code{output} defined in the experiment consists of three 2D plots. The first plot has two different curves (ll. 65-70) and provides the time course of the simulation using the tim mRNA concentrations from both simulation experiments. The second plot shows the \code{per-tim} concentration against the \code{tim} concentration for the oscillating model. And the third plot shows the same plot for the chaotic model. The resulting three plots are shown in Figure \ref{fig:leloupExample}. 
%
\sedfigX[scale=0.8]{xml/leloupSBML}{The simulation result gained from the simulation description given in \lst{leloup1}}{fig:leloupExample}
%


\myXmlImport{LeLoup Model Simulation Description in SED-ML}{lst:leloup1}{xml/leloupSbml.xml}


%%% Local Variables: 
%%% mode: latex
%%% TeX-master: "../sed-ml-L1V1"
%%% End: 

% sed-ml example file
The following example provides a SED-ML description for the simulation of the model based on the publication by Leoup, Gonze and Goldbeter ``Limit Cycle Models for Circadian Rhythms Based on Transcriptional Regulation in Drosophila and Neurospora'' (PubMed ID: 10643740).
The model source code is taken from the CellML Model Repository \citep{LLH+08}. 

The original model used in the simulation experiment is referred to using a URL (\url{http://models.cellml.org/workspace/leloup_gonze_goldbeter_1999/@@rawfile/d6613d7e1051b3eff2bb1d3d419a445bb8c754ad/leloup_gonze_goldbeter_1999_b.cellml}, ll. 15-16).
In order to st up the model some pre-processing needs to be applied: Those are defined in the \code{listOfChanges} from ll. 17-25. All changes defined update particular parameter values in the model.

A second model is defined in l. 28 of the example, using \code{model1} as a source and applying even further changes to it, in this case updating two more model parameters.

One simulation setup is defined in the \code{listOfSimulations}. It is a \code{uniformTimeCourse} over 180 time units, using 1000 simulation points. The algorithm used is the CVODE solver, as denoted by the KiSAO ID \code{KiSAO:0000019}.

A number of \code{dataGenerator}s are defined in ll. 42-92. Those are the prerequisite for defining the output of the simulation. The first dataGenerator named \code{tim1} in l. 45 maps on the \code{Mt} entity in the model that is used in \code{task1} which here is the model with ID \code{model1}. The second dataGenerator named \code{per-tim} in l. 57 maps on the \code{CN} entity in \code{model1}. Finally  the third and fourth dataGenerators map on the \code{Mt} and \code{per-tim} entity respectively in the updated model with ID \code{model2}.

The \code{output} defined in the experiment constists of a 2D plot with two different curves (ll. 96-102). Both curves plot the \code{per-tim} concentration against the \code{tim} concentration. In the first curve the original parametrisation (as given in \code{model1}) is used, in the second curve the updated one is used (as given in \code{model2}).

\myXmlImport{LeLoup Model Simulation Description in SED-ML}{lst:leloup}{xml/leloupCellml.xml}

% \footnotesize
% \begin{myXmlLst}{LeLoup Model Simulation Description in SED-ML}{lst:leloup}
% <?xml version="1.0" encoding="utf-8"?>
% <sedML version="0.1" xmlns="http://www.biomodels.net/sed-ml" 
%        xmlns:math="http://www.w3.org/1998/Math/MathML">
%  <notes><p xmlns="http://www.w3.org/1999/xhtml">Comparing Limit Cycles and strange attractors for
%         oscillation in Drosophila</p></notes> 
%  <listOfSimulations>
%    <uniformTimeCourse id="simulation1" algorithm="KiSAO:0000019" 
%     initialTime="0" outputStartTime="0" outputEndTime="180" 
%     numberOfPoints="1000" >
%      <algorithm kisaoID="KISAO:0000019"/>
%     </uniformTimeCourse>
%  </listOfSimulations>
%  <listOfModels>
%   <model id="model1" name="Circadian Oscillations" language="urn:sedml:language:cellml" source="http://models.cellml.org/workspace/leloup_gonze_goldbeter_1999/@@rawfile/d6613d7e1051b3eff2bb1d3d419a445bb8c754ad/leloup_gonze_goldbeter_1999_a.cellml" >
%    <listOfChanges>
%     <changeAttribute target="/cellml:model/cellml:component[@cmeta:id='MP']/cellml:variable[@name='vsP']/@initial_value" newValue="1"/>
%     <changeAttribute target="/cellml:model/cellml:component[@cmeta:id='MP']/cellml:variable[@name='vmP']/@initial_value" newValue="0.7"/>
%     <changeAttribute target="/cellml:model/cellml:component[@cmeta:id='P2']/cellml:variable[@name='vdP']/@initial_value" newValue="2"/>
%     <changeAttribute target="/cellml:model/cellml:component[@cmeta:id='T2']/cellml:variable[@name='vdT']/@initial_value" newValue="2"/>  
%     <changeAttribute target="/cellml:model/cellml:component[@name='parameters']/cellml:variable[@name='k1']/@initial_value" newValue="0.6"/>
%     <changeAttribute target="/cellml:model/cellml:component[@name='parameters']/cellml:variable[@name='K4P']/@initial_value" newValue="1"/>
%     <changeAttribute target="/cellml:model/cellml:component[@name='parameters']/cellml:variable[@name='K4T']/@initial_value" newValue="1"/>
%    </listOfChanges>
%   </model>
%   <model id="model2" name="Circadian Chaos" language="urn:sedml:language:cellml" source="model1">
%    <listOfChanges>
%     <changeAttribute target="/cellml:model/cellml:component[@cmeta:id='MT']/cellml:variable[@name='vmT']/@initial_value" newValue="0.28"/>
%     <changeAttribute target="/cellml:model/cellml:component[@cmeta:id='T2']/cellml:variable[@name='vdT']/@initial_value" newValue="4.8"/>        
%    </listOfChanges>
%   </model>
%  </listOfModels>
 
%   <listOfTasks>
%     <task id="task1" name="Limit Cycle" modelReference="model1" simulationReference="simulation1"/>
%     <task id="task2" name="Strange attractors" modelReference="model2" simulationReference="simulation1"/>
%   </listOfTasks>
%   <listOfDataGenerators>
%     <dataGenerator id="tim1" name="tim mRNA">
%       <listOfVariables>
%         <variable id="v0" taskReference="task1" target="/cellml:model/cellml:component[@cmeta:id='MT']" />
%       </listOfVariables>
%        <math:math>
%           <math:apply>
%             <math:plus />
%             <math:ci>v0</math:ci>
%           </math:apply>
%         </math:math>
%     </dataGenerator>

%     <dataGenerator id="per-tim" name="nuclear PER-TIM complex">
%       <listOfVariables>
%         <variable id="v1" taskReference="task1" target="/cellml:model/cellml:component[@cmeta:id='CN']" />
%       </listOfVariables>
%       <math:math>
%         <math:apply>
%           <math:plus />
%           <math:ci>v1</math:ci>
%         </math:apply>
%       </math:math>
%     </dataGenerator>
    
%     <dataGenerator id="tim2" name="tim mRNA (changed parameters)">
%       <listOfVariables>
%         <variable id="v2" taskReference="task2" target="/cellml:model/cellml:component[@cmeta:id='MT']" />
%       </listOfVariables>  
%         <math:math>
%           <math:apply>
%             <math:plus />
%             <math:ci>v2</math:ci>
%           </math:apply>
%         </math:math>
%     </dataGenerator>
    
%     <dataGenerator id="per-tim2" name="nuclear PER-TIM complex">
%       <listOfVariables>
%         <variable id="v3" taskReference="task2" target="/cellml:model/cellml:component[@cmeta:id='CN']" />
%       </listOfVariables>
%       <math:math>
%         <math:apply>
%           <math:plus />
%           <math:ci>v3</math:ci>
%         </math:apply>
%       </math:math>
%     </dataGenerator>
%   </listOfDataGenerators>
  
%   <listOfOutputs>
%     <plot2D id="plot1" name="tim mRNA with Oscillation and Chaos">
%       <listOfCurves>
%         <curve id ="c1" logX="false" logY="false" xDataReference="per-tim" yDataReference="tim1" />
%         <curve id ="c2" logX="false" logY="false" xDataReference="per-tim2" yDataReference="tim2" />
%       </listOfCurves>
%     </plot2D>
%   </listOfOutputs>
% </sedML>
% \end{myXmlLst}



%%% Local Variables: 
%%% mode: latex
%%% TeX-master: "../sed-ml-L1V1"
%%% End: 

% sed-ml example file
The following example provides a SED-ML description for the simulation of the IkappaB-NF-kappaB signaling module based on the publication by Hoffmann, Levchenko, Scott and  Baltimore ``The IkappaB-NF-kappaB signaling module: temporal control and selective gene activation. '' (PubMed ID: 12424381)

This model is referenced by a MRIAM URN as \code{model1}. Software applications interpreting this example know to dereference this URN and access the BioModels Database \citep{N+06} in order to retrieve the model. The simulation description speciies one simulation \code{simulation1}, a uniform timecourse simulation that simulates the model for 41 hours. \code{task1} then applies this simulation to the model. 

As output this simulation description collects four parameters: \code{Total\_NFkBn}, \code{Total\_IkBbeta}, \code{Total\_IkBeps} and \code{Total\_IkBalpha}. These variables are to be plotted against the simulation time and displayed in four separate plots. 



\footnotesize
\begin{myXmlLst}{IkappaB-NF-kappaB signaling Model Simulation Description in SED-ML}{lst:ikappab}
<?xml version="1.0" encoding="utf-8"?>
<sedML xmlns="http://www.biomodels.net/sed-ml">
  <listOfSimulations>
    <uniformTimeCourse id="simulation1" algorithm="KISAO:0000019"
    initialTime="0" outputStartTime="0" outputEndTime="2500"
    numberOfPoints="1000" />
  </listOfSimulations>
  <listOfModels>
    <model id="model1" type="SBML" source="urn:miriam:biomodels.db:BIOMD0000000140" />
  </listOfModels>
  <listOfTasks>
    <task id="task1" modelReference="model1"
    simulationReference="simulation1" />
  </listOfTasks>
  <listOfDataGenerators>
    <dataGenerator id="time" name="time">
      <listOfVariables>
        <variable id="time" taskReference="task1" target="time" />
      </listOfVariables>
      <math xmlns="http://www.w3.org/1998/Math/MathML">
        <ci>time</ci>
      </math>
    </dataGenerator>
    <dataGenerator id="Total_NFkBn" name="Total_NFkBn">
      <listOfVariables>
        <variable id="Total_NFkBn" taskReference="task1"
        target="/sbml:sbml/sbml:model/sbml:listOfParameters/sbml:parameter[@id='Total_NFkBn']" />
      </listOfVariables>
      <math xmlns="http://www.w3.org/1998/Math/MathML">
        <ci>Total_NFkBn</ci>
      </math>
    </dataGenerator>
    <dataGenerator id="Total_IkBbeta" name="Total_IkBbeta">
      <listOfVariables>
        <variable id="Total_IkBbeta" taskReference="task1"
        target="/sbml:sbml/sbml:model/sbml:listOfParameters/sbml:parameter[@id='Total_IkBbeta']" />
      </listOfVariables>
      <math xmlns="http://www.w3.org/1998/Math/MathML">
        <ci>Total_IkBbeta</ci>
      </math>
    </dataGenerator>
    <dataGenerator id="Total_IkBeps" name="Total_IkBeps">
      <listOfVariables>
        <variable id="Total_IkBeps" taskReference="task1"
        target="/sbml:sbml/sbml:model/sbml:listOfParameters/sbml:parameter[@id='Total_IkBeps']" />
      </listOfVariables>
      <math xmlns="http://www.w3.org/1998/Math/MathML">
        <ci>Total_IkBeps</ci>
      </math>
    </dataGenerator>
    <dataGenerator id="Total_IkBalpha" name="Total_IkBalpha">
      <listOfVariables>
        <variable id="Total_IkBalpha" taskReference="task1"
        target="/sbml:sbml/sbml:model/sbml:listOfParameters/sbml:parameter[@id='Total_IkBalpha']" />
      </listOfVariables>
      <math xmlns="http://www.w3.org/1998/Math/MathML">
        <ci>Total_IkBalpha</ci>
      </math>
    </dataGenerator>
  </listOfDataGenerators>
  <listOfOutputs>
    <plot2D id="plot1" name="BM140 Total_NFkBn">
      <listOfCurves>
        <curve logX="false" logY="false" xDataReference="time" 
        yDataReference="Total_NFkBn" />
      </listOfCurves>
    </plot2D>
    <plot2D id="plot2" name="BM140 Total_IkBbeta">
      <listOfCurves>
        <curve logX="false" logY="false" xDataReference="time"
        yDataReference="Total_IkBbeta" />
      </listOfCurves>
    </plot2D>
    <plot2D id="plot3" name="BM140 Total_IkBeps">
      <listOfCurves>
        <curve logX="false" logY="false" xDataReference="time"
        yDataReference="Total_IkBeps" />
      </listOfCurves>
    </plot2D>
    <plot2D id="plot4" name="BM140 Total_IkBalpha">
      <listOfCurves>
        <curve logX="false" logY="false" xDataReference="time" 
        yDataReference="Total_IkBalpha" />
      </listOfCurves>
    </plot2D>
  </listOfOutputs>
</sedML>
\end{myXmlLst}



%%% Local Variables: 
%%% mode: latex
%%% TeX-master: "../sed-ml-L1V1"
%%% End: 

The \hyperref[class:repeatedTask]{repeatedTask} introduced in \LoneVtwo makes it possible to encode a large number of different simulation experiments. In this section several simulation experiment are presented that use the repeated tasks construct. 

\subsection{One dimensional steady state parameter scan}
Here the repeated task calls out to a \hyperref[class:oneStep]{oneStep} task (performing a steady state computation). Each time a parameter is carried in order to collect different responses. 

In the description below the range to be used in the \hyperref[class:setValue]{setValue} construct use of the \token{range} attribute.

%
\sedfigX[scale=0.6]{xml/v3-example1-repeated-steady-scan-oscli}{The simulation result gained from the simulation description given in listing \ref{lst:repeated1}}{fig:figrepeated1}
%

\myXmlImport{SED-ML document implementing the one dimensional steady state parameter scan}
{lst:repeated1}
{xml/v3-example1-repeated-steady-scan-oscli.xml}

\subsection{Perturbing a Simulation}
Often it is interesting to see how the dynamic behavior of a model changes when some perturbations are applied to the model. In this example we include one repeated task that makes repeated use of a oneStep task (that advances an ODE integration to the next output step). During the steps one parameter is modified effectively causing the oscillations of a model to stop. Once the value is reset the oscillations recover. 

Note: In the example below we use a functionalRange, although the same result could also be achieved using the setValue element directly.

%
\sedfigX[scale=0.6]{xml/v3-example2-oscli-nested-pulse}{The simulation result gained from the simulation description given in listing \ref{lst:repeated2}}{fig:figrepeated2}
%

\myXmlImport{SED-ML document implementing the perturbation experiment}
{lst:repeated2}
{xml/v3-example2-oscli-nested-pulse.xml}

\subsection{Repeated Stochastic Simulation}
NOTE: This example produces three dimensional results (time, species concentration, multiple repeats). While \LoneVtwo does not include a way to post-processing these values. So it is left to the implementation on how to display them. One example would be to flatten the values by overlaying them onto the desired plot. 

Running just one stochastic trace does not provide a complete picture of the behavior of a system. A large number of traces are needed to provide a result. This example demonstrates the basic use case of running ten traces of a simulation to. This is achieved by running on repeatedTask running ten uniform time course simulations (each of which performing a stochastic simulation run). 

%
\sedfigX[scale=0.6]{xml/v3-example3-repeated-stochastic-runs}{The simulation result gained from the simulation description given in listing \ref{lst:repeated3}}{fig:figrepeated3}
%

\myXmlImport{SED-ML document implementing repeated stochastic runs}
{lst:repeated3}
{xml/v3-example3-repeated-stochastic-runs.xml}

\subsection{One dimensional time course parameter scan}
NOTE: This example produces three dimensional results (time, species concentration, multiple repeats). While \LoneVtwo does not include a way to post-processing these values. So it is left to the implementation on how to display them. One example would be to flatten the values by overlaying them onto the desired plot. 

Here one repeatedTask  runs repeated uniform time course simulations (performing a deterministic simulation run) after each run the parameter value is changed.


%
\sedfigX[scale=0.6]{xml/v3-example4-repeated-scan-oscli}{The simulation result gained from the simulation description given in listing \ref{lst:repeated4}}{fig:figrepeated4}
%

\myXmlImport{SED-ML document implementing the one dimensional time course parameter scan}
{lst:repeated4}
{xml/v3-example4-repeated-scan-oscli.xml}

\subsection{Two dimensional steady state parameter scan}
NOTE: This example produces three dimensional results (time, species concentration, multiple repeats). While \LoneVtwo does not include a way to post-processing these values. So it is left to the implementation on how to display them. One example would be to flatten the values by overlaying them onto the desired plot. 

Here a repeatedTask runs over another repeatedTask which runs over a oneStep task (performing a steady state computation). Each repeated simulation task modifies a different parameter.

%
\sedfigX[scale=0.6]{xml/v3-example5-boris-2d-scan}{The simulation result gained from the simulation description given in listing \ref{lst:repeated5}}{fig:figrepeated5}
%

\myXmlImport{SED-ML document implementing the one dimensional steady state parameter scan}
{lst:repeated5}
{xml/v3-example5-boris-2d-scan.xml}
\section{Referencing external data}
This example demonstrates the use of the data sources in a basic SED-ML description. In this example a model is simulated (using a uniform time course simulation), that simulation result is plotted in one plot, a second plot obtains a stored result (using the data sources), extracts the \token{S1} and \token{time} column from it and renders it.

\sedfigX[scale=0.6]{examples/dataExample1}{The simulation result gained from the simulation description given in \lst{dataExample1}}{fig:dataExample1}

\myXmlImport{SED-ML document using \SedDataSource and \SedDataDescription}
{lst:dataExample1}
{examples/dataExample1.xml}



% ~~~~~~~~~~~~~~~~~~~~~~~~~~~~~~~~~~~~
%% OVERVIEW  (BPMN)
% ~~~~~~~~~~~~~~~~~~~~~~~~~~~~~~~~~~~~
% We've agreed not to use the overview in the L1V2 spec
% \section{Overview of SED-ML}
\label{sec:overview}
% overview
The \emph{Simulation Experiment Description Markup Language} (SED-ML) is an XML-based format for the description of simulation experiments. It serves to store information about the simulation experiment performed on one or more models with a given set of outputs. Support for SED-ML compliant simulation descriptions will enable the exchange of simulation experiments across tools.
\subsection{Conventions}
%
The Busines Process Modeling Notation Version 1.2 (BPMN) was initially intended to describe internal business procedures (processes) in a graphical way. However, we will use BPMN to graphically describe the steps and processes of setting up a simulation experiment description. The major parts of BPMN that are used to specify SED-ML are activities, gateways, events, data, and documentation. 

An \emph{activity} is ``work that is performed on a [..] process'', for example ``Specify the simulation settings''. Activities may be atomic or non-atomic. SED-ML in particular makes use of the \emph{task} activities, \ie specific work units that need to be performed. Non-atomic tasks might be collapsed or expanded in the graphical representation (see \fig{task}). Each collapsed subprocess has a corresponding expanded subprocess definition.

\begin{figure}[h]
\centering
\includegraphics[width=0.5\textwidth]{images/processes.pdf}
\caption{BPMN activities: task, collapsed process, expanded subprocess}
\label{fig:task}
\end{figure}

\emph{Gateways} serve as means to control the flow of sequence in the diagram. As the term already implies, a gateway needs some ``mechanism that either allows or disallows passage through'' \citep{White:2004}. The result of a gateway pass-through can be that processes are merged or splitted. Graphically, a gateway is represented as a diamond. 

\begin{figure}[h]
\centering
\includegraphics[width=0.5\textwidth]{images/gateways.pdf}
\caption{BPML gateway types: Exclusive (left), parallel (right)}
\label{fig:gateways}
\end{figure}

While there exist a number of different gateway types (see \citep{White:2004}, pp. 93), the SED-ML specification only uses the parallel and the exclusive gates  (see \fig{gateways}). 

\emph{Exclusive} gateways -- also denoted as decisions -- allow the sequence flow to take two or more alternative paths (\fig{gateways}, left hand side). However, \emph{only one} of the paths may be chosen (not more). Sometimes two alternative branches need to be merged together again, in which case the exclusive gate must be used as well: The sequence flow continues as soon as \emph{one} of the incoming processes send a signal. An exclusive gateways is marked by an \code{X} in the graphical notation.

\emph{Parallel} gateways, ``provide a mechanism to synchronize parallel flow and to create parallel flow'' \citep{White:2004} (\fig{gateways}, right hand side). They are used to show parallel paths in the workflow; even if sometimes not required they might help in understanding the process. Synchronisation allows to start two processes in parallel at the same time in the sequence flow: The sequence flow will continue with \emph{all} processes leaving the parallel gateway. Joining two processes with a parallel gateway is also possible: the process flow will only continue after a signal has arrived from \emph{all} processes coming in the parallel gateway. A parallel gateway is marked by a \code{+} in the graphical notation.

\emph{Events} mark everything happening during the execution of the sequence flow, usually they interrrupt the business process, having some cause or impact on the execution. From the broad range of events that BPMN offers, SED-ML only uses a small subset, namely the start event and the end event (\fig{connectorEvents}).

\begin{figure}[h]
\centering
\includegraphics[width=0.5\textwidth]{images/connectors.pdf}
\caption{BPML connectors (left) and events (right).}
\label{fig:connectorEvents}
\end{figure}

All events are graphically drawn as small circles. A \emph{start event} is drawn with a single thin line and mark the start of a process, it can not have any incoming sequence flow. Start events may be triggered by different mechanisms, for the case of SED-ML the untyped start event (no marker inside the circle) is used. The trigger to start the process is ``Create new simulation experiment''. The \emph{end event} is marked with a thick line. It indicates the end of a process. SED-ML specification makes use of the untyped end event (no marker inside the circle). The end event is used to show the end of sub-processes as well as processes. If the end of a sub-process is reached, the sequence flow returns to the according parent process.

\emph{Connectors} are used to combine different BPMN objects with each other (\citep{White:2004} page 30 shows the full list of valid connections). SED-ML uses only a subset of available connectors, namely sequence flow, default flow, and unidirectional associations (\fig{connectorEvents}). \emph{Sequence flow} defines the execution order of activities. \emph{Default flow} marks the default branch to be chosen if other conditions leave various possibilities for further execution of the sequence flow. A \emph{unidirectional association} is used to indicate that a data object is modified, i.\,e. read and written during the execution of an activity \citep{bpmnPoster}.

%
The rough SED-ML workflow is shown in Figure \ref{fig:sedmlWorkflow}.
%
\begin{figure}[h]
\centering
\includegraphics[width=\textwidth]{images/bpmn/sedMainOryx.png}
\caption{The process of defining a simulation experiment in SED-ML (overview)}
\label{fig:sedmlWorkflow}
\end{figure}
%
The process of defining a SED-ML simulation experiment starts by initialising the experiment and creating a new SED-ML file. Afterwards, the \concept{models} needed for the simulation are specified and stored into the existing SED-ML file (see section \ref{overview:models}). In a third step, the simulation experiment \concept{setups} are defined and stored into the same file (see section \ref{overview:simulation}). To assign a setup to a number of models used in the experiment, these connections have to be defined and recorded (see section \ref{overview:task}), called \concept{task} in SED-ML. After simulation, the \concept{output} should defined, based on the specified tasks and performed simulation experiment. The information is added to the existing SED-ML file (see section \ref{overview:output}). In the end, the whole experiment is stored in the final SED-ML file.
%
All collapsed processes are described in the following. Examples in XML are provided in the more technical description.

\subsection{Models}
\label{overview:models}
To define a simulation experiment, first of all a new SED-ML file is created. The models to be used in the experiment (zero or many) are referenced, using a link to a model description in some open, curated model base (such as Biomodels Database \cite{LDR+10}, CellML Repository \cite{BBC+09}, or alike). Changes that are necessary to simulate the model correctly are defined, e.\,g. assigning new parameter values or updating the mathematics of the model (Figure \ref{fig:workflowModel}).
%
\begin{figure}[h]
\centering
\includegraphics[width=0.8\textwidth]{images/bpmn/sedModelOryx.png}
\caption{The process of defining model(s) in SED-ML}
\label{fig:workflowModel}
\end{figure}
%
The procedure is repeated until all models participating in the experiment have been described. Each model used gets an internal SED-ML ID and an optional name.

\subsection{Simulation setup}
\label{overview:simulation}
Secondly, the simulation setups (zero or many) used throughout the simulation experiment are described (Figure \ref{fig:workflowSimulation}). 
%
%
\begin{figure}[h]
\centering
\includegraphics[width=0.8\textwidth]{images/bpmn/sedSimulationOryx.png}
\caption{The process of defining simulation(s) in SED-ML}
\label{fig:workflowSimulation}
\end{figure}
%
Those may stem from various different types of simulation, e.\,g. steady state analysis or bifurcation.  Depending on the specific type of experiment, the information encoded for the simulation setup might differ. Thus, the definition of simulation settings is specific to the simulation experiment.

In a simple case the experiment consists of one simulation, but it can get far more comlex. For example, one might define a nested sequence of simulations, in which case every simulation has to be defined separately.
Each simulation setup gets its own internal ID and an optional name. For each of the setups, the simulation algorithm to be used for that simulation is defined through a reference to a well-defined algorithm name, e.\,g. an ontology or controlled vocabulary. One approach to define such a controlled vocabulary of simulation algorihtms is the \emph{Kinetic Simulation Algorithm Ontology} (KiSAO, \cite{CWK+10}). 
%
The setup definition is repeated until all different simulations have been described.

\subsection{Task}
\label{overview:task}
SED-ML allows to apply one defined simulation setting to one defined model at a time. However, any number of \concept{tasks} may be defined inside a simulation experiment description (Figure \ref{fig:workflowTask}). 
%
\begin{figure}[h]
\centering
\includegraphics[width=0.7\textwidth]{images/bpmn/sedTaskOryx.png}
\caption{The process of defining simulation task(s) in SED-ML}
\label{fig:workflowTask}
\end{figure}
%
To do so, each task refers to one of the formerly specified models and to one of the formerly specified simulation setups. Each task has its own ID and an optional name. The process of task definition is repeated until all tasks have been defined.


The current SED-ML does not allow to nest or order tasks. However, these features are evaluated for future versions of SED-ML.

\subsection{Output}
\label{overview:output}
The SED-ML finally consists of output definitions that describe what kind of output the experiment uses to present the simulation result to the user, i.\,e. a plot or a data table (Figure \ref{fig:workflowOutput}), and also which data is part of the output. 
%
\begin{figure}[h]
\centering
\includegraphics[width=0.7\textwidth]{images/bpmn/sedOutputOryx.png}
\caption{The process of defining output(s) in SED-ML}
\label{fig:workflowOutput}
\end{figure}
%
Therefore, SED-ML first defines a set of \concept{data generators} (Figure \ref{fig:workflowDataGenerator}), which are then used to specify a particular result, i.\,e. output (see section \ref{overview:dataGen}). 

The SED-ML specification comes with three pre-defined types of outputs: 2D- and 3D plots, and reports. All use the aforementioned data generators to specify the information to be plotted on the different axes, or in the table comlumns respectively.
\subsection{Data Generator}
\label{overview:dataGen}
%
\begin{figure}[h]
\centering
\includegraphics[width=0.7\textwidth]{images/bpmn/sedDataGeneratorOryx.png}
\caption{The process of defining data generator(s) in SED-ML}
\label{fig:workflowDataGenerator}
\end{figure}
%
A data generator may use data elements, e.\,g. variables or parameters, that either (1) have been taken directly from the model, or (2) have been generated in a post-processing step. If post-processing needs to be applied, variables and parameters from the various, previously defined models may be used, but also existing global parameters, such as \emph{time}.
If the variables are taken from existing models, a reference to the model and the particular variable needs to be given. 
If post-processing is necessary, a reference to an existing variable or parameter, including other data generators, has to be provided. Additional mathematical rules to be applied on the referred variable or parameter must then  be specified. 
%
In a SED-ML file, any number of data generators can be created for later re-use in the output definition.



%%% Local Variables: 
%%% mode: latex
%%% TeX-master: "../sed-ml-L1V1"
%%% End: 


% REFERENCES
% \bibliographystyle{plainnat}
\bibliographystyle{abbrv}
\bibliography{sed-ml-L1V3}
\end{document}