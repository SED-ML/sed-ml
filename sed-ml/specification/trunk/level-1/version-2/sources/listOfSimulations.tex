  \subsubsection{listOfSimulations: The simulation description container}
\label{sec:listOfSimulations}

The \concept{listOfSimulations} element is the container for \concept{simulation} descriptions (\fig{sedListOfSimulations}).
%
\sedfig[width=0.851\textwidth]{pdf/listOfSimulations}{The listOfSimulations container}{fig:sedListOfSimulations}
%

\lsttext{listOfSimulations}{listOfSimulation}
%
\begin{myXmlLst}{The SED-ML \element{listOfSimulations} element, containing two simulation setups}{lst:listOfSimulations}
 <listOfSimulations>
  <simulation id="s1" [..]>
   [UNIFORM TIMECOURSE DEFINITION]
  </simulation>
  <simulation id="s2" [..]>
   [UNIFORM TIMECOURSE DEFINITION]
  </simulation>
 </listOfSimulations>
\end{myXmlLst}
%
The \code{listOfSimulations} is optional and may contain zero to many simulations. However, if the \LoneVtwo document contains one or more \code{Task} elements, at least one \code{Simulation} element must be defined to which  the \code{Task} element refers --- see section \ref{sec:simulationReference} on \refpage{sec:simulationReference}.

% A SED-ML description can be a sole storage container for a \emph{general} simulation setting, comparible to an experiment procudure description. In that case, no particular model needs to be related to the simulation description:
% %
% \begin{myXmlLst}{}{}
% <sedML>
%  <listOfModels />
%   <listOfSimulations>
%    [SIMULATION SETTINGS FOLLOWING]
%   </listOfSimulations>
% </sedML>
% \end{myXmlLst}
% %

%%% Local Variables: 
%%% mode: plain-tex
%%% TeX-master: "../sed-ml-L1V2"
%%% End: 
