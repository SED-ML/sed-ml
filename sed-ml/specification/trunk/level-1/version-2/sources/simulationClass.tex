% simulation class
 \subsection{\element{Simulation}}
\label{class:simulation}

A simulation is the execution of some defined algorithm(s). 
Simulations are described differently depending on the type of simulation experiment to be performed (\fig{sedSimulation}). 
%
\sedfig[width=0.85\textwidth]{pdf/simulationClass}{The SED-ML Simulation class}{fig:sedSimulation}
%
\concept{Simulation} is an abstract class and serves as the container for the different types of simulation experiments.
SED-ML \LoneVtwo offers the predefined simulation classes \hyperref[class:uniformTimeCourse]{UniformTimeCourse}, \hyperref[class:oneStep]{OneStep} and \hyperref[class:steadyState]{SteadyState}.
Further simulation classes are planned for future versions of SED-ML, including simulation classes for bifurcation analysis.
Simulation algorithms used for the execution of a simulation setup are defined in the \hyperref[class:algorithm]{Algorithm} class.

\tabtext{simulation}{simulation}
%
\begin{table}[ht]
\center
\begin{tabular}{|l|l|}
\hline
\textbf{\attribute} & \textbf{\desc}\\
\hline
metaid$^{o}$ & \refpage{sec:metaID}\\
id & \refpage{sec:id} \\
name$^{o}$ & \refpage{sec:name}\\
\hline
\hline
\textbf{\subelements} & \textbf{\desc}\\
\hline
notes$^{o}$ & \refpage{class:notes}\\
annotation$^{o}$ & \refpage{class:annotation}\\
\hline
algorithm & \refpage{class:algorithm}\\
\hline
\end{tabular}
\caption{\tabcap{simulation}}
\label{tab:simulation}
\end{table}
%

\lsttext{simulation}{simulation}
%
\begin{myXmlLst}{The SED-ML \code{listOfSimulations} element, defining two different simulations}{lst:simulation}
<listOfSimulations>
  <uniformTimeCourse [..]>
    [SIMULATION SPECIFICATION]
  </uniformTimeCourse>
  <uniformTimeCourse [..]>
    [SIMULATION SPECIFICATION]
  </uniformTimeCourse>
</listOfSimulations>
\end{myXmlLst}
%
Two timecourses with uniform range are defined.
%%% Local Variables: 
%%% mode: plain-tex
%%% TeX-master: "../sed-ml-L1V2"
%%% End: 
