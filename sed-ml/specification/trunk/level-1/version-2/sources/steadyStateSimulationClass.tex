 \subsubsection{\element{SteadyState}}
\label{class:steadyState}

The \element{SteadyState} class represents a steady state computation (as for example implemented by NLEQ or Kinsolve). 

%
\sedfig[width=0.3\textwidth]{pdf/steadyState}{The \code{SteadyState} class}{fig:steadyStateSimulation}
%

\tabtext{steadyState}{steadyState}
%
\begin{table}[ht]
\center
\begin{tabular}{|l|l|}
\hline
\textbf{attribute} & \textbf{description}\\
\hline
metaid$^{o}$ & \refpage{sec:metaID}\\
id & \refpage{sec:id} \\
name$^{o}$ & \refpage{sec:name}\\
\hline
\hline
\hline
\textbf{\subelements} & \textbf{\desc}\\
\hline
notes$^{o}$ & \refpage{class:notes}\\
annotation$^{o}$ & \refpage{class:annotation}\\
\hline
algorithm & \refpage{class:algorithm}\\
\hline
\end{tabular}
\caption{\tabcap{steadyState}}
\label{tab:steadyState}
\end{table}
%

\lsttext{steadyState}{steadyState}

%
\begin{myXmlLst}{The SED-ML \code{steadyState} element, defining a steady state simulation with id \code{steady}.}{lst:steadyState}
<listOfSimulations>
  <steadyState id="steady"> 
    <algorithm kisaoID="KISAO:0000282" />
  </steadyState > 
</listOfSimulations>
\end{myXmlLst}



%%% Local Variables: 
%%% mode: latex
%%% TeX-master: "../sed-ml-L1V2"
%%% End: 
