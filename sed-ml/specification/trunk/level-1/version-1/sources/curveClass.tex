\label{class:curve}
One or more instances of the \concept{Curve} class define a 2D plot. 
A \concept{curve} needs a data generator reference to refer to the data that will be plotted on the x-axis, using the \concept{xDataReference}. A second data generator reference is needed to refer to the data that will be plotted on the y-axis, using the \concept{yDataReference}. 
% 
\sedfig[width=0.35\textwidth]{curveClass}{The SED-ML Curve class}{fig:curve}
%

\tabtext{curve}{curve}
%
\begin{table}[ht]
\center
\begin{tabular}{|l|l|}
\hline
\textbf{\attribute} & \textbf{\desc}\\
\hline
metaid$^{o}$ & \refpage{sec:metaID}\\
id & \refpage{sec:id} \\
name$^{o}$ & \refpage{sec:name}\\
\hline
logX & \refpage{sec:logX}\\
xDataReference & \refpage{sec:xDataReference}\\
logY & \refpage{sec:logY}\\
yDataReference & \refpage{sec:yDataReference}\\
\hline
\hline
\textbf{\subelements} & \textbf{\desc}\\
\hline
notes$^{o}$ & \refpage{class:notes}\\
annotation$^{o}$ & \refpage{class:annotation}\\
\hline
\end{tabular}
\label{tab:curve}
\caption{\tabcap{curve}}
\end{table}
%

\lsttext{curve}{curve}
%
\begin{myXmlLst}{The SED-ML \code{curve} element, defining the output curve showing the result of simulation for the referenced dataGenerators}{lst:curve}
<listOfCurves>
  <curve id="c1" name="v1 / time" xDataReference="dg1" yDataReference="dg2" logX="true" logY="false" />
</listOfCurves>
\end{myXmlLst}
Here, only one curve is created, results shown on the x-axis are generated by the data generator \code{dg1}, results shown on the y-axis are generated by the data generator \code{dg2}. Both \code{dg1} and \code{dg2} need to be already defined in the \hyperref[sec:listOfDataGenerators]{listOfDataGenerators}. The x-axis is plotted logarithmically.

\subsubsection{The \element{logX} attribute}
\label{sec:logX}
\concept{logX} is a required attribute of the \hyperref[class:curve]{Curve} class and defines whether or not the data output on the x-axis is logarithmic. The data type of \concept{logX} is \code{boolean}. 
To make the output on the x-axis of a plot logarithmic, \concept{logX} must be set to ``true'', as shown in the sample listing \ref{lst:curve}.

\concept{logX} is also used in the definition of a \hyperref[class:surface]{Surface} output.

\subsubsection{The \element{logY} attribute}
\label{sec:logY}
\concept{logY} is a required attribute of the \hyperref[class:curve]{Curve} class and defines whether or not the data output on the y-axis is logarithmic. The data type of \concept{logY} is \code{boolean}. 
To make the output on the y-axis of a plot logarithmic, \concept{logY} must be set to ``true'', as shown in the sample listing \ref{lst:curve}. 

\concept{logY} is also used in the definition of a \hyperref[class:surface]{Surface} output.

\subsubsection{The \element{xDataReference} attribute}
\label{sec:xDataReference}
The \concept{xDataReference} is a mandatory attribute of the \hyperref[class:curve]{Curve} object. Its content refers to a dataGenerator ID which denotes the \hyperref[class:dataGenerator]{DataGenerator} object that is used to generate the output on the x-axis of a \hyperref[class:curve]{Curve} in a \hyperref[class:plot2D]{2D Plot}. 
The \concept{xDataReference} data type is \code{string}. However, the valid values for the \concept{xDataReference} are restricted to the IDs of already defined \hyperref[class:dataGenerator]{DataGenerator objects}.

An example for the definition of a curve is given in listing \ref{lst:curve}.
\concept{xDataReference} is also used in the definition of the x-axis of a \hyperref[class:surface]{Surface} in a \hyperref[class:plot3D]{3D Plot}.

\subsubsection{The \element{yDataReference} attribute}
\label{sec:yDataReference}
The \concept{yDataReference} is a mandatory attribute of the \hyperref[class:curve]{Curve} object. Its content refers to a dataGenerator ID which denotes the \hyperref[class:dataGenerator]{DataGenerator} object that is used to generate the output on the y-axis of a \hyperref[class:curve]{Curve} in a \hyperref[class:plot2D]{2D Plot}.
The \concept{yDataReference} data type is \code{string}. However, the number of valid values for the \concept{yDataReference} is restricted to the IDs of already defined \hyperref[class:dataGenerator]{DataGenerator objects}.

An example for the definition of a curve is given in listing \ref{lst:curve}.
\concept{yDataReference} is also used in the definition of the y-axis of a \hyperref[class:surface]{Surface} in a \hyperref[class:plot3D]{3D Plot}.

%%% Local Variables: 
%%% mode: latex
%%% TeX-master: "../sed-ml-L1V1"
%%% End: 
