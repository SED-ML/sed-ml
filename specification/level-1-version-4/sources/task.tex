% ~~~~~~~~~~~~~~~~~~~~~~~~~~~~~~~~~~~~~~~~
% ABSTRACT TASK
% ~~~~~~~~~~~~~~~~~~~~~~~~~~~~~~~~~~~~~~~~
\subsection{\element{AbstractTask}}
\label{class:abstractTask}
In SED-ML the subclasses of \concept{AbstractTask} define which \hyperref[class:simulation]{Simulations} should be executed with which \hyperref[class:model]{Models} in the simulation experiment. \concept{AbstractTask} is the base class of all SED-ML tasks, i.e.\ \hyperref[class:task]{Task} and \hyperref[class:repeatedTask]{RepeatedTask}.

\sedfig[width=0.9\textwidth]{images/uml/abstractTask}{The SED-ML Abstract Task class}{fig:abstractTask}

% ~~~~~~~~~~~~~~~~~~~~~~~~~~~~~~~~~~~~~~~~
% TASK
% ~~~~~~~~~~~~~~~~~~~~~~~~~~~~~~~~~~~~~~~~
\subsubsection{\element{Task}}
\label{class:task}

A \concept{Task} links a \hyperref[class:model]{Model} to a certain \hyperref[class:simulation]{Simulation} description via their respective identifiers (\fig{abstractTask}), using the \hyperref[sec:modelReference]{\element{modelReference}} and the \hyperref[sec:simulationReference]{\element{simulationReference}}. The task class receives the \hyperref[sec:id]{\element{id}} and \hyperref[sec:name]{\element{name}} attributes from \hyperref[class:abstractTask]{AbstractTask}.

In SED-ML it is only possible to link one \hyperref[class:simulation]{Simulation} description to one \hyperref[class:model]{Model} at a time. However, one can define as many tasks as needed within one experiment description. Please note that the tasks may be executed in any order, as determined by the implementation.

\lsttext{task}{task} 
In the example, a simulation setting \code{simulation1} is applied first to \code{model1} and then to \code{model2}.
\begin{myXmlLst}{The \code{task} element}{lst:task}
<listOfTasks>
	<task id="t1" name="task definition" modelReference="model1" 
		simulationReference="simulation1" />
	<task id="t2" name="another task definition" modelReference="model2" 
		simulationReference="simulation1" />
</listOfTasks>
\end{myXmlLst}

\subsubsection{\element{CalculateEigenvectors}}
\label{class:calculateEigenvectors}
\hl{TODO: fill in, merge with other tasks below}

\subsubsection{\element{CalculateEigenvalues}}
\label{class:calculateEigenvalues}
\hl{TODO: fill in}

\subsubsection{\element{CalculateJacobian}}
\label{class:calculateJacobian}
\hl{TODO: fill in}

\subsubsection{\element{CalculateStoichiometryMatrix}}
\label{class:calculateStoichiometryMatrix}
\hl{TODO: fill in}

\subsubsection{\element{CalculateElasticity}}
\label{class:calculateElasticity}
\hl{TODO: fill in}

\subsubsection{\element{CalculateLocalSensitivity}}
\label{class:calculateLocalSensitivity}
\hl{TODO: fill in}

\subsubsection{\element{CalculateMoietyConservationLaws}}
\label{class:calculateMoietyConservationLaws}
\hl{TODO: fill in}

\subsubsection{Other Tasks}
There are a range of other tasks that can be specified. These are generally related to model specific properties. Figure~\ref{fig:abstractTask} lists the current set of such tasks. 

{\bf CalculateJacobian} The {\tt CalculateJacobian} task is used to compute the Jacobian matrix at the current state of the model. The matrix will be $m$ by $m$ where $m$ is the number state variables in the model. For those models where there are dependencies among the state variables (eg in reaction network models) it is also possible to specify the calculation of a reduced Jacobian matrix. The attribute {\tt reduced} is Boolean value than can specify a reduced of full matrix. 

{\bf CalculateStoichiometryMatrix} The {\tt CalculateStoichiometryMatrix} task is used to retrieve the stoichiometry matrix from the current model. This calculation is only applicable for reaction network models. The Boolean attribute {\tt reduced} can be used to specify either a reduced or full stoichiometry matrix. 

{\bf CalculateEigenvalues} The {\tt CalculateEigenvalues} task is used to compute the eigenvalues for a model at the last computed state of the model. The task generates two vectors, one vector contains the real numbers and the other the imaginary numbers. 

{\bf CalculateEigenvectors} The {\tt CalculateEigenvalues} task is used to compute the eigenvectors for a model at the last computed state of the model. The task generates a matrix where the columns are the corresponding eigenvectors. 

{\bf CalculateLocalSensitivities} The {\tt CalculateLocalSensitivities} is used to specify sensitivity of a model variable with respect to a parameter. 
%
$$ \frac{dx}{dp} $$
%
where $x$ is a model variable and $p$ a model parameter. For reaction network models, the variable can be a species concentration, reaction flux, or other output of the model. The parameter can be a rate constant or boundary species or other value that the model is dependent upon.

The Boolean attribute {\tt scaled} can be used to compute scaled sensitivities as defined below:
%
$$ \frac{dx}{dp} \frac{x}{p} $$
%
In metabolic control analysis the scaled sensitivities are called the control coefficients and in biochemical systems theory, the logarithmic gains. 

The Boolean attribute {\tt all} can be used to specify that all sensitivities should be computed. In this case the parameter independent sensitivities are computed (See ref). 

{\bf CalculateMoietyConservationLaws} The {\tt CalculateMoietyConservationLaws} can be used to obtain the moiety concervation laws for a model. This is applicable to reaction network based models where it is possible for a linear combination of species amounts to be fixed throughout a simulation.

The task will return a matrix where each row corresponds to a conservation law. The columns will corresponds to the floating species in the model and indicate the amount of each species in a given conservation law. The last column will include the total amount represented by the conservation law. 

For example the model {\tt S1 -> S2; S2 -> S1} has a single conservation laws $S1 + S2$, this would be represented using the matrix:

$$
\left[
\begin{array}{ll}
1 & 1 \\
\end{array}
\right]
$$

% ~~~~~~~~~~~~~~~~~~~~~~~~~~~~~~~~~~~~~~~~
% REPEATED TASK
% ~~~~~~~~~~~~~~~~~~~~~~~~~~~~~~~~~~~~~~~~
\subsubsection{\element{Repeated Task}}
\label{class:repeatedTask}
The \concept{RepeatedTask} (\fig{repeatedTask}) provides a generic looping construct, allowing complex tasks to be composed from individual steps. The \concept{RepeatedTask} performs a specified task (or sequence of tasks as defined in the \hyperref[sec:listOfSubTasks]{\element{listOfSubTasks}}) multiple times (where the exact number is specified through a \hyperref[class:range]{Range} construct as defined in \hyperref[sec:rangeAttribute]{\element{range}}), while allowing specific quantities in the model to be altered at each iteration (as defined in the \hyperref[sec:changesRepeatedTask]{\element{listOfChanges}}).

\sedfig[width=.8\textwidth]{images/uml/repeatedTask}{The SED-ML RepeatedTask class}{fig:repeatedTask}

The \concept{RepeatedTask} inherits the required attribute \hyperref[sec:id]{\element{id}} and optional attribute \hyperref[sec:name]{\element{name}} from \hyperref[class:abstractTask]{AbstractTask}. Additionally it has the two required attributes \hyperref[sec:rangeAttribute]{\element{range}} and \hyperref[sec:resetModel]{\element{resetModel}} and the child elements \hyperref[sec:listOfRanges]{\element{listOfRanges}}, \hyperref[sec:changesRepeatedTask]{\element{listOfChanges}} and \hyperref[class:subTask]{\element{listOfSubTasks}}. Of these \hyperref[sec:listOf]{\element{listOf*}} only \hyperref[sec:changesRepeatedTask]{\element{listOfChanges}} is optional.

The order of activities within each iteration of a \concept{RepeatedTask} is as follows:
\begin{itemize} 
	\item The \hyperref[class:model]{Model} is reset if specified by the \hyperref[sec:resetModel]{\element{resetModel}} attribute. 
	\item Any changes to the model specified by \hyperref[class:setValue]{SetValue} objects in the \hyperref[sec:changesRepeatedTask]{\element{listOfChanges}} are applied to the \hyperref[class:model]{Model}. 
	\item Finally, all \hyperref[class:subTask]{{subTasks}} in the \hyperref[sec:listOfSubTasks]{\element{listOfSubtasks}} are executed in the order specified by their \hyperref[sec:subTaskOrder]{\element{order}} element.
\end{itemize}

\lsttext{repeatedTask}{repeatedTask}
In the example, \code{task1} is repeated three times, each time with a different value for a model parameter \code{w}.
\begin{myXmlLst}{The \code{repeatedTask} element}{lst:repeatedTask}
<task id="task1" modelReference="model1" simulationReference="simulation1" />
<repeatedTask id="task3" resetModel="false" range="current"
    xmlns:s='http://www.sbml.org/sbml/level3/version1/core'>
  <listOfRanges>
    <vectorRange id="current"> 
        <value> 1 </value> 
        <value> 4 </value> 
        <value> 10 </value> 
    </vectorRange> 
  </listOfRanges>
  <listOfChanges>
     <setValue target="/s:sbml/s:model/s:listOfParameters/s:parameter[@id='w']" modelReference="model1">
       <listOfVariables> 
         <variable id="val" name="current range value" target="#current" /> 
       </listOfVariables> 
       <math xmlns="http://www.w3.org/1998/Math/MathML"> 
         <ci> val </ci> 
       </math> 
     </setValue> 
  </listOfChanges>
  <listOfSubTasks>
    <subTask task="task1" />
  </listOfSubTasks>
</repeatedTask>
\end{myXmlLst}
 
% ~~~ RANGE ~~~
\paragraph*{\element{range}}
\label{sec:rangeAttribute}
The \hyperref[class:repeatedTask]{RepeatedTask} has a required attribute \concept{\element{range}} of type \hyperref[type:sidref]{\element{SIdRef}}. It specifies which \concept{\element{range}} defined in the \hyperref[sec:listOfRanges]{\element{listOfRanges}} this repeated task iterates over. Listing~\ref{lst:repeatedTask} shows an example of a \hyperref[class:repeatedTask]{repeatedTask} iterating over a single range comprising the values: \code{1}, \code{4} and \code{10}.
If there are multiple ranges in the \hyperref[sec:listOfRanges]{\element{listOfRanges}}, then only the master \concept{\element{range}} identified by this attribute determines how many iterations there will be in the \hyperref[class:repeatedTask]{repeatedTask}. All other ranges must allow for at least as many iterations as the master range, and will be moved through in lock-step; their values can be used in \hyperref[class:setValue]{setValue} constructs.

% ~~~ RESET MODEL ~~~
\paragraph*{\element{resetModel}}
\label{sec:resetModel}
The \hyperref[class:repeatedTask]{repeatedTask} has a required attribute \concept{\element{resetModel}} of type \code{boolean}. It specifies whether the model should be reset to the initial state before processing an iteration of the defined \hyperref[class:subTask]{subTasks}. Here initial state refers to the state of the model as given in the \hyperref[sec:listOfModels]{\element{listOfModels}}.

In the example in  Listing~\ref{lst:repeatedTask} the repeated task is not to be reset, so a change is made, \code{task1} is carried out, another change is made, then \code{task1} continues from there, another change is applied, and \code{task1} is carried out a last time.

%% ~~~ LIST OF CHANGES ~~~
\paragraph*{\element{listOfChanges}}
\label{sec:changesRepeatedTask}
The optional \concept{\element{listOfChanges}} element contains one or many \hyperref[class:setValue]{SetValue} elements. These elements allow the modification of values in the model prior to the next iteration of the \hyperref[class:repeatedTask]{RepeatedTask}.

%% ~~~ REPEATED TASK : LIST OF SUBTASKS ~~~
\paragraph*{\element{listOfSubTasks}}
\label{sec:listOfSubTasks}
The required \concept{\element{listOfSubTasks}} contains one or more \hyperref[class:subTask]{subTasks} that specify which \hyperref[class:abstractTask]{Tasks} are performed in every iteration of the \hyperref[class:repeatedTask]{RepeatedTask}. All \hyperref[class:subTask]{subTasks} have to be carried out sequentially, each continuing from the current model state (i.e.\ as at the end of the previous \hyperref[class:subTask]{subTask}, assuming it simulates the same model), and with their results concatenated (thus appearing identical to a single complex simulation). The order in which to run multiple \hyperref[class:subTask]{subTasks} must be specified using the \hyperref[sec:subTaskOrder]{\element{order}} attribute on the \hyperref[class:subTask]{subTask}. 

\begin{myXmlLst}{The \code{subTask} element. In this example the task \code{task2} must be executed before \code{task1}.}{lst:subTask}
<listOfSubTasks>
	<subTask task="task1" order="2"/> 
	<subTask task="task2" order="1"/> 
</listOfSubTasks>
\end{myXmlLst}

% ~~~ LIST OF RANGES ~~~
\paragraph*{\element{listOfRanges}}
\label{sec:listOfRanges}
The \concept{\element{listOfRanges}} defines one or more \hyperref[class:range]{ranges} used in the \hyperref[class:repeatedTask]{repeatedTask}.

\hyperref[class:range]{Ranges} are the iterative element of the repeated simulation experiment. Each \hyperref[class:range]{Range} defines a collection of values to iterate over. The \hyperref[sec:id]{\element{id}} attribute of the ranges can be used to refer to the current value of a range. When the \hyperref[sec:id]{\element{id}} attribute is used in a \hyperref[sec:listOfVariables]{listOfVariables} within the \hyperref[class:repeatedTask]{RepeatedTask} its value is to be replaced with the current value of the \hyperref[class:range]{Range}.
%% ? IS THIS CORRECT ? I.e. is it necessary to define a variable with the range id, or is it 
% sufficient to use math with the id.


% ~~~~~~~~~~~~~~~~~~~~~~~~~~~~~~~~~~~~~~~~
% TASK COMPONENTS
% ~~~~~~~~~~~~~~~~~~~~~~~~~~~~~~~~~~~~~~~~
\subsection{\element{Task} components}
\label{class:taskComponents}

% ~~~ SUBTASK ~~~
\subsubsection{\element{SubTask}}
\label{class:subTask}
A \concept{SubTask} (\fig{repeatedTask}) defines the subtask which is executed in every iteration of the enclosing \hyperref[class:repeatedTask]{RepeatedTask}. The \concept{SubTask} has a required attribute \hyperref[sec:subTaskTask]{\element{task}} that references the \hyperref[sec:id]{\element{id}} of another \hyperref[class:abstractTask]{AbstractTask}. The order in which to run multiple \concept{subTasks} must be specified via the required attribute \hyperref[sec:subTaskOrder]{\element{order}}. 

\paragraph*{\element{task}}
\label{sec:subTaskTask}
The required element \concept{\element{task}} of data type \hyperref[type:sidref]{\element{SIdRef}} specifies the \hyperref[class:abstractTask]{AbstractTask} executed by this \hyperref[class:subTask]{SubTask}.

\paragraph*{\element{order}}
\label{sec:subTaskOrder}
The required attribute \concept{\element{order}} of data type \element{integer} specifies the order in which to run multiple \concept{subTasks} in the \hyperref[sec:listOfSubTasks]{\element{listOfSubTasks}}. To specify that one \concept{subTask} should be executed before another its \concept{\element{order}} attribute must have a lower number (e.g.\ in Listing~\ref{lst:subTask}).


% ~~~ SET VALUE ~~~
\subsubsection{\element{SetValue}}
\label{class:setValue}
The \concept{SetValue} class (\fig{repeatedTask}) allows the modification of the \hyperref[class:model]{model} prior to the next execution of the \hyperref[class:subTask]{subTasks}. The changes to the model are defined in the \hyperref[sec:changesRepeatedTask]{\element{listOfChanges}} of the \hyperref[class:repeatedTask]{RepeatedTask}.

\concept{SetValue} inherits from the \hyperref[class:computeChange]{ComputeChange} class, which allows it to compute arbitrary expressions involving a number of \hyperref[class:variable]{variables} and \hyperref[class:parameter]{parameters}. \concept{SetValue} has a mandatory \element{modelReference} attribute, and the optional attributes \element{range} and \element{symbol}.

The value to be changed is identified via the combination of the attributes \code{modelReference} and either \code{symbol} or \code{target}, in order to select an implicit or explicit variable within the referenced model.

As in \hyperref[class:functionalRange]{functionalRange}, the attribute \code{range} may be used as a shorthand to specify the \code{id} of another \concept{Range}. The current value of the referenced range may then be used within the function defining this \concept{FunctionalRange}, just as if that range had been referenced using a \hyperref[class:variable]{variable} element, except that the \code{id} of the range is used directly. In other words, whenever the expression contains a \code{ci} element that contains the value specified in the \code{range} attribute, the value of the referenced range is to be inserted.

The \hyperref[sec:math]{\element{math}} contains the expression computing the value by referring to optional \hyperref[class:parameter]{parameters}, \hyperref[class:variable]{variables} or \hyperref[class:range]{ranges}.
Again as for \hyperref[class:functionalRange]{functionalRange}, variable references retrieve always the current value of the model variable or range at the current iteration of the enclosing \hyperref[class:repeatedTask]{repeatedTask}.

\begin{myXmlLst}{A \code{setValue} element setting \code{w} to the values of the range with id \code{current}.}{lst:setValue}
<listOfChanges>
	<setValue target="/s:sbml/s:model/s:listOfParameters/s:parameter[@id='w']"
		range="current" modelReference="model1">
		<math xmlns="http://www.w3.org/1998/Math/MathML">
			<ci> current </ci>
		</math>
	</setValue>
</listOfChanges>
\end{myXmlLst}

% missing attribute descriptions for consistency
% \paragraph*{range}
% \paragraph*{symbol}
% \paragraph*{range}
% \paragraph*{modelReference}


% ~~~ RANGE ~~~
\subsubsection{\element{Range}}
\label{class:range}
The \concept{Range} class is the abstract base class for the different types of ranges, i.e. \hyperref[class:uniformRange]{UniformRange}, \hyperref[class:vectorRange]{VectorRange}, and \hyperref[class:functionalRange]{FunctionalRange} (\fig{range}). 

\sedfig[width=1.0\textwidth]{images/uml/range}{The SED-ML Range class}{fig:range}

\paragraph{\element{UniformRange}}
\label{class:uniformRange}
The \concept{UniformRange} (\fig{range}) allows the definition of a \hyperref[class:range]{Range} with uniformly spaced values. In this it is quite similar to what is used in the \hyperref[class:uniformTimeCourse]{UniformTimeCourse}. The \concept{UniformRange} is defined via three mandatory attributes: \element{start}, the start value; \element{end}, the end value and \code{numberOfPoints} which defines defines the number of points in addition to the start value (the actual items in the range are \code{numberOfPoints+1}). A fourth attribute \code{type} that can take the values \code{linear} or \code{log} determines whether to draw the values logarithmically (with a base of $10$) or linearly.

For example, the following \concept{UniformRange} will produce \code{101} values uniformly spaced on the interval \code{$[0, 10]$} in ascending order.
\begin{myXmlLst}{The \code{UniformRange} element}{lst:uniformRange}
<uniformRange id="current" start="0.0" end="10.0" numberOfPoints="100" type="linear" /> 
\end{myXmlLst}

The following logarithmic example generates the three values \code{1}, \code{10} and \code{100}.
\begin{myXmlLst}{The \code{UniformRange} element with a logarithmic range.}{lst:uniformRangeLog}
<uniformRange id="current" start="1.0" end="100.0" numberOfPoints="2" type="log" />
\end{myXmlLst}

\paragraph{\element{VectorRange}}
\label{class:vectorRange}

The \concept{VectorRange} (\fig{range}) describes an ordered collection of real values, listing them explicitly within child \element{value} elements .

For example, the range below iterates over the values $1$, $4$ and $10$ in that order.
\begin{myXmlLst}{The \code{VectorRange} element}{lst:vectorRange}
<vectorRange id="current"> 
	<value> 1 </value> 
	<value> 4 </value> 
	<value> 10 </value> 
</vectorRange> 
\end{myXmlLst}

\paragraph{\element{DataRange}}
\label{class:dataRange}
\hl{TODO: decide and fill in}

\paragraph{\element{FunctionalRange}}
\label{class:functionalRange}
The \concept{FunctionalRange} (\fig{range}) constructs a range through calculations that determine the next value based on the value(s) of other range(s) or model variables. In this it is similar to the \hyperref[class:computeChange]{ComputeChange} element, and shares some of the same child elements (but is no subclass of \hyperref[class:computeChange]{ComputeChange}). It consists of an optional attribute \code{range}, two optional elements \hyperref[sec:listOfVariables]{\element{listOfVariables}} and \hyperref[sec:listOfParameters]{\element{listOfParameters}}, and a required element \hyperref[sec:math]{\element{math}}.

The optional attribute \code{range} of type \hyperref[type:sidref]{SIdRef} may be used as a shorthand to specify the \hyperref[sec:id]{\element{id}} of another \hyperref[class:range]{Range}. The current value of the referenced range may then be used within the function defining this \concept{FunctionalRange}, just as if that range had been referenced using a \hyperref[class:variable]{variable} element, except that the \hyperref[sec:id]{\element{id}} of the range is used directly. In other words, whenever the expression contains a \code{ci} element that contains the value specified in the \code{range} attribute, the value of the referenced range is to be inserted.

In the \hyperref[sec:listOfVariables]{\element{listOfVariables}}, the \hyperref[class:variable]{variable} elements define identifiers referring to model variables or range values, which may then be used within the \hyperref[sec:math]{\element{math}} expression. These references always retrieve the current value of the model variable or range at the current iteration of the enclosing \element{repeatedTask}.

The \hyperref[sec:math]{\element{math}} encompasses the mathematical expression that is used to compute the value for the \concept{FunctionalRange} at each iteration of the enclosing \hyperref[class:repeatedTask]{repeatedTask}.

For example:

\begin{myXmlLst}{An example of a \code{functionalRange} where a parameter \code{w} of model \code{model2} is multiplied by \code{index} each time it is called.}{lst:functionalRange}
<functionalRange id="current" range="index"
	xmlns:s='http://www.sbml.org/sbml/level3/version1/core'>
	<listOfVariables>
		<variable id="w" name="current parameter value" modelReference="model2"
			target="/s:sbml/s:model/s:listOfParameters/s:parameter[@id='w']" />
	</listOfVariables>
	<math xmlns="http://www.w3.org/1998/Math/MathML">
	  <apply>
	    <times/>
		  <ci> w </ci>
          <ci> index </ci>
      </apply>
	</math>
</functionalRange>
\end{myXmlLst}

Here is another example, this time using the values in a piecewise expression: 

\begin{myXmlLst}{A \code{functionalRange} element that returns \code{8} if \code{index} is smaller than \code{1}, \code{0.1} if \code{index} is between \code{4} and \code{6}, and \code{8} otherwise.}{lst:functionalRange2}
<uniformRange id="index" start="0" end="10" numberOfPoints="100" />
<functionalRange id="current" range="index">
	<math xmlns="http://www.w3.org/1998/Math/MathML">
		<piecewise>
			<piece>
				<cn> 8 </cn>
				<apply>
					<lt />
					<ci> index </ci>
					<cn> 1 </cn>
				</apply>
			</piece>
			<piece>
				<cn> 0.1 </cn>
				<apply>
					<and />
					<apply>
						<geq />
                    	<ci> index </ci>
                    	<cn> 4 </cn>
					</apply>
					<apply>
						<lt />
						<ci> index </ci>
						<cn> 6 </cn>
					</apply>
				</apply>
			</piece>
			<otherwise>
				<cn> 8 </cn>
			</otherwise>
		</piecewise>
	</math>
</functionalRange>
\end{myXmlLst}
