\label{class:report}
The \concept{Report} class defines a data table consisting of several single instances of the \hyperref[class:dataSet]{DataSet} class (\fig{report}).
Its output returns the simulation result in actual \emph{numbers}. The particular columns of the report table are defined by creating an instance of the \hyperref[class:dataSet]{DataSet} class for each column. 
% 
\sedfig[width=0.75\textwidth]{pdf/report}{The SED-ML Report class}{fig:report}
%


\tabtext{report}{report}
%
\begin{table}[ht]
\center
\begin{tabular}{|l|l|}
\hline
\textbf{\attribute} & \textbf{\desc}\\
\hline
metaid$^{o}$ & \refpage{sec:metaID}\\
id & \refpage{sec:id} \\
name$^{o}$ & \refpage{sec:name}\\
\hline
\hline
\textbf{\subelements} & \textbf{\desc}\\
\hline
notes$^{o}$ & \refpage{class:notes}\\
annotation$^{o}$ & \refpage{class:annotation}\\
\hline
dataSet & \refpage{class:dataSet}\\
\hline
\end{tabular}
\caption{\tabcap{report}}
\label{tab:report}
\end{table}
%

\lsttext{listOfDataSets}{listOfDataSets}
%
\begin{myXmlLst}{The \code{report} element with the nested \code{listOfDataSets} element}{lst:listOfDataSets}
<report>
 <listOfDataSets>
  <dataSet>
   [DATA REFERENCE]
  </dataSet>
 </listOfDataSets>
</report>
\end{myXmlLst}
%

The simulation result itself, i.\,e. concrete result numbers, are not stored in SED-ML, but the directive how to \emph{calculate} them from the output of the simulator is provided through the \concept{dataGenerator}.

The encoding of simulation results is outside the scope of SED-ML, but other efforts exist, for example the \emph{Systems Biology Result Markup Language} (SBRML, \citep{DSM10}).

%%% Local Variables: 
%%% mode: latex
%%% TeX-master: "../sed-ml-L1V2"
%%% End: 
