\label{class:plot2D}

% Fig: 2DPlot
\sedfig[width=0.7\textwidth]{plot2DClass.png}{The SED-ML Plot2D class}{fig:plot2D}
%

\tabtext{plot2D}{Plot2D}
%
\begin{table}[ht]
\center
\begin{tabular}{|l|l|}
\hline
\textbf{\attribute} & \textbf{\desc}\\
\hline
metaid & \refpage{sec:metaID}\\
id & \refpage{sec:id} \\
name & \refpage{sec:name}\\
\hline
\hline
\textbf{\subelements} & \textbf{\desc}\\
\hline
notes & \refpage{class:notes}\\
annotation & \refpage{class:annotation}\\
\hline
\end{tabular}
\label{tab:plot2D}
\caption{\tabcap{Plot2D}}
\end{table}
%

A 2D plot needs a data generator to be assigned to each of its two axes using the \concept{xDataReference} to refer to a \concept{dataGenerator} for the x-axis and using the \concept{yDataReference} to refer to \concept{dataGenerator} for the y-axis. Additionally, the curve can be logarithmic on the x-axis (\concept{logX} set to true) as well as on the y-axis (\concept{logY} set to true).


An example for the definition of a 2D-plot with 2 curves in it is given in listing \ref{lst:listOfCurves}.
%
\begin{myXmlLst}{The listOfCurves element}{lst:listOfCurves}
<listOfCurves>
  <curve logX="true" logY="false" xDataReference="datagenerator1" 
   yDataReference="datagenerator2" /> 
  <curve logX="true" logY="false" xDataReference="datagenerator3" 
   yDataReference="datagenerator4"/> 
</listOfCurves>
\end{myXmlLst}
%
%%% Local Variables: 
%%% mode: latex
%%% TeX-master: "../sed-ml-L1V1"
%%% End: 
