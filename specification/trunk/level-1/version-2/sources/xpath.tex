% XPath
\subsection{XPath usage}  

\label{sec:xpath} 
XPath is a language for finding information in an XML document \citep{xpath:1999}. Within \LoneVtwo, XPath version 1 expressions are  used to identify nodes and attributes within an XML representation of a biological model in the following ways:
%
\begin{enumerate}
\item {Within a \hyperref[class:variable]{Variable} definition, where XPath identifies the model variable required for manipulation in SED-ML.}
\item {Within a  \hyperref[class:change]{Change} definition, where XPath is used to identify the target XML to which a change should be applied.} 

\end{enumerate}

For proper application, XPath expressions should contain prefixes that allow their resolution to the correct XML namespace within an XML document. For example, the XPath expression referring to a species \emph{X} in an SBML model:
\begin{alltt}
/sbml:sbml/sbml:model/sbml:listOfSpecies/sbml:species[@id=`X'] {\color{green} \tickYes -\emph{CORRECT}}
\end{alltt}
is preferable to 
\begin{alltt}
/sbml/model/listOfSpecies/species[@id=`X'] {\color{red} \tickNo -\emph{INCORRECT} }
\end{alltt}

which will only be interpretable by standard XML software tools if the SBML file declares no namespaces (and hence is invalid SBML).

Following the convention of other XPath host languages such as XPointer and XSLT, the prefixes used within XPath expressions must be declared using namespace declarations within the SED-ML document, and be in-scope for the relevant expression.
Thus for the correct example above, there must also be an ancestor element of the node containing the XPath expression that has an attribute like:
\begin{alltt}
xmlns:sbml=`http://www.sbml.org/sbml/level3/version1/core'
\end{alltt}
(a different namespace URI may be used; the key point is that the prefix `sbml' must match that used in the XPath expression).


%%% Local Variables: 
%%% mode: latex
%%% TeX-master: "../sed-ml-L1V2"
%%% End: 
