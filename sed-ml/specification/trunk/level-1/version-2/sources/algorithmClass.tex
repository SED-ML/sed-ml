% algorithm class
 \subsubsection{\element{Algorithm}}
\label{class:algorithm}

SED-ML makes use of the \hyperref[sec:kisao]{KiSAO ontology} (Section~\ref{sec:kisao} on \refpage{sec:kisao}) to refer to a term in the controlled vocabulary identifying the particular simulation algorithm to be used in the simulation. 

Each instance of the \hyperref[class:simulation]{Simulation} class must contain one reference to a simulation algorithm (\fig{algorithm}). 
%
\sedfig[width=0.35\textwidth]{pdf/algorithm}{The \code{Algorithm} class}{fig:algorithm}
%

Each instance of the \concept{Algorithm} class must contain a \hyperref[sec:kisao]{KiSAO} reference to a simulation algorithm. The reference should define the  simulation algorithm to be used in the simulation as precisely as possible, and should be defined in the correct syntax, as defined by the regular expression \code{KISAO:[0-9]\{7\}}.

The \concept{Algorithm} class contains an optional list of parameters (\hyperref[class:algorithmParameter]{algorithmParameter}) that are used to parameterize the particular algorithm used in the simulation. 

\tabtext{algorithm}{Algorithm}
%
\begin{table}[ht]
\center
\begin{tabular}{|l|l|}
\hline
\textbf{attribute} & \textbf{description}\\
\hline
metaid$^{o}$ & \refpage{sec:metaID}\\
kisaoID & \refpage{sec:kisao}\\
\hline
\hline
\textbf{\subelements} & \textbf{\desc}\\
\hline
notes$^{o}$ & \refpage{class:notes}\\
annotation$^{o}$ & \refpage{class:annotation}\\
algorithmParameter$^{o}$ & \refpage{class:algorithmParameter}\\
\hline
\end{tabular}
\caption{\tabcap{algorithm}}
\label{tab:algorithm}
\end{table}
%

The example given in code snippet in Listing~\ref{lst:simulation}, completed by algorithm definitions results in the code given in Listing \ref{lst:algorithm}.
%
\begin{myXmlLst}{The SED-ML \code{algorithm} element for two different time course simulations, defining two different algorithms. KISAO:0000030 refers to the \emph{Euler forward method} ; KISAO:0000021 refers to the \emph{StochSim nearest neighbor algorithm}.}{lst:algorithm}
<listOfSimulations>
 <uniformTimeCourse id="s1" name="time course simulation over 100 minutes" [..]>
  <algorithm kisaoID="KISAO:0000030" />
 </uniformTimeCourse>
 <uniformTimeCourse id="s2" name="time course definition for concentration of p" [..]>
  <algorithm kisaoID="KISAO:0000021" />
 </uniformTimeCourse>
</listOfSimulations>
\end{myXmlLst}
%
For both simulations, one algorithm is defined. In the first simulation \code{s1} a deterministic approach has been chosen (Euler forward method), in the second simulation \code{s2} a stochastic approach is used (Stochsim nearest neighbor).

%%% Local Variables: 
%%% mode: latex
%%% TeX-master: "../sed-ml-L1V2"
%%% End: 
