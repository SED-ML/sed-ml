% SED-Base class
\label{class:sedBase}
\concept{SEDBase} represents the base class for all elements of the SED-ML \LoneVone language. That is, all elements are derived from it. It provides means for additional information to be attached on all other classes  (\fig{sedBase}). That information can be specified in form of human readable \hyperref[class:note]{Notes} or custom \hyperref[class:annotation]{Annotation} classes. 
%
\sedfig[width=0.9\textwidth]{sedBaseClass}{The SEDBase class}{fig:sedBase}
%

%SEDBase has one optional attribute \hyperref[sec:metaID]{metaID}. 

\tabtext{sedbase}{SEDBase}
%
\begin{table}[ht]
\center
\begin{tabular}{|l|l|}
\hline
\textbf{\attribute} & \textbf{\desc}\\
\hline
metaID$^{o}$ & \refpage{sec:metaID} \\
\hline
\hline
\textbf{\subelements} & \textbf{\desc}\\
\hline
notes$^{o}$ & \refpage{class:notes}\\
annotation$^{o}$ & \refpage{class:annotation}\\
\hline
\end{tabular}
\label{tab:sedbase}
\caption{\tabcap{SEDBase}}
\end{table}
%
\subsubsection{\code{metaid} Attribute}
\label{sec:metaID}
The main purpose of the \element{metaid} attribute is to attach \hyperref[class:annotation]{Annotation}s to SED-ML elements. Thus, 
the \element{metaID} attribute has to be globally unique throughout the whole SED-ML document. 

\element{metaID} is of type XML ID.


%%% Local Variables: 
%%% mode: latex
%%% TeX-master: "../sed-ml-L1V1"
%%% End: 