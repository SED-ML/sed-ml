\subsection{\element{id}}
\label{sec:id}
%

Most objects in SED-ML carry an \concept{id} attribute. 
The \hyperref[sec:id]{id} attribute, if it exists for an object, is always required and identifies SED-ML constituents unambiguously.   
The data type for \code{id} is \code{SId} which is a datatype derived from the basic XML type \code{string}, but with restrictions about the characters permitted and the sequences in which those characters may appear. The definition is shown in
Figure~\vref{fig:sid}.

\begin{figure}[hbt]
  \ttfamily
  \small
  \centering
  \begin{tabular}{lll}
    letter & ::= & 'a'..'z','A'..'Z'\\
    digit  & ::= & '0'..'9'\\
    idChar & ::= & letter | digit | '\_'\\
    SId    & ::= & ( letter | '\_' ) idChar*\\
  \end{tabular}
  \vspace*{-1ex}
  \caption{The definition of the type \code{SId}}
  \label{fig:sid}
\end{figure}

For a detailed description see also the SBML specification on the ``Type SId'' \citep[p. 11]{HBH+10}.

All \code{id}s have a global scope, i.\,e.\ the \code{id} must be unambiguous throughout a whole SED-ML document. As such it identifies the constituent it is related to.

An example for a defined \concept{id} is given in Listing~\ref{lst:id}.
%
\begin{myXmlLst}{SED-ML identifier definition, e.\,g.\ for a model}{lst:id}
<model id="m00001" language="urn:sedml:language:sbml" source="urn:miriam:biomodels.db:BIOMD0000000012">
 [MODEL DEFINITION]
</model>
\end{myXmlLst}
%
The defined model carries the  \code{id} \code{m00001}. If the model is referenced elsewhere in the SED-ML document, it is referred to by that  \code{id}.

%%% Local Variables: 
%%% mode: latex
%%% TeX-master: "../sed-ml-L1V1"
%%% End: 
