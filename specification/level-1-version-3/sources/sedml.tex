% ~~~~~~~~~~~~~~~~~~~~~~~~~~~~~~~~~~~~
% SED-ML
% ~~~~~~~~~~~~~~~~~~~~~~~~~~~~~~~~~~~~
\subsection{\element{SED-ML} top level element}
\label{class:sed-ml}
Each SED-ML \currentLV document has a main class called \concept{SED-ML} which defines the document's structure and content (\fig{sed-mlMain}). It consists of several parts connected to the \concept{SED-ML} class via \hyperref[sec:listOf]{\element{listOf*}} constructs through aggregation: 
\begin{itemize}
	\item \hyperref[class:dataDescription]{DataDescription} (for resolving external data), 
	\item \hyperref[class:model]{Model} (for models specifications),
	\item \hyperref[class:simulation]{Simulation} (for simulation setup specification, see Section~\ref{class:simulation}), 
	\item \hyperref[class:abstractTask]{AbstractTask} (for the linkage of models and simulation setups), 
	\item \hyperref[class:dataGenerator]{DataGenerator} (for the definition of post-processing),
	\item \hyperref[class:output]{Output} (for the specification of plots and reports).
\end{itemize}
These parts will be explained in more detail in the relevant sections of this document.

\sedfig[width=0.8\textwidth]{images/uml/sed-ml}{The SED-ML class}{fig:sed-mlMain}

\tabtext{sed-ml}{SED-ML}

\begin{table}[ht]
\center
\begin{tabular}{ll}
\toprule
\textbf{\attribute} & \textbf{\desc}\\
\midrule
metaid$^{o}$ & \refpage{sec:metaid}\\
xmlns & \refpage{sec:xmlns}\\
level & \refpage{sec:level}\\
version & \refpage{sec:version}\\
\midrule
\textbf{\subelements} & \textbf{\desc}\\
\midrule
notes$^{o}$ & \refpage{class:notes}\\
annotation$^{o}$ & \refpage{class:annotation}\\
listOfDataDescriptions$^{o}$ & \refpage{sec:listOfDataDescriptions}\\
listOfModels$^{o}$ & \refpage{sec:listOfModels}\\
listOfSimulations$^{o}$ & \refpage{sec:listOfSimulations} \\
listOfTasks$^{o}$ & \refpage{sec:listOfTasks} \\
listOfDataGenerators$^{o}$ & \refpage{sec:listOfDataGenerators} \\
listOfOutputs$^{o}$ & \refpage{sec:listOfOutputs} \\
\bottomrule
\end{tabular}
\caption{\tabcap{SED-ML}}
\label{tab:sed-ml}
\end{table}

A SED-ML document needs to have the SED-ML namespace defined through the mandatory \hyperref[sec:xmlns]{xmlns} attribute. In addition, the SED-ML \hyperref[sec:level]{level} and \hyperref[sec:version]{version} attributes are required.

The basic XML structure of a SED-ML file is shown in listing \vref{lst:sedmlRoot}.

\begin{myXmlLst}{The SED-ML root element}{lst:sedmlRoot}
<?xml version="1.0" encoding="utf-8"?>
<sedML xmlns:math="http://www.w3.org/1998/Math/MathML" 
       xmlns="http://sed-ml.org/sed-ml/level1/version3" level="1" version="3">
	<listOfDataDescriptions>
	  	[DATA REFERENCES AND TRANSFORMATIONS]
  	</listOfDataDescriptions>
	<listOfModels>
		[MODEL REFERENCES AND APPLIED CHANGES]	 	
 	</listOfModels>
	<listOfSimulations>
		[SIMULATION SETUPS]
	</listOfSimulations>
	<listOfTasks>
		[MODELS LINKED TO SIMULATIONS]
	</listOfTasks>
	<listOfDataGenerators>
		[DEFINITION OF POST-PROCESSING]
	</listOfDataGenerators>
	<listOfOutputs>
		[DEFINITION OF OUTPUT]
	</listOfOutputs>
</sedML>
\end{myXmlLst}

The root element of each SED-ML XML file is the \code{sedML} element, encoding \hyperref[sec:level]{\element{level}} and \hyperref[sec:version]{\element{version}} of the file, and setting the necessary namespaces. Nested inside the \code{sedML} element are the six lists serving as containers for the encoded information: \hyperref[sec:listOfDataGenerators]{\concept{listOfDataDescriptions}} for all external data sources, \hyperref[sec:listOfModels]{\concept{listOfModels}} for all models, \hyperref[sec:listOfSimulations]{\concept{listOfSimulations}} for all simulations, \hyperref[sec:listOfTasks]{\concept{listOfTasks}} for all tasks, \hyperref[sec:listOfDataGenerators]{\concept{listOfDataGenerators}} for all post-processing definitions, and \hyperref[sec:listOfOutputs]{\concept{listOfOutputs}} for all output definitions.

% ~~~ XMLNS ~~~
\subsubsection{\element{xmlns}}
\label{sec:xmlns}
The \concept{xmlns} attribute declares the namespace for the SED-ML document. The pre-defined namespace for SED-ML documents is \url{http://sed-ml.org/sed-ml/level1/version3}. 

In addition, SED-ML makes use of the \concept{MathML} namespace \url{http://www.w3.org/1998/Math/MathML} to enable the encoding of mathematical expressions in MathML 2.0. SED-ML uses a subset of MathML as described in Section~\ref{sec:mathML} on page \pageref{sec:mathML}.

SED-ML \concept{notes} use the XHTML namespace \url{http://www.w3.org/1999/xhtml}.  The \hyperref[class:notes]{Notes} class is described in Section~\ref{class:notes} on page \pageref{class:notes}.

Additional external namespaces might be used in \hyperref[class:annotation]{annotations}. 

% ~~~ LEVEL ~~~
\subsubsection{\element{level}}
\label{sec:level}
The current SED-ML \concept{level} is ``level \level''. Major revisions containing substantial changes will lead to the definition of forthcoming levels.

The level attribute is \code{required} and its value is a \code{fixed} decimal. For SED-ML \currentLV the value is set to \code{1}, as shown in the example in Listing~\ref{lst:sedmlRoot}.

% ~~~ VERSION ~~~
\subsubsection{\element{version}}
\label{sec:version}
The current SED-ML \concept{version} is ``version \version''. Minor revisions containing corrections and refinements of SED-ML elements, or new constructs which do not affect backwards compatibility, will lead to the definition of forthcoming versions.

The version attribute is \code{required} and its value is a \code{fixed} decimal. For SED-ML \currentLV the value is set to \code{\version}, as shown in the example in Listing~\ref{lst:sedmlRoot}.

% ~~~ LIST OF DATA DESCRIPTIONS ~~~
\subsubsection{\element{listOfDataDescriptions}}
\label{sec:listOfDataDescriptions}
In order to reference data in a simulation experiment, the data files along with a description on how to access such files and what information to extract from it have to be defined. SED-ML uses the \concept{listOfDataDescriptions} container for all necessary data (\fig{sed-mlMain}). The \code{listOfDataDescriptions} is optional and may contain zero to many data files.

\lsttext{listOfDataDescriptions}{listOfDataDescriptions}

\begin{myXmlLst}{SED-ML listOfDataDescriptions element}{lst:listOfDataDescriptions}
<listOfDataDescriptions>
	<dataDescription id="Data1" name="Oscli Time Course Data" source="http://svn.code.sf.net/p/libsedml/code/trunk/Samples/data/oscli.numl">
		<dimensionDescription>
			<compositeDescription indexType="double" id="time" name="time" xmlns="http://www.numl.org/numl/level1/version1">
        			<compositeDescription indexType="string" id="SpeciesIds" name="SpeciesIds">
         			<atomicDescription valueType="double" name="Concentrations" />
          		</compositeDescription>
      		</compositeDescription>
		</dimensionDescription>
		<listOfDataSources>
			<dataSource id="dataS1">
				<listOfSlices>
					<slice reference="SpeciesIds" value="S1" />
				</listOfSlices>
			</dataSource>
			<dataSource id="dataTime" indexSet="time" />
		</listOfDataSources>
	</dataDescription>
</listOfDataDescriptions>
\end{myXmlLst}


% ~~~ LIST OF MODELS ~~
\subsubsection{\element{listOfModels}}
\label{sec:listOfModels}
In order to specify a simulation experiment, the participating models have to be defined. SED-ML uses the \concept{listOfModels} container for all necessary models (\fig{sed-mlMain}). 

The \code{listOfModels} is optional and may contain zero to many models. However, if the \currentLV document contains  one or more \code{Task} elements, at least one \code{Model} element must be defined to which the \code{Task} element refers (c.f.\ Section~\ref{sec:modelReference} on \refpage{sec:modelReference}).

\lsttext{listOfModels}{listOfModels}

\begin{myXmlLst}{SED-ML listOfModels element}{lst:listOfModels}
<listOfModels>
	<model id="m0001" language="urn:sedml:language:sbml" 
		source="urn:miriam:biomodels.db:BIOMD0000000012" />
	<model id="m0002" language="urn:sedml:language:cellml" 
		source="http://models.cellml.org/workspace/leloup_gonze_goldbeter_1999/@@rawfile/d6613d7e1051b3eff2bb1d3d419a445bb8c754ad/leloup_gonze_goldbeter_1999_a.cellml" />
</listOfModels>
\end{myXmlLst}


% ~~~ LIST OF CHANGES ~~
\subsubsection{\element{listOfChanges}}
\label{sec:listOfChanges}
The \concept{listOfChanges} contains the defined changes to be applied to a particular \hyperref[class:model]{model} (\fig{model}). It always occurs as an optional subelement of the \element{model} element. The \code{listOfChanges} is nested inside the \code{model} element. The \concept{listOfChanges} is optional and may contain zero to many models.

\lsttext{listOfChanges}{listOfChanges}

\begin{myXmlLst}{The SED-ML \element{listOfChanges} element, defining a change on a model}{lst:listOfChanges}
<model id="m0001" [..]>
	<listOfChanges>
		[CHANGE DEFINITION]
	</listOfChanges>
</model>
\end{myXmlLst}


% ~~~ LIST OF SIMULATIONS ~~
\subsubsection{\element{listOfSimulations}}
\label{sec:listOfSimulations}
The \concept{listOfSimulations} element is the container for \hyperref[class:simulation]{simulation} descriptions (\fig{sed-mlMain}). The \code{listOfSimulations} is optional and may contain zero to many simulations. However, if the \currentLV document contains one or more \code{Task} elements, at least one \code{Simulation} element must be defined to which  the \code{Task} element refers --- see section \ref{sec:simulationReference}.

\lsttext{listOfSimulations}{listOfSimulation}

\begin{myXmlLst}{The SED-ML \element{listOfSimulations} element, containing two simulation setups}{lst:listOfSimulations}
<listOfSimulations>
	<simulation id="s1" [..]>
		[UNIFORM TIMECOURSE DEFINITION]
	</simulation>
	<simulation id="s2" [..]>
   		[UNIFORM TIMECOURSE DEFINITION]
	</simulation>
</listOfSimulations>
\end{myXmlLst}
 
% ~~~ LIST OF TASKS ~~~
\subsubsection{\element{listOfTasks}}
\label{sec:listOfTasks}
The \concept{listOfTasks} element contains the defined \hyperref[class:task]{tasks} for the simulation experiment (\fig{sed-mlMain}).

\lsttext{listOfTasks}{listOfTasks}

\begin{myXmlLst}{The SED-ML \code{listOfTasks} element, defining one task}{lst:listOfTasks}
<listOfTasks>
	<task id="t1" name="simulating v1" modelReference="m1" simulationReference="s1">
	[FURTHER TASK DEFINITIONS]
</listOfTasks>
\end{myXmlLst}

The \code{listOfTasks} is optional and may contain zero to many tasks, each of which is an instance of a subclass of \hyperref[class:abstractTask]{AbstractTask}. However, if the \currentLV document contains a \code{DataGenerator} element with at least one \code{Variable} element, at least one \concept{task} must be defined to which variable(s) in the \code{DataGenerator} element refer --- see Section~\ref{sec:taskReference} on \refpage{sec:taskReference}.


% ~~~ LIST OF DATA GENERATORS ~~~
\subsubsection{\element{listOfDataGenerators}}
\label{sec:listOfDataGenerators}
In SED-ML, all variable and parameter values that are used in the \hyperref[class:output]{Output} class need to be defined as a \hyperref[class:dataGenerator]{dataGenerator} beforehand. The container for those data generators is the \concept{listOfDataGenerators} (\fig{sed-mlMain}).

The \code{listOfDataGenerators} is optional and in general may contain zero to many DataGenerators. 

% THIS IS NOT TRUE:
% However, if the \currentLV document contains an \code{Output} element, at least one \code{DataGenerator} must be defined to which the \code{Output} element refers -  see section \ref{sec:dataReference} on \refpage{sec:dataReference}.

\lsttext{listOfDataGenerators}{listOfDataGenerators}

\begin{myXmlLst}{The \code{listOfDataGenerators} element, defining two data generators \emph{time} and \emph{LaCI repressor}}{lst:listOfDataGenerators}
<listOfDataGenerators>
	<dataGenerator id="d1" name="time">
		[DATA GENERATOR DEFINITION FOLLOWING]
	</dataGenerator>
	<dataGenerator id="LaCI" name="LaCI repressor">
		[DATA GENERATOR DEFINITION FOLLOWING]
	</dataGenerator>
</listOfDataGenerators>
\end{myXmlLst}


% ~~ LIST OF OUTPUTS ~~~
\subsubsection{\element{listOfOutputs}}
\label{sec:listOfOutputs}
The \concept{listOfOutputs} container holds the \hyperref[class:output]{output} definitions of a simulation experiment (\fig{sed-mlMain}). The \concept{listOfOutputs} is optional and may contain zero to many outputs.

The \hyperref[class:output]{Output} can be either a \hyperref[class:report]{report}, a \hyperref[class:plot2D]{plot2D} or as a \hyperref[class:plot3D]{plot3D}. 

\lsttext{listOfOutputs}{listOfOutputs}
\begin{myXmlLst}{The \concept{listOfOutput} element}{lst:listOfOutputs}
<listOfOutputs>
	<report id="report1">
		[REPORT DEFINITION FOLLOWING]
	</report>
	<plot2D id="plot1">
		[2D PLOT DEFINITION FOLLOWING] 
	</plot2D>
</listOfOutputs>
\end{myXmlLst}