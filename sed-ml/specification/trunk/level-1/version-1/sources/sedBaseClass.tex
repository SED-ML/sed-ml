% SED-Base class
\label{class:sedBase}
\concept{SEDBase} is the base class  of SED-ML \LoneVone. All other classes are derived from it. As such it provides means to attach additional information on all other classes  (\fig{sedBase}). That information can be specified in form of human readable \hyperref[class:notes]{Notes} or custom \hyperref[class:annotation]{Annotations}. 
%
\sedfig[width=0.9\textwidth]{sedBaseClass}{The SEDBase class}{fig:sedBase}
%

%SEDBase has one optional attribute \hyperref[sec:metaID]{metaID}. 

\tabtext{sedbase}{SEDBase}
%
\begin{table}[ht]
\center
\begin{tabular}{|l|l|}
\hline
\textbf{\attribute} & \textbf{\desc}\\
\hline
metaID$^{o}$ & \refpage{sec:metaID} \\
\hline
\hline
\textbf{\subelements} & \textbf{\desc}\\
\hline
notes$^{o}$ & \refpage{class:notes}\\
annotation$^{o}$ & \refpage{class:annotation}\\
\hline
\end{tabular}
\label{tab:sedbase}
\caption{\tabcap{SEDBase}}
\end{table}
%
\subsubsection{\code{metaid} Attribute}
\label{sec:metaID}
The main purpose of the \element{metaid} attribute is to attach semantic annotations in form of the \hyperref[class:annotation]{Annotation} class to SED-ML elements.  The type of \code{metaid} is XML ID and as such the \code{metaid} attribute is globally unique throughout the whole SED-ML document. 

For an example showing how to link a semantic annotation to a SED-ML object via the \element{metaid} is given in the \hyperref[class:annotation]{Annotation} class description.


%%% Local Variables: 
%%% mode: latex
%%% TeX-master: "../sed-ml-L1V1"
%%% End: 