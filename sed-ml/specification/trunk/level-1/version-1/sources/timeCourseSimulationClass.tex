\label{class:uniformTimeCourse}
SED-ML \LoneVone so far only supports uniform time courses. 
%A \concept{time course} is defined by the initial time, and the \concept{range} of the simulation (see the \hyperref[class:range]{Range} class for further definition of the range). 
\tabtext{uniformTimeCourse}{UniformTimeCourse}
%
\begin{table}[ht]
\center
\begin{tabular}{|l|l|}
\hline
\textbf{attribute} & \textbf{description}\\
\hline
metaid$^{o}$ & \refpage{sec:metaID}\\
id & \refpage{sec:id} \\
name$^{o}$ & \refpage{sec:name}\\
initialTime & \refpage{sec:initialTime}\\
outputStartTime & \refpage{sec:outputStartTime}\\
outputEndTime & \refpage{sec:outputEndTime}\\
numberOfPoints & \refpage{sec:numberOfPoints}\\
\hline
\hline
\textbf{\subelements} & \textbf{\desc}\\
\hline
notes$^{o}$ & \refpage{class:notes}\\
annotation$^{o}$ & \refpage{class:annotation}\\
algorithm & \refpage{class:algorithm}\\
\hline
\end{tabular}
\label{tab:uniformTimeCourse}
\caption{\tabcap{UniformTimeCourse}}
\end{table}
%


An example for the definition of a uniform time course simulation is given in the XML snippet in listing \ref{lst:timecourse}.
%
\begin{myXmlLst}{The SED-ML \code{uniformTimeCourse} element, defining a uniform time course simulation over 2500 time units with 1000 simulation points, using the CVODE solver.}{lst:timecourse}
<listOfSimulations>
 <uniformTimeCourse id="s1"  name="time course simulation of variable v1 over 100 minutes"  initialTime="0" outputStartTime="0" outputEndTime="2500" numberOfPoints="1000">
    <algorithm kisaoID="KiSAO:0000030" />
 </uniformTimeCourse>
</listOfSimulations>
\end{myXmlLst}

\subsubsection{The \element{initialTime} attribute}
\label{sec:initialTime}

tbw

\subsubsection{The \element{outputStartTime} attribute}
\label{sec:outputStartTime}

tbw

\subsubsection{The \element{outputEndTime} attribute}
\label{sec:outputEndTime}

tbw


\subsubsection{The \element{numberOfPoints} attribute}
\label{sec:numberOfPoints}

tbw

%\subsubsection{Bifurcation Search}
%\alert{tbc}

% \subsubsection{Parameter Scan}
% A parameter scan can be described in SED-ML using the \code{SteadyStateParameterScan1D} class. It allows to scan one parameter which is defined in the \code{parameter} attribute of the according XML element through reference via XPath.
% An example for the specification of a parameter scan in the one-dimensional space is given in example \ref{lst:parameterScan}.
% %
% \begin{myXmlLst}{The \code{steadyStateParameterScan1D} element \alert{to be validated}}{lst:parameterScan}
% </listOfSimulations>
%  <steadyStateParameterScan1D id="s1" name="parameter scan of p1" algorithm="KiSAO:ID" parameter="path/to/parameter">
%   [definitionOfRange]
%  </steadyStateParameterScan1D>
% </listOfSimulations>
% \end{myXmlLst}

% I vote for leaving those out as we need nested simulation tasks here to pre-define the steady state analysis?!

%%% Local Variables: 
%%% mode: latex
%%% TeX-master: "../sed-ml-L1V1"
%%% End: 
