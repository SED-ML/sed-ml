% model class
\label{class:model}
The \concept{Model} class defines the models to be used in the simulation experiment (\fig{sedModel}).
% Fig: sed model
\sedfig[width=0.85\textwidth]{listOfChanges}{The SED-ML Model class}{fig:sedModel}
%

Each instance of the Model class has an unambiguous and mandatory \hyperref[sec:id]{id}. An additional, optional \hyperref[sec:name]{name} may be given to the model. 

The \hyperref[sec:language]{language} may be specified, defining the format the model is encoded in, if such a format exists. Example formats are SBML or CellML.

The \concept{Model} class refers to the particular model of interest through the \hyperref[sec:source]{source} attribute. The restrictions on the model reference are
\begin{itemize}
 \item{The model must be encoded in an XML format.}
 \item{To refer to the model encoding language, a reference to a valid definition of that XML format must be given (\hyperref[sec:language]{language} attribute).}
 \item{To refer to a particular model in an external resource, an unambiguous reference must be given (\hyperref[sec:source]{source} attribute).}
\end{itemize}

A model might need to undergo pre-processings before simulation. Those pre-processings are specified in the SED-ML \hyperref[class:change]{Change} class.

\tabtext{model}{model}
%
\begin{table}[ht]
\center
\begin{tabular}{|l|l|}
\hline
\textbf{\attribute} & \textbf{\desc}\\
\hline
metaid$^{o}$ & \refpage{sec:metaID}\\
id & \refpage{sec:id} \\
name$^{o}$ & \refpage{sec:name}\\
\hline
language$^{o}$ & \refpage{sec:language}\\
source & \refpage{sec:source}\\
\hline
\hline
\textbf{\subelements} & \textbf{\desc}\\
\hline
notes$^{o}$ & \refpage{class:notes}\\
annotation$^{o}$ & \refpage{class:annotation}\\
\hline
change$^{o}$ & \refpage{class:change}\\
\hline
\end{tabular}
\label{tab:model}
\caption{\tabcap{model}}
\end{table}
%

\lsttext{model}{model}
%
\begin{myXmlLst}{SED-ML \code{model} element}{lst:model}
<listOfModels>
 <model id="m0001" language="urn:sedml:language:sbml" 
  source="urn:miriam:biomodels.db:BIOMD0000000012">
  <listOfChanges>
   <change>
    [MODEL PRE-PROCESSING]
   </change>
   </listOfChanges> 
 </model>
 <model id="m0002" language="urn:sedml:language:sbml" source="m0001">
  <listOfChanges>
   [MODEL PRE-PROCESSING]
  </listOfChange>
 </model>
 <model id="m0003" language="urn:sedml:language:cellml" source="http://www.cellml.org/models/leloup_gonze_goldbeter_1999_version02">
  [MODEL PRE-PROCESSING]
 </model>
</listOfModels>
\end{myXmlLst} 
%

The above \code{listOfModels} contains three models: 
The first model \code{m0001} is the Repressilator model taken from \biom. 
The original model is available from \url{urn:miriam:biomodels.db:BIOMD0000000012}. 
For the SED-ML simulation, the model might undergo pre-processings, described in the \hyperref[class:change]{change} element (lines 5-7).
Based on the description of the first model \code{m0001}, the second model is built. 
It refers to the model \code{m001} in the \code{source} attribute, that is the modified version of the Repressilator model.
\code{m0002} might then have even further changes applied to it on top of the changes defined in the pre-processing of \code{m0001}.
The third model in the code example above (lines 13-15) is a different model in CellML representation. \code{m0003} is the model available from the given URL in the \code{source} attribute. Again, it might have additional pre-processing applied to it before used in the simulation.


\subsubsection{The \code{language} attribute}
\label{sec:language}
The evaluation of a SED-ML document is required in order for software to decide whether or not it can be used for a particular simulation environment. One crucial criterion is the particular model representation language used to encode the model. A simulation software usually only supports a small subset of the representation formats available to model biological systems computationally. 

To help  software decide whether or not it supports a SED-ML description file, the information on the model encoding for each referenced model can be provided through the \concept{language} attribute, as the description of a language name and version through an unrestricted \code{String} is error-prone. 
A prerequisite for a language to be fully supported by SED-ML is that a formalised language definition, e.\,g. an XML Schema, is provided online. SED-ML also defines a set of standard URIs to refer to particular language definitions. 
The list of URNs for languages so far associated with SED-ML is available from the SED-ML web site on \url{http://biomodels.net/sed-ml}  (see again section \ref{sec:languageURI} on \refpage{sec:languageURI}). 
To specify language and version, following the idea of MIRIAM URNs, the SED-ML URN scheme \code{urn:sedml:language:}\emph{language name} is used. A model's language being ``SBML Level 2 Version 2'' can be referred to, for example, through the URN \code{urn:sedml:language:sbml.level-2.version-2}.

The \concept{language} attribute is optional in the XML representation of a SED-ML file. 
If it is not explicitly defined in the SED-ML file, the default value for the \concept{language} attribute is \code{urn:sedml:language:xml}, referring to any XML based model representation. 

However, the use of the \concept{language} attribute is strongly encouraged for two reasons. 
Firstly, it helps a user decide whether or not he is able to run the simulation, that is to parse the model referenced in the SED-ML file. 
Secondly, the language attribute is also needed to decide how to handle the implicit variables in the \hyperref[class:variable]{Variable} class, as the interpretation of implicit variables depends on the language of the representation format. The concept of implicit variables has been introduced in section \ref{sec:implicitVariable} on \refpage{sec:implicitVariable}.


\subsubsection{The \code{source} attribute}
\label{sec:source}
To make a model available for the execution of a SED-ML file, the model \element{source} must be specified through an XLink. 
The XLink should preferably point to a public, consistent URI that contains the model description file and follows the proposed \hyperref[sec:uriScheme]{URI Scheme}.
References to curated, open model bases are recommended, such as the BioModels Database. However, any resource registered with MIRIAM resources\footnote{\url{http://www.ebi.ac.uk/miriam/main/}} can easily be referenced. Even without a MIRIAM URN, SED-ML can be used (see again section \ref{sec:modelURI} on \refpage{sec:modelURI}).

An example for the definition of a model, and using the  \hyperref[sec:uriScheme]{URI scheme} is given in listing \ref{lst:sourceA}.
%
\begin{myXmlLst}{The SED-ML \code{source} element, using the URI scheme}{lst:sourceA}
 <model id="m1" name="repressilator" language="urn:sedml:language:sbml" 
  source="urn:miriam:biomodels.db:BIOMD0000000012">
  <listOfChanges>
   [MODEL PRE-PROCESSING]
  </listOfChanges>
 </model>
\end{myXmlLst}
%
The example defines one model \code{m1}. \code{urn:miriam:biomodels.db:BIOMD0000000012} defines the source of the model code. The MIRIAM URN can be resolved into the SBML model stored in BioModels Database under ID \element{BIOMD0000000012} using the MIRIAM web service. The resulting URL is \url{http://www.ebi.ac.uk/biomodels-main/BIOMD0000000012}.

An example for the definition of a model and using a URL is given in listing \ref{lst:sourceB}.
%
\begin{myXmlLst}{The SED-ML \code{source} element, using a URL}{lst:sourceB}
 <model id="m1" name="repressilator" language="urn:sedml:language:cellml" 
  source="http://models.cellml.org/exposure/bba4e39f2c7ba8af51fd045463e7bdd3/aguda_b_1999.cellml">
  <listOfChanges />
 </model>
\end{myXmlLst}
%
In the example one model is defined. The \element{language} of the model is \element{CellML}. As the CellML model repository currently does not provide a MIRIAM URI for model reference, the \emph{URL} pointing to the model code is used to refer to the model. The URL is given in the \element{source} element.

%%% Local Variables: 
%%% mode: plain-tex
%%% TeX-master: "../sed-ml-L1V1"
%%% End: 