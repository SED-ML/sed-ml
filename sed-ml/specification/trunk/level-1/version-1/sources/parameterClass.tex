% parameter class
\label{class:parameter}
An instance of the \concept{Parameter} class in SED-ML defines a symbol associated with a constant value (see \fig{parameter}).
%
\sedfig[width=0.3\textwidth]{parameterClass}{The Parameter class}{fig:parameter}
%
A parameter can unambiguously be identified through it's given \concept{id}. It may additionally carry an optional \concept{name}. Each parameter has one associated \concept{value}. \tab{parameter} shows all attributes and derived classes for \concept{Parameter}.

\tabtext{parameter}{Parameter}
%
\begin{table}[ht]
\center
\begin{tabular}{|l|l|}
\hline
\textbf{\attribute} & \textbf{\desc}\\
\hline
metaID$^{o}$ & \refpage{sec:metaID} \\
id & \refpage{sec:id}\\
name$^{o}$ & \refpage{sec:name}\\
value & \refpage{sec:value}\\
\hline
\hline
\textbf{\subelements} & \textbf{\desc}\\
\hline
notes$^{o}$ & \refpage{class:notes}\\
annotation$^{o}$ & \refpage{class:annotation}\\
\hline
\end{tabular}
\label{tab:parameter}
\caption{\tabcap{Parameter}}
\end{table}
%

SED-ML uses parameters in two ways: 
First, they may occur in the \hyperref[class:change]{Change} class for describing the mathematical computation of a change of a model's observable through predefined parameters.
The second use of the \concept{Parameter} class is to refer from a \hyperref[class:dataGenerator]{DataGenerator} definition to a constant value. 

In both cases the parameter definitions are local to the according particular class defining them. 

\subsubsection{The \element{value} attribute}
\label{sec:value}
Each instance of \concept{Parameter} defines a particular thing with a fixed \concept{value}. The value in the XML representation is of the data type \code{String}. The \code{value} attribute is required for each \code{parameter} element. A parameter gets exactly one constant value assigned, meaning that the value is fixed and cannot be changed. 

An example for a parameter definition is given in listing \ref{lst:parameter}.
%
\begin{myXmlLst}{{The definition of a parameter in SED-ML}}{lst:parameter}
<listOfParameters>
 <parameter id="p1" name="KM" value="40" />
</listOfParameters>
\end{myXmlLst}
%


%%% Local Variables: 
%%% mode: plain-tex
%%% TeX-master: "../sed-ml-L1V1"
%%% End: 
