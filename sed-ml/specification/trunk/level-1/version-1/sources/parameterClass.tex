% parameter class
\label{class:parameter}
The SED-ML \concept{Parameter} class creates instances with a constant value (\fig{parameter}).
%
\sedfig[width=0.35\textwidth]{parameterClass}{The Parameter class}{fig:parameter}
%
SED-ML uses parameters in two ways: 
Firstly, parameters may be defined in the \hyperref[class:computeChange]{ComputeChange} class for describing the mathematical computation of a change of a model's observable.
Secondly, parameters may be part of a \hyperref[class:dataGenerator]{DataGenerator} specification. 
In both cases the parameter definitions are local to the particular class defining them. 

\tabtext{parameter}{parameter}
%
\begin{table}[h!]
\center
\begin{tabular}{|l|l|}
\hline
\textbf{\attribute} & \textbf{\desc}\\
\hline
metaID$^{o}$ & \refpage{sec:metaID} \\
id & \refpage{sec:id}\\
name$^{o}$ & \refpage{sec:name}\\
\hline
value & \refpage{sec:value}\\
\hline
\hline
\textbf{\subelements} & \textbf{\desc}\\
\hline
notes$^{o}$ & \refpage{class:notes}\\
annotation$^{o}$ & \refpage{class:annotation}\\
\hline
\end{tabular}
\label{tab:parameter}
\caption{\tabcap{parameter}}
\end{table}
%

A parameter can unambiguously be identified through its given \hyperref[sec:id]{id}. It may additionally carry an optional \hyperref[sec:name]{name}. Each parameter has one associated \hyperref[sec:value]{value}. 

\lsttext{parameter}{parameter}
The listing shows the definition of a parameter \code{p1} with the \code{value="40"} assigned. 
%
\begin{myXmlLst}{The definition of a parameter in SED-ML}{lst:parameter}
<listOfParameters>
 <parameter id="p1" name="KM" value="40" />
</listOfParameters>
\end{myXmlLst}
%


\subsubsection{The \code{value} attribute}
\label{sec:value}
Each \concept{parameter} has exactly one fixed \concept{value}. The \code{value} attribute of XML data type \code{Double} is required for each \code{parameter} element. 

%%% Local Variables: 
%%% mode: plain-tex
%%% TeX-master: "../sed-ml-L1V1"
%%% End: 