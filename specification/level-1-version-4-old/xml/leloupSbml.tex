% sed-ml example file
The following example provides a SED-ML description for the simulation of the model based on the publication by Leoup, Gonze and Goldbeter ``Limit Cycle Models for Circadian Rhythms Based on Transcriptional Regulation in Drosophila and Neurospora'' (PubMed ID: 10643740).

This model is referenced by its SED-ML ID  \code{model1} and refers to the model with the MIRIAM URN \url{urn:miriam:biomodels.db:BIOMD0000000021}. 
Software applications interpreting this example know how to dereference this URN and access the model in \biom \citep{N+06}.

A second model is defined in l. 11 of the example, using \code{model1} as a source and applying even further changes to it, in this case updating two model parameters.

One simulation setup is defined in the \code{listOfSimulations}. It is a \code{uniformTimeCourse} over 380 time units, providing 1000 output points. The algorithm used is the CVODE solver, as denoted by the KiSAO ID \code{KiSAO:0000019}.

A number of \code{dataGenerator}s are defined in ll. 23-62. Those are the prerequisite for defining the outputs of the simulation. The first dataGenerator named \code{time} collects the simulation time. \code{tim1} in l. 31 maps on the \code{Mt} entity in the model that is used in \code{task1} which here is the model with ID \code{model1}. The dataGenerator named \code{per-tim1} in l. 39 maps on the \code{Cn} entity in \code{model1}. Finally  the fourth and fifth dataGenerators map on the \code{Mt} and \code{per-tim} entity respectively in the updated model with ID \code{model2}.

The \code{output} defined in the experiment consists of three 2D plots. The first plot has two different curves (ll. 65-70) and provides the time course of the simulation using the tim mRNA concentrations from both simulation experiments. The second plot shows the \code{per-tim} concentration against the \code{tim} concentration for the oscillating model. And the third plot shows the same plot for the chaotic model. The resulting three plots are shown in Figure \ref{fig:leloupExample}. 
%
\sedfigX[scale=0.8]{xml/leloupSBML}{The simulation result gained from the simulation description given in \lst{leloup1}}{fig:leloupExample}
%


\myXmlImport{LeLoup Model Simulation Description in SED-ML}{lst:leloup1}{xml/leloupSbml.xml}


%%% Local Variables: 
%%% mode: latex
%%% TeX-master: "../sed-ml-L1V1"
%%% End: 
