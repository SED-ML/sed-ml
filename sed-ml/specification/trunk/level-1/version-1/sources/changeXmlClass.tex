% ChangeAttribute Class
\label{class:changeXml}
The \concept{ChangeXML} element is used to change any XML element in the model that can be addressed by a valid XPath expression. 

\tabtext{changeXml}{changeXml}
%
\begin{table}[ht]
\center
\begin{tabular}{|l|l|}
\hline
\textbf{\attribute} & \textbf{\desc}\\
\hline
metaid & \refpage{sec:metaID}\\
id & \refpage{sec:id} \\
name & \refpage{sec:name}\\
target & \refpage{sec:target}\\
newXML & \refpage{sec:newXml}\\
\hline
\hline
\textbf{\subelements} & \textbf{\desc}\\
\hline
notes & \refpage{class:notes}\\
annotation & \refpage{class:annotation}\\
\hline
\end{tabular}
\label{tab:changeXml}
\caption{\tabcap{ChangeXML}, \alert{newXML is an attribute, probably should be made an element!? (see problems in listing \ref{lst:changeXML})}}
\end{table}
%

The XPath is specified in the required \hyperref[sec:target]{target} (see definition on page \pageref{sec:target}). 

\subsubsection{The \element{newXML} attribute}
\label{sec:newXml}

The new piece of XML code that is to substitute the XML element addressed by the XPAth is provided in the required \element{newXML} attribute. 

An example that adds an additional parameter to a model is given in listing \ref{lst:changeXML}.
%
\begin{myXmlLst}{The changeXML element}{lst:changeXML}
<model [..]>
 <listOfChanges>
  <changeXML target="/sbml:sbml/sbml:model/sbml:listOfParameters
                     /sbml:parameter[@id='V_mT']" 
   newXML="<parameter metaid="metaid_0000010" id="V_mT" value="0.7">
           <parameter metaid="metaid_0000050" id="V_mT_2" value="4.6">" />
 </listOfChanges>
</model>
\end{myXmlLst}
%




%%% Local Variables: 
%%% mode: latex
%%% TeX-master: "../sed-ml-L1V1"
%%% End: 
