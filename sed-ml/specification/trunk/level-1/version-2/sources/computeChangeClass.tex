% ChangeAttribute Class
  \subsubsection{\element{ComputeChange}}
\label{class:computeChange}
The \concept{ComputeChange} class permits to change, prior to the experiment, the value of any element or attribute of a model addressable by an XPath expression, based on a calculation (\fig{computeChange}). 
%
\sedfig[width=\textwidth]{computeChangeClass}{The \code{ComputeChange} class}{fig:computeChange}
%
The changes are described by mathematical expressions using a \hyperref[sec:mathML]{subset of MathML} (see section \ref{sec:mathML} on \refpage{sec:mathML}). The computation can use the value of variables from any model defined in the simulation experiment. Those \hyperref[class:variable]{variables} need to be defined, and can then be addressed by their ID. A variable used in a \concept{ComputeChange} must carry a modelReference attribute (\refpage{sec:modelReference}) but no taskReference attribute (\refpage{sec:taskReference}). To carry out the calculation it may be necessary to introduce additional parameters, that are not defined in any of the model used by the experiment. This is done through the \hyperref[class:parameter]{parameter} class, thereafter refered to by their ID.  Finally, the change itself is specified using an instance of the \hyperref[sec:math]{Math} class.


\tabtext{computeChange}{computeChange}
%
\begin{table}[ht]
\center
\begin{tabular}{|l|l|}
\hline
\textbf{\attribute} & \textbf{\desc}\\
\hline
metaid$^{o}$ & \refpage{sec:metaID}\\
id & \refpage{sec:id} \\
name$^{o}$ & \refpage{sec:name}\\
\hline
target & \refpage{sec:target}\\
\hline
\hline
\textbf{\subelements} & \textbf{\desc}\\
\hline
notes$^{o}$ & \refpage{class:notes}\\
annotation$^{o}$ & \refpage{class:annotation}\\
\hline
listOfVariables$^{o}$ & \refpage{sec:listOfVariables}\\
listOfParameters$^{o}$ & \refpage{sec:listOfParameters}\\
math &\refpage{sec:math}\\
\hline
\end{tabular}
\caption{\tabcap{computeChange}}
\label{tab:computeChange}
\end{table}
%


\paragraph{\element{Math}}
\label{sec:math}

The \element{Math} element encodes mathematical functions. 
If used as an element of the \concept{ComputeChange} class, it computes the change of the element or attribute addressed by the \hyperref[sec:target]{target} attribute.
\LoneVtwo supports the subset of MathML 2.0 shown in section \ref{sec:mathML}.

\lsttext{computeChange}{computeChange}
%
\begin{myXmlLst}{The computeChange element}{lst:computeChange}
<model [..]>
    <computeChange target="/sbml:sbml/sbml:model/sbml:listOfParameters/sbml:parameter[@id='sensor']">
      <listOfVariables>
        <variable modelReference="model1" id="R" name="regulator" 
                  target="/sbml:sbml/sbml:model/sbml:listOfSpecies/sbml:species[@id='regulator']" />
        <variable modelReference="model2" id="S" name="sensor"
                  target="/sbml:sbml/sbml:model/sbml:listOfParameters/sbml:parameter[@id='sensor']" />
      <listOfVariables/>
      <listOfParameters>
        <parameter id="n" name="cooperativity" value="2">
        <parameter id="K" name="sensitivity" value="1e-6">
      <listOfParameters/>
      <math  xmlns="http://www.w3.org/1998/Math/MathML>
        <apply>
          <times />
          <ci>S</ci>
          <apply>
            <divide />
            <apply>
              <power />
              <ci>R</ci>
              <ci>n</ci>
            </apply>
            <apply>
              <plus />
              <apply>
                <power />
                <ci>K</ci>
                <ci>n</ci>
              </apply>
              <apply>
                <power />
                <ci>R</ci>
                <ci>n</ci>
              </apply>
            </apply> 
          </apply>
        </math>
    </computeChange>
  </listOfChanges>
</model>
\end{myXmlLst}
%

The example in listing \ref{lst:computeChange} computes a change of the variable \code{sensor} of the model \code{model2}. To do so, it uses the value of the variable \code{regulator} coming from model \code{model1}. In addition, the calculation used two additional parameters, the cooperativity \code{n}, and the sensitivity \code{K}.
The mathematical expression in the mathML then computes the new initial value of \code{sensor} using the equation:

\begin{math}
S =  S \times \frac{R^{n}}{K^{n}+R^{n}}
\end{math}
.
%A problem arises, because the individual supported model exchange languages allow different subsets of MathML. Thus, when an instance of ComputeChange replaces a %mathematical expression of  an SBML reaction, only the MathML subset allowed by SBML should be used here.


%%% Local Variables: 
%%% mode: latex
%%% TeX-master: "../sed-ml-L1V2"
%%% End: 
