 \subsection{\element{Repeated Task}}
\label{class:repeatedTask}

Instead of redefining a large number of simulation types to suite various simulation tasks, the \concept{repeatedTask} breaks down complex tasks into separate steps. The \concept{repeatedTask} performs a specified simulation multiple times (where the exact number is specified through a \hyperref[class:range]{range} construct while allowing specific aspects of the model to change (in the simplest case that would be by advancing the simulation time). The \concept{repeatedTask} defines a list of subtasks to be carried out sequentially. 

%
\sedfig[width=0.35\textwidth]{repeatedTaskClass}{The SED-ML RepeatedTask class}{fig:sedTask}
%

\tabtext{repeatedTask}{repeatedTask}
%
\begin{table}[ht]
\center
\begin{tabular}{|l|l|}
\hline
\textbf{\attribute} & \textbf{\desc}\\
\hline
metaid$^{o}$ & \refpage{sec:metaID}\\
id & \refpage{sec:id} \\
name$^{o}$ & \refpage{sec:name}\\
\hline
range$^{o}$ & \refpage{sec:rangeAttribute}\\
resetModel$^{o}$ & \refpage{sec:resetModel}\\
\hline
\hline
\textbf{\subelements} & \textbf{\desc}\\
\hline
notes$^{o}$ & \refpage{class:notes}\\
annotation$^{o}$ & \refpage{class:annotation}\\
\hline
listOfRanges$^{o}$ & \refpage{class:ranges}\\
listOfChanges$^{o}$ & \refpage{class:changes}\\
listOfSubTasks$^{o}$ & \refpage{class:subTasks}\\
\hline
\hline
\end{tabular}
\caption{\tabcap{repeatedTask}}
\label{tab:repeatedTask}
\end{table}
%


\subsubsection{ The \element{range} attribute}
\label{sec:rangeAttribute}
The \element{repeatedTask} class has a required attribute \element{range} of type \code{SId}. It specifies which \hyperref[class:range]{range} defined in the \element{listOfRanges} that this repeated task iterates over. Listing~\ref{lst:repeatedTask} shows an example of a \element{repeatedTask} iterating over the values: \code{1}, \code{4} and \code{10}. 


\subsubsection{ The \element{resetModel} attribute}
\label{sec:resetModel}
The \element{repeatedTask} class has a required attribute \element{resetModel} of type \code{boolean}. It specifies whether the model should be reset to the initial state before processing an iteration of the defined \hyperlink[sec:subTasks]{subTasks}. Here initial state refers to the state of the model as given in the \element{listOfModels}.  In the example in  Listing~\ref{lst:repeatedTask} the repeated task is not to be reset, so between the three executions of \code{task1}, the model is not to be reset. 


\lsttext{repeatedTask}{repeatedTask}

%
\begin{myXmlLst}{The \code{repeatedTask} element}{lst:repeatedTask}
<task id="task1" modelReference="model1" simulationReference="simulation1" />
<repeatedTask id="task3" resetModel="false" range="current"> 
  <listOfRanges>
    <vectorRange id="current"> 
        <value> 1 </value> 
        <value> 4 </value> 
        <value> 10 </value> 
    </vectorRange> 
  </listOfRanges>
  <listofChanges>
    <setValue target="/sbml/model/listOfParameters/parameter[@id='w']" modelReference="model1" > 
     <listOfVariables> 
         <variable id="val" name="current range value" target="#current" /> 
     </listOfVariables> 
     <math> 
         <ci> val </ci> 
     </math> 
    </setValue> 
</listOfChanges>
<listOfSubTasks>
  <subTask task="task1" />
</listOfSubTasks>
</repeatedTask>

\end{myXmlLst}
%
In the example, \code{task1} is repeated three times, each time with a different value for a model parameter \code{w}. 


%%% Local Variables: 
%%% mode: latex
%%% TeX-master: "../sed-ml-L1V2"
%%% End: 