\subsection{\element{DataDescription}}
% datadescription class
\label{class:dataDescription}
The \concept{DataDescription} class (\fig{sedDataDescription}) references a file containing data points, along with a description on how to access that file, and what information to extract from it. 

The \concept{DataDescription} class introduces three attributes: the required attributes \token{id} and \token{source} and the optional attribute \token{name}. Additionally two elements are defined: \concept{dimensionDescription} and \concept{listOfDataSources}. 

% Fig: sed model
\sedfig[width=1.0\textwidth]{pdf/dataDescriptionClass}{The SED-ML DataDescription class}{fig:sedDataDescription}
%

\tabtext{dataDescription}{dataDescription}
%
\begin{table}[ht]
\center
\begin{tabular}{|l|l|}
\hline
\textbf{\attribute} & \textbf{\desc}\\
\hline
metaid$^{o}$ & \refpage{sec:metaID}\\
id & \refpage{sec:id} \\
name$^{o}$ & \refpage{sec:name}\\
\hline
source & \refpage{sec:sedsource}\\
\hline
\hline
\textbf{\subelements} & \textbf{\desc}\\
\hline
notes$^{o}$ & \refpage{class:notes}\\
annotation$^{o}$ & \refpage{class:annotation}\\
\hline
dimensionDescription$^{o}$ & \refpage{class:dimensionDescription}\\
listOfDataSources$^{o}$ & \refpage{class:listOfDataSources}\\
\hline
\end{tabular}
\caption{\tabcap{dataDescription}}
\label{tab:dataDescription}
\end{table}
%

\subsubsection{The \token{id} and \token{name} attributes}
The attribute \token{id} of type \token{SId} is meant to uniquely identify the \token{dataDescription} element, while the optional \token{name} attribute of type \token{string}, is there to provide a human readable representation if desired.

\subsubsection{The \token{source} attribute}
\label{sec:sedsource}
Analog to how the \token{source} attribute on the \SedModel is handled, this attribute provides a location of a data file. In order to resolve the \token{source} attribute, the same mechanisms are allowed as for \SedModel element: be it a local file system, a relative link or an online resource.

\note{The community decided, that the file will always be encoded in the NuML format.}

\lsttext{dataDescription}{dataDescription}
%
\begin{myXmlLst}{SED-ML \code{dataDescription} element}{lst:dataDescription}
   <dataDescription id="Data1" name="Oscli Time Course Data" 
     source="http://svn.code.sf.net/p/libsedml/code/trunk/Samples/data/oscli.numl" >
     ...
    </dataDescription>
\end{myXmlLst} 
%

\subsubsection{\element{dimensionDescription}}
\label{class:dimensionDescription}

The \concept{dimensionDescription} element is the data description from an NuML file. Consider for example:

%
\begin{myXmlLst}{SED-ML \code{dimensionDescription} element}{lst:dataDescription}
    <dimensionDescription>
        <compositeDescription indexType="double" id="time" name="time" 
          xmlns="http://www.numl.org/numl/level1/version1">
        <compositeDescription indexType="string" id="SpeciesIds" name="SpeciesIds">
            <atomicDescription valueType="double" name="Concentration" />
        </compositeDescription>
        </compositeDescription>
    </dimensionDescription>
\end{myXmlLst} 
%

Here a nested NuML \token{componentDescription} with \token{time} spanning one dimension and \token{SpeciesIds} another. This two dimensional space is then filled with \token{double} values representing concentrations.

\subsubsection{\element{listOfDataSources}}
\label{class:listOfDataSources}
The \concept{listOfDataSources} contains one or more \SedDataSource elements that are then used in the remainder of the SED-ML document.


\subsection{\element{DataSource}}
% datasource class
\label{class:dataSource}
The \concept{DataSource} class (\fig{sedDataSource}) extracts chunks out of the data file provided by the outer \SedDataDescription element. 

The \concept{DataSource} class introduces three attributes: the required attribute \token{id} and the optional attribute \token{indexSet} and \token{name}. Additionally the \concept{listOfSlices} element is defined. 

% Fig: sed model
\sedfig[width=0.9\textwidth]{pdf/dataSourceClass}{The SED-ML DataSource class}{fig:sedDataSource}
%

\tabtext{dataSource}{dataSource}
%
\begin{table}[ht]
\center
\begin{tabular}{|l|l|}
\hline
\textbf{\attribute} & \textbf{\desc}\\
\hline
metaid$^{o}$ & \refpage{sec:metaID}\\
id & \refpage{sec:id} \\
name$^{o}$ & \refpage{sec:name}\\
\hline
indexSet & \refpage{sec:indexSet}\\
\hline
\hline
\textbf{\subelements} & \textbf{\desc}\\
\hline
notes$^{o}$ & \refpage{class:notes}\\
annotation$^{o}$ & \refpage{class:annotation}\\
\hline
listOfSlices$^{o}$ & \refpage{class:listOfSlices}\\
\hline
\end{tabular}
\caption{\tabcap{dataSource}}
\label{tab:dataSource}
\end{table}
%

\subsubsection{The \token{id} and \token{name} attributes}
The attribute \token{id} of type \token{SId} is meant to uniquely identify the \token{dataSource} element, while the optional \token{name} attribute of type \token{string}, is there to provide a human readable representation if desired.

\subsubsection{The \token{indexSet} attribute}
\label{sec:indexSet}
Since data elements in NuML are either values, or indices, the \SedDataSource element needs two ways of addressing those elements. The \token{indexSet} attribute allows to address all indices privided by NuML elements with \token{indexType}. For example in for the \token{time} \element{componentDescription} above, a \token{dataSource}:

%
\begin{myXmlLst}{}{lst:indexSet}
        <dataSource id="dataTime" indexSet="time" />
\end{myXmlLst} 
%

would extract the set of all timepoints stored in the index. Similarly: 


%
\begin{myXmlLst}{}{lst:indexSet2}
        <dataSource id="allIds" indexSet="SpeciesIds" />
\end{myXmlLst} 
%

would extract all the species id strings stored in that index set. Valid values for \token{indexSet} are all NuML Id elements declared in the \token{dimensionDescription}. If the \token{indexSet} attribute is specified the corresponding \token{dataSource} may not define any \token{slice} elements.

\subsubsection{\element{listOfSlices}}
\label{class:listOfSlices}
The \concept{listOfSlices} contains one or more \SedSlice elements that are then used in the remainder of the SED-ML document.

\subsubsection{Using the \token{dataSource} elements}
Once the \SedDataSource elements are defined, they can be reused anywhere in the SED-ML Description. Specifically their \token{id} attribute can be referenced within the \element{listOfVariables} of \element{DataGenerators}, \element{computeChange} or \element{setValue} objects. Here an example that re-uses the data source \token{dataS1}:


%
\begin{myXmlLst}{}{lst:indexSet2}
   <dataGenerator id="dgDataS1" name="S1 (data)">
     <listOfVariables>
       <variable id="varS1" modelReference="model1" target="#dataS1" />
     </listOfVariables>
     <math xmlns="http://www.w3.org/1998/Math/MathML">
       <ci> varS1 </ci>
     </math>
   </dataGenerator>
\end{myXmlLst} 
%

This represents a change from \LoneVone and \LoneVtwo, in which a \token{taskReference} was always present for a \token{variable} in a data generator.

To indicate that the target is an entity defined within the current SED-ML description the hashtag (\#) with the reference to an \token{id} was used. Additionally, this example uses the \token{modelReference}, in order to facilitate a mapping of the data encoded in the NuML document with a given model. 





%%% Local Variables: 
%%% mode: plain-tex
%%% TeX-master: "../sed-ml-L1V3"
%%% End: 

%%% Local Variables: 
%%% mode: plain-tex
%%% TeX-master: "../sed-ml-L1V3"
%%% End: 