\label{sec:id}
%

Most objects in SED-ML carry the \concept{id} and \hyperref[sec:name]{name} attributes. 
The \concept{id} attribute, if existent for an object, is always required and can be used to identify SED-ML constituents unambiguously.  The {id} attribute can be used to refer to a constituent from other constituents. 
The \code{id} data type is \code{String}. All \code{id}s have a global scope, meaning that throughout a whole SED-ML document, the \code{id} should be unambiguous and as such identifying the constituent it is related to.
An example for a defined \concept{id} is given in listing \ref{lst:id}.
%
\begin{myXmlLst}{SED-ML identifier definition, e.\,g. for a model}{lst:id}
<model id="m00001" type="SBML" source="urn:miriam:biomodels.db:BIOMD0000000012">
 [MODEL DEFINITION]
</model>
\end{myXmlLst}
%

%%% Local Variables: 
%%% mode: latex
%%% TeX-master: "../sed-ml-L1V1"
%%% End: 
