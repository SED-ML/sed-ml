\label{surfaceClass}

A surface is a three-dimensional Figure representing a simulation result. \fig{surface} shows the definition of the \concept{Surface} class used to encode the information needed for the creation of such Figures in SED-ML. 
% 
\sedfig[width=0.8\textwidth]{surfaceClass}{The SED-ML Surface class}{fig:surface}
%

\tabtext{surface}{Surface}
%
\begin{table}[ht]
\center
\begin{tabular}{|l|l|}
\hline
\textbf{\attribute} & \textbf{\desc}\\
\hline
metaid & \refpage{sec:metaID}\\
id & \refpage{sec:id} \\
name & \refpage{sec:name}\\
logX & \refpage{sec:logX}\\
xDataReference & \refpage{sec:xDataReference}\\
logY & \refpage{sec:logY}\\
yDataReference & \refpage{sec:yDataReference}\\
logZ & \refpage{sec:logZ}\\
zDataReference & \refpage{sec:zDataReference}\\
\hline
\hline
\textbf{\subelements} & \textbf{\desc}\\
\hline
notes & \refpage{class:notes}\\
annotation & \refpage{class:annotation}\\
\hline
\end{tabular}
\label{tab:surface}
\caption{\tabcap{Surface}}
\end{table}
%
To define a surface, the three different axes have to be defined, that is which data to plot on which axis and in which way.
The aforementioned \hyperref[sec:xDataReference]{xDataReference} and \hyperref[sec:yDataReference]{yDataReference} attributes define the according dataGenerators for both the x- and y-axis of a surface. In addition, the \hyperref[sec:zDataReference]{zDataReference} attribute defines the output for the z-axis. All axes might be logarithmic or not. This can be specified through the \hyperref[sec:logX]{logX}, \hyperref[sec:logY]{logY}, and the \hyperref[sec:logZ]{logZ} attributes in the according dataReference elements.

\subsubsection{The \element{logZ} attribute}
\label{sec:logZ}
\concept{logZ} is an optional attribute of the \hyperref[class:surface]{Surface} class and defines whether or not the data output on the z-axis is logarithmic or not. The data type of \concept{logZ} is boolean, the standard value is ``false''.
To make the output on the z-axis of a surface plot logarithmic, \concept{logZ} must be set to ``true'', as shown in the sample listing \ref{lst:logZ}: 
\begin{myXmlLst}{The SED-ML  logZ attribute, defining a logarithmic output on the z-axis of the according output}{lst:logZ}
<listOfSurfaces logZ="true">
  <surface id="s1" [..]>
</listOfSurfaces>
\end{myXmlLst}

\subsubsection{The \element{zDataReference} attribute}
\label{sec:zDataReference}
The \concept{zDataReference} is a mandatory attribute of the \hyperref[class:surface]{Surface} object. It's content refers to a dataGenerator ID which denotes the \hyperref[class:dataGenerator]{DataGenerator} object that is used to generate the output on the z-axis of a \hyperref[class:3DPlot]{3D Plot}.
The \concept{zDataReference} data type is string. However, the number of valid values for the \concept{zDataReference} is restricted to the IDs of already defined \hyperref[class:dataGenerator]{DataGenerator objects}.

An example for the definition of a surface is given in the XML snippet in listing \ref{lst:zDataReference}.
%
\begin{myXmlLst}{The SED-ML curve element, defining the output curve showing the result of the referenced task}{lst:zDataReference}
<listOfSurfaces>
  <surface id="s1" name="surface" algorithm ="KiSAO:ID" xDataReference="dg1" 
   yDataReference="dg2" zDataReference="dg3">
</listOfSurfaces>
\end{myXmlLst}
Here, only one curve is created, results shown on the x-axis are generated by the data generator \code{dg1}, results shown on the y-axis are generated by the data generator \code{dg2}, results shown on the z-axis are generated by the data generator \code{dg3}. All \code{dg1}, \code{dg2} and \code{dg3} need to be already defined in the \hyperref[sec:listOfDataGenerators]{listOfDataGenerators}.
%%% Local Variables: 
%%% mode: latex
%%% TeX-master: "../sed-ml-L1V1"
%%% End: 
