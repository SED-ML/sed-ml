\subsubsection{\element{DataSet}}
\label{class:dataSet}
The \concept{DataSet} class holds definitions of data to be used in the \hyperref[class:report]{Report} class (\fig{dataSet}).
% 
\sedfig[width=0.75\textwidth]{pdf/dataSetClass}{The SED-ML DataSet class}{fig:dataSet}
%
Data sets are labeled references to instances of the \hyperref[class:dataGenerator]{DataGenerator} class.

\tabtext{dataSet}{dataSet}
%
\begin{table}[h!t]
\center
\begin{tabular}{|l|l|}
\hline
\textbf{\attribute} & \textbf{\desc}\\
\hline
metaid$^{o}$ & \refpage{sec:metaID}\\
id & \refpage{sec:id} \\
name$^{o}$ & \refpage{sec:name}\\
\hline
dataReference & \refpage{sec:dataReference1}\\
label & \refpage{sec:label}\\
\hline
\hline
\textbf{\subelements} & \textbf{\desc}\\
\hline
notes$^{o}$ & \refpage{class:notes}\\
annotation$^{o}$ & \refpage{class:annotation}\\
\hline
\end{tabular}
\caption{\tabcap{dataSet}}
\label{tab:dataSet}
\end{table}
%

\paragraph{\element{label}}
\label{sec:label}
Each data set in a \hyperref[class:report]{Report} does have to carry an unambiguous \concept{label}. A label is a human readable descriptor of a data set for use in a  \hyperref[class:report]{report}. For example, for a tabular data set of time series results, the label could be the column heading. 

\paragraph{\element{dataReference}}
\label{sec:dataReference1}

The \concept{dataReference} attribute contains the ID of a \concept{dataGenerator} element and as such represents a link to it. The data produced by that particular data generator fills the according data set in the \hyperref[class:report]{report}.
\lsttext{dataSet}{dataSet}
%
\begin{myXmlLst}{The SED-ML \code{dataSet} element, defining a data set containing the result of the referenced task}{lst:dataSet}
<listOfDataSets>
  <dataSet id="d1" name="v1 over time" dataReference="dg1" label="_1">
</listOfDataSets>
\end{myXmlLst}


%%% Local Variables: 
%%% mode: latex
%%% TeX-master: "../sed-ml-L1V3"
%%% End: 
