  \subsubsection{listOfTasks: The task specification container}
\label{sec:listOfTasks}
The \concept{listOfTasks} element contains the defined tasks for the simulation experiment (\fig{listOfTasks}).
%
\sedfig[width=0.85\textwidth]{listOfTasks}{The SED-ML listOfTasks container}{fig:listOfTasks}
%

\lsttext{listOfTasks}{listOfTasks}
%
\begin{myXmlLst}{The SED-ML \code{listOfTasks} element, defining one task}{lst:listOfTasks}
<listOfTasks>
 <task id="t1" name="simulating v1" modelReference="m1" simulationReference="s1">
 [FURTHER TASK DEFINITIONS]
</listOfTasks>
\end{myXmlLst}
The \code{listOfTasks} is optional and may contain zero to many tasks. However, if the \LoneVtwo document contains  a  \code{DataGenerator}  element with at least one  \code{Variable} element, at least one  \code{Task} must be defined to which variable(s) in the   \code{DataGenerator} element refers -  see  section \ref{sec:taskReference} on \refpage{sec:taskReference}.

% TODO: clarify that you have concrete instances of AbstractTask in the list.
%
%The example shows the definition of one task \code{t1} in the \code{listOfTasks} container.

%%% Local Variables: 
%%% mode: latex
%%% TeX-master: "../sed-ml-L1V2"
%%% End: 
