\label{class:notes}

A \concept{note} is considered a  human-readable description of the element it is assigned to. It serves to display information to the user. 
Instances of the \concept{Notes} class may contain any valid XHTML \citep{P+02}, ranging from short comments to whole HTML pages for display in a Web browser. 
The namespace URL for \code{XHTML} content inside the \hyperref[class:notes]{Notes} class is \url{http://www.w3.org/1999/xhtml}. It may either be declared in the \hyperref[class:sed-ml]{\code{sedML} XML element}, or directly used in top level XHTML elements contained within the  \code{notes} element. For further options of how to set the namespace and detailed examples, please refer to \citep[p. 14]{HBH+10}.

\tabtext{notes}{Notes}
%
\begin{table}[ht]
\center
\begin{tabular}{|l|l|}
\hline
\textbf{\attribute} & \textbf{\desc}\\
\hline
xmlns: string \\{ ``http://www.w3.org/1999/xhtml" } & \refpage{sec:xmlns} \\
\hline
\hline
\textbf{\subelements} & \textbf{ }\\
\hline
\emph{well-formed content permitted in XHTML} & \\
\hline
\end{tabular}
\label{tab:notes}
\caption{\tabcap{Notes}}
\end{table}
%
\code{Notes}  does not have any further sub-elements defined in SED-ML, nor attributes associated with it.
%

\lsttext{notes}{notes}
%
\begin{myXmlLst}{The \element{notes} element}{lst:notes}
<sedML [..]>
 <notes >
  <p xmlns="http://www.w3.org/1999/xhtml">The enclosed simulation description shows the oscillating behaviour of 
     the Repressilator model using deterministic and stochastic simulators.</p>
 </notes>
</sedML>
\end{myXmlLst}
%
In this example, the namespace declaration is inside the \element{notes} element and the note is related to the \element{sedML} root element of the SED-ML file. A note may, however, occur inside \emph{any} SED-ML XML element, except \code{note} itself and \hyperref[class:annotation]{\code{annotation}}.

%%% Local Variables: 
%%% mode: latex
%%% TeX-master: "../sed-ml-L1V1"
%%% End: 
