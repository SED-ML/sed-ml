% ~~~~~~~~~~~~~~~~~~~~~~~~~~~~~~~~~~~~~~~~
%% OUTPUT
% ~~~~~~~~~~~~~~~~~~~~~~~~~~~~~~~~~~~~~~~~ 
\subsection{\element{Output}}
\label{class:output}

The abstract \concept{Ouput} class describes how the results of a simulation are presented. (\fig{output}). The available output classes are plots (\hyperref[class:plot2D]{2D plot} or \hyperref[class:plot3D]{3D plot}) and \hyperref[class:report]{reports}. The data used in the output plots or report is provided via \hyperref[class:dataGenerator]{dataGenerators}.

\sedfig{images/uml/output}{The SED-ML Output class. Note that ListOfCurves, Curve, ListOfSurfaces, Surface, ListOfDataSets, DataSet and DataGenerator are subclasses of \emph{SEDBase}; the respective inheritance connections are not shown in the figure.}{fig:output}

Note that even though the terms \code{Plot2D} and \code{Plot3D} are used, the exact type of plot is not specified. In other words, whether the 3D plot represents a surface plot, or three dimensional lines in space, cannot be distinguished by SED-ML SED-ML \currentLV alone.


% ~~~ PLOT2D ~~~
\subsubsection{\element{Plot2D}}
\label{class:plot2D}
The \concept{Plot2D} class is used for two dimensional plot outputs (\fig{output}). The \concept{Plot2D} contains a number of \hyperref[class:curve]{\code{Curve}} definitions, which define the curves which should be plotted in the the 2D plot.
%
\tabtext{plot2D}{plot2D}
\begin{table}[ht]
\center
\begin{tabular}{ll}
\toprule
\textbf{\attribute} & \textbf{\desc}\\
\midrule
metaid$^{o}$ & \refpage{sec:metaID}\\
id & \refpage{sec:id} \\
name$^{o}$ & \refpage{sec:name}\\
\midrule
\textbf{\subelements} & \textbf{\desc}\\
\midrule
notes$^{o}$ & \refpage{class:notes}\\
annotation$^{o}$ & \refpage{class:annotation}\\
\midrule
listOfCurves$^{o}$ & \refpage{class:curve}\\
\bottomrule
\end{tabular}
\caption{\tabcap{plot2D}}
\label{tab:plot2D}
\end{table}

\lsttext{listOfCurves}{listOfCurves}
The example shows the definition of a \element{Plot2D} containing one \hyperref[class:curve]{\code{Curve}} inside the \code{listOfCurves}.
\begin{myXmlLst}{The \code{plot2D} element with the nested \code{listOfCurves} element.}{lst:listOfCurves}
<plot2D>
	<listOfCurves>
		<curve>
			[CURVE DEFINITION]
		</curve>
	</listOfCurves>
</plot2D>
\end{myXmlLst}


% ~~~ PLOT3D ~~~
\subsubsection{\element{Plot3D}}
\label{class:plot3D}
The \code{Plot3D} class is used for three dimensional plot outputs (\fig{output}). The \code{Plot3D} contains a number of \hyperref[class:surface]{\code{Surface}} definitions. 
%
\tabtext{plot3D}{plot3D}
\begin{table}[ht]
\center
\begin{tabular}{ll}
\toprule
\textbf{\attribute} & \textbf{\desc}\\
\midrule
metaid$^{o}$ & \refpage{sec:metaID}\\
id & \refpage{sec:id} \\
name$^{o}$ & \refpage{sec:name}\\
\midrule
\textbf{\subelements} & \textbf{\desc}\\
\midrule
notes$^{o}$ & \refpage{class:notes}\\
annotation$^{o}$ & \refpage{class:annotation}\\
\midrule
listOfSurfaces$^{o}$ & \refpage{class:surface}\\
\bottomrule
\end{tabular}
\caption{\tabcap{plot3D}}
\label{tab:plot3D}
\end{table}

\lsttext{plot3D}{plot3D}
The example shows the definition of a \hyperref[class:surface]{\element{Surface}} for the three dimensional plot inside the \code{listOfSurfaces}.
\begin{myXmlLst}{The \code{plot3D} element with the nested \element{listOfSurfaces} element}{lst:plot3D}
<plot3D>
	<listOfSurfaces>
		<surface> 
			[SURFACE DEFINITION]
		</surface>
		[FURTHER SURFACE DEFINITIONS]
	</listOfSurfaces>
</plot3D>
\end{myXmlLst}


% ~~~ REPORT ~~~
\subsubsection{\element{Report}}
\label{class:report}
The \concept{Report} class defines a data table consisting of several single instances of the \hyperref[class:dataSet]{\element{DataSet}} class (\fig{output}). Its output returns the simulation result processed via \element{DataGenerators} in actual \emph{numbers}. The columns of the report table are defined by creating an instance of the \hyperref[class:dataSet]{DataSet} class for each column. 
%
\tabtext{report}{report}
\begin{table}[ht]
\center
\begin{tabular}{ll}
\toprule
\textbf{\attribute} & \textbf{\desc}\\
\midrule
metaid$^{o}$ & \refpage{sec:metaID}\\
id & \refpage{sec:id} \\
name$^{o}$ & \refpage{sec:name}\\
\midrule
\textbf{\subelements} & \textbf{\desc}\\
\midrule
notes$^{o}$ & \refpage{class:notes}\\
annotation$^{o}$ & \refpage{class:annotation}\\
\midrule
listOfDataSets$^{o}$ & \refpage{class:dataSet}\\
\bottomrule
\end{tabular}
\caption{\tabcap{report}}
\label{tab:report}
\end{table}

\lsttext{listOfDataSets}{listOfDataSets}

\begin{myXmlLst}{The \code{report} element with the nested \code{listOfDataSets} element}{lst:listOfDataSets}
<report>
	<listOfDataSets>
		<dataSet>
			[DATA REFERENCE]
		</dataSet>
	</listOfDataSets>
</report>
\end{myXmlLst}

The simulation result itself, i.\,e. concrete result numbers, are not stored in SED-ML, but the directive how to \emph{calculate} them from the output of the simulator is provided through the \concept{dataGenerator}.

The encoding of simulation results is not part of SED-ML \currentLV.

% ~~~~~~~~~~~~~~~~~~~~~~~~~~~~~~~~~~~~~~~~
% OUTPUT COMPONENTS
% ~~~~~~~~~~~~~~~~~~~~~~~~~~~~~~~~~~~~~~~~ 
\subsection{Output components}

% ~~~ CURVE ~~~
\subsubsection{\element{Curve}}
\label{class:curve}
One or more instances of the \concept{Curve} class define a 2D plot. A \concept{curve} needs a data generator reference to refer to the data that will be plotted on the x-axis, using the \concept{xDataReference}. A second data generator reference is needed to refer to the data that will be plotted on the y-axis, using the \concept{yDataReference} (\fig{output}). 
%
\tabtext{curve}{curve}
\begin{table}[ht]
\center
\begin{tabular}{ll}
\toprule
\textbf{\attribute} & \textbf{\desc}\\
\midrule
metaid$^{o}$ & \refpage{sec:metaID}\\
id & \refpage{sec:id} \\
name$^{o}$ & \refpage{sec:name}\\
\midrule
logX & \refpage{sec:logX}\\
xDataReference & \refpage{sec:xDataReference}\\
logY & \refpage{sec:logY}\\
yDataReference & \refpage{sec:yDataReference}\\
\midrule
\textbf{\subelements} & \textbf{\desc}\\
\midrule
notes$^{o}$ & \refpage{class:notes}\\
annotation$^{o}$ & \refpage{class:annotation}\\
\bottomrule
\end{tabular}
\caption{\tabcap{curve}}
\label{tab:curve}
\end{table}

\lsttext{curve}{curve}
In the example a single curve is created. Results for the x-axis are generated by the dataGenerator \code{dg1}, results for the y-axis are generated by the dataGenerator \code{dg2}. Both \code{dg1} and \code{dg2} need to be defined in the \hyperref[sec:listOfDataGenerators]{listOfDataGenerators}. The x-axis is plotted logarithmically.
\begin{myXmlLst}{The SED-ML \code{curve} element, defining the output curve showing the result of simulation for the referenced dataGenerators}{lst:curve}
<listOfCurves>
	<curve id="c1" name="v1 / time" xDataReference="dg1" yDataReference="dg2" logX="true" logY="false" />
</listOfCurves>
\end{myXmlLst}

\paragraph{\element{logX}}
\label{sec:logX}
\concept{logX} is a required attribute of the \hyperref[class:curve]{Curve} class and defines whether or not the data output on the x-axis is logarithmic. The data type of \concept{logX} is \code{boolean}. 
To make the output on the x-axis of a plot logarithmic, \concept{logX} must be set to ``true'', as shown in the sample Listing~\ref{lst:curve}.

\paragraph{\element{logY}}
\label{sec:logY}
\concept{logY} is a required attribute of the \hyperref[class:curve]{Curve} class and defines whether or not the data output on the y-axis is logarithmic. The data type of \concept{logY} is \code{boolean}. 
To make the output on the y-axis of a plot logarithmic, \concept{logY} must be set to ``true'', as shown in the sample Listing~\ref{lst:curve}. 

\paragraph{\element{xDataReference}}
\label{sec:xDataReference}
The \concept{xDataReference} is a mandatory attribute of the \hyperref[class:curve]{Curve} object. Its content refers to a dataGenerator ID which denotes the \hyperref[class:dataGenerator]{DataGenerator} object that is used to generate the output on the x-axis of a \hyperref[class:curve]{Curve} in a \hyperref[class:plot2D]{2D Plot}. 
The \concept{xDataReference} data type is \code{SIdRef}. However, the valid values for the \concept{xDataReference} are restricted to the IDs of already defined \hyperref[class:dataGenerator]{DataGenerator objects}.

\paragraph{\element{yDataReference}}
\label{sec:yDataReference}
The \concept{yDataReference} is a mandatory attribute of the \hyperref[class:curve]{Curve} object. Its content refers to a dataGenerator ID which denotes the \hyperref[class:dataGenerator]{DataGenerator} object that is used to generate the output on the y-axis of a \hyperref[class:curve]{Curve} in a \hyperref[class:plot2D]{2D Plot}.
The \concept{yDataReference} data type is \code{SIdRef}. However, the valid values for the \concept{yDataReference} are restricted to the IDs of already defined \hyperref[class:dataGenerator]{DataGenerator objects}.


% ~~~ SURFACE ~~~
\subsubsection{\element{Surface}}
\label{class:surface}
A \concept{surface} is a three-dimensional figure representing a (processed) simulation result (\fig{output}). \element{Surface} is a subclass of \hyperref[class:curve]{\element{Curve}} inheriting among others the elements \hyperref[sec:xDataReference]{\element{xDataReference}}, \hyperref[sec:yDataReference]{\element{yDataReference}}, \hyperref[sec:logX]{\element{logX}}, and \hyperref[sec:logY]{\element{logY}}.
 
Creating an instance of the \concept{Surface} class requires the definition of data on three different axis. The aforementioned \hyperref[sec:xDataReference]{\element{xDataReference}} and \hyperref[sec:yDataReference]{\element{yDataReference}} attributes define the \hyperref[class:dataGenerator]{dataGenerators} for the x- and y-axis of a surface. In addition, the \hyperref[sec:zDataReference]{zDataReference} attribute defines the output for the z-axis. All axes might be logarithmic or not. This can be specified through the \hyperref[sec:logX]{logX}, \hyperref[sec:logY]{logY}, and the \hyperref[sec:logZ]{logZ} attributes in the according dataReference elements.

\tabtext{surface}{surface}

\begin{table}[ht]
\center
\begin{tabular}{ll}
\toprule
\textbf{\attribute} & \textbf{\desc}\\
\midrule
metaid$^{o}$ & \refpage{sec:metaID}\\
id & \refpage{sec:id} \\
name$^{o}$ & \refpage{sec:name}\\
\midrule
logX & \refpage{sec:logX}\\
xDataReference & \refpage{sec:xDataReference}\\
logY & \refpage{sec:logY}\\
yDataReference & \refpage{sec:yDataReference}\\
logZ & \refpage{sec:logZ}\\
zDataReference & \refpage{sec:zDataReference}\\
\midrule
\textbf{\subelements} & \textbf{\desc}\\
\midrule
notes$^{o}$ & \refpage{class:notes}\\
annotation$^{o}$ & \refpage{class:annotation}\\
\bottomrule
\end{tabular}
\caption{\tabcap{surface}}
\label{tab:surface}
\end{table}

\lsttext{surface}{surface}
In the example a single surface is created. Results shown on the x-axis are generated by the data generator \code{dg1}, results on the y-axis by dataGenerator \code{dg2}, results on the z-axis by dataGenerator \code{dg3}. All used dataGenerators, i.e. \code{dg1}, \code{dg2} and \code{dg3}, must be defined in the \hyperref[sec:listOfDataGenerators]{listOfDataGenerators}.
\begin{myXmlLst}{The SED-ML \code{surface} element, defining the output showing the result of the referenced task}{lst:surface}
<listOfSurfaces>
	<surface id="s1" name="surface" xDataReference="dg1" yDataReference="dg2" zDataReference="dg3" 
		logX="true"  logY="false" logZ="false" />
	[FURTHER SURFACE DEFINITIONS]
</listOfSurfaces>
\end{myXmlLst}

\paragraph{\element{logZ}}
\label{sec:logZ}
\concept{logZ} is a required attribute of the \hyperref[class:surface]{Surface} class and defines whether or not the data output on the z-axis is logarithmic. The data type of \concept{logZ} is \code{boolean}.
To make the output on the z-axis of a surface plot logarithmic, \concept{logZ} must be set to ``true'', as shown in the sample Listing~\ref{lst:surface}.

\paragraph{\element{zDataReference}}
\label{sec:zDataReference}
The \concept{zDataReference} is a mandatory attribute of the \hyperref[class:surface]{Surface} object. Its content refers to a dataGenerator ID which denotes the \hyperref[class:dataGenerator]{DataGenerator} object that is used to generate the output on the z-axis of a \hyperref[class:plot3D]{3D Plot}.
The \concept{zDataReference} data type is \code{SIdRef}. However, the valid values for the \concept{zDataReference} are restricted to the IDs of already defined \hyperref[class:dataGenerator]{DataGenerator objects}.

%% ~~~ DATASET ~~~
\subsubsection{\element{DataSet}}
\label{class:dataSet}
The \concept{DataSet} class holds definitions of data to be used in the \hyperref[class:report]{Report} class (\fig{output}). DataSets are labeled references to instances of the \hyperref[class:dataGenerator]{DataGenerator} class.

\tabtext{dataSet}{dataSet}

\begin{table}[h!t]
\center
\begin{tabular}{ll}
\toprule
\textbf{\attribute} & \textbf{\desc}\\
\midrule
metaid$^{o}$ & \refpage{sec:metaID}\\
id & \refpage{sec:id} \\
name$^{o}$ & \refpage{sec:name}\\
\midrule
dataReference & \refpage{sec:dataReference1}\\
label & \refpage{sec:label}\\
\midrule
\textbf{\subelements} & \textbf{\desc}\\
\midrule
notes$^{o}$ & \refpage{class:notes}\\
annotation$^{o}$ & \refpage{class:annotation}\\
\bottomrule
\end{tabular}
\caption{\tabcap{dataSet}}
\label{tab:dataSet}
\end{table}

\paragraph{\element{label}}
\label{sec:label}
Each data set in a \hyperref[class:report]{Report} must have an unambiguous \concept{label}. A label is a human readable descriptor of a data set for use in a \hyperref[class:report]{report}. For example, for a tabular data set of time series results, the label could be the column heading. 

\paragraph{\element{dataReference}}
\label{sec:dataReference1}
The \concept{dataReference} attribute contains the ID of a \concept{dataGenerator} element and as such represents a link to it. The data produced by that particular data generator fills the according dataSet in the \hyperref[class:report]{report}.

\lsttext{dataSet}{dataSet}
The example shows the definition of a dataSet. The referenced dataGenerator \element{dg1} must be defined in the listOfDataGenerators.
\begin{myXmlLst}{The SED-ML \code{dataSet} element, defining a data set containing the result of the referenced task}{lst:dataSet}
<listOfDataSets>
	<dataSet id="d1" name="v1 over time" dataReference="dg1" label="_1">
</listOfDataSets>
\end{myXmlLst}
