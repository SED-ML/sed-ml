\label{class:plot3D}


% Fig: 2DPlot
\sedfig[width=0.7\textwidth]{plot3DClass.png}{The SED-ML Plot3D class}{fig:plot3D}
%


\tabtext{plot3D}{Plot3D}
%
\begin{table}[ht]
\center
\begin{tabular}{|l|l|}
\hline
\textbf{\attribute} & \textbf{\desc}\\
\hline
metaid & \refpage{sec:metaID}\\
id & \refpage{sec:id} \\
name & \refpage{sec:name}\\
\hline
\hline
\textbf{\subelements} & \textbf{\desc}\\
\hline
notes & \refpage{class:notes}\\
annotation & \refpage{class:annotation}\\
\hline
\end{tabular}
\label{tab:plot3D}
\caption{\tabcap{Plot3D}}
\end{table}
%

A 3D plot is defined similarly to a 2D plot, but having 3 axes. To every axis a data generator is assigned using the \concept{xDataReference}, \concept{yDataReference}, and the \concept{zDataReference}. As well, each axis can be defined as logarithmix (\concept{logX/logY/logZ} set to true). An example for a 3D plot with one defined surface is given in listing \ref{lst:listOfSurfaces}.
%
\begin{myXmlLst}{The listOfSurfaces element}{lst:listOfSurfaces}
<listOfSurfaces>
 <surface logX="true" logY="false" logZ="false" 
  xDataReference="datagenerator1" yDataReference="datagenerator2" 
  zDataReference="datagenerator3" /> 
</listOfSurfaces>
\end{myXmlLst}
%

%%% Local Variables: 
%%% mode: latex
%%% TeX-master: "../sed-ml-L1V1"
%%% End: 
