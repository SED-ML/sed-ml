\label{sec:range}
Most simulation types demand a range to be defined over which to perform the analysis. SED-ML defines three different types of ranges, i.\,e. \concept{uniform range}, \concept{vector range}, and \concept{functional range}.
A \code{uniformRange} expects evenly distributed time steps from a certain \code{start} to a certain \code{end} time, with a defined \code{numberOfPoints} to measure. An example is given in \lst{uniformRange}. %and Figure \ref{fig:uniformRange}.
\begin{myXmlLst}{The \code{uniformRange} element \alert{to be validated}}{lst:uniformRange}
</listOfSimulations>
 <timeCourse id="s1" name="time course definition for concentration of p" algorithm="KiSAO:ID">
  <uniformRange start="0" end="100" numberOfPoints="10" />
 </timeCourse>
</listOfSimulations>
\end{myXmlLst}

%\myfigure[width=0.5\textwidth]{images/uniformRange.pdf}{Time course with uniform range}{fig:uniformRange}
% to be redone in gnuplot!

A \code{vectorRange} defines a set of values, representing the time points at which the according entity shall be simulated. % can one say "simulated" here?
The construction in a SED-ML file is shown in \lst{functionalRange}. Each \code{value} element defines a time point in simulation at which the result of simulation shall be passed on for later use.
%
\begin{myXmlLst}{The \code{vectorRange} element \alert{to be validated}}{lst:vectorRange}
</listOfSimulations>
 <timeCourse id="s1" name="time course definition for concentration of p" algorithm="KiSAO:ID">
  <vectorRange>
   [list of values]
   <value> 1  </value>
   <value> 4  </value>  
   <value> 10 </value>
   <value> 23 </value>
   <value> 42 </value>
  </vectorRange>
 </timeCourse>
</listOfSimulations>
\end{myXmlLst}
% concrete example!

If the time points can be described by a particular function, the \code{functionalRange} should be used. The construction in a SED-ML file is shown in \lst{functionalRange}.
\begin{myXmlLst}{The \code{functionalRange} element \alert{to be validated}}{lst:functionalRange}
</listOfSimulations>
 <timeCourse id="s1" name="time course definition for concentration of p" algorithm="KiSAO:ID">
  <functionalRange>
   <function>
    [mathML expression]
   </function>
  </functionalRange>
 </timeCourse>
</listOfSimulations>
\end{myXmlLst}
% concrete example!

%%% Local Variables: 
%%% mode: latex
%%% TeX-master: "../sed-ml-L1V2"
%%% End: 

