% sed-ml Class
\subsection{\element{SED-ML} top level element}
\label{class:sed-ml}
Each SED-ML \LoneVtwo document has a main class called SED-ML which defines the document's structure and content (\fig{sed-mlMain}).
%
%\sedfig[width=0.35\textwidth]{sed-mlClass}{The SED-ML class}{fig:sed-ml}
%
It consists of several parts; the parts are all connected to the SED-ML class through aggregation: 
the \hyperref[class:model]{Model} class (for model specification, see Section~\ref{class:model}), the \hyperref[class:simulation]{Simulation} class (for simulation setup specification, see Section~\ref{class:simulation}), the \hyperref[class:task]{Task} class (for the linkage of models and simulation setups, see Section~\ref{class:task}), the \hyperref[class:dataGenerator]{DataGenerator} class (for the definition of post-processing, see Section~\ref{class:dataGenerator}), and the \hyperref[class:output]{Output} class (for the output specification, see Section~\ref{class:output}). All of them are shown in \fig{sed-mlMain} and will be explained in more detail in the relevant sections of this document.
%
\sedfig[width=0.8\textwidth]{sed-mlMain}{The sub-classes of SED-ML}{fig:sed-mlMain}
%

\tabtext{sed-ml}{SED-ML}

%
\begin{table}[ht]
\center
\begin{tabular}{|l|l|}
\hline
\textbf{\attribute} & \textbf{\desc}\\
\hline
metaID$^{o}$ & \refpage{sec:metaID}\\
xmlns & \refpage{sec:xmlns}\\
level & \refpage{sec:level}\\
version & \refpage{sec:version}\\
\hline
\hline
\textbf{\subelements} & \textbf{\desc}\\
\hline
notes$^{o}$ & \refpage{class:notes}\\
annotation$^{o}$ & \refpage{class:annotation}\\
listOfModels$^{o}$ & \refpage{sec:listOfModels}\\
listOfSimulations$^{o}$ & \refpage{sec:listOfSimulations} \\
listOfTasks$^{o}$ & \refpage{sec:listOfTasks} \\
listOfDataGenerators$^{o}$ & \refpage{sec:listOfDataGenerators} \\
listOfOutputs$^{o}$ & \refpage{sec:listOfOutputs} \\
\hline
\end{tabular}
\caption{\tabcap{SED-ML}}
\label{tab:sed-ml}
\end{table}
%
A SED-ML document needs to have the SED-ML namespace defined through the mandatory \hyperref[sec:xmlns]{xmlns} attribute. In addition, the SED-ML \hyperref[sec:level]{level} and \hyperref[sec:version]{version} attributes are mandatory.

The basic XML structure of a SED-ML file is shown in listing  \vref{lst:sedmlRoot}.
%
\begin{myXmlLst}{The SED-ML root element}{lst:sedmlRoot}
<?xml version="1.0" encoding="utf-8"?>
<sedML xmlns:math="http://www.w3.org/1998/Math/MathML" 
       xmlns="http://sed-ml.org/" level="1" version="1">
 <listOfModels />
  [MODEL REFERENCES AND APPLIED CHANGES]
 <listOfSimulations />
  [SIMULATION SETUPS]
 <listOfTasks />
  [MODELS LINKED TO SIMULATIONS]
 <listOfDataGenerators />
  [DEFINITION OF POST-PROCESSING]
 <listOfOutputs />
  [DEFINITION OF OUTPUT]
</sedML>
\end{myXmlLst}
%
The root element of each SED-ML XML file is the \code{sedML} element, encoding \hyperref[sec:version]{version} and \hyperref[sec:level]{level} of the file, and setting the necessary namespaces. Nested inside the \code{sedML} element are the five lists serving as containers for the encoded data (\concept{listOfModels} for all models, \concept{listOfSimulations} for all simulations, \concept{listOfTasks} for all tasks, \concept{listOfDataGenerators} for all post-processing definitions, and \concept{listOfOutputs} for all output definitions).

\subsubsection{\element{xmlns}}
\label{sec:xmlns}
The \concept{xmlns} attribute declares the namespace for the SED-ML document. The pre-defined namespace for SED-ML documents is \url{http://sed-ml.org/}. 

In addition, SED-ML makes use of the \concept{MathML} namespace \url{http://www.w3.org/1998/Math/MathML} to enable the encoding of mathematical expressions in MathML 2.0. SED-ML uses a subset of MathML as described in Section~\ref{sec:mathML} on page \pageref{sec:mathML}.

SED-ML \concept{notes} use the XHTML namespace \url{http://www.w3.org/1999/xhtml}.  The \hyperref[class:notes]{Notes} class is described in Section~\ref{class:notes} on page \pageref{class:notes}.

Additional external namespaces might be used in \hyperref[class:annotation]{annotations}. 

%All namespace declarations will be omitted in the following SED-ML sample code snippets.

%%% Local Variables: 
%%% mode: latex
%%% TeX-master: "../sed-ml-L1V2"
%%% End: 


\subsubsection{\element{level}}
\label{sec:level}

The current SED-ML \concept{level} is ``level \level''. Major revisions containing substantial changes will lead to the definition of forthcoming levels.

The level attribute is \code{required} and its value is a \code{fixed} decimal. For SED-ML \currentLV the value is set to \code{1}, as shown in the example in Listing~\ref{lst:sedmlRoot}.

%%% Local Variables: 
%%% mode: latex
%%% TeX-master: "../sed-ml-L1V3"
%%% End: 


\subsubsection{\element{version}}
\label{sec:version}
The current SED-ML \concept{version} is ``version \version''. Minor revisions containing corrections and refinements of SED-ML elements will lead to the definition of forthcoming versions.

The version attribute is \code{required} and its value is a \code{fixed} decimal. For SED-ML \LoneVtwo the value is set to \code{1}, as shown in the example in Listing~\ref{lst:sedmlRoot}.




%%% Local Variables: 
%%% mode: latex
%%% TeX-master: "../sed-ml-L1V2"
%%% End: 
