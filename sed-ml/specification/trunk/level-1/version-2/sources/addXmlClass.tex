% ChangeAttribute Class
  \subsubsection{\element{AddXML}}
\label{class:addXml}
The \concept{AddXML} class specifies a snippet of XML that is to be added as a child of the element selected by the XPath expression in the \hyperref[sec:target]{target} attribute (\fig{addXMLClass}).
The new piece of XML code is provided by the \hyperref[sec:newXml]{NewXML} class.
%
\sedfig[width=0.75\textwidth]{addXMLClass}{The SED-ML \code{AddXML} class}{fig:addXMLClass}
%

\tabtext{addXml}{addXml}
%
\begin{table}[ht]
\center
\begin{tabular}{|l|l|}
\hline
\textbf{\attribute} & \textbf{\desc}\\
\hline
metaid$^{o}$ & \refpage{sec:metaID}\\
id & \refpage{sec:id} \\
name$^{o}$ & \refpage{sec:name}\\
target & \refpage{sec:target}\\
\hline
\hline
\textbf{\subelements} & \textbf{\desc}\\
\hline
notes$^{o}$ & \refpage{class:notes}\\
annotation$^{o}$ & \refpage{class:annotation}\\
\hline
newXML & \refpage{sec:newXml}\\
\hline
\end{tabular}
\caption{\tabcap{addXML}}
\label{tab:addXml}
\end{table}
%

An example for a change that adds an additional parameter to a model is given in listing \ref{lst:addXML}.
%
\begin{myXmlLst}{The \code{addXML} element with its \code{newXML} sub-element}{lst:addXML}
<model language="urn:sedml:language:sbml" [..]>
 <listOfChanges>
  <addXML target="/sbml:sbml/sbml:model/sbml:listOfParameters" >
   <newXML>
     <parameter metaid="metaid_0000010" id="V_mT" value="0.7" />
  </newXML>
  </addXML>
 </listOfChanges>
</model>
\end{myXmlLst}
%

The code of the model is changed so that a parameter with ID \code{V\_mT} is added to its list of parameters. The \code{newXML} element adds an additional XML element to the original model. The element's name is \code{parameter} and it is added to the existing parent element \code{listOfParameters} that is addressed by the XPath expression in the \code{target} attribute.

%%% Local Variables: 
%%% mode: latex
%%% TeX-master: "../sed-ml-L1V2"
%%% End: 
