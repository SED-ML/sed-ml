\label{sec:id}
%

Most objects in SED-ML carry an \concept{id} attribute. 
The \hyperref[sec:id]{id} attribute, if existent for an object, is always required and identifies SED-ML constituents unambiguously.  
It is used to refer to a constituent from other constituents. 
The \code{id} data type is \code{String}. All \code{id}s have a global scope, i.\,e. the \code{id} must be unambiguous throughout a whole SED-ML document. As such it identifies the constituent it is related to.
An example for a defined \code{id} is given in listing \ref{lst:id}.
%
\begin{myXmlLst}{SED-ML identifier definition, e.\,g. for a model}{lst:id}
<model id="m00001" language="urn:sedml:language:sbml" source="urn:miriam:biomodels.db:BIOMD0000000012">
 [MODEL DEFINITION]
</model>
\end{myXmlLst}
%
The defined model carries the ID \code{m00001}. If the model is used somewhere else in the SED-ML document, it is referred to by that ID.

%%% Local Variables: 
%%% mode: latex
%%% TeX-master: "../sed-ml-L1V1"
%%% End: 
