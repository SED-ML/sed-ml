% XPath
\subsection{XPath usage}  

\label{sec:xpath} 
XPath is a language for finding information in an XML document \citep{xpath:1999}. Within \LoneVone, XPath version 1 expressions are  used to identify nodes and attributes within an XML representation of a biological model in the following ways:
%
\begin{enumerate}
\item {Within a \hyperref[class:variable]{Variable} definition, where XPath identifies the model variable required for manipulation in SED-ML.}
\item {Within a  \hyperref[class:change]{Change} definition, where XPath is used to identify the target XML to which a change should be applied.} 

\end{enumerate}

For proper application, XPath expressions should contain prefixes that allow their resolution to the correct XML namespace within an XML document. For example, the XPath expression referring to a species \emph{X} in an SBML model:
\begin{alltt}
/sbml:sbml/sbml:model/sbml:listOfSpecies/sbml:species[@id=`X'] {\color{green} \tickYes -\emph{CORRECT}}
\end{alltt}
is preferable to 
\begin{alltt}
/sbml/model/listOfSpecies/species[@id=`X'] {\color{red} \tickNo -\emph{INCORRECT} }
\end{alltt}

which will only be interpretable by standard XML software tools  if the SBML file declares no namespaces. 




%%% Local Variables: 
%%% mode: latex
%%% TeX-master: "../sed-ml-L1V1"
%%% End: 
