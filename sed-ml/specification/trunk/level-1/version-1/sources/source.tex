\subsubsection{\element{source}}
\label{sec:source}
To make a model available for the execution of a SED-ML file, the model \element{source} must be specified through a URI. 
The URI should preferably point to a public, consistent location that provides the model description file and follows the proposed \hyperref[sec:uriScheme]{URI Scheme}.
References to curated, open model bases are recommended, such as the BioModels Database. However, any resource registered with MIRIAM resources\footnote{\url{http://www.ebi.ac.uk/miriam/main/}} can easily be referenced. Even without a MIRIAM URN, SED-ML can be used (see again section \ref{sec:modelURI} on \refpage{sec:modelURI}).

An example for the definition of a model, and using the  \hyperref[sec:uriScheme]{URI scheme} is given in listing \ref{lst:sourceA}.
%
\begin{myXmlLst}{The SED-ML \code{source} element, using the URI scheme}{lst:sourceA}
 <model id="m1" name="repressilator" language="urn:sedml:language:sbml" 
  source="urn:miriam:biomodels.db:BIOMD0000000012">
  <listOfChanges>
   [MODEL PRE-PROCESSING]
  </listOfChanges>
 </model>
\end{myXmlLst}
%
The example defines one model \code{m1}. \code{urn:miriam:biomodels.db:BIOMD0000000012} defines the source of the model code. The MIRIAM URN can be resolved into the SBML model stored in BioModels Database under ID \element{BIOMD0000000012} using the MIRIAM web service. The resulting URL is \url{http://www.ebi.ac.uk/biomodels-main/BIOMD0000000012}.

An example for the definition of a model and using a URL is given in listing \ref{lst:sourceB}.
%
\begin{myXmlLst}{The SED-ML \code{source} element, using a URL}{lst:sourceB}
 <model id="m1" name="repressilator" language="urn:sedml:language:cellml" 
  source="http://models.cellml.org/exposure/bba4e39f2c7ba8af51fd045463e7bdd3/aguda_b_1999.cellml">
  <listOfChanges />
 </model>
\end{myXmlLst}
%
In the example one model is defined. The \element{language} of the model is \element{CellML}. As the CellML model repository currently does not provide a MIRIAM URI for model reference, the \emph{URL} pointing to the model code is used to refer to the model. The URL is given in the \element{source} attribute.
