% sed-ml Class
\label{class:sed-ml}
Each SED-ML \LoneVone document has a main class called SED-ML which defines the document's structure and content (\fig{sed-ml}).
%
\sedfig[width=0.4\textwidth]{sed-mlClass}{The SED-ML class}{fig:sed-ml}
%
A SED-ML document needs to have the SED-ML namespace defined through the mandatory \hyperref[sec:xmlns]{xmlns} attribute. In addition, the SED-ML \hyperref[sec:level]{level} and \hyperref[sec:version]{version} attributes are mandatory.

The SED-ML document consists of several parts which are all connected to the SED-ML class through aggregation: 
the \hyperref[class:model]{Model} class (for model specification, see section \ref{class:model}), the \hyperref[class:simulation]{Simulation} class (for simulation setup specification, see section \ref{class:simulation}), the \hyperref[class:task]{Task} class (for the linkage of models and simulation setups, see section \ref{class:task}), the \hyperref[class:dataGenerator]{DataGenerator} class (for the definition of post-processing, see section \ref{class:dataGenerator}), and the \hyperref[class:output]{Output} class (for the output specification, see section \ref{class:output}). All of them are shown in \fig{sed-mlMain} and will be explained in more detail in the according sections of this document.
%
\sedfig[width=0.8\textwidth]{sed-mlMain}{The sub-classes of SED-ML}{fig:sed-mlMain}
%

\tabtext{sed-ml}{SED-ML}
%
\begin{table}[ht]
\center
\begin{tabular}{|l|l|}
\hline
\textbf{\attribute} & \textbf{\desc}\\
\hline
metaID$^{o}$ & \refpage{sec:metaID}\\
xmlns & \refpage{sec:xmlns}\\
level & \refpage{sec:level}\\
version & \refpage{sec:version}\\
\hline
\hline
\textbf{\subelements} & \textbf{\desc}\\
\hline
notes$^{o}$ & \refpage{class:notes}\\
annotation$^{o}$ & \refpage{class:annotation}\\
model$^{o}$ & \refpage{class:model}\\
simulation$^{o}$ & \refpage{class:simulation} \\
task$^{o}$ & \refpage{class:task} \\
dataGenerator$^{o}$ & \refpage{class:dataGenerator} \\
output$^{o}$ & \refpage{class:output} \\
\hline
\end{tabular}
\label{tab:sed-ml}
\caption{\tabcap{SED-ML}}
\end{table}
%

The basic XML structure of a SED-ML file is shown in listing  \ref{lst:sedmlRoot}.
%
\begin{myXmlLst}{The SED-ML root element}{lst:sedmlRoot}
<?xml version="1.0" encoding="utf-8"?>
<sedML xmlns:math="http://www.w3.org/1998/Math/MathML" 
       xmlns="http://www.biomodels.net/sed-ml" level="1" version="1">
 <listOfModels />
  [MODEL REFERENCES AND APPLIED CHANGES]
 <listOfSimulations />
  [SIMULATION SETUPS]
 <listOfTasks />
  [MODELS LINKED TO SIMULATIONS]
 <listOfDataGenerators />
  [DEFINITION OF POST-PROCESSING]
 <listOfOutputs />
  [DEFINITION OF OUTPUT]
</sedML>
\end{myXmlLst}
%
The root element of each SED-ML XML file is the \code{sedML} element, encoding \hyperref[sec:version]{version} and \hyperref[sec:level]{level} of the file, and setting the necessary namespaces. Nested inside the \code{sedML} element are the five lists serving as containers for the encoded data (\concept{listOfModels} for all models, \concept{listOfSimulations} for all simulations, \concept{listOfTasks} for all tasks, \concept{listOfDataGenerators} for all post-processing definitions, and \concept{listOfOutputs} for all output definitions).

\subsubsection{The \element{level} attribute}
\label{sec:level}

The current SED-ML \concept{level} is  \emph{level \level}. Major revisions containing substantial changes will lead to the definition of forthcoming levels.

The level attribute is \code{required} and its value is a \code{fixed} decimal. For SED-ML \LoneVone the value is set to \code{1}, as shown in the example in listing \ref{lst:sedmlRoot}.

\subsubsection{The \element{version} attribute}
\label{sec:version}
The current SED-ML \concept{version} is \emph{version \version}. Minor revisions containing corrections and refinements of SED-ML elements will lead to the definition of forthcoming versions.

The version attribute is \code{required} and its value is a \code{fixed} decimal. For SED-ML \LoneVone the value is set to \code{1}, as shown in the example in listing \ref{lst:sedmlRoot}.



%%% Local Variables: 
%%% mode: latex
%%% TeX-master: "../sed-ml-L1V1"
%%% End: 
