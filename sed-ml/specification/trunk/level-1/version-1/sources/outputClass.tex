% Output 
\label{class:output}

The \concept{Ouput} class describes how the results of a simulation should be presented to the user (\fig{sedOutput}). 
%
\sedfig{outputClass}{The SED-ML Output class}{fig:sedOutput}
%
It does not contain the data itself, but the type of output and the \hyperref[class:dataGenerator]{data generators} used to produce a particular output.

The types of output pre-defined in SED-ML \LoneVone are plots and \hyperref[class:report]{reports}. The output can be defined as a \hyperref[class:plot2D]{2D plot} or alternatively as a \hyperref[class:plot3D]{3D plot}. 

Note that even though the terms ``2D plot" and ``3D plot" are used, the exact type of plot is not specified. In other words, whether the 3D plot represents a surface plot, or three dimensional lines in space, cannot be distinguished by SED-ML alone. It is expected that applications use \hyperref[class:annotation]{annotations} for this purpose.

\tabtext{output}{output}
%
\begin{table}[ht]
\center
\begin{tabular}{|l|l|}
\hline
\textbf{\attribute} & \textbf{description}\\
\hline
metaid$^{o}$ & \refpage{sec:metaID}\\
id & \refpage{sec:id} \\
name$^{o}$ & \refpage{sec:name}\\
\hline
\hline
\textbf{sub-elements} & \textbf{description}\\
\hline
notes$^{o}$ & \refpage{class:notes}\\
annotation$^{o}$ & \refpage{class:annotation}\\
\hline
plot2D$^{o}$ & \refpage{class:plot2D}\\
plot3D$^{o}$ & \refpage{class:plot3D}\\
report$^{o}$ & \refpage{class:report}\\
\hline
\end{tabular}
\label{tab:output}
\caption{\tabcap{output}}
\end{table}
%


%An example for a 2D plot and a report defined in one output is given in listing \ref{lst:listOfOutputs}.
%
%\begin{myXmlLst}{The listOfOutput element}{lst:listOfOutputs}
%<listOfOutputs>
% <plot2D id="plot1" name="sample 2D plot">
%  <listOfCurves> 
%   \alert{[..]}
%  </listOfCurves>
% </plot2D>
%</listOfOutputs>
%\end{myXmlLst}
%


%%% Local Variables: 
%%% mode: latex
%%% TeX-master: "../sed-ml-L1V1"
%%% End: 
