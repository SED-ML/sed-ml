\pagebreak
\chapter{Concepts used in SED-ML}
\label{chp:concepts}
% ~~~~~~~~~~~~~~~~~~~~~~~~~~~~~~~~~~~~
% MATHML
% ~~~~~~~~~~~~~~~~~~~~~~~~~~~~~~~~~~~~
\section{MathML}
\label{sec:mathML}
SED-ML encodes mathematical expressions using a subset of \concept{MathML} 2.0 \citep{CIM+01}. \concept{MathML} is an international standard for encoding mathematical expressions using XML. It is also used as a representation of mathematical expressions in other formats, such as SBML and CellML, two of the model languages supported by SED-ML. 

SED-ML files can use mathematical expressions to encode for example pre-processing steps applied to the computational model (\hyperref[class:computeChange]{ComputeChange}), or post processing steps applied to the raw simulation data before output (\hyperref[class:dataGenerator]{DataGenerator}). 

SED-ML classes reference \concept{MathML} expressions via the element \hyperref[sec:math]{\element{math}} of data type \hyperref[type:mathml]{\element{MathML}}.

% ~~~ MATHML ELEMENTS ~~~
\subsection{MathML elements}
The allowed MathML in SED-ML is restricted to the following subset: 

\begin{itemize}\setlength{\parskip}{-0.1ex}

\item \emph{token}: \token{cn}, \token{ci}, \token{csymbol},
  \token{sep}
  
\item \emph{general}: \token{apply}, \token{piecewise},
  \token{piece}, \token{otherwise}, \token{lambda} 

\item \emph{relational operators}: \token{eq}, \token{neq},
  \token{gt}, \token{lt}, \token{geq}, \token{leq}

\item \emph{arithmetic operators}: \token{plus}, \token{minus},
  \token{times}, \token{divide}, \token{power}, \token{root},
  \token{abs}, \token{exp}, \token{ln}, \token{log},
  \token{floor}, \token{ceiling}, \token{factorial}

\item \emph{logical operators}: \token{and}, \token{or},
  \token{xor}, \token{not}

\item \emph{qualifiers}: \token{degree}, \token{bvar},
  \token{logbase}

\item \emph{trigonometric operators}: \token{sin}, \token{cos},
  \token{tan}, \token{sec}, \token{csc}, \token{cot},
  \token{sinh}, \token{cosh}, \token{tanh}, \token{sech},
  \token{csch}, \token{coth}, \token{arcsin}, \token{arccos},
  \token{arctan}, \token{arcsec}, \token{arccsc}, \token{arccot},
  \token{arcsinh}, \token{arccosh}, \token{arctanh},
  \token{arcsech}, \token{arccsch}, \token{arccoth}

\item \emph{constants}: \token{true}, \token{false},
  \token{notanumber}, \token{pi}, \token{infinity},
  \token{exponentiale}

\item \emph{MathML annotations}: \token{semantics},
  \token{annotation}, \token{annotation-xml}
\end{itemize}


% ~~~ MATHML SYMBOLS ~~~
\subsection{MathML symbols}
All the operations listed above only operate on \emph{scalar} values. However, as one of SED-ML's aims is to provide post processing on the results of simulation experiments, this basic set needs to be extended by some aggregate functions. Therefore a defined set of \concept{MathML symbols} that represent vector values are supported by SED-ML. The only allowed symbols to be used in aggregate functions are the identifiers of \hyperref[class:variable]{Variables} defined in the \hyperref[sec:listOfVariables]{\element{listOfVariables}} of \hyperref[class:dataGenerator]{DataGenerators}. These \hyperref[class:variable]{Variables} represent the data collected from the simulation experiment in the associated \hyperref[class:task]{Task}. 


% ~~~ MATHML FUNCTIONS ~~~
\subsection{MathML functions}
The only aggregate \concept{MathML functions} available in SED-ML are \hyperref[fun:min]{\element{min}}, \hyperref[fun:max]{\element{max}}, \hyperref[fun:sum]{\element{sum}}, and \hyperref[fun:product]{\element{product}}. These represent the only exceptions. At this point SED-ML does not define a complete algebra of vector values.

\subsubsection*{\element{min}}
\label{fun:min}
The \concept{\element{min}} of a variable represents the smallest value the simulation experiment for that variable (Listing~\ref{lst:minFunction}). 
\begin{myXmlLst}{Example for the use of the MathML \code{min} function.}{lst:minFunction}
<apply>
 	<csymbol encoding="text" definitionURL="http://sed-ml.org/#min">
 		min
 	</csymbol>
 	<ci> variableId </ci>
</apply>
\end{myXmlLst}

\subsubsection*{\element{max}}
\label{fun:max}
The \concept{\element{max}} of a variable represents the largest value the simulation experiment for that variable (\lst{maxFunction}).
\begin{myXmlLst}{Example for the use of the MathML \code{max} function.}{lst:maxFunction}
<apply>
 	<csymbol encoding="text" definitionURL="http://sed-ml.org/#max">
 		max
 	</csymbol>
 	<ci> variableId </ci>
</apply>
\end{myXmlLst}

\subsubsection*{\element{sum}}
\label{fun:sum}
The \concept{\element{sum}} of a variable represents the sum of all values of the variable returned by the simulation experiment (\lst{sumFunction}).
\begin{myXmlLst}{Example for the use of the MathML \code{sum} function.}{lst:sumFunction}
<apply>
 	<csymbol encoding="text" definitionURL="http://sed-ml.org/#sum">
 		sum
 	</csymbol>
 	<ci> variableId </ci>
</apply>
\end{myXmlLst}

\subsubsection*{\element{product}}
\label{fun:product}
The \concept{\element{product}} of a variable represents the multiplication of all values of the variable returned by the simulation experiment (\lst{productFunction}).
\begin{myXmlLst}{Example for the use of the MathML \code{product} function.}{lst:productFunction}
<apply>
 	<csymbol encoding="text" definitionURL="http://sed-ml.org//#product">
 		product
 	</csymbol>
 	<ci> variableId </ci>
</apply>
\end{myXmlLst}
  
  
% ~~~~~~~~~~~~~~~~~~~~~~~~~~~~~~~~~~~~
% URI SCHEME
% ~~~~~~~~~~~~~~~~~~~~~~~~~~~~~~~~~~~~
\section{URI scheme}  
\label{sec:uriScheme}
URIs are used in SED-ML as a mechanism
\begin{itemize}
	\item to reference models (\ref{sec:modelURI} \hyperref[sec:modelURI]{Model references})
	\item to reference data files (\ref{sec:dataURI} \hyperref[sec:dataURI]{Data references})
	\item to specify the language of the referenced model (\ref{sec:languageURI} \hyperref[sec:languageURI]{Language references})
	\item to specify the format of the referenced dataset (\ref{sec:dataFormatURI} \hyperref[sec:dataFormatURI]{Data format references})
	\item to enable addressing implicit model variables (\ref{sec:implicitVariable}  \hyperref[sec:implicitVariable]{Symbols})
\end{itemize}

In addition, annotation of SED-ML elements should use a standardised URI \hyperref[sec:annotations]{Annotations Scheme} to ensure long-time availability of information that can unambiguously be identified.


% ~~~ MODEL REFERENCES ~~~
\subsection{Model references}
\label{sec:modelURI}
The URI of a \hyperref[class:model]{model} should preferably point to a public, consistent location that provides the model description file. References to curated, open model bases are recommended, such as the BioModels Database. However, any resource registered with MIRIAM resources\footnote{\url{http://www.ebi.ac.uk/miriam/main/}} can easily be referenced.

One way for referencing a model from a SED-ML file is adopted from the \concept{MIRIAM URI Scheme}. MIRIAM enables identification of a data resource (in this case a model resource) by a predefined URN. A data entry inside that resource is identified by an ID. That way each single  model in a particular model repository can be unambiguously referenced. One model repository that is part of MIRIAM resources is the \concept{BioModels Database} \citep{LDR+10}. Its data resource name in MIRIAM is \code{urn:miriam:biomodels.db}. To refer to a particular model, a standardised identifier scheme is defined in \concept{MIRIAM Resources}\footnote{\url{http://www.ebi.ac.uk/miriam/}}. The ID entry maps to a particular model in the model repository. That model is never deleted. A sample BioModels Database ID is \code{BIOMD0000000048}. Together with the data resource name it becomes unambiguously identifiable by the URN \code{urn:miriam:biomodels.db:BIOMD0000000048}. 

SED-ML does not specify how to resolve the URNs. However, MIRIAM Resources offers web services to do so\footnote{\url{http://www.ebi.ac.uk/miriam/}}. For the above example of the \code{urn:miriam:biomodels.db:BIOMD0000000048} model, the resolved URL may look like \code{\url{http://www.ebi.ac.uk/biomodels-main/BIOMD0000000048}}.

For additional information see the \hyperref[sec:model_source]{\element{source}} attribute on \hyperref[class:model]{Model}.

An alternative means to obtain a model may be to provide a single resource containing necessary models and a SED-ML file. Although a specification of such a resource is beyond the scope of this document, the recommended means is the \hyperref[sec:archive]{COMBINE archive}.


% ~~~ DATA REFERENCES ~~~
\subsection{Data references}
\label{sec:dataURI}
One way for referencing a data file from a SED-ML file is adopted from the \concept{MIRIAM URI Scheme}. MIRIAM enables identification of a data resource by a predefined URN. 

For additional information see the \hyperref[sec:data_source]{\element{source}} attribute on \hyperref[class:dataDescription]{DataDescription}.

An alternative means to obtain a data file may be to provide a single resource containing necessary data files and the SED-ML file is the \hyperref[sec:archive]{COMBINE archive}. 


% ~~~ LANGUAGE REFERENCES ~~~
\subsection{Language references}
\label{sec:languageURI}
The evaluation of a SED-ML document is required in order for software to decide whether or not it can be used in a particular simulation environment. One crucial criterion is the particular model representation language used to encode the \hyperref[class:model]{model}. A simulation software usually only supports a small subset of the representation formats available to model biological systems computationally. 

To help  software decide whether or not it supports a SED-ML description file, the information on the \hyperref[class:model]{model} encoding for each referenced \hyperref[class:model]{model} can be provided through the \hyperref[sec:language]{\element{language}} attribute, as the description of a language name and version through an unrestricted \code{String} is error-prone. 
A prerequisite for a language to be fully supported by SED-ML is that a formalised language definition, e.g., an XML Schema, is provided online. SED-ML also defines a set of standard URIs to refer to particular language definitions. 

To specify the language a model is encoded in, a set of pre-defined SED-ML URNs can be used (\tab{languageURI}). The structure of SED-ML language URNs is \element{urn:sedml:language:}\emph{\element{name.version}}. SED-ML allows to specify a model representation format very generally as being \code{XML}, if no standardised representation format has been used to encode the model. On the other hand, one can be as specific as defining a model being in a particular version of a language, e.g., SBML Level 3 Version 1 as \code{urn:sedml:language:sbml.level-3.version-1}.

For additional information see the \hyperref[sec:language]{\element{language}} attribute on \hyperref[class:model]{Model}.

\begin{table}[ht]
\center
\begin{tabular}{p{5cm}p{10cm}}
\toprule
\textbf{Language} & \textbf{URN}\\
\midrule
CellML (generic) & \code{urn:sedml:language:cellml} \\
CellML 1.0 & \code{urn:sedml:language:cellml.1\_0} \\
CellML 1.1 & \code{urn:sedml:language:cellml.1\_1} \\
NeuroML (generic) & \code{urn:sedml:language:neuroml} \\
NeuroML Version 1.8.1 Level 1 &	\code{urn:sedml:language:neuroml.version-1\_8\_1.level-1} \\
NeuroML Version 1.8.1 Level 2 &	\code{urn:sedml:language:neuroml.version-1\_8\_1.level-2} \\
SBML (generic) & \code{urn:sedml:language:sbml} \\
SBML Level 1 Version 1 & \code{urn:sedml:language:sbml.level-1.version-1} \\
SBML Level 1 Version 2 & \code{urn:sedml:language:sbml.level-1.version-2} \\
SBML Level 2 Version 1 & \code{urn:sedml:language:sbml.level-2.version-1} \\
SBML Level 2 Version 2 & \code{urn:sedml:language:sbml.level-2.version-2} \\
SBML Level 2 Version 3 & \code{urn:sedml:language:sbml.level-2.version-3} \\
SBML Level 2 Version 4 & \code{urn:sedml:language:sbml.level-2.version-4} \\
SBML Level 3 Version 1 & \code{urn:sedml:language:sbml.level-3.version-1} \\
SBML Level 3 Version 2 & \code{urn:sedml:language:sbml.level-3.version-2} \\
VCML (generic) & \code{urn:sedml:language:vcml} \\
\bottomrule
\end{tabular}
\caption{Predefined model language URNs. The latest list of language URNs is available from \url{http://sed-ml.org/}.}
\label{tab:languageURI}
\end{table}

% ~~~ DATA FORMAT REFERENCES ~~~
\subsection{Data format references}
\label{sec:dataFormatURI}
To help software decide whether or not it supports a SED-ML file, the information on the \hyperref[class:dataDescription]{dataDescription} encoding for each referenced \hyperref[class:dataDescription]{dataDescription} can be provided through the \hyperref[sec:format]{\element{format}} attribute.

To specify the format of a \hyperref[class:dataDescription]{dataDescription}, a set of pre-defined SED-ML URNs can be used (\tab{dataFormatURI}). The structure of SED-ML format URNs is \element{urn:sedml:format:}\emph{\element{name.version}}. 

If it is not explicitly defined the default value for \element{format} is \code{urn:sedml:format:numl}, referring to NuML representation of the data. However, the use of the \element{format} attribute is strongly encouraged.

For additional information see the \hyperref[sec:format]{\element{format}} attribute on \hyperref[class:dataDescription]{DataDescription} and the description of individual formats and their use in SED-ML below.

\begin{table}[ht]
\center
\begin{tabular}{p{5cm}p{10cm}}
\toprule
\textbf{Data Format} & \textbf{URN}\\
\midrule
NuML (generic) & \code{urn:sedml:format:numl} \\
NuML Level 1 Version 1 & \code{urn:sedml:format:numl.level-1.version-1} \\
CSV & \code{urn:sedml:format:csv} \\
TSV & \code{urn:sedml:format:tsv} \\
\bottomrule
\end{tabular}
\caption{Predefined dataDescription format URNs. The latest list of format URNs is available from \url{http://sed-ml.org/}.}
\label{tab:dataFormatURI}
\end{table}

\subsubsection{NuML (Numerical Markup Language)}
\label{sec:dataFormatNUML}
\hyperref[sec:numl]{NuML} is an exchange format for numerical data. Data in the \concept{NuML} format (\code{urn:sedml:format:numl}) is defined via \code{resultComponents} with a single dataset corresponding to a single \code{resultComponent}. In the case that a \concept{NuML} file consists of multiple \code{resultComponents} the first \code{resultComponent} contains the data used in the \hyperref[class:dataDescription]{DataDescription}. There is currently no mechanism in SED-ML to reference the additional \code{resultComponents}.

If a \hyperref[sec:dimensionDescription]{\element{dimensionDescription}} is set on the \hyperref[class:dataDescription]{DataDescription}, than this \hyperref[sec:dimensionDescription]{\element{dimensionDescription}} must be identical to the \hyperref[sec:dimensionDescription]{\element{dimensionDescription}} of the \concept{NuML} file.

\subsubsection{CSV (Comma Separated Values)}
\label{sec:dataFormatCSV}
Data in the \concept{CSV} format (\code{urn:sedml:format:csv}) must follow the following rules when used in combination with SED-ML: 

\begin{itemize}
	\item Each record is one line - Line separator may be LF (0x0A) or CRLF (0x0D0A), a line separator may also be embedded in the data (making a record more than one line but still acceptable).
    \item Fields are separated with commas.
    \item Embedded commas - Field must be delimited with double-quotes.
    \item Leading and trailing whitespace is ignored - Unless the field is delimited with double-quotes in that case the whitespace is preserved.
    \item Embedded double-quotes - Embedded double-quote characters must be doubled, and the field must be delimited with double-quotes.
    \item Embedded line-breaks - Fields must be surounded by double-quotes.
    \item Always Delimiting - Fields may always be delimited with double quotes, the delimiters will be parsed and discarded by the reading applications.
	\item The first record is the header record defining the unique column ids 
	\item Lines starting with "\#" are treated as comment lines and ignored
	\item Empty lines are allowed and ignored
	\item For numerical data the "." decimal separator is used
	\item The following strings are interpreted as NaN: "", "\#N/A", "\#N/A N/A", "\#NA", "-1.\#IND", "-1.\#QNAN", "-NaN", "-nan", "1.\#IND", "1.\#QNAN", "N/A", "NA", "NULL", "NaN", "nan".
\end{itemize}

A dataset in \concept{CSV} is always encoding two dimensional data. 

When using data in the \concept{CSV} format  SED-ML, the \hyperref[sec:dimensionDescription]{\element{dimensionDescription}} is required on the \hyperref[class:dataDescription]{DataDescription}. 

The {\element{dimensionDescription}} must consist of an outer \code{compositeDescription} with \code{indexType="integer"} which allows to reference the rows of the \concept{CSV} by index and a inner \code{compositeDescription} which allows to reference the columns of the \concept{CSV} by their column header id. Within the inner \code{compositeDescription} exactly one \code{atomicDescription} must exist. All data in the \concept{CSV} must have the same type  which is defined via the \element{valueType} on the \element{atomicDescription}.

Below an example of the required \hyperref[sec:dimensionDescription]{\element{dimensionDescription}} for a \concept{CSV} is provided. In the example the \code{time} and \code{S1} columns are read from the \concept{CSV} file
\begin{myXmlLst}{Example CSV}{lst:exampleCSV}
# ./example.csv
time, S1, S2
0.0, 10.0, 0.0
0.1, 9.9, 0.1
0.2, 9.8, 0.2
\end{myXmlLst}

\begin{myXmlLst}{SED-ML \code{dimensionDescription} element for the example.csv}{lst:dimensionDescriptionCSV}

<dataDescription id="datacsv" name="Example CSV dataset" source="./example.csv" format="urn:sedml:format:csv">
  <dimensionDescription>
    <compositeDescription indexType="integer" name="Index">
      <compositeDescription indexType="string" name="ColumnIds">
        <atomicDescription valueType="double" name="Values" />
      </compositeDescription>
    </compositeDescription>
  </dimensionDescription>
  <listOfDataSources>
    <dataSource id="dataTime">
      <listOfSlices>
        <slice reference="ColumnIds" value="time" />
      </listOfSlices>
    </dataSource>
    <dataSource id="dataS1">
      <listOfSlices>
        <slice reference="ColumnIds" value="S1" />
      </listOfSlices>
    </dataSource>
  </listOfDataSources>
  ...          
</dataDescription>
\end{myXmlLst} 

\subsubsection{TSV (Tab Separated Values)}
\label{sec:dataFormatTSV}
The format \concept{TSV} (\code{urn:sedml:format:tsv}) is defined identical to \hyperref[sec:dataFormatCSV]{\element{CSV}} with the exceptions listed below
\begin{itemize}
    \item Fields are separated with tabs instead of commas.
    \item Embedded tab - Field must be delimited with double-quotes (embedded comma field must not be delimited with double quotes)
\end{itemize}


% ~~~ SYMBOLS ~~~
\subsection{Symbols}
\label{sec:implicitVariable}
Some variables used in a simulation experiment are not explicitly defined in the model, but may be implicitly contained in it. For example, to plot a variable's behaviour over time, that variable is defined in an SBML model, whereas time is not explicitly defined. 

SED-ML can refer to such implicit variables via the \concept{Symbol} concept. Such implicit variables are defined using the SED-ML URN scheme \element{urn:sedml:symbol:}\emph{\element{implicitVariable}}. 

For example, to refer in a SED-ML file to the definition of time, the URN \element{urn:sedml:symbol:time} is used.

\tab{symbols}~lists the predefined symbols in SED-ML.
\begin{table}[ht]
\center
\begin{tabular}{p{2cm}p{4cm}p{7cm}}
\toprule
\textbf{Language} & \textbf{URN} & \textbf{Definition}\\
\midrule
SBML & \code{urn:sedml:symbol:time} & Time in SBML is an intrinsic model variable that is addressable in model equations via a csymbol \code{time}.\\
\bottomrule
\end{tabular}
\caption{Predefined symbols in SED-ML. The latest list of symbols is available from \url{http://sed-ml.org}.}
\label{tab:symbols}
\end{table}


% ~~~ ANNOTATIONS ~~~
\subsection{Annotation Scheme}
\label{sec:annotations}
When annotating SED-ML elements with semantic \hyperref[class:annotation]{annotations}, the \concept{MIRIAM URI Scheme} should be used. In addition to providing the data type (e.g., PubMed) and the particular data entry inside that data type (e.g., \code{10415827}), the relation of the annotation to the annotated element should be described using the standardized \concept{biomodels.net qualifier}. The list of qualifiers, as well as further information about their usage, is available from \url{http://www.biomodels.net/qualifiers/}.


% ~~~~~~~~~~~~~~~~~~~~~~~~~~~~~~~~~~~~
% XPATH
% ~~~~~~~~~~~~~~~~~~~~~~~~~~~~~~~~~~~~
\section{XPath}  
\label{sec:xpath}
\concept{XPath} is a language for finding and referencing information in an XML document \citep{xpath:1999}. Within SED-ML \currentLV, XPath version 1 expressions are used to identify nodes and attributes within an XML representation in the following ways:

\begin{itemize}
	\item {Within a \hyperref[class:variable]{Variable} definition, where \concept{XPath} identifies the model variable required for manipulation in SED-ML.}
	\item {Within a \hyperref[class:change]{Change} definition, where \concept{XPath} is used to identify the target XML to which a change should be applied.}
\end{itemize}

For proper application, \concept{XPath} expressions should contain prefixes that allow their resolution to the correct XML namespace within an XML document. For example, the \concept{XPath} expression referring to a species \emph{X} in an SBML model:
\begin{alltt}
/sbml:sbml/sbml:model/sbml:listOfSpecies/sbml:species[@id=`X'] {\tickYes -\emph{CORRECT}}
\end{alltt}
is preferable to 
\begin{alltt}
/sbml/model/listOfSpecies/species[@id=`X'] {\tickNo -\emph{INCORRECT} }
\end{alltt}

which will only be interpretable by standard XML software tools if the SBML file declares no namespaces (and hence is invalid SBML).

Following the convention of other \concept{XPath} host languages such as XPointer and XSLT, the prefixes used within \concept{XPath} expressions must be declared using namespace declarations within the SED-ML document, and be in-scope for the relevant expression. Thus for the correct example above, there must also be an ancestor element of the node containing the XPath expression that has an attribute like:
\begin{alltt}
xmlns:sbml=`http://www.sbml.org/sbml/level3/version1/core'
\end{alltt}
(a different namespace URI may be used; the key point is that the prefix `sbml' must match that used in the XPath expression).


% ~~~~~~~~~~~~~~~~~~~~~~~~~~~~~~~~~~~~
%% NUML
% ~~~~~~~~~~~~~~~~~~~~~~~~~~~~~~~~~~~~
\section{NuML}
\label{sec:numl}
The Numerical Markup Language (\concept{NuML}) aims to standardize the exchange and archiving of numerical results. Additional information including the \concept{NuML} specification is available from \url{https://github.com/NuML/NuML}.

\concept{NuML} constructs are used in SED-ML for referencing external data sets in the \hyperref[class:dataDescription]{DataDescription} class. NuML is used to define the \hyperref[sec:dimensionDescription]{DimensionDescription} of external datasets in the \hyperref[class:dataDescription]{DataDescription}. In addition, \hyperref[type:numlsid]{\element{NuMLSIds}} are used for retrieving subsets of data via either the \hyperref[sec:indexSet]{\element{indexSet}} element in the \hyperref[class:dataSource]{DataSource} or within the \hyperref[class:slice]{Slice} class.

% ~~~~~~~~~~~~~~~~~~~~~~~~~~~~~~~~~~~~
%% KISAO
% ~~~~~~~~~~~~~~~~~~~~~~~~~~~~~~~~~~~~
\section{KiSAO}
\label{sec:kisao}
The Kinetic Simulation Algorithm Ontology (\concept{KiSAO} \citep{CWK+10}) is used in SED-ML to specify simulation \hyperref[class:algorithm]{algorithms} and \hyperref[class:algorithmParameter]{algorithmParameters}. \concept{KiSAO} is a community-driven approach of classifying and structuring simulation approaches by model characteristics and numerical characteristics. The ontology is available in OWL format from \concept{BioPortal} at \url{http://purl.bioontology.org/ontology/KiSAO}. 

Defining simulation \hyperref[class:algorithm]{algorithms} through \concept{KISAO} terms not only identifies the simulation algorithm used for the SED-ML simulation, it also enables software to find related algorithms, if the specific implementation is not available. For example, software could decide to use the CVODE integration library for an analysis instead of a specific Runge Kutta 4,5 implementation. 

Should a particular simulation algorithm or algorithm parameter not exist in \concept{KiSAO}, please request one via \url{http://www.biomodels.net/kisao/}.

% ~~~~~~~~~~~~~~~~~~~~~~~~~~~~~~~~~~~~~~~~
%% COMBINE ARCHIVE
% ~~~~~~~~~~~~~~~~~~~~~~~~~~~~~~~~~~~~~~~~
\section{COMBINE archive}
\label{sec:archive}

A \concept{COMBINE archive} \citep{Bergmann2014} is a single file that supports the exchange of all the information necessary for a modeling and simulation experiment in biology. A \concept{COMBINE archive} file is a ZIP container that includes a manifest file, listing the content of the archive, an optional metadata file adding information about the archive and its content, and the files describing the model. The content of a \concept{COMBINE archive} consists of files encoded in COMBINE standards whenever possible, but may include additional files defined by an Internet Media Type. Several tools that support the \concept{COMBINE archive} are available, either as independent libraries or embedded in modeling software.

The \concept{COMBINE archive} is described at \url{http://co.mbine.org/documents/archive} and 
in \citep{Bergmann2014}.

\concept{COMBINE archives} are the recommended means for distributing simulation experiment descriptions in SED-ML, the respective data and model files, and the \hyperref[class:output]{Outputs} of the simulation experiment (figures and reports). All SED-ML specification examples in Appendix~\ref{app:examples} are available as \concept{COMBINE archive} from \url{http://sed-ml.org}.

% ~~~~~~~~~~~~~~~~~~~~~~~~~~~~~~~~~~~~
%% RESOURCES
% ~~~~~~~~~~~~~~~~~~~~~~~~~~~~~~~~~~~~
\section{SED-ML resources}
\label{sec:resources}

Information on SED-ML can be found on \url{http://sed-ml.org}. The SED-ML XML Schema, the UML schema, SED-ML examples, and additional information is available from \url{https://github.com/sed-ml}.
