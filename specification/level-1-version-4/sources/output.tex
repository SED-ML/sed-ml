% ~~~~~~~~~~~~~~~~~~~~~~~~~~~~~~~~~~~~~~~~
%% OUTPUT
% ~~~~~~~~~~~~~~~~~~~~~~~~~~~~~~~~~~~~~~~~ 
\subsection{\element{Output}}
\label{class:output}

The abstract \Output class describes how the results of a simulation are presented (\fig{output}). \changed{The available output classes are \Plot, \Report, and \Figure.} The data used in an \Output is provided via the \DataGenerator class.

\sedfig[width=0.8\textwidth]{images/uml/output}{The definition of the SED-ML \Output class. The subclasses are defined below.}{fig:output}

\begin{blockChanged}
The \Output class inherits the \element{id} and \element{name} attributes from \SedBase, as well as the optional \element{annotation} and \element{notes} chidren.  When producing a printed table or figure, users may want to use the \element{name} as the title, and the \element{notes} as the legend.
\end{blockChanged}

\begin{blockChanged}
% ~~~ PLOT ~~~
\subsubsection{\element{Plot}}
\label{class:plot}
The \Plot class is used for two- and three-dimensional plot outputs.  It defines the axes of a plot, as well as whether or not a legend should be displayed.

\sedfig[width=0.75\textwidth]{images/uml/plot}{The definition of the SED-ML \Plot, \PlotTwo, \PlotThree, \Axis, \ListOfCurves, and \ListOfSurfaces classes.  The \AbstractCurve and \Surface classes are defined below.}{fig:plot}

The \Plot class inherits the \element{id} and \element{name} attributes from \SedBase, and may additionally contain a \element{legend} attribute of type Boolean that defines whether a legend should be displayed (\val{true}) or not (\val{false}).  If the attribute is missing, it is up to the tool whether to display the legend or not.  It does not mean that the attribute has a default value of \val{false}.

A \Plot may define optional \element{xAxis} and \element{yAxis} children, each of type \Axis.  Each axis defines the axis by which the \Curve or \Surface children are to be interpreted.  If either child is omitted, that axis is undefined, and it is up to the tool whether and how to display any necessary axes, and to decide what type that axis should be (linear or logarithmic).

\end{blockChanged}

\subsubsection{\element{Plot2D}}
\label{class:plot2D}
\label{class:listOfCurves}
The \PlotTwo class is used for two dimensional plot outputs. The \PlotTwo may contain a number of \Curve definitions in the \element{listOfCurves}, defining the \hyperref[class:curve]{curves} to be plotted in the the 2D plot.  \changed{It may also contain a child \element{rightYAxis}, defining an alternate vertical axis that corresponds to some of the \Curve children.  Each \Curve defines for itself which y axis it corresponds to, if there are two.  A \PlotTwo with no \element{yAxis} may not have a \element{rightYAxis}.}


% ~~~ PLOT3D ~~~
\subsubsection{\element{Plot3D}}
\label{class:plot3D}
\label{class:listOfSurfaces}
The \PlotThree class is used for three dimensional plot outputs (\fig{output}). The \PlotThree contains a number of \hyperref[class:surface]{Surface} definitions in the \element{listOfSurfaces}, \changed{and may additionally define a \element{zAxis} child, of type \Axis, defining the characteristics of the z axis.  If no \element{zAxis} is provided, those characteristics are undefined, and the tool may choose how and whether to display that axis, as well as what type it is (linear or logarithmic).}
%


% ~~~ Axis ~~~
\begin{blockChanged}
\subsubsection{\element{Axis}}
\label{class:axis}
The \Axis class is used to define whether an axis for a given \Plot is linear or logarithmic, and how to display it.  It inherits the optional attributes \element{id} and \element{name} from \SedBase (the latter of which can be used as a label), and adds the required attribute 'type' of type \AxisKind (either 'linear' or 'log10'), as well as the optional attributes \element{min} and \element{max}, both of type \element{double}, \element{grid} of type \element{Boolean}, and \element{style} of type \element{SIdRef}.

The \element{min} and \element{max} values indicate the minimum and maximum values for the axis.  Data points outside of this range should not be shown on the parent \Plot.  Either value may be set or not, and if not set, a value must be chosen for display that is less than (for \element{min}) or greater than (for \element{max}) the most extreme value along that axis for any \Curve or \Surface in that \Plot.  Do note that in some cases, a given \Curve may not have any data points associated with one Y \Axis, as its data may be associated with the alternative Y \Axis.

The \element{grid} attribute indicates whether grid lines should be displayed in the \Plot for tick marks along that axis.

The \element{style} attribute, if present, must be the \element{SId} of a \Style in the same \SedDocument, and can be used to indicate things like color and/or line thickness for that axis and its labels.  If not present, any style may be used.  Note that it is possible to suppress an axis from being displayed at all if the corresponding \Style of an \Axis has a \element{line} with a \element{style} of \val{none}.



\end{blockChanged}
%


% ~~~ REPORT ~~~
\subsubsection{\element{Report}}
\label{class:report}
The \concept{Report} class defines a data table consisting of several single instances of the \DataSet in the child \element{listOfDataSets} (\fig{output}). Its output returns the simulation result processed via \hyperref[class:dataGenerator]{DataGenerators} in actual numbers. The columns of the report table are defined by creating an instance of the \DataSet for each column. 

The simulation result itself, i.e.\ concrete result numbers, are not stored in SED-ML, but the directive how to calculate them from the output of the simulator is provided through the \hyperref[class:dataGenerator]{dataGenerator}. The encoding of simulation results is not part of SED-ML \currentLV.

% ~~~~~~~~~~~~~~~~~~~~~~~~~~~~~~~~~~~~~~~~
% OUTPUT COMPONENTS
% ~~~~~~~~~~~~~~~~~~~~~~~~~~~~~~~~~~~~~~~~ 
\begin{blockChanged}
\subsubsection{\element{AbstractCurve}}
\label{class:abstractCurve}
An \AbstractCurve is a two-dimensional \Output component representing a (processed) simulation result (\fig{output}). Zero or more \AbstractCurve instances define a \PlotTwo (\fig{output}).  The \AbstractCurve class defines the attributes common to the \Curve and \ShadedArea child classes.  In addition to the optional \element{id} and \element{name} attributes it inherits from \SedBase (the latter of which may be used as the label in the \Plot legend, if present), it also defines the optional attributes \element{xDataReference}, \element{order}, \element{style}, and \element{yAxis}.  It is also legal but discouraged to include an attribute \element{logX}.


The \element{xDataReference} attribute, if present, must be the \element{SId} of a \DataGenerator in the same \SedDocument.  The referenced \DataGenerator will contain the information for the x coordinates for the data to be plotted.  This attribute is optional because in the case of a \Curve, an absence of x-coordinate data means that the y-coordinate data is ordinal or categorical, and can simply be plotted in order.

The \element{order} attribute is of type \element{non-negative integer} and, if present, defines the order in which this \Curve must be displayed relative to other \Curve elements in the same \Plot.  A \Curve with a lower \element{order} will be added earlier to the displayed curves.  This means that for lines, the curve with the highest \element{order} will be fully visible, while a \Curve with a lower \element{order} may be hidden by a \Curve with a higher \element{order}.  A \Curve with no \element{order} may be displayed in front or behind any other \Curve.  For adjacent bars, the bar with the lower \element{order} is presented to the left of any bar with a higher \element{order}.  For stacked bars, the bar with the lower \element{order} is presented underneath any bar with a higher \element{order}.

The \element{style} attribute is of type \element{SIdRef} and, if present, must reference a \Style in the same \SedDocument.  It can be used to indicate things like color, marker, and/or line thickness for this \Curve.  If not present, any style may be used.  A \Curve may be displayed as only a set of markers if the \Line from its \Style is set to have a \element{type} of \val{none}.  Similarly, a \Curve may be displayed as a line only with no markers if the \Marker from its \Style is set to have a \element{type} of \val{none}.  (If both are set to \val{none}, the curve will not be displayed at all!)  The \Fill of a \Style has no meaning for a \Curve, and, if present, will be ignored.

The \element{yAxis} attribute is of type \element{string} and must be defined if the parent \Plot defines both a \element{yAxis} and a \element{rightYAxis}.  If it has the value of \val{left}, it means that the data is to be displayed corresponding to the \element{yAxis} of the parent \Plot, and if it has the value of \val{right}, it means that the data is to be disaplyed corresponding to the \element{rightYAxis} of the parent \Plot.  If the parent \Plot has no defined \element{rightYAxis}, this attribute must not be defined.

The \element{logX} attribute, of type \element{Boolean}, was used in previous versions of SED-ML to indicate whether the x axis of the \Plot should be linear or log10.  This allowed mutliple \Curve objects in the same \Plot to contradict each other, and has therefore been moved to \Plot.  The \element{logX} attribute on \Curve has therefore been deprecated, and will always be ignored.

\end{blockChanged}


% ~~~ CURVE ~~~
\subsubsection{\element{Curve}}
\label{class:curve}
A \Curve is a two-dimensional \Output component representing a (processed) simulation result (\fig{output}). Zero or more \Curve instances define a \PlotTwo (\fig{output}). \changed{In addition to the attributes  it inherits from \AbstractCurve (and \SedBase), it also defines the required attribute \element{yDataReference} of type \element{SIdRef}.  It also defines the optional attribute \element{type} of type \CurveKind, and the optional attributes \element{xError}, \element{xErrorLower}, \element{yError}, and \element{yErrorLower}, all of type \element{SIdRef}.}

\begin{blockChanged}
Like the \element{xDataReference}, the \element{yDataReference} must be the \element{SId} of a \DataGenerator in the same \SedDocument.  The referenced \DataGenerator will contain the information for the y coordinates for the data to be plotted.  The dimensions of the y data sould match the x data, if present.  If the y data is multi-dimensional (such as time course data over several stochastic replicates), one dimension should match the x data (time, in our example), and the other dimension should simply be replicated as separate curves on the same plot (with the same style and label).

The optional \element{type} attribute is of type \CurveKind, and determines the kind of curve being displayed.  The possible values are:

\begin{itemize}
\item \textbf{points}: The curve is plotted as points, which can mean markers and/or a line, depending on the \element{style}.
\item \textbf{bar}: The curve is plotted as bars with the height of the bars defined via the \element{yDataGenerator} values.  The middle of the bars are plotted at the \element{xDataGenerator} position, if present, or sequentially, if not.  The fill of the bars is defined via the \element{style}. 
\item \textbf{barStacked}: The curve is plotted as with \element{bar}, but stacked instead of adjacent.
\item \textbf{horizontalBar}: The curve is plotted as a bar plot, as above, but the y axis is vertical and the x axis is horizontal.
\item \textbf{horizontalBarStacked}: The curve is plotted as a stacked bar plot, as above, but the y axis is vertical and the x axis is horizontal.
\item \textbf{polarPoints}: The curve is plotted as points as above, but on polar coordinates.
\item \textbf{polarBar}: The curve is plotted as a bar plot, as above, but on polar coordinates.
\item \textbf{polarBarStacked}: The curve is plotted as a stacked bar plot, as above, but on polar coordinates.
\end{itemize}

The optional attributes \element{xError}, \element{xErrorLower}, \element{yError}, and \element{yErrorLower} may be declared to define the error in the data present in the \Curve.  Each attribute must, if defined, point to a \DataGenerator in the same \SedDocument.  The \element{xError} and \element{xErrorLower} must have the same dimensionality as the \element{xDataReference}, and the \element{yError} and \element{yErrorLower} must have the same dimensionality as the \element{yDataReference}.  Each set of data represents the error in that dimension, in distance from the given data point.  If \element{xError} is defined and \element{xErrorLower} is not defined, the error in the x dimension is assumed to be symmetrical (i.e. '1.1 +/- 0.4').  If both attributes are defined, \element{xError} refers to the error in the positive direction, and \element{xErrorLower} refers to the error in the negative direction.  The same is true for \element{yError} and \element{yErrorLower}.


% ~~~ SHADEDAREA ~~~
\subsubsection{\element{ShadedArea}}
\label{class:shadedArea}
A \ShadedArea is an \AbstractCurve that defines an area instead of a series of points.  In addition to what is inherited from \AbstractCurve, a \ShadedArea defines the required attributes \element{yDataReferenceFrom} and \element{yDataReferenceTo}, both of which must be the \element{SId} of a \DataGenerator in the same \SedDocument.  The area between these two sets of points is then filled for display.  If the \element{style} is defined, the \Fill of that \Style is used to color the fill.  If both \element{color} and \element{secondColor} are defined, the first is associated with the \element{yDataReferenceFrom}, and the second is associated with the \element{yDataReferenceTo}.

\end{blockChanged}


% ~~~ SURFACE ~~~
\subsubsection{\element{Surface}}
\label{class:surface}
\begin{blockChanged}
A \Surface is a parallel class to \AbstractCurve that defines a three-dimensional surface instead of a two-dimensional curve.  It defines the attributes \element{xDataReference}, \element{yDataReference}, and \element{zDataReference}, the first two of which are optional and the last of which is required, along with the optional attributes \element{style} of type \element{SIdRef}, and \element{type}, of type \SurfaceKind.

The three data reference attributes, if defined, must point to \DataGenerator elements in the same \SedDocument, which define the surface to be plotted.  If the \element{zDataReference} is two-dimensional, the x and y data may be omitted: if so, the z data points are displayed in an ordinal or categorical manner, with x and y values drawn from the data's position in the matrix.

The \element{style} attribute, if defined, must contain the \SId of a \Style object in the same \SedDocument.  This \Style determines how any lines, markers, or fills on that surface should be displayed, if present for that type of \Surface.

The \element{type} attribute, if present, determines the type of surface and how it should be displayed.  The options are:

\begin{itemize}
\item \textbf{parametricCurve}: Each successive data point is plotted in order, potentially joined by a line.  If the z data is 2-dimensional instead of a vector, the last point of the first vector should not be connected to the first point of the next.  The line and marker styles can be set from the \element{style} (including removing them if the \element{type} of either is set to \val{none}).
\item \textbf{surfaceMesh}: The data are plotted as a wireframe, with adjacent-in-space data points connected with lines.  The line style can be set from the \element{style}.
\item \textbf{surfaceContour}: The data is plotted as a continuous surface.  The fill color can be set from the \element{style}, as can the lines and/or markers, if displaying those elements are desired.
\item \textbf{contour}:  The 3D data are plotted as a 2D surface, with contour lines (similar to elevation plots).  The line style can be set from the \element{style}.
\item \textbf{heatMap}:  The 3D data are plotted as a 2D surface, with color representing the values.  The colors can be set from the \element{fill} of the \element{style}.
\item \textbf{stackedCurves}:  
\item \textbf{bar}: The data 
\end{itemize}
parametricCurve (surface is plotted as points; points can be connected via a line via the style)
surfaceMesh (surface is plotted as a 3D surface mesh, Wireframe)
surfaceContour (surface if plotted as a 3D, Surface)
contour (surface if plotted as a 2D, Surface)
heatMap (surface is plotted as a 2D heatmap)
stackedCurves (line, marker, fill)
bar (surface is plotted as a 3D barplot)


\end{blockChanged}



%% ~~~ DATASET ~~~
\subsubsection{\element{DataSet}}
\label{class:dataSet}
The \concept{DataSet} class holds definitions of data to be used in the \hyperref[class:report]{Report} class (\fig{output}). DataSets are labeled references to instances of the \hyperref[class:dataGenerator]{DataGenerator} class.

\tabtext{dataSet}{dataSet}

\begin{table}[h!t]
\center
\begin{tabular}{ll}
\toprule
\textbf{\attribute} & \textbf{\desc}\\
\midrule
metaid$^{o}$ & \refpage{sec:metaid}\\
id & \refpage{sec:id} \\
name$^{o}$ & \refpage{sec:name}\\
\midrule
dataReference & \refpage{sec:dataReference1}\\
label & \refpage{sec:label}\\
\midrule
\textbf{\subelements} & \textbf{\desc}\\
\midrule
notes$^{o}$ & \refpage{class:notes}\\
annotation$^{o}$ & \refpage{class:annotation}\\
\bottomrule
\end{tabular}
\caption{\tabcap{dataSet}}
\label{tab:dataSet}
\end{table}

\paragraph*{\element{label}}
\label{sec:label}
Each data set in a \hyperref[class:report]{Report} must have an unambiguous \concept{\element{label}}. A \concept{\element{label}} is a human readable descriptor of a data set for use in a \hyperref[class:report]{report}. For example, for a tabular data set of time series results, the \concept{\element{label}} could be the column heading. 

\paragraph*{\element{dataReference}}
\label{sec:dataReference1}
The \concept{\element{dataReference}} attribute contains the ID of a \concept{dataGenerator} element and as such represents a link to it. The data produced by that particular \hyperref[class:dataGenerator]{dataGenerator} fills the according \hyperref[class:dataSet]{dataSet} in the \hyperref[class:report]{report}.

\lsttext{dataSet}{dataSet}
The example shows the definition of a dataSet. The referenced dataGenerator \element{dg1} must be defined in the \hyperref[sec:listOfDataGenerators]{\element{listOfDataGenerators}}.
\begin{myXmlLst}{The SED-ML \code{dataSet} element, defining a data set containing the result of the referenced task}{lst:dataSet}
<listOfDataSets>
	<dataSet id="d1" name="v1 over time" dataReference="dg1" label="_1">
</listOfDataSets>
\end{myXmlLst}
