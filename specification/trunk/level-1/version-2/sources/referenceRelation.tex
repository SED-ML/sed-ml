\subsection{Reference relations}
\label{sec:reference}

The \concept{reference} concept is used to refer to a particular element inside the SED-ML document. It may occur in six different ways in the SED-ML document:
%
\begin{enumerate}
\item{as an association between two \hyperref[class:model]{Model}s (\hyperref[sec:modelReference]{modelReference}),}
\item{as an association between a \hyperref[class:variable]{Variable} and a \hyperref[class:model]{Model} (\hyperref[sec:modelReference]{modelReference}),}
\item{as an association between a \hyperref[class:variable]{Variable} and an \hyperref[class:abstractTask]{AbstractTask} (\hyperref[sec:taskReference]{taskReference}),}
\item{as an association between a \hyperref[class:task]{Task} and the simulated \hyperref[class:model]{Model} (\hyperref[sec:modelReference]{modelReference}),}
\item{as an association between a \hyperref[class:task]{Task} and the \hyperref[class:simulation]{Simulation} run (\hyperref[sec:simulationReference]{simulationReference}), or}
\item{as an association between an \hyperref[class:output]{Output} and a \hyperref[class:dataGenerator]{DataGenerator} (\hyperref[sec:dataReference]{dataReference}).}
\end{enumerate}
%
The definition of a \hyperref[class:task]{Task} object requires a reference to a particular Model object (\hyperref[sec:modelReference]{modelReference}, see Section~\ref{sec:modelReference} on page \pageref{sec:modelReference}); furthermore, the Task object must be associated with a particular Simulation object (\hyperref[sec:simulationReference]{simulationReference}, see Section~\ref{sec:simulationReference} on page \pageref{sec:simulationReference}).

Depending on the use of the \concept{reference} relation in connection with a \hyperref[class:variable]{Variable} object, it may take different roles: 
\begin{enumerate}
\item[a.]{The \concept{reference} association might occur between a Variable object and a Model object, e.g.\ if the variable is to define a \hyperref[class:change]{Change}. 
In that case the \code{variable} element contains a \hyperref[sec:modelReference]{modelReference} to refer to the particular model that contains the variable used to define the change (see Section~\ref{sec:modelReference} on page \pageref{sec:modelReference}). }
\item[b.]{If the \concept{reference} is used as an association between a Variable object and an AbstractTask object inside the \hyperref[class:dataGenerator]{dataGenerator} class, then the \code{variable} element contains a \hyperref[sec:taskReference]{taskReference} to unambiguously refer to an observable in a given task (see Section~\ref{sec:taskReference} on page \pageref{sec:taskReference}).}
\end{enumerate}

Four different types of \concept{data references} exist in SED-ML \LoneVtwo. They are used depending on the \emph{type} of output for the simulation. A 2d plot has an \hyperref[sec:xDataReference]{xDataReference} and a \hyperref[sec:yDataReference]{yDataReference} assigned. A 3D plot has in addition a \hyperref[sec:zDataReference]{zDataReference} assigned. To define a report, each data column has a \hyperref[sec:dataReference1]{dataReference} assigned.

\subsubsection{modelReference}
\label{sec:modelReference}
%
The \concept{modelReference} either represents a relation between two \hyperref[class:model]{Model} objects, a \hyperref[class:variable]{Variable} object and a \hyperref[class:model]{Model} object, or  a relation between a \hyperref[class:task]{Task} object and a \hyperref[class:model]{Model} object.

The \code{source} attribute of a \hyperref[class:model]{Model} is allowed to reference either a URI or an \code{SId} to a second
\hyperref[class:model]{Model}. Constructs where a model \code{A} refers to a model \code{B} and \code{B} to \code{A} (directly or indirectly) are invalid.

If pre-processing needs to be applied to a model before simulation, then the model update can be specified by creating a \hyperref[class:change]{Change} object. In the particular case that a change must be calculated with a mathematical function, variables need to be defined. To refer to an existing entity in a defined \hyperref[class:model]{Model}, the \concept{modelReference} is used. 

The \code{modelReference} attribute of the \code{variable} element contains the \concept{id} of a model that is defined in the document. 
\lsttext{modelReference1}{modelReference} 
%
\begin{myXmlLst}{SED-ML \code{modelReference} attribute inside a variable definition of a  \code{computeChange} element}{lst:modelReference1}
<model id="m0001" [..]>
 <listOfChanges>
   <computeChange>
    <listOfVariables>
     <variable id="v1" modelReference="cellML" target="/cellml:model/cellml:component[@cmeta:id='MP']/cellml:variable[@name='vsP']/@initial_value" />
     [..]
    </listOfVariables>
    <listOfParameters [..] />
    <math>
     [CALCULATION OF CHANGE]
    </math>
   </computeChange>
 </listOfChanges>
 [..]
</model>
\end{myXmlLst}
%
In the example, a change is  applied on model \code{m0001}. In the \code{computeChange} element a list of variables is defined. One of those variable is \code{v1} which is defined in another model (\code{cellML}). 
The XPath expression given in the \hyperref[sec:target]{target} attribute identifies the variable in the model which carries the ID \code{cellML}.

The \concept{modelReference} is also used to indicate that a \hyperref[class:model]{Model} object is used in a particular  \hyperref[class:task]{Task}. Listing \ref{lst:modelReference2} shows how this can be done for a sample SED-ML document.
%
\begin{myXmlLst}{SED-ML \code{modelReference} definition inside a \element{task} element}{lst:modelReference2}
<listOfTasks>
 <task id="t1" name="Baseline" modelReference="model1" simulationReference="simulation1" />
 <task id="t2" name="Modified" modelReference="model2" simulationReference="simulation1" />
</listOfTasks>
\end{myXmlLst}
%
The example defines two different tasks; the first one applies the simulation settings of \code{simulation1} on \code{model1}, the second one applies the same simulation settings on \code{model2}.

%%% Local Variables: 
%%% mode: latex
%%% TeX-master: "../sed-ml-L1V2"
%%% End: 



\subsubsection{taskReference}
\label{sec:taskReference}
\hyperref[class:dataGenerator]{DataGenerator} objects are created to apply post-processing to the simulation results before simulation output. 

For certain types of post-processing \hyperref[class:variable]{Variable} objects need to be created. These link to a defined \hyperref[class:task]{Task} from which the model that contains the variable of interest can be inferred. 
A \concept{taskReference} association is used to realise that link from a \hyperref[class:variable]{Variable} object inside a \hyperref[class:dataGenerator]{DataGenerator} to a \hyperref[class:task]{Task} object. 
Listing \ref{lst:reference3} gives an example.
%
\begin{myXmlLst}{SED-ML \code{taskReference} definition inside a \element{dataGenerator} element}{lst:reference3}
<listOfDataGenerators>
 <dataGenerator id="tim3" name="tim mRNA (difference v1-v2+20)">
  <listOfVariables>
   <variable id="v1" taskReference="t1" [..] />
  </listOfVariables>
  <math [..]/>
 </dataGenerator>
</listOfDataGenerators>
\end{myXmlLst}
%
The example shows the definition of a variable \code{v1} in a \code{dataGenerator} element. The variable appears in the model that is used in task \code{t1}. The task definition of \code{t1} might look as shown in Listing~\ref{lst:taskReferences}.
\begin{myXmlLst}{Use of the reference relations in a task definition}{lst:taskReferences}
<listOfTasks>
  <task id="t1" name="task definition" modelReference="model1" simulationReference="simulation1" />
</listOfTasks>
\end{myXmlLst}
Task \code{t1} references the model \code{model1}. Therefore we can conclude that the variable \code{v1} defined in listing \ref{lst:reference3} targets an element of the model with ID \code{model1}. The targeting process itself will be explained in section \ref{sec:target} on \refpage{sec:target}.


\subsubsection{simulationReference}
\label{sec:simulationReference}
The \concept{simulationReference} is used to refer to a particular \hyperref[class:simulation]{Simulation} in a \hyperref[class:task]{Task}. 
Listing \ref{lst:modelReference2} shows the reference to a defined simulation for a sample SED-ML document. In the example, both tasks \code{t1} and \code{t2} use the simulation settings defined in \code{simulation1} to run the experiment.

%%% Local Variables: 
%%% mode: latex
%%% TeX-master: "../sed-ml-L1V3"
%%% End: 


\subsubsection{dataReference}
\label{sec:dataReference}
The \concept{dataReference} is used to refer to a particular \hyperref[class:dataGenerator]{DataGenerator} instance from an \hyperref[class:output]{Output} instance. 
Listing \ref{lst:dataReference} shows the reference to a defined data set for a sample SED-ML document. 
%
\begin{myXmlLst}{Example for the use of data references in a curve definition}{lst:dataReference}
<listOfOutputs>
  <plot2D id="p1" [..] >
    <curve id="c1" xDataReference="dg1" yDataReference="dg2" />
    [..]
  </plot>
</listOfOutputs>
\end{myXmlLst}
%
In the example, the output type is a 2D plot, which defines one curve with id \code{c1}. A curve must refer to two different data generators which describe how to procure the data that is to be plotted on the x-axis and y-axis respectively. 

%%% Local Variables: 
%%% mode: latex
%%% TeX-master: "../sed-ml-L1V3"
%%% End: 



%%% Local Variables: 
%%% mode: latex
%%% TeX-master: "../sed-ml-L1V2"
%%% End: 