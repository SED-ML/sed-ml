%% ----------------------------------------------------------------------------
%% Load up packages we need.
%% ----------------------------------------------------------------------------

% Warning: more \usepackage commands are used later in this file.

\usepackage[a4paper,centering,margin=1in,marginparwidth=0.75in]{geometry}
\usepackage[utf8]{inputenc} % unicode support
\usepackage{shadow}
\usepackage{supertabular}
\usepackage{multirow}
\usepackage{multicol}
\usepackage{enumitem}
\usepackage{lscape}
\usepackage{layout}
\usepackage{array}
\usepackage{fancybox}
\usepackage{xspace}
\usepackage{ifpdf}

% Hyperref, xcolor, graphicx and possibly others have a flag "pdftex"
% that needs to be used if pdflatex is being used.  The following puts
% these inside a conditional for that situation.

\ifpdf
  % Case: using pdflatex
  \usepackage[pdftex,rgb,dvipsnames,svgnames,hyperref,table]{xcolor}

\else
  % Case: not using pdflatex
  \usepackage[rgb,dvipsnames,svgnames,hyperref,table]{xcolor}

\fi


\usepackage{float}
\usepackage[american]{varioref}
\usepackage[normalem]{ulem}
\usepackage{eso-pic}                   
\usepackage{natbib}
\usepackage{listings} % enabling listings

\usepackage{url} % enabling URLs


\usepackage{pifont}% enable dingbats
\usepackage{alltt}


% package to enanble metapost for drawing the UML classes
\usepackage{emp}
% use: mpost spec.mp to create the UML diagrams

%\usepackage{enumerate}
% to change enumeration starting numbers and so on

%\usepackage{color} % enabling text colors

% using pdflatex with emp package
 \ifx\pdftexversion\undefined
 \usepackage[dvips]{graphicx}
 \else
 \usepackage[pdftex]{graphicx}
 \graphicspath{{./}{./images/}}
\DeclareGraphicsExtensions{.png}
 \DeclareGraphicsRule{*}{mps}{*}{}
 \fi

% Hyperref, xcolor, graphicx and possibly others have a flag "pdftex"
% that needs to be used if pdflatex is being used.  The following puts
% these inside a conditional for that situation.

\ifpdf
  % Case: using pdflatex

  \usepackage[pdftex]{graphicx}
  \DeclareGraphicsExtensions{.pdf,.png}

  % Options get even more complicated.  If we're producing grayscale output,
  % we don't want to bother with coloring links, but we still want to load
  % hyperref so that its macros are defined (and we don't have to redefine
  % everything that uses hyperref).  So:

  \usepackage[pdftex,breaklinks=true,colorlinks=true,plainpages=false,
  pdfpagelabels,bookmarks=true,bookmarksopen=true,bookmarksopenlevel=2,
  pdfhighlight=/O,linkcolor={RoyalBlue},citecolor={RoyalBlue},
  urlcolor={RoyalBlue}]{hyperref}

  \usepackage[pdftex,rgb,dvipsnames,svgnames,hyperref,table]{xcolor}

\else
  % Case: not using pdflatex

  \usepackage{graphicx}
  \DeclareGraphicsExtensions{.eps,.png}

  \usepackage[breaklinks,plainpages=false,pdfpagelabels]{hyperref}
  \usepackage[rgb,dvipsnames,svgnames,hyperref,table]{xcolor}

\fi

% Package 'rotating' needs to come after the others.

\usepackage{rotating}

% Booktabs for sharp-looking \toprule, \midrule, \bottomrule in tables.

\usepackage{booktabs}
\setlength{\cmidrulewidth}{0.3 pt}
\setlength{\lightrulewidth}{0.3 pt}
\setlength{\heavyrulewidth}{0.9 pt}
\setlength{\parindent}{0pt} % no indents in the whole document

%% ----------------------------------------------------------------------------
%% Special customizations
%% ----------------------------------------------------------------------------


\makeatletter

% Adjust bottom-of-page behavior and page line spacing.

\raggedbottom
\renewcommand{\baselinestretch}{0.98}

% Numbers the paragraph, but does not lists them in the ToC

\setcounter{secnumdepth}{4}
\setcounter{tocdepth}{2}

% Fix placement of figures & tables.  This keeps latex from shoving big
% floats to the end of a document when they are somewhat big, which it will
% do even if you put [htb] as the argument.

\setcounter{topnumber}{3}
\renewcommand\topfraction{1.0}
\setcounter{bottomnumber}{2}
\renewcommand\bottomfraction{1.0}
\renewcommand\textfraction{0.0}
\renewcommand\floatpagefraction{0.9}

% Adjustment to the spacing around tables and figures.

\renewcommand{\intextsep}{2em}

% This sets up Helvetica for headings.  The font scaling is
% because the default Helvetica size is too big.

\RequirePackage{helvet}
\def\Hv@scale{0.87}

% The following sets up txtt for the typewriter font.

\renewcommand{\ttdefault}{txtt}
\DeclareMathAlphabet{\mathtt}{OT1}{txtt}{m}{n}
\SetMathAlphabet{\mathtt}{bold}{OT1}{txtt}{b}{n}

% The next bit is an adaption of code from ot1phv.fd and adapted to the txtt
% fonts.  The txtt fonts are just a tad too big, so this tries to rescale
% them down a tiny bit.  This isn't completely right because I couldn't
% figure out the right syntax when the DeclareFontShape uses ssub below.
% (Notice how the ones with ssub don't have the \Txtt@@scale factor.)

\def\Txtt@scale{0.975}
\edef\Txtt@@scale{s*[\csname Txtt@scale\endcsname]}%

\DeclareFontFamily{OT1}{txtt}{\hyphenchar \font\m@ne}
\DeclareFontShape{OT1}{txtt}{m}{n}{	%rebular
     <-> \Txtt@@scale txtt%
}{}
\DeclareFontShape{OT1}{txtt}{m}{sc}{	%cap & small cap
     <-> \Txtt@@scale txttsc%
}{}
\DeclareFontShape{OT1}{txtt}{m}{sl}{	%slanted
     <-> \Txtt@@scale txttsl%
}{}
\DeclareFontShape{OT1}{txtt}{m}{it}{	%italic
     <-> ssub * txtt/m/sl%
}{}
\DeclareFontShape{OT1}{txtt}{m}{ui}{	%unslanted italic
     <-> ssub * txtt/m/sl%
}{}
\DeclareFontShape{OT1}{txtt}{b}{n}{	%bold
     <-> \Txtt@@scale txbtt%
}{}
\DeclareFontShape{OT1}{txtt}{b}{sc}{	%bold cap & small cap
     <-> \Txtt@@scale txbttsc%
}{}
\DeclareFontShape{OT1}{txtt}{b}{sl}{	%bold slanted
     <-> \Txtt@@scale txbttsl%
}{}
\DeclareFontShape{OT1}{txtt}{b}{it}{	%bold italic
     <-> ssub * txtt/b/sl%
}{}
\DeclareFontShape{OT1}{txtt}{b}{ui}{	%bold unslanted italic
     <-> ssub * txtt/b/sl%
}{}
\DeclareFontShape{OT1}{txtt}{bx}{n}{	%bold extended
     <-> ssub * txtt/b/n%
}{}
\DeclareFontShape{OT1}{txtt}{bx}{sc}{	%bold extended cap & small cap
     <-> ssub * txtt/b/sc%
}{}
\DeclareFontShape{OT1}{txtt}{bx}{sl}{	%bold extended slanted
     <-> ssub * txtt/b/sl%
}{}
\DeclareFontShape{OT1}{txtt}{bx}{it}{	%bold extended italic
     <-> ssub * txtt/b/sl%
}{}
\DeclareFontShape{OT1}{txtt}{bx}{ui}{	%bold extended unslanted italic
     <-> ssub * txtt/b/sl%
}{}

% This next set of commands uses the sectsty package to change
% section labels to be in a sans-serif font and put the numbers
% into the left margin.

\usepackage{sectsty}
\allsectionsfont{\sffamily\raggedright}
\def\@seccntformat#1{\protect\makebox[0pt][r]{\csname the#1\endcsname\quad}}

% The next definitions change the style of the section headings.

% Definition of paragraph style.

\setlength{\parindent}{0 pt}            % Unindented paragraphs, separated ...
\setlength{\parskip}{1.3 ex}            % ... by roughly one blank line.
\setlength{\partopsep}{-1ex plus 0.1ex minus -0.2ex}
\setlength{\itemsep}{-0.25ex plus 0.15ex}

% \topsep is supposed to affect list environments like itemize,
% but does nothing there.  Instead, it affects environments like tabular.

\setlength{\topsep}{0.3ex plus 0.1ex minus -0.2ex}

% Definition of section heading style.

\renewcommand{\section}{\newpage\@startsection%
  {section}{1}{0pt}{-1.8ex \@plus -1ex \@minus -.2ex}%
  {0.8ex}{\clearpage\normalfont\Large\bfseries\sffamily}}

\renewcommand{\subsection}{\@startsection%
  {subsection}{2}{0pt}{-2ex \@plus 1ex \@minus -.2ex}%
  {0.8ex}{\large\bfseries\sffamily}}

\renewcommand{\subsubsection}{\@startsection%
  {subsubsection}{3}{0pt}{-1.5ex \@plus 1ex \@minus -.2ex}%
  {0.5ex}{\slshape\normalsize\bfseries\sffamily}}

\renewcommand{\paragraph}{\@startsection%
  {paragraph}{4}{0pt}{-1.25ex \@plus 1ex \@minus -.2ex}%
  {0.5ex}{\slshape\normalsize\sffamily}}

% Just for now I replaced those definitions
% 
% \renewcommand{\section}{\@startsection%
%   {section}{1}{0pt}{-3ex \@plus -0.5ex \@minus -.2ex}%
%   {1ex}{\Large\bfseries\sffamily}}
% 
% \renewcommand{\subsection}{\@startsection%
%   {subsection}{2}{0pt}{-2.5ex \@plus -1ex \@minus -.2ex}%
%   {1ex}{\large\bfseries\sffamily}}
% 
% \renewcommand{\subsubsection}{\@startsection%
%   {subsubsection}{3}{0pt}{-1.5ex \@plus -1ex \@minus -.2ex}%
%   {0.8ex}{\slshape\normalsize\bfseries\sffamily}}
% 
% \renewcommand{\paragraph}{\@startsection%
%   {paragraph}{4}{0pt}{-1.25ex \@plus -1ex \@minus -.2ex}%
%   {0.5ex}{\slshape\normalsize\sffamily}}


% Below, we take out the heading embedded by the \tableofcontents
% command so that we can control where it's put.

\renewcommand\tableofcontents{%
    \if@twocolumn
      \@restonecoltrue\onecolumn
    \else
      \@restonecolfalse
    \fi
    \@starttoc{toc}%
    \if@restonecol\twocolumn\fi
}

% There seems to be no way to redefine the font used by the table of contents
% line printing command except to modify the source from latex.ltx.

\def\@dottedtocline#1#2#3#4#5{%
  \ifnum #1>\c@tocdepth \else
    \vskip \z@ \@plus.2\p@
    {\leftskip #2\relax \rightskip \@tocrmarg \parfillskip -\rightskip
     \parindent #2\relax\@afterindenttrue
     \interlinepenalty\@M
     \leavevmode
     \@tempdima #3\relax
     \advance\leftskip \@tempdima \null\nobreak\hskip -\leftskip
     {#4}\nobreak
     \leaders\hbox{$\m@th
        \mkern \@dotsep mu\hbox{.}\mkern \@dotsep
        mu$}\hfill
     \nobreak
     \hb@xt@\@pnumwidth{\hfil\normalfont\sffamily \normalcolor #5}%
     \par}%
  \fi}

% The following was ripped out of caption.sty, version 1.4b.
% Copyright (C) 1994-95 Harald Axel Sommerfeldt
% The first few lines set up the parameters for the layout created
% by this style file.

\newcommand{\captionsize}{\small}
\newcommand{\captionfont}{\captionsize\itshape}
\newcommand{\captionlabelfont}{\captionsize\sffamily\bfseries\upshape}
\newlength{\captionmargin}
\setlength{\captionmargin}{6ex}

\newsavebox{\as@captionbox}
\newlength{\as@captionwidth}
\newcommand{\as@normalcaption}[2]{%
  #1 #2\par}
\let\as@caption\as@normalcaption
\newcommand{\as@centercaption}[2]{%
  \parbox[t]{\as@captionwidth}{{\centering#1 #2\par}}}
\let\as@shortcaption\as@centercaption
\newcommand{\as@makecaption}[2]{%
  \setlength{\leftskip}{\captionmargin}%
  \setlength{\rightskip}{\captionmargin}%
  \addtolength{\as@captionwidth}{-2\captionmargin}%
  \renewcommand{\baselinestretch}{0.9}
  \captionfont%
  \sbox{\as@captionbox}{{\captionlabelfont #1:} #2}%
  \ifdim \wd\as@captionbox >\as@captionwidth
    \as@caption{{\captionlabelfont #1:}}{#2}%
  \else%
    \as@shortcaption{{\captionlabelfont #1:}}{#2}%
  \fi}
\renewcommand{\@makecaption}[2]{%
  \vskip\abovecaptionskip%
  \setlength{\as@captionwidth}{\linewidth}%
  \as@makecaption{#1}{#2}%
  \vskip\belowcaptionskip}
\ifx\@makerotcaption\undefined
\else
  \typeout{\space\space\space\space\space\space\space\space\space
           `rotating' package detected}
  \renewcommand{\@makerotcaption}[2]{%
    \renewcommand{\baselinestretch}{0.9}
    \captionfont%
    \sbox{\as@captionbox}{{\captionlabelfont #1:} #2}%
    \ifdim \wd\as@captionbox > .8\vsize
      \rotatebox{90}{%
        \setlength{\as@captionwidth}{.8\textheight}%
        \begin{minipage}{\as@captionwidth}%
          \as@caption{{\captionlabelfont #1:}}{#2}%
        \end{minipage}}\par
    \else%
      \rotatebox{90}{\usebox{\as@captionbox}}%
    \fi
    \hspace{12pt}}
\fi
\ifx\floatc@plain\undefined
\else
  \typeout{\space\space\space\space\space\space\space\space\space
           `float' package detected}
  \renewcommand\floatc@plain[2]{%
    \setlength{\as@captionwidth}{\linewidth}%
    \as@makecaption{#1}{#2}}
  \ifx\as@ruled\undefined
  \else
    \renewcommand\floatc@ruled[2]{%
      \setlength{\as@captionwidth}{\linewidth}%
      \renewcommand{\baselinestretch}{0.9}
      \captionfont%
      \as@caption{{\captionlabelfont #1:}}{#2}}
  \fi
\fi

\renewcommand{\@makechapterhead}[1]{%
\vspace*{50 pt}%
{\setlength{\parindent}{0pt} \raggedright% \normalfont
\sffamily\bfseries\Huge\thechapter.\ #1
\par\nobreak\vspace{40 pt}}}


%% ----------------------------------------------------------------------------
%% The end.
%% ----------------------------------------------------------------------------

\makeatother

%%% Local Variables: 
%%% mode: latex
%%% TeX-master: "../spec"
%%% End: 
