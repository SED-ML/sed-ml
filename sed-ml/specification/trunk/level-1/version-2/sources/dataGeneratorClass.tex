\subsection{\element{DataGenerator}}
\label{class:dataGenerator}

The \concept{DataGenerator} class prepares the raw simulation results for later output (\fig{sedDG}). It encodes the post-processing  to be applied to the simulation data. The post-processing steps could be anything, from simple normalisations of data to mathematical calculations. 
%
\sedfig[width=0.85\textwidth]{dataGeneratorClass}{The SED-ML DataGenerator class}{fig:sedDG}
%
Each instance of the \concept{DataGenerator} class is identifiable within the experiment by its unambiguous \concept{id}. It can be further characterised by an optional \concept{name}. The related \hyperref[sec:math]{Math} class contains a mathML expression for the calculation of the data generator. Mathematical functions available for the specification of data generators are given in Section~\ref{sec:mathML} on \refpage{sec:mathML}. \hyperref[class:variable]{Variable} and \hyperref[class:parameter]{Parameter} instances can be used to encode the mathematical expression.

\tabtext{dataGenerator}{dataGenerator}
%
\begin{table}[ht]
\center
\begin{tabular}{|l|l|}
\hline
\textbf{\attribute} & \textbf{\desc}\\
\hline
metaid$^{o}$ & \refpage{sec:metaID}\\
id & \refpage{sec:id} \\
name$^{o}$ & \refpage{sec:name}\\
\hline
%listOfVariables$^{o}$ & \refpage{sec:listOfVariables}\\
%listOfParameters$^{o}$ & \refpage{sec:listOfParameters}\\
\hline
\hline
\textbf{\subelements} & \textbf{\desc}\\
\hline
math & \refpage{sec:math}\\
notes$^{o}$ & \refpage{class:notes}\\
annotation$^{o}$ & \refpage{class:annotation}\\
\hline
variable$^{o}$ & \refpage{class:variable}\\
parameter$^{o}$ & \refpage{class:parameter}\\
\hline
\end{tabular}
\caption{\tabcap{dataGenerator}}
\label{tab:dataGenerator}
\end{table}
%

\lsttext{dataGenerator}{dataGenerator}
%
\begin{myXmlLst}{Definition of two \code{dataGenerator} elements, \emph{time} and \emph{LaCI repressor}}{lst:dataGenerator}
<listOfDatGenerators>
 <dataGenerator id="d1" name="time">
  <listOfVariables>
   <variable id="time" taskReference="task1" symbol="urn:sedml:symbol:time" />
  </listOfVariables >
  <listOfParameters />
  <math xmlns="http://www.w3.org/1998/Math/MathML">
   <ci> time </ci>
  </math>
 </dataGenerator>
 <dataGenerator id="d2" name="LaCI repressor">
  <listOfVariables>
   <variable id="v1" taskReference="task1" 
    target="/sbml:sbml/sbml:model/sbml:listOfSpecies/
            sbml:species[@id='PX']" />
  </listOfVariables>
  <math:math>
   <math:ci>v1</math:ci>
  </math:math>
 </dataGenerator>
</listOfDataGenerators>
\end{myXmlLst}
%
The \code{listOfDataGenerator} contains two \code{dataGenerator} elements. 
The first one, \code{d1}, refers to the task definition \code{t1} (which itself refers to a particular model), and from the corresponding model it reuses the symbol \code{time}.
The second one, \code{d2}, references a particular species defined in the same model (and referred to via the \code{taskReference="t1"}). The model species with ID \code{PX} is reused for the data generator \code{d2} without further post-processing.

%% Change in schema: The math attribute in the UML would become an attribuet in the XMLS -> maybe better to define a Math class (for consistent conversion and reuse of the Math class)

%% Comment: listOfVariables and listOfParameters should be global lists which are then referred to from inside the listOfDataGenerator/listOfChanges and so on -> propose a change in the XML schema


%%% Local Variables: 
%%% mode: latex
%%% TeX-master: "../sed-ml-L1V1"
%%% End: 