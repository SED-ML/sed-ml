% ~~~~~~~~~~~~~~~~~~~~~~~~~~~~~~~~~~~~
% SED-ML
% ~~~~~~~~~~~~~~~~~~~~~~~~~~~~~~~~~~~~
\subsection{\element{SED-ML} top level element}
\label{class:sed-ml}
Each SED-ML \currentLV document has a main class called \concept{SED-ML} which defines the document's structure and content (\fig{sed-mlMain}). It consists of several parts connected to the \concept{SED-ML} class via \hyperref[sec:listOf]{\element{listOf*}} constructs: 
\begin{itemize}
	\item \hyperref[class:dataDescription]{DataDescription} (for specification of external data), 
	\item \hyperref[class:model]{Model} (for specification of models),
	\item \hyperref[class:simulation]{Simulation} (for specification of simulation setups), 
	\item \hyperref[class:abstractTask]{AbstractTask} (for the linkage of models and simulation setups), 
	\item \hyperref[class:dataGenerator]{DataGenerator} (for the definition of post-processing),
	\item \hyperref[class:output]{Output} (for the specification of plots and reports).
\end{itemize}

A SED-ML document needs to have the SED-ML namespace defined through the mandatory \hyperref[sec:xmlns]{\element{xmlns}} attribute. In addition, the SED-ML \hyperref[sec:level]{\element{level}} and \hyperref[sec:version]{\element{version}} attributes are required.

The root element of each SED-ML XML file is the \code{sedML} element, encoding \hyperref[sec:level]{\element{level}} and \hyperref[sec:version]{\element{version}} of the file, and setting the necessary namespaces. Nested inside the \code{sedML} element are the six optional lists serving as containers for the encoded information: \hyperref[sec:listOfDataDescriptions]{\element{listOfDataDescriptions}} for all external data, \hyperref[sec:listOfModels]{\element{listOfModels}} for all models, \hyperref[sec:listOfSimulations]{\element{listOfSimulations}} for all simulations, \hyperref[sec:listOfTasks]{\element{listOfTasks}} for all tasks, \hyperref[sec:listOfDataGenerators]{\element{listOfDataGenerators}} for all post-processing definitions, and \hyperref[sec:listOfOutputs]{\element{listOfOutputs}} for all output definitions.

\sedfig[width=0.8\textwidth]{images/uml/sed-ml}{The SED-ML class}{fig:sed-mlMain}

The basic XML structure of a SED-ML file is shown in listing \vref{lst:sedmlRoot}.

\begin{myXmlLst}{The SED-ML root element}{lst:sedmlRoot}
<?xml version="1.0" encoding="utf-8"?>
<sedML xmlns:math="http://www.w3.org/1998/Math/MathML" 
       xmlns="http://sed-ml.org/sed-ml/level1/version3" level="1" version="3">
	<listOfDataDescriptions>
	  	[DATA REFERENCES AND TRANSFORMATIONS]
  	</listOfDataDescriptions>
	<listOfModels>
		[MODEL REFERENCES AND APPLIED CHANGES]	 	
 	</listOfModels>
	<listOfSimulations>
		[SIMULATION SETUPS]
	</listOfSimulations>
	<listOfTasks>
		[MODELS LINKED TO SIMULATIONS]
	</listOfTasks>
	<listOfDataGenerators>
		[DEFINITION OF POST-PROCESSING]
	</listOfDataGenerators>
	<listOfOutputs>
		[DEFINITION OF OUTPUT]
	</listOfOutputs>
</sedML>
\end{myXmlLst}

% ~~~ XMLNS ~~~
\subsubsection{\element{xmlns}}
\label{sec:xmlns}
The \concept{\element{xmlns}} attribute declares the namespace for the SED-ML document. The pre-defined namespace for SED-ML documents is \url{http://sed-ml.org/sed-ml/level1/version3}. 

In addition, SED-ML makes use of the \hyperref[sec:mathML]{MathML} namespace \url{http://www.w3.org/1998/Math/MathML} to enable the encoding of mathematical expressions. SED-ML \hyperref[class:notes]{notes} use the XHTML namespace \url{http://www.w3.org/1999/xhtml}. Additional external namespaces might be used in \hyperref[class:annotation]{annotations}.

% ~~~ LEVEL ~~~
\subsubsection{\element{level}}
\label{sec:level}
The current SED-ML \concept{\element{level}} is \code{\level}. Major revisions containing substantial changes will lead to the definition of forthcoming levels. The \concept{\element{level}} attribute is required and its value is a fixed \code{decimal}. For SED-ML \currentLV the value is set to \code{\level}, as shown in the example in Listing~\ref{lst:sedmlRoot}.

% ~~~ VERSION ~~~
\subsubsection{\element{version}}
\label{sec:version}
The current SED-ML \concept{\element{version}} is \code{\version}. Minor revisions containing corrections and refinements of SED-ML elements, or new constructs which do not affect backwards compatibility, will lead to the definition of forthcoming versions.

The \concept{\element{version}} attribute is required and its value is a fixed \code{decimal}. For SED-ML \currentLV the value is set to \code{\version}, as shown in the example in Listing~\ref{lst:sedmlRoot}.

% ~~~ LIST OF DATA DESCRIPTIONS ~~~
\subsubsection{\element{listOfDataDescriptions}}
\label{sec:listOfDataDescriptions}
In order to reference data in a simulation experiment, the data files along with a description on how to access such files and what information to extract from them have to be defined. The \hyperref[class:sed-ml]{SED-ML document} uses the \concept{\element{listOfDataDescriptions}} container to define \hyperref[class:dataDescription]{DataDescription}s for referencing external data (\fig{sed-mlMain}). The \concept{\element{listOfDataDescriptions}} is optional and may contain zero to many \hyperref[class:dataDescription]{DataDescriptions}.

\lsttext{listOfDataDescriptions}{listOfDataDescriptions}

\begin{myXmlLst}{SED-ML listOfDataDescriptions element}{lst:listOfDataDescriptions}
<listOfDataDescriptions>
	<dataDescription id="Data1" name="Oscli Time Course Data" source="./oscli.numl">
		<dimensionDescription>
			<compositeDescription indexType="double" id="time" name="time" xmlns="http://www.numl.org/numl/level1/version1">
        			<compositeDescription indexType="string" id="SpeciesIds" name="SpeciesIds">
         			<atomicDescription valueType="double" name="Concentrations" />
          		</compositeDescription>
      		</compositeDescription>
		</dimensionDescription>
		<listOfDataSources>
			<dataSource id="dataS1">
				<listOfSlices>
					<slice reference="SpeciesIds" value="S1" />
				</listOfSlices>
			</dataSource>
			<dataSource id="dataTime" indexSet="time" />
		</listOfDataSources>
	</dataDescription>
</listOfDataDescriptions>
\end{myXmlLst}


% ~~~ LIST OF MODELS ~~
\subsubsection{\element{listOfModels}}
\label{sec:listOfModels}
The models used in a simulation experiment are defined in the \concept{\element{listOfModels}} container (\fig{sed-mlMain}). The \concept{\element{listOfModels}} is optional and may contain zero to many \hyperref[class:model]{Models}. However, if a \hyperref[class:sed-ml]{SED-ML document} contains one or more \hyperref[class:abstractTask]{Tasks}, at least one \hyperref[class:model]{Model} must be defined to which the \hyperref[class:abstractTask]{Task} elements refer (see Section~\ref{sec:modelReference}).

\lsttext{listOfModels}{listOfModels}

\begin{myXmlLst}{SED-ML listOfModels element}{lst:listOfModels}
<listOfModels>
	<model id="m0001" language="urn:sedml:language:sbml" 
		source="urn:miriam:biomodels.db:BIOMD0000000012" />
	<model id="m0002" language="urn:sedml:language:cellml" 
		source="http://models.cellml.org/workspace/leloup_gonze_goldbeter_1999/@@rawfile/d6613d7e1051b3eff2bb1d3d419a445bb8c754ad/leloup_gonze_goldbeter_1999_a.cellml" />
</listOfModels>
\end{myXmlLst}


% ~~~ LIST OF SIMULATIONS ~~
\subsubsection{\element{listOfSimulations}}
\label{sec:listOfSimulations}
The \concept{\element{listOfSimulations}} element is the container for \hyperref[class:simulation]{Simulation} descriptions (\fig{sed-mlMain}). The \concept{\element{listOfSimulations}} is optional and may contain zero to many \hyperref[class:simulation]{Simulations}. However, if the \hyperref[class:sed-ml]{SED-ML document} contains one or more \hyperref[class:abstractTask]{Tasks}, at least one \hyperref[class:simulation]{Simulation} element must be defined to which the \hyperref[class:abstractTask]{Task} elements refer (see Section \ref{sec:simulationReference}).

\lsttext{listOfSimulations}{listOfSimulation}
\begin{myXmlLst}{The SED-ML \element{listOfSimulations} element, containing two simulation setups}{lst:listOfSimulations}
<listOfSimulations>
	<simulation id="s1" [..]>
		[UNIFORM TIMECOURSE DEFINITION]
	</simulation>
	<simulation id="s2" [..]>
   		[UNIFORM TIMECOURSE DEFINITION]
	</simulation>
</listOfSimulations>
\end{myXmlLst}
 
% ~~~ LIST OF TASKS ~~~
\subsubsection{\element{listOfTasks}}
\label{sec:listOfTasks}
The \concept{\element{listOfTasks}} element contains the defined \hyperref[class:task]{tasks} for the simulation experiment (\fig{sed-mlMain}). The \concept{\element{listOfTasks}} is optional and may contain zero to many tasks, each of which is an instance of a subclass of \hyperref[class:abstractTask]{AbstractTask}.

\lsttext{listOfTasks}{listOfTasks}
\begin{myXmlLst}{The SED-ML \code{listOfTasks} element, defining one task}{lst:listOfTasks}
<listOfTasks>
	<task id="t1" name="simulating v1" modelReference="m1" simulationReference="s1">
	[FURTHER TASK DEFINITIONS]
</listOfTasks>
\end{myXmlLst}


% ~~~ LIST OF DATA GENERATORS ~~~
\subsubsection{\element{listOfDataGenerators}}
\label{sec:listOfDataGenerators}
The \concept{\element{listOfDataGenerators}} container holds the \hyperref[class:dataGenerator]{dataGenerator} definitions of a simulation experiment (\fig{sed-mlMain}). The \concept{\element{listOfDataGenerators}} is optional and in general may contain zero to many \hyperref[class:dataGenerator]{DataGenerators}.

In SED-ML, all variable and parameter values used in the \hyperref[class:output]{Output} class need to be defined as a \hyperref[class:dataGenerator]{DataGenerator} beforehand.

% THIS IS NOT TRUE:
% However, if the \currentLV document contains an \code{Output} element, at least one \code{DataGenerator} must be defined to which the \code{Output} element refers -  see section \ref{sec:dataReference} on \refpage{sec:dataReference}.

\lsttext{listOfDataGenerators}{listOfDataGenerators}

\begin{myXmlLst}{The \code{listOfDataGenerators} element, defining two data generators \emph{time} and \emph{LaCI repressor}}{lst:listOfDataGenerators}
<listOfDataGenerators>
	<dataGenerator id="d1" name="time">
		[DATA GENERATOR DEFINITION FOLLOWING]
	</dataGenerator>
	<dataGenerator id="LaCI" name="LaCI repressor">
		[DATA GENERATOR DEFINITION FOLLOWING]
	</dataGenerator>
</listOfDataGenerators>
\end{myXmlLst}


% ~~ LIST OF OUTPUTS ~~~
\subsubsection{\element{listOfOutputs}}
\label{sec:listOfOutputs}
The \concept{\element{listOfOutputs}} container holds the \hyperref[class:output]{Output} definitions of a simulation experiment (\fig{sed-mlMain}). The \concept{\element{listOfOutputs}} is optional and may contain zero to many outputs.

\lsttext{listOfOutputs}{listOfOutputs}
\begin{myXmlLst}{The \concept{listOfOutput} element}{lst:listOfOutputs}
<listOfOutputs>
	<report id="report1">
		[REPORT DEFINITION FOLLOWING]
	</report>
	<plot2D id="plot1">
		[2D PLOT DEFINITION FOLLOWING] 
	</plot2D>
</listOfOutputs>
\end{myXmlLst}