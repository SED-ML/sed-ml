% The proposed URI scheme to use
  \subsection{URI Scheme}  
\label{sec:uriScheme}

URIs are needed at different points in SED-ML \LoneVone: 
Firstly, they are the preferred mechanism to refer to model encodings. 
Secondly, they are used to specify the language of the referenced model.
Thirdly, they enable addressing implicit model variables.
Finally, annotations of SED-ML elements should be provided with a standardised annotation scheme.

The use of a standardised URI Scheme ensures long-time availability  of particular information that can unambiguously be identified. 

\subsubsection{Model references}
\label{sec:modelURI}
The preferred way for referencing a model from a SED-ML file is adopted from the \concept{MIRIAM URI Scheme}.
MIRIAM enables identification of a data resource (in this case a model resource) by a predefined URN. A data entry inside that resource is identified by an ID. 
That way each single  model  in a particular model repository can be unambiguously referenced. To become part of MIRIAM resources, a model repository must ensure permanent and consistent model references, that is stable IDs.

One model repository that is part of MIRIAM resources is the \concept{BioModels Database} \citep{LDR+10}. Its data resource name in MIRIAM is \code{urn:miriam:biomodels.db}. To refer to a particular model, a standardised identifier scheme is defined in \concept{MIRIAM Resources}\footnote{\url{http://www.ebi.ac.uk/miriam/}}. The ID entry maps to a particular model in the model repository. That model is never deleted. 
A sample BioModels Database ID is \code{BIOMD0000000048}. Together with the data resource name it becomes unambiguously identifiable by the URN \code{urn:miriam:biomodels.db:BIOMD0000000048} (in this case referring to the 1999 Kholodenko model on EGFR signaling). 
%

SED-ML recommends to follow the above scheme for model references, if possible. 
SED-ML does not specify how to resolve the URNs. However, MIRIAM Resources offers web services to do so\footnote{\url{http://www.ebi.ac.uk/miriam/}}. For the above example of the \code{urn:miriam:biomodels.db:BIOMD0000000048} model, the resolved URL may look like: 
\begin{itemize}
 \item{\code{http://biomodels.caltech.edu/BIOMD0000000048} or}
 \item{\code{http://www.ebi.ac.uk/biomodels-main/BIOMD0000000048}}
\end{itemize}
depending on the physical location of the resource chosen to resolve the URN.

An alternative means to obtain a model is provided by a MIASE archive. This is a stand-alone, self contained file containing necessary models and a single SED-ML file.  Further information is provided in section \ref{sec:archive}.
Further information on the \hyperref[sec:source]{source} attribute referencing the model location is provided in section \ref{sec:source}.

\subsubsection{Language references}
\label{sec:languageURI}
To specify the language a model is encoded in, a set of pre-defined SED-ML URNs can be used. 
The structure of SED-ML language URNs is \element{urn:sedml:language:}\emph{name.version}. 
SED-ML allows to specify a model representation format very generally as being \code{XML}, if no standardised representation format has been used to encode the model. On the other hand, one can be as specific as defining
a model being in a particular version of a language, as ``SBML Level 2, Version 2, Revision 1''.

The list of URNs is available from \url{http://sed-ml.org/}. 
Further information on the \hyperref[sec:language]{language} attribute is provided in section \ref{sec:language}.

\subsubsection{Implicit variables}
\label{sec:implicitVariable}

Some variables used in an experiment are not explicitly defined in the model, but may be implicitly contained in it. 
For example, to plot a variable's behaviour over time, that variable is defined in an SBML model, while \emph{time} is not explicitly defined. 

To overcome this issue and allow SED-ML to refer to such variables in a common way, the notion of \emph{implicit variables} is used.
Those variables are called \code{symbols} in SED-ML. They are defined following the idea of MIRIAM URNs and using the SED-ML URN scheme. The structure of the URNs is \element{urn:sedml:symbol:}\emph{implicit variable}.
To refer from a SED-ML file to the definition of \emph{time}, for example, the URN is \element{urn:sedml:symbol:time}.

The list of predefined symbols is available from the SED-ML site on \url{http://sed-ml.org/}.
From that source, a mapping of SED-ML symbols on possibly existing concepts in the single languages supported by SED-ML is provided.

\subsubsection{Annotations}
\label{sec:annotations}
When annotating SED-ML elements with semantic \hyperref[class:annotation]{annotation}s, the \concept{MIRIAM URI Scheme} should be used. In addition to providing the data type (e.\,g.\ PubMed) and the particular data entry inside that data type (e.\,g.\ \code{10415827}), the relation of the annotation to the annotated element should be described using the standardised \concept{biomodels.net qualifier}. The list of qualifiers, as well as further information about their usage, is available from \url{http://www.biomodels.net/qualifiers/}.


%%% Local Variables: 
%%% mode: latex
%%% TeX-master: "../sed-ml-L1V1"
%%% End: 
