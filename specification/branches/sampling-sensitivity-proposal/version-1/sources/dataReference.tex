\subsubsection{dataReference}
\label{sec:dataReference}
The \concept{dataReference} is used to refer to a particular \hyperref[class:dataGenerator]{DataGenerator} instance from an \hyperref[class:output]{Output} instance. 
Listing \ref{lst:dataReference} shows the reference to a defined data set for a sample SED-ML document. 
%
\begin{myXmlLst}{Example for the use of data references in a curve definition}{lst:dataReference}
<listOfOutputs>
  <plot2D id="p1" [..] >
    <curve id="c1" xDataReference="dg1" yDataReference="dg2" />
    [..]
  </plot>
</listOfOutputs>
\end{myXmlLst}
%
In the example, the output type is a 2D plot, which defines one curve with id \code{c1}. A curve must refer to two different data generators which describe how to procure the data that is to be plotted on the x-axis and y-axis respectively. 
