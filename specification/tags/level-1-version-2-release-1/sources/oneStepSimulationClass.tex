 \subsubsection{\element{OneStep}}
\label{class:oneStep}
%
\sedfig[width=0.3\textwidth]{pdf/oneStep}{The \code{OneStep} class}{fig:oneStepSimulation}
%

The SED-ML \code{oneStep} calculates one further output step for the model from its current state. Note that this does NOT necessarily equate to one integration step. The simulator is allowed to take as many steps as needed. However, after running oneStep, the desired output time is reached.

\tabtext{oneStep}{oneStep}
%
\begin{table}[ht]
\center
\begin{tabular}{|l|l|}
\hline
\textbf{attribute} & \textbf{description}\\
\hline
metaid$^{o}$ & \refpage{sec:metaID}\\
id & \refpage{sec:id} \\
name$^{o}$ & \refpage{sec:name}\\
\hline
step & \refpage{sec:step}\\
\hline
\hline
\textbf{\subelements} & \textbf{\desc}\\
\hline
notes$^{o}$ & \refpage{class:notes}\\
annotation$^{o}$ & \refpage{class:annotation}\\
\hline
algorithm & \refpage{class:algorithm}\\
\hline
\end{tabular}
\caption{\tabcap{oneStep}}
\label{tab:oneStep}
\end{table}
%

\lsttext{oneStep}{oneStep}

%
\begin{myXmlLst}{The SED-ML \code{oneStep} element, specifying to apply the simulation algorithm for another output step of size 0.1.}{lst:oneStep}
<listOfSimulations> 
  <oneStep id="s1" step="0.1"> 
    <algorithm kisaoID="KISAO:0000019" />
  </oneStep> 
</listOfSimulations>

\end{myXmlLst}

\paragraph{\element{step}}
\label{sec:step}
The \element{oneStep} class has one required attribute \element{step} of type \code{double}.
It defines the next output point that should be reached by the algorithm, by specifying the increment from the current output point.
Listing~\ref{lst:oneStep} shows an example. 


%%% Local Variables: 
%%% mode: latex
%%% TeX-master: "../sed-ml-L1V2"
%%% End: 
