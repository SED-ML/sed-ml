% ChangeAttribute Class
  \subsubsection{\element{ChangeXML}}
\label{class:changeXml}
The \concept{ChangeXML} class defines changes of any XML element in the model that can be addressed by a valid XPath expression (\fig{changeXml}). 
%
\sedfig[width=0.75\textwidth]{changeXmlClass}{The \code{ChangeXML} class}{fig:changeXml}
%
The XPath is specified in the required \hyperref[sec:target]{target} attribute (Section~\ref{sec:target} on page \refpage{sec:target}). 
The change of XML is specified in the \hyperref[sec:newXml]{NewXML} class.

\tabtext{changeXml}{changeXml}
%
\begin{table}[ht]
\center
\begin{tabular}{|l|l|}
\hline
\textbf{\attribute} & \textbf{\desc}\\
\hline
metaid$^{o}$ & \refpage{sec:metaID}\\
id & \refpage{sec:id} \\
name$^{o}$ & \refpage{sec:name}\\
target & \refpage{sec:target}\\
\hline
\hline
\textbf{\subelements} & \textbf{\desc}\\
\hline
notes$^{o}$ & \refpage{class:notes}\\
annotation$^{o}$ & \refpage{class:annotation}\\
\hline
newXML & \refpage{sec:newXml}\\
\hline
\end{tabular}
\caption{\tabcap{changeXML}}
\label{tab:changeXml}
\end{table}
%

An example for a change that adds an additional parameter to a model is given in listing \ref{lst:changeXML}.
%
\begin{myXmlLst}{The \code{changeXML} element}{lst:changeXML}
<model [..]>
 <listOfChanges>
  <changeXML target="/sbml:sbml/sbml:model/sbml:listOfParameters/sbml:parameter[@id='V_mT']" >
   <newXML>
     <parameter metaid="metaid_0000010" id="V_mT_1" value="0.7" />
     <parameter metaid="metaid_0000050" id="V_mT_2" value="4.6"> />
   </newXML>
  </changeXML>
 </listOfChanges>
</model>
\end{myXmlLst}
%
The code of the model is changed in the way that its parameter with ID \code{V\_mT} is substituted by two other parameters \code{V\_mT\_1} and \code{V\_mT\_2}.
The \code{target} attribute defines that the parameter with ID \code{V\_mT} is to be changed. The \code{newXML} element then specifies the XML that is to be  exchanged for  that parameter.


%%% Local Variables: 
%%% mode: latex
%%% TeX-master: "../sed-ml-L1V2"
%%% End: 
