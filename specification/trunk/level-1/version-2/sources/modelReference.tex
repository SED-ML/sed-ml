\subsubsection{modelReference}
\label{sec:modelReference}
%
The \concept{modelReference} either represents a relation between two \hyperref[class:model]{Model} objects, a \hyperref[class:variable]{Variable} object and a \hyperref[class:model]{Model} object, or  a relation between a \hyperref[class:task]{Task} object and a \hyperref[class:model]{Model} object.

The \code{source} attribute of a \hyperref[class:model]{Model} is allowed to reference either a URI or an \code{SId} to a second
\hyperref[class:model]{Model}. Constructs where a model \code{A} refers to a model \code{B} and \code{B} to \code{A} (directly or indirectly) are invalid.

If pre-processing needs to be applied to a model before simulation, then the model update can be specified by creating a \hyperref[class:change]{Change} object. In the particular case that a change must be calculated with a mathematical function, variables need to be defined. To refer to an existing entity in a defined \hyperref[class:model]{Model}, the \concept{modelReference} is used. 

The \code{modelReference} attribute of the \code{variable} element contains the \concept{id} of a model that is defined in the document. 
\lsttext{modelReference1}{modelReference} 
%
\begin{myXmlLst}{SED-ML \code{modelReference} attribute inside a variable definition of a  \code{computeChange} element}{lst:modelReference1}
<model id="m0001" [..]>
 <listOfChanges>
   <computeChange>
    <listOfVariables>
     <variable id="v1" modelReference="cellML" target="/cellml:model/cellml:component[@cmeta:id='MP']/cellml:variable[@name='vsP']/@initial_value" />
     [..]
    </listOfVariables>
    <listOfParameters [..] />
    <math>
     [CALCULATION OF CHANGE]
    </math>
   </computeChange>
 </listOfChanges>
 [..]
</model>
\end{myXmlLst}
%
In the example, a change is  applied on model \code{m0001}. In the \code{computeChange} element a list of variables is defined. One of those variable is \code{v1} which is defined in another model (\code{cellML}). 
The XPath expression given in the \hyperref[sec:target]{target} attribute identifies the variable in the model which carries the ID \code{cellML}.

The \concept{modelReference} is also used to indicate that a \hyperref[class:model]{Model} object is used in a particular  \hyperref[class:task]{Task}. Listing \ref{lst:modelReference2} shows how this can be done for a sample SED-ML document.
%
\begin{myXmlLst}{SED-ML \code{modelReference} definition inside a \element{task} element}{lst:modelReference2}
<listOfTasks>
 <task id="t1" name="Baseline" modelReference="model1" simulationReference="simulation1" />
 <task id="t2" name="Modified" modelReference="model2" simulationReference="simulation1" />
</listOfTasks>
\end{myXmlLst}
%
The example defines two different tasks; the first one applies the simulation settings of \code{simulation1} on \code{model1}, the second one applies the same simulation settings on \code{model2}.

%%% Local Variables: 
%%% mode: latex
%%% TeX-master: "../sed-ml-L1V2"
%%% End: 

