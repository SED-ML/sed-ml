\label{class:task}

% Fig: sed task
\sedfig[width=0.35\textwidth]{taskClass}{The SED-ML Task class}{fig:sedTask}
%

A task in SED-ML links a \concept{model} as defined in the \concept{listOfModels} to a certain \concept{simulation} description as defined in the \concept{listOfSimulations} via the two according IDs (model ID and simulation ID).


\tabtext{task}{Task}
%
\begin{table}[ht]
\center
\begin{tabular}{|l|l|}
\hline
\textbf{\attribute} & \textbf{\desc}\\
\hline
metaid$^{o}$ & \refpage{sec:metaID}\\
id & \refpage{sec:id} \\
name$^{o}$ & \refpage{sec:name}\\
modelReference & \refpage{sec:modelReference}\\
simulationReference & \refpage{sec:simulationReference}\\
\hline
\hline
\textbf{\subelements} & \textbf{\desc}\\
\hline
notes$^{o}$ & \refpage{class:notes}\\
annotation$^{o}$ & \refpage{class:annotation}\\
\hline
\end{tabular}
\label{tab:task}
\caption{\tabcap{Task}}
\end{table}
%


Each task does have its own task \concept{id} for later reference and an optional \concept{name}.


An example linking a simulation experiment to two different models  using two task definitions within the list of tasks is given in listing \ref{lst:listOfTasks}.
%
\begin{myXmlLst}{The listOfTasks element}{lst:listOfTasks}
<listOfTasks>
  <task id="t1" name="task definition" modelReference="model1" 
        simulationReference="simulation 1" />
  <task id="t2" name="another task definition" modelReference="model2" 
        simulationReference="simulation 1" />
</listOfTasks>
\end{myXmlLst}
%
In the example, a simulation setting \emph{simulation1} is applied first to \emph{model1} and then is applied to \emph{model2}. Please note, that the tasks may be executed in any order, as XML does not have an ordering concept.

In SED-ML \version it is only possible to link one simulation description to one model at a time. However, one can define as many tasks as needed within one experiment description, i.\,e. one SED-ML file.


%%% Local Variables: 
%%% mode: latex
%%% TeX-master: "../sed-ml-L1V1"
%%% End: 