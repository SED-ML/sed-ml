% ~~~~~~~~~~~~~~~~~~~~~~~~~~~~~~~~~~~~~~~~
% SIMULATION
% ~~~~~~~~~~~~~~~~~~~~~~~~~~~~~~~~~~~~~~~~
\subsection{\element{Simulation}}
\label{class:simulation}
A simulation is the execution of some defined algorithm(s). Simulations are described differently depending on the type of simulation experiment to be performed (\fig{simulation}). 

\sedfig[width=0.9\textwidth]{images/uml/simulation}{The SED-ML Simulation class}{fig:simulation}

\concept{Simulation} is an abstract class and serves as the container for the different types of simulation experiments. SED-ML \currentLV provides the predefined simulation classes \hyperref[class:uniformTimeCourse]{UniformTimeCourse}, \hyperref[class:oneStep]{OneStep}, \hyperref[class:steadyState]{SteadyState}, and \Analysis. 

Each instance of the \concept{Simulation} class has an unambiguous and mandatory \hyperref[sec:id]{\code{id}}. An additional, optional \hyperref[sec:name]{\code{name}} may be given to the simulation. Every simulation has a required element \hyperref[class:algorithm]{\code{algorithm}} describing the simulation \hyperref[class:algorithm]{Algorithm}.

\begin{myXmlLst}{The SED-ML \code{listOfSimulations} element, defining two different UniformTimecourse simulations}{lst:simulation}
<listOfSimulations>
	<uniformTimeCourse [..]>
		[SIMULATION SPECIFICATION]
	</uniformTimeCourse>
	<uniformTimeCourse [..]>
		[SIMULATION SPECIFICATION]
	</uniformTimeCourse>
</listOfSimulations>
\end{myXmlLst}

\paragraph*{\element{algorithm}}
\label{sec:algorithm}
The mandatory attribute \concept{\element{algorithm}} defines the simulation algorithms used for the execution of the \hyperref[class:simulation]{simulation}. The algorithms are defined via the \hyperref[class:algorithm]{Algorithm} class.


%% ~~~ UNIFORM TIMECOURSE SIMULATION ~~~
\subsubsection{\element{UniformTimeCourse}}
\label{class:uniformTimeCourse}
The \concept{UniformTimeCourse} class calculates a time course output with equidistant time points. Each instance of the \concept{UniformTimeCourse} class has, in addition to the elements from \Simulation, the mandatory elements \hyperref[sec:initialTime]{\code{initialTime}}, \hyperref[sec:outputStartTime]{\code{outputStartTime}}, \hyperref[sec:outputEndTime]{\code{outputEndTime}}, and \hyperref[sec:numberOfSteps]{\code{numberOfSteps}} (\fig{simulation}).

\lsttext{timecourse}{uniformTimeCourse}

\begin{myXmlLst}{The SED-ML \code{uniformTimeCourse} element, defining a uniform time course simulation over 2500 time units with 1000 simulation points.}{lst:timecourse}
<listOfSimulations>
	<uniformTimeCourse id="s1"  name="time course simulation of variable v1 over 100 minutes"  
		initialTime="0" outputStartTime="0" outputEndTime="2500" numberOfSteps="1000">
		<algorithm [..] />
 	</uniformTimeCourse>
</listOfSimulations>
\end{myXmlLst}

\paragraph*{\element{initialTime}}
\label{sec:initialTime}
The attribute \concept{\element{initialTime}} of type \code{double} represents what the time is at the start of the simulation, for purposes of output variables, and for calculating the \element{outputStartTime} and \element{outputEndTime}.  In most cases, this will be \code{0.0}.  The model must be set up such that \element{intialTime} is correct internally with respect to any output variables that may be produced.
Listing~\ref{lst:timecourse} shows an example. 

\paragraph*{\element{outputStartTime}}
\label{sec:outputStartTime}
Sometimes a researcher is not interested in simulation results at the start of the simulation, i.e., the initial time. The \hyperref[class:uniformTimeCourse]{\element{UniformTimeCourse}} class uses the attribute \concept{\element{outputStartTime}} of type \code{double}, and describes the time (relative to the \element{intialTime}) that output is to be collected. To be valid the \concept{\element{outputStartTime}} cannot be before \hyperref[sec:initialTime]{\element{initialTime}}. For an example, see Listing~\ref{lst:timecourse}. 

\paragraph*{\element{outputEndTime}}
\label{sec:outputEndTime}
The attribute \concept{\element{outputEndTime}} of type \code{double} marks the end time of the simulation, relative to the \element{initialTime}. See Listing~\ref{lst:timecourse} for an example. 

\paragraph*{\element{numberOfSteps}}
\label{sec:numberOfSteps}
When executed, the \hyperref[class:uniformTimeCourse]{\element{UniformTimeCourse}} simulation produces an output on a regular grid starting with \hyperref[sec:outputStartTime]{\element{outputStartTime}} and ending with \hyperref[sec:outputEndTime]{\element{outputEndTime}}. The attribute \concept{\element{numberOfSteps}} of type \code{integer} describes the number of steps expected to produce the result. Software interpreting the \hyperref[class:uniformTimeCourse]{\element{UniformTimeCourse}} is expected to produce a first outputPoint at time \hyperref[sec:outputStartTime]{\element{outputStartTime}} and then \concept{\element{numberOfSteps}} output points with the results of the simulation. Thus a total of \code{numberOfSteps + 1} output points will be produced.

Just because the output points lie on the regular grid described above, does not mean that the simulation algorithm has to work with the same step size. Usually the step size the simulator chooses will be adaptive and much smaller than the required output step size. On the other hand a stochastic simulator might not have any new events occurring between two grid points. Nevertheless the simulator has to produce data on this regular grid. For an example, see Listing~\ref{lst:timecourse}.

\changed{This attribute used to be named \element{numberOfPoints}, but was defined to be `the number of output points minus one', which was confusing.  The old name is thus deprecated, and the new name is more in line with its definition.}


% ~~~ ONESTEP SIMULATION ~~~
\subsubsection{\element{OneStep}}
\label{class:oneStep}

The \concept{OneStep} class calculates one further output step for the model from its current state. Each instance of the \concept{OneStep} class has, in addition to the elements from \Simulation, the mandatory element \hyperref[sec:step]{\code{step}} (\fig{simulation}).

\lsttext{oneStep}{oneStep}

\begin{myXmlLst}{The SED-ML \code{oneStep} element, specifying to apply the simulation algorithm for another output step of size 0.1.}{lst:oneStep}
<listOfSimulations> 
	<oneStep id="s1" step="0.1"> 
		<algorithm kisaoID="KISAO:0000019" />
	</oneStep> 
</listOfSimulations>
\end{myXmlLst}

\paragraph*{\element{step}}
\label{sec:step}
The \hyperref[class:oneStep]{OneStep} class has one required attribute \concept{\element{step}} of type \code{double}. It defines the next output point that should be reached by the algorithm, by specifying the increment from the current output point. Listing~\ref{lst:oneStep} shows an example. 

Note that the \concept{\element{step}} does not necessarily equate to one integration step. The simulator is allowed to take as many steps as needed. However, after running oneStep, the desired output time is reached.


% ~~~ STEADYSTATE SIMULATION ~~~
\subsubsection{\element{SteadyState}}
\label{class:steadyState}
The \concept{SteadyState} represents a steady state computation (as for example implemented by NLEQ or Kinsolve). The \concept{SteadyState} class has no additional elements than the elements from \Simulation (\fig{simulation}).

\lsttext{steadyState}{steadyState}

\begin{myXmlLst}{The SED-ML \code{steadyState} element, defining a steady state simulation with id \code{steady}.}{lst:steadyState}
<listOfSimulations>
	<steadyState id="steady"> 
		<algorithm kisaoID="KISAO:0000282" />
	</steadyState > 
</listOfSimulations>
\end{myXmlLst}


\begin{blockChanged}
% ~~~ ANALYSIS SIMULATION ~~~
\subsubsection{\element{Analysis}}
\label{class:analysis}
The \concept{Analysis} represents any sort of analysis or simulation of a \Model, entirely defined by its child \Algorithm.  If a simulation can be defined by a different \Simulation, that should be used instead, so that tools are more likely to recognize the request.  But for any simultion or any analysis not covered by \SteadyState, \OneStep, or \UniformTimeCourse, the only thing necessary is a KiSAO term for the \Algorithm defining what to do.  The following examples illustrate analyses that could not be created with other SED-ML \Simulation classes:

\lsttext{analysis}{analysis}

\begin{myXmlLst}{The SED-ML \code{analysis} element, defining a time course with a stop condition ($ObsA<9$).}{lst:analysis}
<listOfSimulations>
    <analysis id="time_course_to_stop_condition">
        <algorithm kisaoID="KISAO:0000263"name="NFSim">
            <algorithmParameter kisaoID="KISAO:0000525" value="ObsA&gt;9"/>
            <algorithmParameter kisaoID="KISAO:0000840" value="0" name="start time"/>
            <algorithmParameter kisaoID="KISAO:0000841" value="10000" name="max end time"/>
            <algorithmParameter kisaoID="KISAO:0000842" value="0.5" name="observed step size"/>
        </algorithm>
    </analysis >
</listOfSimulations>
\end{myXmlLst}

\lsttext{analysis2}{analysis}

\begin{myXmlLst}{The SED-ML \code{analysis} element, defining a non-uniform time course.}{lst:analysis2}
<listOfSimulations>
    <analysis id="non_uniform_time_course">
        <algorithm kisaoID="KISAO:0000057" name="Brownian diffusion Smoluchowski method">
            <algorithmParameter kisaoID="KISAO:0000525" value="ObsA&gt;9" name="stop condition"/>
            <algorithmParameter kisaoID="KISAO:0000840" value="0" name="start time"/>
            <algorithmParameter kisaoID="KISAO:0000841" value="100" name="max end time"/>
        </algorithm>
    </analysis >
</listOfSimulations>
\end{myXmlLst}

\lsttext{analysis3}{analysis}

\begin{myXmlLst}{The SED-ML \code{analysis} element, defining the Klarner ASP logical model trap space identification method, using the reduced model.}{lst:analysis3}
<listOfSimulations>
    <analysis id="non_uniform_time_course">
        <algorithm kisaoID="KISAO:0000662" name="Klarner ASP logical model trap space identification method">
            <algorithmParameter kisaoID="KISAO:0000216" value="true" name="use reduced model"/>
        </algorithm>
    </analysis >
</listOfSimulations>
\end{myXmlLst}
\end{blockChanged}


% ~~~~~~~~~~~~~~~~~~~~~~~~~~~~~~~~~~~~~~~~
% SIMULATION COMPONENTS
% ~~~~~~~~~~~~~~~~~~~~~~~~~~~~~~~~~~~~~~~~
\subsection{\element{Simulation} components}
\label{class:simulationComponents}

% ~~~ ALGORITHM ~~~
\subsubsection{\element{Algorithm}}
\label{class:algorithm}
The \concept{Algorithm} class has a mandatory element \hyperref[sec:kisaoid]{\element{kisaoID}} which contains a \hyperref[sec:kisao]{KiSAO} reference to the particular simulation algorithm used in the \hyperref[class:simulation]{simulation}. In addition, the \concept{Algorithm} has an optional \hyperref[class:listOfAlgorithmParameters]{\element{listOfAlgorithmParameters}}, a collection of \hyperref[class:algorithmParameter]{algorithmParameter}, which are used to parameterize the \concept{algorithm}.

The example given in Listing~\ref{lst:simulation}, completed by algorithm definitions results in the code given in Listing \ref{lst:algorithm}. In the example, for both simulations a algorithm is defined. In the first simulation \code{s1} a deterministic approach is used (Euler forward method), in the second simulation \code{s2} a stochastic approach is used (Stochsim nearest neighbor).

\begin{myXmlLst}{The SED-ML \code{algorithm} element for two different time course simulations, defining two different algorithms. KISAO:0000030 refers to the \emph{Euler forward method} ; KISAO:0000021 refers to the \emph{StochSim nearest neighbor algorithm}.}{lst:algorithm}
<listOfSimulations>
	<uniformTimeCourse id="s1" name="time course simulation over 100 minutes" [..]>
		<algorithm kisaoID="KISAO:0000030" />
	</uniformTimeCourse>
	<uniformTimeCourse id="s2" name="time course definition for concentration of p" [..]>
		<algorithm kisaoID="KISAO:0000021" />
	</uniformTimeCourse>
</listOfSimulations>
\end{myXmlLst}


% ~~~ LIST OF ALGORITHM PARAMETERS ~~~
\paragraph*{\element{listOfAlgorithmParameters}}
\label{class:listOfAlgorithmParameters}
The \concept{\element{listOfAlgorithmParameters}} contains the settings for the simulation algorithm used in a \hyperref[class:simulation]{simulation} (\fig{simulation}). It may list several instances of the \hyperref[class:algorithmParameter]{AlgorithmParameter} class. The \concept{\element{listOfAlgorithmParameters}} is optional and may contain zero to many parameters.

\lsttext{listOfAlgorithmParameters}{listOfAlgorithmParameters}
\begin{myXmlLst}{SED-ML listOfAlgorithmParameters element}{lst:listOfAlgorithmParameters}
<listOfAlgorithmParameters>
	<algorithmParameter name="absolute tolerance" kisaoID="KISAO:0000211" value="23"/> 
</listOfAlgorithmParameters>
\end{myXmlLst}


% ~~~ ALGORITHM PARAMETER ~~~
\subsubsection{\element{AlgorithmParameter}}
\label{class:algorithmParameter}
The \concept{AlgorithmParameter} class allows to parameterize a particular simulation \hyperref[class:algorithm]{algorithm}. The set of possible parameters for a particular instance is determined by the algorithm that is referenced by the \element{kisaoID} of the enclosing \hyperref[class:algorithm]{algorithm} element (\fig{simulation}). Parameters of simulation algorithms are unambiguously referenced by the mandatory \hyperref[sec:kisaoid]{\element{kisaoID}} attribute. Their value is set in the mandatory \hyperref[sec:algorithmParameterValue]{\element{value}} attribute.  \changed{An \AlgorithmParameter may also have child \AlgorithmParameter elements through a \ListOfAlgorithmParameters.}

\changed{Any \AlgorithmParameter defined in a \Simulation overrides any global \AlgorithmParameter defined in the \SedDocument.  And in the reverse, any \AlgorithmParameter left undefined in a \Simulation may be defined by a global \AlgorithmParameter element.  Any \AlgorithmParameter child of a \Simulation applies only to that individual \Simulation, and assumes its previous value (if set globally) or becomes unset (if not) outside of the context of that \Simulation (for example, within a \RepeatedTask).}

\changed{NOTE:  the global \ListOfAlgorithmParameters was added to SED-ML in \currentLV.  As such, the only place to define a random seed (\code{KISAO:0000488)} for the simulation experiment as a whole in previous versions was in a \Simulation, which might be part of a \RepeatedTask.  Rather than indicating that each repeat was to receive the same seed, resulting in identical traces, users would generally use the 'seed' parameter to indicate that the experiment as a whole was to be replicable from one run to the next.  Current users of SED-ML should use a global \AlgorithmParameter for this purpose, but older versions or older files may use the previous scheme.}

\paragraph*{\element{value}}
\label{sec:algorithmParameterValue}
The \concept{\element{value}} sets the value of the \hyperref[class:algorithmParameter]{AlgorithmParameter}.  For XML purposes, it is of type \element{string}, but should contain a value that makes sense for the \element{kisaoID} in question:  if the KiSAO term is a value, the string should contain a number; if the KiSAO term is a Boolean, the string should contain the string \val{true} or \val{false}, etc.  \changed{The string must be non-empty; to explicitly state that a value is not set, this should be encoded in the string as a value such as \val{null} or \val{unset}, as makes sense for the term in question.}

\begin{myXmlLst}{The SED-ML \concept{\code{algorithmParameter}} element setting the parameter value for the simulation algorithm. \code{KISAO:0000032} refers to the \emph{explicit fourth-order Runge-Kutta method}; \code{KISAO:00000211} refers to the \emph{absolute tolerance}.}{lst:algorithmParameter}
<algorithm kisaoID="KISAO:0000032"> 
	<listOfAlgorithmParameters> 
		<algorithmParameter name="absolute tolerance" kisaoID="KISAO:0000211" value="23"/> 
	</listOfAlgorithmParameters>
</algorithm>
\end{myXmlLst}

\begin{blockChanged}
\paragraph*{\element{listOfAlgorithmParameters}}
The child \element{listOfAlgorithmParameters} of an \AlgorithmParameter may contain parameters that modify or refine the parent parameter, depending on the KiSAO term used.  

\begin{myXmlLst}{A SED-ML \code{algorithmParameter} element defining a mixed simulator, each with their own set of algorithm parameters.}{lst:algorithmParameter2}
<algorithm name="hybrid method" kisaoID="KISAO:0000352">
    <listOfAlgorithmParameters>
        <algorithmParameter name="modeling and simulation algorithm" kisaoID="KISAO:0000000" value="KISAO:0000019">
            <listOfAlgorithmParameters>
                <algorithmParameter name="absolute tolerance"   kisaoID="KISAO:0000211" value="1e-07"/>
                <algorithmParameter name="integration method"   kisaoID="KISAO:0000475" value="BDF"/>
                <algorithmParameter name="interpolate solution" kisaoID="KISAO:0000481" value="true"/>
                <algorithmParameter name="iteration type"       kisaoID="KISAO:0000476" value="Newton"/>
                <algorithmParameter name="linear solver"        kisaoID="KISAO:0000477" value="Dense"/>
                <algorithmParameter name="lower half-bandwidth" kisaoID="KISAO:0000480" value="0"/>
                <algorithmParameter name="maximum number of steps" kisaoID="KISAO:0000415" value="500"/>
                <algorithmParameter name="maximum step size"    kisaoID="KISAO:0000467" value="0"/>
                <algorithmParameter name="preconditioner"       kisaoID="KISAO:0000478" value="Banded"/>
                <algorithmParameter name="relative tolerance"   kisaoID="KISAO:0000209" value="1e-07"/>
                <algorithmParameter name="upper half-bandwidth" kisaoID="KISAO:0000479" value="0"/>
            </listOfAlgorithmParameters>
        </algorithmParameter>
        <algorithmParameter name="modeling and simulation algorithm" kisaoID="KISAO:0000000" value="KISAO:0000282">
            <listOfAlgorithmParameters>
                <algorithmParameter name="linear solver"        kisaoID="KISAO:0000477" value="Dense"/>
                <algorithmParameter name="lower half-bandwidth" kisaoID="KISAO:0000480" value="0"/>
                <algorithmParameter name="maximum iterations"   kisaoID="KISAO:0000486" value="200"/>
                <algorithmParameter name="upper half-bandwidth" kisaoID="KISAO:0000479" value="0"/>
            </listOfAlgorithmParameters>
        </algorithmParameter>
    </listOfAlgorithmParameters>
</algorithm>
\end{myXmlLst}


\end{blockChanged}



