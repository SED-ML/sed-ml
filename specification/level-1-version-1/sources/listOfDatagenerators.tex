 \subsubsection{listOfDataGenerators: The post-processing container}
\label{sec:listOfDataGenerators}

In SED-ML, all variable- and parameter values that shall be used in the \hyperref[class:output]{Output} class need to be defined as a \hyperref[class:dataGenerator]{dataGenerator} beforehand. The container for those data generators is the \concept{listOfDataGenerators} (\fig{listOfDataGenerators}). 
% Fig: DG
\sedfig[width=0.85\textwidth]{listOfDataGenerators}{The SED-ML listOfDataGenerators container}{fig:listOfDataGenerators}
%

\lsttext{listOfDataGenerators}{listOfDataGenerators}
%
\begin{myXmlLst}{The \code{listOfDataGenerators} element, defining two data generators \emph{time} and \emph{LaCI repressor}}{lst:listOfDataGenerators}
<listOfDatGenerators>
 <dataGenerator id="d1" name="time">
  [DATA GENERATOR DEFINITION FOLLOWING]
 </dataGenerator>
 <dataGenerator id="LaCI" name="LaCI repressor">
  [DATA GENERATOR DEFINITION FOLLOWING]
 </dataGenerator>
</listOfDataGenerators>
\end{myXmlLst}

The \code{listOfDataGenerators} is optional and in general may contain zero to many DataGenerators. However, if the \LoneVone document contains  an  \code{Output}  element, at least one  \code{DataGenerator} must be defined to which the \code{Output} element refers -  see  section \ref{sec:dataReference} on \refpage{sec:dataReference}.
%



%%% Local Variables: 
%%% mode: latex
%%% TeX-master: "../sed-ml-L1V1"
%%% End: 
