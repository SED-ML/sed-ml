% ~~~~~~~~~~~~~~~~~~~~~~~~~~~~~~~~~~~~~~~~
%% OUTPUT
% ~~~~~~~~~~~~~~~~~~~~~~~~~~~~~~~~~~~~~~~~ 
\subsection{\element{Output}}
\label{class:output}

The abstract \concept{Output} class describes how the results of a simulation are presented (\fig{output}). The available output classes are plots (\hyperref[class:plot2D]{Plot2D} and \hyperref[class:plot3D]{Plot3D}) and reports (\hyperref[class:report]{Report}). The data used in \concept{Outputs} is provided via \hyperref[class:dataGenerator]{DataGenerators}.

\sedfig[width=0.6\textwidth]{images/uml/output}{The SED-ML Output class. Note that ListOfCurves, Curve, ListOfSurfaces, Surface, ListOfDataSets, DataSet and DataGenerator are subclasses of \emph{SEDBase}; the respective inheritance connections are not shown in the figure.}{fig:output}

\sedfig[width=0.6\textwidth]{images/uml/plot}{The SED-ML Plot class}{fig:plot}

\sedfig[width=0.6\textwidth]{images/uml/report}{The SED-ML Report class}{fig:report}

\sedfig[width=0.6\textwidth]{images/uml/abstractCurve}{The SED-ML AbstractCurve class}{fig:abstractCurve}

\sedfig[width=1.0\textwidth]{images/uml/abstractSurface}{The SED-ML AbstractSurface class}{fig:abstractSurface}


Note that even though the terms \code{Plot2D} and \code{Plot3D} are used, the exact type of plot is not specified. In other words, whether the 3D plot represents a surface plot, or three dimensional lines in space, cannot be distinguished by SED-ML SED-ML \currentLV alone.

% ~~~ PLOT ~~~
\subsubsection{\element{Plot}}
\label{class:plot}
\hl{TODO: fill in}

\paragraph*{\element{title}}
\hl{TODO: fill in}

\paragraph*{\element{subtitle}}
\hl{TODO: fill in}

\paragraph*{\element{legend}}
The attribute {\tt Legend} is of type Boolean that indicates whether a legend is drawn or not. 

\paragraph*{\element{border}}
\hl{TODO: fill in}

% ~~~ PLOT2D ~~~
\subsubsection{\element{Plot2D}}
\label{class:plot2D}
The \concept{Plot2D} class is used for two dimensional plot outputs (\fig{output}). The \concept{Plot2D} contains a number of \hyperref[class:curve]{Curve} definitions in the \element{listOfCurves}, defining the \hyperref[class:curve]{curves} to be plotted in the the 2D plot.

\lsttext{listOfCurves}{listOfCurves}
The example shows the definition of a \element{Plot2D} containing one \hyperref[class:curve]{Curve} inside the \element{listOfCurves}.
\begin{myXmlLst}{The \code{plot2D} element with the nested \code{listOfCurves} element.}{lst:listOfCurves}
<plot2D>
	<listOfCurves>
		<curve>
			[CURVE DEFINITION]
		</curve>
	</listOfCurves>
</plot2D>
\end{myXmlLst}


% ~~~ PLOT3D ~~~
\subsubsection{\element{Plot3D}}
\label{class:plot3D}
The \concept{Plot3D} class is used for three dimensional plot outputs (\fig{output}). The \concept{Plot3D} contains a number of \hyperref[class:surface]{Surface} definitions in the \element{listOfSurfaces}. 

\lsttext{plot3D}{plot3D}
The example shows the definition of a \hyperref[class:surface]{\element{Surface}} for the three dimensional plot inside the \code{listOfSurfaces}.
\begin{myXmlLst}{The \code{plot3D} element with the nested \element{listOfSurfaces} element}{lst:plot3D}
<plot3D>
	<listOfSurfaces>
		<surface> 
			[SURFACE DEFINITION]
		</surface>
		[FURTHER SURFACE DEFINITIONS]
	</listOfSurfaces>
</plot3D>
\end{myXmlLst}


% ~~~ REPORT ~~~
\subsubsection{\element{Report}}
\label{class:report}
The \concept{Report} class defines a data table consisting of several single instances of the \hyperref[class:dataSet]{\element{DataSet}} in the \element{listOfDataSets} (\fig{output}). Its output returns the simulation result processed via \hyperref[class:dataGenerator]{DataGenerators} in actual numbers. The columns of the report table are defined by creating an instance of the \hyperref[class:dataSet]{DataSet} for each column. 

\lsttext{listOfDataSets}{listOfDataSets}

\begin{myXmlLst}{The \code{report} element with the nested \code{listOfDataSets} element}{lst:listOfDataSets}
<report>
	<listOfDataSets>
		<dataSet>
			[DATA REFERENCE]
		</dataSet>
	</listOfDataSets>
</report>
\end{myXmlLst}

The simulation result itself, i.e.\ concrete result numbers, are not stored in SED-ML, but the directive how to calculate them from the output of the simulator is provided through the \hyperref[class:dataGenerator]{dataGenerator}. The encoding of simulation results is not part of SED-ML \currentLV.

% ~~~~~~~~~~~~~~~~~~~~~~~~~~~~~~~~~~~~~~~~
% OUTPUT COMPONENTS
% ~~~~~~~~~~~~~~~~~~~~~~~~~~~~~~~~~~~~~~~~ 
\subsection{\element{Output} components}
In this section the \hyperref[class:output]{Output} components \hyperref[class:curve]{Curve}, \hyperref[class:surface]{Surface}, and \hyperref[class:dataSet]{DataSet} are described.


% ~~~ AXIS ~~~
\subsubsection{\element{Axis}}
\label{class:axis}
\hl{TODO: fill in}

\paragraph*{\element{log}}
The attribute {\tt log} of type \element{boolean} defines if the corresponding axis will be drawn with a log scale or not (linear scale).

\paragraph*{\element{min} and \element{max}}
The {\tt min} and {\tt max} attributes in the {\tt Axis} class indicate the minimum and maximum values for the axis. 

\paragraph*{\element{relativeDimension}}
The attribute {\tt relativeDimension} indicates the relative size of the axis relative to all other axes defined for the plot.
\hl{TODO: clarify, how this works with 3 axes}


\paragraph*{\element{label}}
The attribute \concept{\element{label}} defines the label that will be drawn next to each axis. It is usual that the $y$ axis label will be drawn vertically (parallel to the axis direction in 3D).  

\paragraph*{\element{majorGrid}}
\hl{TODO: fill in}

\paragraph*{\element{minorGrid}}
\hl{TODO: fill in}


% TODO: This section must be formated in correct style and tags, see the robust sections like model for how to --->
\hl{TODO: style and clarify the next 2 sections}
\subsubsection{\element{Marker}}
\label{class:marker}
A marker is a symbol used to indicate a data point on a graph. It has four attributes: the marker style, of type {\tt MarkerType}~\ref{type:markerType}; the fill color and the line color, of type {\tt ColorType}~\ref{type:colorType}; and the line thickness of the outline, of type double. To make a marker invisible, use the style of {\tt none}.

\begin{myXmlLst}{The \element{marker} element}{lst:marker}
<sedML [..]>
 <marker style="square" fillColor="#FF0000" lineColor="#FF0000" lineThickness="2.0"/>
</sedML>
\end{myXmlLst}

\paragraph*{\element{fillColor}}
The {\tt fillColor} of a marker is of type {\tt ColorType} and is used to define the interior color of the marker. 

\paragraph*{\element{lineColor}}
The {\tt lineColor} of a marker is of type {\tt ColorType} and is used to define the outline color of the marker. The thickness of the outline is described by the thickness attribute.

\paragraph*{\element{lineThickness}}
The {\tt lineThickness} of a marker is of type {\tt double} and is used to define the thickness of the marker outline. The units of the thickness are not defined.

\subsubsection{\element{Line}}
\label{class:Line}
A line defines the styling characteristics of a plotted curve. It consists of three attributes: the line style, of type {\tt LineType}; the line color, of type {\tt ColorType}; and the line thickness, of type {\tt double}.

\begin{myXmlLst}{The \element{line} element}{lst:line}
<sedML [..]>
 <line style="solid" color="#00FF00" thickness="2.0"/>
</sedML>
\end{myXmlLst}

\paragraph*{\element{style}}
The {\tt style} is of type {\tt LineStyle} and describes the appearance of the line, whether it be solid, dashed, none etc. To make a line invisible, use a value of {\tt none} for the style. 

\paragraph*{\element{color}}
The {\tt color} is of type {\tt ColorType} and defines the color of a line.

\paragraph*{\element{thickness}}
The {\tt thickness} is of type {\tt double} and defines the thickness of a drawn line. The units of the thickness are not defined.
% <-------------- until here

% ~~~ ABSTRACT CURVE ~~~
\subsubsection{\element{AbstractCurve}}
\label{class:axis}
\hl{TODO: fill in and remove Curve}

% ~~~ ABSTRACT SURFACE ~~~
\subsubsection{\element{AbstractSurface}}
\label{class:axis}
\hl{TODO: fill in and remove Surface}

% ~~~ CURVE ~~~
\subsubsection{\element{Curve}}
\label{class:curve}
A \concept{Curve} is a two-dimensional \hyperref[class:output]{Output} component representing a (processed) simulation result (\fig{output}). Zero or more \concept{Curve} instances define a \hyperref[class:plot2D]{plot2D} (\fig{output}). A \concept{curve} needs a \hyperref[class:dataGenerator]{dataGenerator} reference to refer to the data that will be plotted on the x-axis, using the \hyperref[sec:xDataReference]{\element{xDataReference}}. A second \hyperref[class:dataGenerator]{dataGenerator} reference is needed to refer to the data that will be plotted on the y-axis, using the \hyperref[sec:yDataReference]{\element{yDataReference}}. 

\lsttext{curve}{curve}
In the example a single curve is created. Results for the x-axis are generated by the dataGenerator \code{dg1}, results for the y-axis are generated by the \hyperref[class:dataGenerator]{dataGenerator} \code{dg2}. Both \code{dg1} and \code{dg2} need to be defined in the \hyperref[sec:listOfDataGenerators]{\element{listOfDataGenerators}}. The x-axis is plotted logarithmically.
\begin{myXmlLst}{The SED-ML \code{curve} element, defining the output curve showing the result of simulation for the referenced dataGenerators}{lst:curve}
<listOfCurves>
	<curve id="c1" name="v1 / time" xDataReference="dg1" yDataReference="dg2" logX="true" logY="false" />
</listOfCurves>
\end{myXmlLst}

\paragraph*{\element{logX}}
\label{sec:logX}
\concept{\element{logX}} is a required attribute of the \hyperref[class:curve]{Curve} class and defines whether or not the data output on the x-axis is logarithmic. The data type of \concept{\element{logX}} is \code{boolean}. To make the output on the x-axis of a plot logarithmic, \concept{\element{logX}} must be set to \code{true}, as shown in the sample Listing~\ref{lst:curve}.

\paragraph*{\element{xDataReference}}
\label{sec:xDataReference}
The \concept{\element{xDataReference}} is a mandatory attribute of the \hyperref[class:curve]{Curve} object. Its content refers to a \hyperref[class:dataGenerator]{dataGenerator} which denotes the \hyperref[class:dataGenerator]{DataGenerator} object that is used to generate the output on the x-axis of a \hyperref[class:curve]{Curve} in a \hyperref[class:plot2D]{plot2D}. 
The \concept{\element{xDataReference}} data type is \hyperref[type:sidref]{\element{SIdRef}}. However, the valid values for the \concept{\element{xDataReference}} are restricted to the \hyperref[sec:id]{\element{id}} of already defined \hyperref[class:dataGenerator]{DataGenerators}.

\paragraph*{\element{logY}}
\label{sec:logY}
\concept{\element{logY}} is a required attribute of the \hyperref[class:curve]{Curve} class and defines whether or not the data output on the y-axis is logarithmic. The data type of \concept{\element{logY}} is \code{boolean}. To make the output on the y-axis of a plot logarithmic, \concept{\element{logY}} must be set to \code{true}, as shown in the sample Listing~\ref{lst:curve}. 

\paragraph*{\element{yDataReference}}
\label{sec:yDataReference}
The \concept{\element{yDataReference}} is a mandatory attribute of the \hyperref[class:curve]{Curve} object. Its content refers to a \hyperref[class:dataGenerator]{dataGenerator} which denotes the \hyperref[class:dataGenerator]{DataGenerator} object that is used to generate the output on the y-axis of a \hyperref[class:curve]{Curve} in a \hyperref[class:plot2D]{plot2D}.
The \concept{\element{yDataReference}} data type is \code{SIdRef}. However, the valid values for the \concept{\element{yDataReference}} are restricted to the \hyperref[sec:id]{\element{id}} of already defined \hyperref[class:dataGenerator]{DataGenerators}.


% ~~~ SURFACE ~~~
\subsubsection{\element{Surface}}
\label{class:surface}
A \concept{Curve} is a three-dimensional \hyperref[class:output]{Output} component representing a (processed) simulation result (\fig{output}). Zero or more \concept{Surface} instances define a \hyperref[class:plot3D]{plot3D} (\fig{output}).

\concept{Surface} is a subclass of \hyperref[class:curve]{\element{Curve}} inheriting among others the elements \hyperref[sec:xDataReference]{\element{xDataReference}}, \hyperref[sec:yDataReference]{\element{yDataReference}}, \hyperref[sec:logX]{\element{logX}}, and \hyperref[sec:logY]{\element{logY}}.
 
Creating an instance of the \concept{Surface} class requires the definition of data on three different axis. The aforementioned \hyperref[sec:xDataReference]{\element{xDataReference}} and \hyperref[sec:yDataReference]{\element{yDataReference}} attributes define the \hyperref[class:dataGenerator]{dataGenerators} for the x- and y-axis of a surface. In addition, the \hyperref[sec:zDataReference]{\element{zDataReference}} attribute defines the output for the z-axis. All axes might be logarithmic or not. This can be specified through the \hyperref[sec:logX]{\element{logX}}, \hyperref[sec:logY]{\element{logY}}, and the \hyperref[sec:logZ]{\element{logZ}} attributes in the according dataReference elements.

\tabtext{surface}{surface}

\begin{table}[ht]
\center
\begin{tabular}{ll}
\toprule
\textbf{\attribute} & \textbf{\desc}\\
\midrule
metaid$^{o}$ & \refpage{sec:metaid}\\
id & \refpage{sec:id} \\
name$^{o}$ & \refpage{sec:name}\\
\midrule
logX & \refpage{sec:logX}\\
xDataReference & \refpage{sec:xDataReference}\\
logY & \refpage{sec:logY}\\
yDataReference & \refpage{sec:yDataReference}\\
logZ & \refpage{sec:logZ}\\
zDataReference & \refpage{sec:zDataReference}\\
\midrule
\textbf{\subelements} & \textbf{\desc}\\
\midrule
notes$^{o}$ & \refpage{class:notes}\\
annotation$^{o}$ & \refpage{class:annotation}\\
\bottomrule
\end{tabular}
\caption{\tabcap{surface}}
\label{tab:surface}
\end{table}

\lsttext{surface}{surface}
In the example a single surface is created. Results shown on the x-axis are generated by the data generator \code{dg1}, results on the y-axis by dataGenerator \code{dg2}, results on the z-axis by dataGenerator \code{dg3}. All used \hyperref[class:dataGenerator]{dataGenerators}, i.e. \code{dg1}, \code{dg2} and \code{dg3}, must be defined in the \hyperref[sec:listOfDataGenerators]{\element{listOfDataGenerators}}.
\begin{myXmlLst}{The SED-ML \code{surface} element, defining the output showing the result of the referenced task}{lst:surface}
<listOfSurfaces>
	<surface id="s1" name="surface" xDataReference="dg1" yDataReference="dg2" zDataReference="dg3" 
		logX="true"  logY="false" logZ="false" />
	[FURTHER SURFACE DEFINITIONS]
</listOfSurfaces>
\end{myXmlLst}

\paragraph*{\element{logZ}}
\label{sec:logZ}
\concept{\element{logZ}} is a required attribute of the \hyperref[class:surface]{Surface} class and defines whether or not the data output on the z-axis is logarithmic. The data type of \concept{\element{logZ}} is \code{boolean}. To make the output on the z-axis of a surface plot logarithmic, \concept{\element{logZ}} must be set to \code{true}, as shown in the sample Listing~\ref{lst:surface}.

\paragraph*{\element{zDataReference}}
\label{sec:zDataReference}
The \concept{\element{zDataReference}} is a mandatory attribute of the \hyperref[class:surface]{Surface} object. Its content refers to a \hyperref[class:dataGenerator]{dataGenerator} which denotes the \hyperref[class:dataGenerator]{DataGenerator} object that is used to generate the output on the z-axis of a \hyperref[class:plot3D]{plot3D}. The \concept{\element{zDataReference}} data type is \hyperref[type:sidref]{\element{SIdRef}}. However, the valid values for the \concept{\element{zDataReference}} are restricted to the \hyperref[sec:id]{\element{id}} of already defined \hyperref[class:dataGenerator]{DataGenerators}.

%% ~~~ DATASET ~~~
\subsubsection{\element{DataSet}}
\label{class:dataSet}
The \concept{DataSet} class holds definitions of data to be used in the \hyperref[class:report]{Report} class (\fig{output}). DataSets are labeled references to instances of the \hyperref[class:dataGenerator]{DataGenerator} class.

\tabtext{dataSet}{dataSet}

\begin{table}[h!t]
\center
\begin{tabular}{ll}
\toprule
\textbf{\attribute} & \textbf{\desc}\\
\midrule
metaid$^{o}$ & \refpage{sec:metaid}\\
id & \refpage{sec:id} \\
name$^{o}$ & \refpage{sec:name}\\
\midrule
dataReference & \refpage{sec:dataReference1}\\
label & \refpage{sec:label}\\
\midrule
\textbf{\subelements} & \textbf{\desc}\\
\midrule
notes$^{o}$ & \refpage{class:notes}\\
annotation$^{o}$ & \refpage{class:annotation}\\
\bottomrule
\end{tabular}
\caption{\tabcap{dataSet}}
\label{tab:dataSet}
\end{table}

\paragraph*{\element{label}}
\label{sec:label}
Each data set in a \hyperref[class:report]{Report} must have an unambiguous \concept{\element{label}}. A \concept{\element{label}} is a human readable descriptor of a data set for use in a \hyperref[class:report]{report}. For example, for a tabular data set of time series results, the \concept{\element{label}} could be the column heading. 

\paragraph*{\element{dataReference}}
\label{sec:dataReference1}
The \concept{\element{dataReference}} attribute contains the ID of a \concept{dataGenerator} element and as such represents a link to it. The data produced by that particular \hyperref[class:dataGenerator]{dataGenerator} fills the according \hyperref[class:dataSet]{dataSet} in the \hyperref[class:report]{report}.

\lsttext{dataSet}{dataSet}
The example shows the definition of a dataSet. The referenced dataGenerator \element{dg1} must be defined in the \hyperref[sec:listOfDataGenerators]{\element{listOfDataGenerators}}.
\begin{myXmlLst}{The SED-ML \code{dataSet} element, defining a data set containing the result of the referenced task}{lst:dataSet}
<listOfDataSets>
	<dataSet id="d1" name="v1 over time" dataReference="dg1" label="_1">
</listOfDataSets>
\end{myXmlLst}
